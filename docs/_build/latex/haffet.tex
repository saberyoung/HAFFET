%% Generated by Sphinx.
\def\sphinxdocclass{report}
\documentclass[letterpaper,10pt,english]{sphinxmanual}
\ifdefined\pdfpxdimen
   \let\sphinxpxdimen\pdfpxdimen\else\newdimen\sphinxpxdimen
\fi \sphinxpxdimen=.75bp\relax

\PassOptionsToPackage{warn}{textcomp}
\usepackage[utf8]{inputenc}
\ifdefined\DeclareUnicodeCharacter
% support both utf8 and utf8x syntaxes
  \ifdefined\DeclareUnicodeCharacterAsOptional
    \def\sphinxDUC#1{\DeclareUnicodeCharacter{"#1}}
  \else
    \let\sphinxDUC\DeclareUnicodeCharacter
  \fi
  \sphinxDUC{00A0}{\nobreakspace}
  \sphinxDUC{2500}{\sphinxunichar{2500}}
  \sphinxDUC{2502}{\sphinxunichar{2502}}
  \sphinxDUC{2514}{\sphinxunichar{2514}}
  \sphinxDUC{251C}{\sphinxunichar{251C}}
  \sphinxDUC{2572}{\textbackslash}
\fi
\usepackage{cmap}
\usepackage[T1]{fontenc}
\usepackage{amsmath,amssymb,amstext}
\usepackage{babel}



\usepackage{times}
\expandafter\ifx\csname T@LGR\endcsname\relax
\else
% LGR was declared as font encoding
  \substitutefont{LGR}{\rmdefault}{cmr}
  \substitutefont{LGR}{\sfdefault}{cmss}
  \substitutefont{LGR}{\ttdefault}{cmtt}
\fi
\expandafter\ifx\csname T@X2\endcsname\relax
  \expandafter\ifx\csname T@T2A\endcsname\relax
  \else
  % T2A was declared as font encoding
    \substitutefont{T2A}{\rmdefault}{cmr}
    \substitutefont{T2A}{\sfdefault}{cmss}
    \substitutefont{T2A}{\ttdefault}{cmtt}
  \fi
\else
% X2 was declared as font encoding
  \substitutefont{X2}{\rmdefault}{cmr}
  \substitutefont{X2}{\sfdefault}{cmss}
  \substitutefont{X2}{\ttdefault}{cmtt}
\fi


\usepackage[Bjarne]{fncychap}
\usepackage{sphinx}

\fvset{fontsize=\small}
\usepackage{geometry}


% Include hyperref last.
\usepackage{hyperref}
% Fix anchor placement for figures with captions.
\usepackage{hypcap}% it must be loaded after hyperref.
% Set up styles of URL: it should be placed after hyperref.
\urlstyle{same}

\usepackage{sphinxmessages}
\setcounter{tocdepth}{0}



\title{HAFFET Documentation}
\date{August 31, 2022}
\release{0.0.0.1}
\author{Sheng Yang}
\newcommand{\sphinxlogo}{\sphinxincludegraphics{logo.png}\par}
\renewcommand{\releasename}{Release}
\makeindex
\begin{document}

\pagestyle{empty}
\sphinxmaketitle
\pagestyle{plain}
\sphinxtableofcontents
\pagestyle{normal}
\phantomsection\label{\detokenize{index::doc}}



\chapter{Background}
\label{\detokenize{index:background}}
The progenitor scenarios of supernovae (SNe) are still open questions, and one approach to diagnose their physical origins is to investigate the bolometric light curves of a large set of SNe, fit them to theoretical models to estimate physical parameter distributions.
Such analysis from different studies often use different approaches and codes which makes the comparisons more difficult. A generic code\sphinxhyphen{}package to handle light curve fitting for transients with different types, from different surveys, in different cadences, is therefore useful to provide reliable results for comparison. For this purpose, we develop \sphinxtitleref{HAFFET}, a data\sphinxhyphen{}driven model fitter for transients.


\chapter{What is HAFFET?}
\label{\detokenize{index:what-is-haffet}}
\sphinxhref{https://sngyang.com/haffet}{HAFFET: Hybrid Analytic Flux FittEr for Transients} is an open source Python package to help analyze SN photometric and spectroscopic data.

The aim of \sphinxtitleref{HAFFET} is to handle observational data for a set of targets, to estimate their physical parameters, and visualize the population of inferred parameters. Therefore, there’re two classes defined, i.e. \sphinxtitleref{snobject} is to deal with data and fittings for one specific object, and \sphinxtitleref{snelist} is to organise the overall running for a list of objects. The inheritance scheme of \sphinxtitleref{HAFFET} is shown as directed flowchart as followed:

\noindent\sphinxincludegraphics[width=800\sphinxpxdimen]{{sdapy}.png}

As shown, \sphinxtitleref{HAFFET} provides utilities to:
\begin{itemize}
\item {} 
download SNe data from online sources:
\begin{enumerate}
\sphinxsetlistlabels{\arabic}{enumi}{enumii}{}{.}%
\item {} 
ZTF alert photometry/spectra from Growth marshal/fritz via \sphinxhref{https://github.com/MickaelRigault/ztfquery/tree/master/ztfquery}{ztfquery} (for ZTF internal collaborators).

\item {} 
\sphinxhref{https://ztfweb.ipac.caltech.edu/cgi-bin/requestForcedPhotometry.cgi/}{ZTF forced photometry services}.

\item {} 
\sphinxhref{https://fallingstar-data.com/forcedphot/}{ATLAS forced phtometry services}.

\item {} 
\sphinxhref{https://github.com/astrocatalogs/OACAPI}{Open Astronomy Catalog}

\item {} 
private lightcurves/spectra from users (take careful of the formats and keywords).

\end{enumerate}

\item {} 
intepolated multi band lightcurves with:
\begin{enumerate}
\sphinxsetlistlabels{\arabic}{enumi}{enumii}{}{.}%
\item {} 
Gaussian Process (via \sphinxhref{https://george.readthedocs.io/en/latest//}{george}).

\item {} 
fittings to analytic models, e.g. \sphinxhref{https://ui.adsabs.harvard.edu/abs/2009A\%26A...499..653B/abstract}{Bazin et al 2009}, \sphinxhref{https://iopscience.iop.org/article/10.3847/1538-4357/ab418c}{Villar et al 2019}.

\end{enumerate}

\item {} 
characterise the first light and rising of SNe with power law fits:
\begin{enumerate}
\sphinxsetlistlabels{\arabic}{enumi}{enumii}{}{.}%
\item {} 
on multi band photometry simultaneously (developed based on \sphinxurl{https://github.com/adamamiller/ztf\_early\_Ia\_2018}).

\item {} 
on different bands seperately.

\end{enumerate}

\item {} 
match epochs of different bands for colours, or the spectra energy distribution (SED) via:
\begin{enumerate}
\sphinxsetlistlabels{\arabic}{enumi}{enumii}{}{.}%
\item {} 
binning

\item {} 
GP interpolation

\item {} 
model fittings

\end{enumerate}

\item {} 
estimate bolometric LCs via:
\begin{enumerate}
\sphinxsetlistlabels{\arabic}{enumi}{enumii}{}{.}%
\item {} 
bolometric corrections defined in \sphinxhref{https://academic.oup.com/mnras/article/437/4/3848/1011706}{Lyman et al 2014} for stripped envolop SNe or SNe II, and \sphinxhref{https://ui.adsabs.harvard.edu/abs/2022arXiv220202059C}{Chen et al 2022} for SLSNe.

\item {} 
\sphinxhref{https://en.wikipedia.org/wiki/Black\_body}{diluted black body fits}.

\end{enumerate}

\item {} 
estimate host galaxy extinction by:
\begin{enumerate}
\sphinxsetlistlabels{\arabic}{enumi}{enumii}{}{.}%
\item {} 
comparing colours to intrinsic colours (e.g. \sphinxhref{https://ui.adsabs.harvard.edu/abs/2018A\&A...609A.135S}{Stritzinger et al 2018}).

\item {} 
the \sphinxhref{https://ui.adsabs.harvard.edu/abs/2012MNRAS.426.1465P}{pEW of Na ID line doublets}.

\end{enumerate}

\item {} 
fit the contructed bolometric lightcurves to different models, e.g.:
\begin{enumerate}
\sphinxsetlistlabels{\arabic}{enumi}{enumii}{}{.}%
\item {} 
the \sphinxhref{https://ui.adsabs.harvard.edu/abs/1982ApJ...253..785A}{Arnett models} for SNe Ia and core collapes during their main peaks.

\item {} 
the \sphinxhref{https://ui.adsabs.harvard.edu/abs/2019MNRAS.484.3941W}{gamma ray leakage tail model} for SNe Ia and core collapes at tail phases.

\item {} 
the shock cooling emission model \sphinxurl{https://arxiv.org/pdf/2007.08543.pdf} for SNe IIb or some Ibc that have early shock cooling tails.

\item {} 
the magnetar and CSM model (that were borrowed from \sphinxhref{https://mosfit.readthedocs.io/en/latest/}{MOSFIT}) for SLSNe, that is unlikely to be explained by the Arnett models.

\end{enumerate}

\item {} 
identify and fit the absorption minima of spectral lines with:
\begin{enumerate}
\sphinxsetlistlabels{\arabic}{enumi}{enumii}{}{.}%
\item {} 
Gaussian

\item {} 
Viglot

\end{enumerate}

\item {} 
scatter all above features into parameter spaces for sample exploration

\end{itemize}

There’re 3 modes to run \sphinxtitleref{HAFFET}:
\begin{itemize}
\item {} 
\sphinxtitleref{HAFFET} can be called as a python package:

\begin{sphinxVerbatim}[commandchars=\\\{\}]
\PYGZgt{}\PYGZgt{}\PYGZgt{} python
Python \PYG{l+m}{3}.6.7 \PYG{p}{|} packaged by conda\PYGZhy{}forge \PYG{p}{|} \PYG{o}{(}default, Nov  \PYG{l+m}{6} \PYG{l+m}{2019}, \PYG{l+m}{16}:03:31\PYG{o}{)}
\PYG{o}{[}GCC Clang \PYG{l+m}{9}.0.0 \PYG{o}{(}tags/RELEASE\PYGZus{}900/final\PYG{o}{)}\PYG{o}{]} on darwin
Type \PYG{l+s+s2}{\PYGZdq{}help\PYGZdq{}}, \PYG{l+s+s2}{\PYGZdq{}copyright\PYGZdq{}}, \PYG{l+s+s2}{\PYGZdq{}credits\PYGZdq{}} or \PYG{l+s+s2}{\PYGZdq{}license\PYGZdq{}} \PYG{k}{for} more information.
\PYGZgt{}\PYGZgt{}\PYGZgt{} from sdapy import snerun
\PYGZgt{}\PYGZgt{}\PYGZgt{} \PYG{n+nv}{ztfp} \PYG{o}{=} snerun.snobject\PYG{o}{(}objid, \PYG{n+nv}{z}\PYG{o}{=}.1, \PYG{n+nv}{ra}\PYG{o}{=}\PYG{l+s+s1}{\PYGZsq{}13:26:29.65\PYGZsq{}}, \PYG{n+nv}{dec}\PYG{o}{=}\PYG{l+s+s1}{\PYGZsq{}+36:00:31.1\PYGZsq{}}\PYG{o}{)}
\end{sphinxVerbatim}

\item {} 
\sphinxtitleref{HAFFET} provides a Graphical user interface (GUI):

\begin{sphinxVerbatim}[commandchars=\\\{\}]
\PYGZgt{}\PYGZgt{}\PYGZgt{} sdapy\PYGZus{}gui \PYGZhy{}h
\end{sphinxVerbatim}

\item {} 
\sphinxtitleref{HAFFET} provides a executable Python file:

\begin{sphinxVerbatim}[commandchars=\\\{\}]
\PYGZgt{}\PYGZgt{}\PYGZgt{} sdapy\PYGZus{}run \PYGZhy{}h
\end{sphinxVerbatim}

\end{itemize}

For fittings, there’re two approaches implemented as hyperparameter optimization routine:
\begin{itemize}
\item {} 
scipy.optimize (\sphinxurl{https://docs.scipy.org/doc/scipy/reference/optimize.html})

\item {} 
emcee (\sphinxurl{https://emcee.readthedocs.io/en/stable/})

\end{itemize}


\chapter{Documentation}
\label{\detokenize{index:documentation}}

\section{Installation}
\label{\detokenize{install:installation}}\label{\detokenize{install::doc}}

\subsection{Requirements}
\label{\detokenize{install:requirements}}
\sphinxtitleref{HAFFET} depends on several python libraries, e.g.
\sphinxhref{https://matplotlib.org/}{matplotlib}, \sphinxhref{https://www.astropy.org/}{astropy}, etc
(see \sphinxhref{https://github.com/saberyoung/sn\_data\_analysis/blob/master/requirements.txt}{full list}),
and all of them could be installed via \sphinxtitleref{pip}, e.g.:

\begin{sphinxVerbatim}[commandchars=\\\{\}]
\PYG{n}{pip} \PYG{n}{install} \PYG{o}{\PYGZhy{}}\PYG{n}{r} \PYG{n}{requirements}\PYG{o}{.}\PYG{n}{txt}
\end{sphinxVerbatim}

\begin{sphinxadmonition}{note}{Note:}\begin{itemize}
\item {} 
Linux and MAC OS systems have been tested, and for Windows users, there’s a \sphinxhref{https://github.com/saberyoung/sn\_data\_analysis/tree/Docker}{docker version} in preparation.

\item {} 
\sphinxtitleref{HAFFET} is tested under Python 3.

\end{itemize}
\end{sphinxadmonition}


\subsection{Source installation with Pypi (RECOMMENDED but not available yet)}
\label{\detokenize{install:source-installation-with-pypi-recommended-but-not-available-yet}}
It is possible to build the latest \sphinxcode{\sphinxupquote{HAFFET}} with \sphinxhref{http://www.pip-installer.org}{pip}

\begin{sphinxVerbatim}[commandchars=\\\{\}]
\PYG{n}{pip} \PYG{n}{install} \PYG{o}{\PYGZhy{}}\PYG{o}{\PYGZhy{}}\PYG{n}{user} \PYG{n}{haffet}
\end{sphinxVerbatim}

If you have installed with \sphinxcode{\sphinxupquote{pip}}, you can keep your installation up to date
by upgrading from time to time:

\begin{sphinxVerbatim}[commandchars=\\\{\}]
\PYG{n}{pip} \PYG{n}{install} \PYG{o}{\PYGZhy{}}\PYG{o}{\PYGZhy{}}\PYG{n}{user} \PYG{o}{\PYGZhy{}}\PYG{o}{\PYGZhy{}}\PYG{n}{upgrade} \PYG{n}{haffet}
\end{sphinxVerbatim}


\subsection{Almost\sphinxhyphen{}as\sphinxhyphen{}quick installation from official source release}
\label{\detokenize{install:almost-as-quick-installation-from-official-source-release}}
\sphinxcode{\sphinxupquote{HAFFET}} is also available in the
\sphinxhref{https://github.com/saberyoung/sn\_data\_analysis}{Github}. You can
download and build it with:

\begin{sphinxVerbatim}[commandchars=\\\{\}]
\PYG{n}{python} \PYG{n}{setup}\PYG{o}{.}\PYG{n}{py} \PYG{n}{install} \PYG{o}{\PYGZhy{}}\PYG{o}{\PYGZhy{}}\PYG{n}{user}
\end{sphinxVerbatim}


\subsection{Check}
\label{\detokenize{install:check}}
If everything goes fine, you can test it:

\begin{sphinxVerbatim}[commandchars=\\\{\}]
\PYG{n}{python}
\PYG{o}{\PYGZgt{}\PYGZgt{}}\PYG{o}{\PYGZgt{}} \PYG{k+kn}{import} \PYG{n+nn}{sdapy}
\PYG{o}{\PYGZgt{}\PYGZgt{}}\PYG{o}{\PYGZgt{}} \PYG{n}{sdapy}\PYG{o}{.}\PYG{n}{\PYGZus{}\PYGZus{}version\PYGZus{}\PYGZus{}}
\PYG{l+s+s1}{\PYGZsq{}}\PYG{l+s+s1}{0.0.0.1}\PYG{l+s+s1}{\PYGZsq{}}
\end{sphinxVerbatim}


\subsection{Clean}
\label{\detokenize{install:clean}}
When you run “python setup.py”, temporary build products are placed in the
“build” directory. If you want to clean out and remove the \sphinxcode{\sphinxupquote{build}} directory,
then run:

\begin{sphinxVerbatim}[commandchars=\\\{\}]
\PYG{n}{python} \PYG{n}{setup}\PYG{o}{.}\PYG{n}{py} \PYG{n}{clean} \PYG{o}{\PYGZhy{}}\PYG{o}{\PYGZhy{}}\PYG{n+nb}{all}
\end{sphinxVerbatim}


\subsection{Uninstall}
\label{\detokenize{install:uninstall}}
For uninstallation, one can easily delete the files directly.
In order to know the file path, you should start python in correct environment and do:

\begin{sphinxVerbatim}[commandchars=\\\{\}]
\PYG{g+gp}{\PYGZgt{}\PYGZgt{}\PYGZgt{} }\PYG{k+kn}{import} \PYG{n+nn}{sdapy}
\PYG{g+gp}{\PYGZgt{}\PYGZgt{}\PYGZgt{} }\PYG{n}{sdapy}\PYG{o}{.}\PYG{n}{\PYGZus{}\PYGZus{}path\PYGZus{}\PYGZus{}}
\end{sphinxVerbatim}

Another approach is to remove it via pip:

\begin{sphinxVerbatim}[commandchars=\\\{\}]
\PYG{n}{pip} \PYG{n}{uninstall} \PYG{n}{haffet}
\end{sphinxVerbatim}


\section{Getting started}
\label{\detokenize{tutorial:getting-started}}\label{\detokenize{tutorial:getstart}}\label{\detokenize{tutorial::doc}}
Once installed, \sphinxtitleref{HAFFET} can be run from any directory, but a data directory should be defined to deal with cached data and fitting samples. The default data directory is a folder in the current directory, i.e. ./data/. It would be more convenient to make a new directory for your project, and refer it in your shell profile:

\begin{sphinxVerbatim}[commandchars=\\\{\}]
\PYGZgt{}\PYGZgt{}\PYGZgt{} mkdir /xxx/yyy/haffet\PYGZus{}data
\PYGZgt{}\PYGZgt{}\PYGZgt{} vi \PYGZti{}/.bashrc

\PYG{c+c1}{\PYGZsh{} add the line below to your shell profiles, e.g. bash, zsh, etc.}
\PYGZgt{}\PYGZgt{}\PYGZgt{} \PYG{n+nb}{export} \PYG{n+nv}{ZTFDATA}\PYG{o}{=}\PYG{l+s+s2}{\PYGZdq{}/xxx/yyy/haffet\PYGZus{}data\PYGZdq{}}
\PYGZgt{}\PYGZgt{}\PYGZgt{} \PYG{n+nb}{source} \PYGZti{}/.bashrc
\end{sphinxVerbatim}

The variable \sphinxtitleref{ZTFDATA} is then defined serving as the data directory for \sphinxtitleref{HAFFET}. The structure of the data directory would be:

\begin{sphinxVerbatim}[commandchars=\\\{\}]
\PYG{l+s+sb}{`}ZTFDATA\PYG{l+s+sb}{`}/
   auth.txt
   bc\PYGZus{}table.txt
   bts\PYGZus{}meta.txt
   c10\PYGZus{}template.txt
   csm\PYGZus{}table.txt
   default\PYGZus{}par.txt
   individual\PYGZus{}par.txt
   logo.txt
   oac\PYGZus{}meta.txt
   marshal/
      lightcurves/
         \PYG{l+s+sb}{`}objid\PYG{l+s+sb}{`}/
            ***.csv
      spectra/
         \PYG{l+s+sb}{`}objid\PYG{l+s+sb}{`}/
            ***.ascii
   fritz/
      lightcurves/
         \PYG{l+s+sb}{`}objid\PYG{l+s+sb}{`}/
            ***.csv
      sample/
         fritz\PYGZus{}groups.json
      spectra/
         \PYG{l+s+sb}{`}objid\PYG{l+s+sb}{`}/
            ***.ascii
   ForcePhot/
      \PYG{l+s+sb}{`}objid\PYG{l+s+sb}{`}/
         ***.csv
   ForcePhot\PYGZus{}atlas/
      \PYG{l+s+sb}{`}objid\PYG{l+s+sb}{`}/
         ***.csv
         ***.txt
   oac/
      \PYG{l+s+sb}{`}objid\PYG{l+s+sb}{`}/
         ***.csv
\end{sphinxVerbatim}

\sphinxtitleref{HAFFET} can be invoked via three approaches:


\subsection{1. Run with the GUI}
\label{\detokenize{tutorial:run-with-the-gui}}

\subsection{2. Run with the executable file}
\label{\detokenize{tutorial:run-with-the-executable-file}}

\subsection{3. Run as a Python package}
\label{\detokenize{tutorial:run-as-a-python-package}}
\begin{DUlineblock}{0em}
\item[] \DUrole{xref,std,std-ref}{\sphinxhyphen{}\textgreater{} snobject}
\item[] \DUrole{xref,std,std-ref}{\sphinxhyphen{}\textgreater{} snelist}
\end{DUlineblock}

\begin{sphinxadmonition}{note}{Note:}
An example data directory can be found at \sphinxurl{https://stockholmuniversity.box.com/s/2c3z8yrgvd9zumm4c35u8jtvapyva3e1}.
\end{sphinxadmonition}


\section{Issues}
\label{\detokenize{issues:issues}}\label{\detokenize{issues::doc}}
Please report any issues \sphinxhref{https://github.com/saberyoung/sn\_data\_analysis/issues}{here},
or drop us an \sphinxhref{mailto:saberyoung@gmail.com}{email}.


\section{Reference}
\label{\detokenize{reference:reference}}\label{\detokenize{reference::doc}}

\subsection{cite HAFFET}
\label{\detokenize{reference:cite-haffet}}\begin{itemize}
\item {} 
\sphinxhref{https://ui.adsabs.harvard.edu/abs/2021A\%26A...655A..90Y/abstract}{paper}

\item {} 
\sphinxcode{\sphinxupquote{bib file}}

\item {} 
\sphinxhref{https://github.com/saberyoung/HAFFET}{github}

\item {} 
\sphinxhref{https://haffet.readthedocs.io/en/latest/}{tutorial}

\end{itemize}


\subsection{Papers using HAFFET}
\label{\detokenize{reference:papers-using-haffet}}\begin{itemize}
\item {} 
\sphinxurl{https://ui.adsabs.harvard.edu/abs/}….

\item {} 
\sphinxurl{https://ui.adsabs.harvard.edu/abs/}….

\end{itemize}


\section{Frequently Asked Questions}
\label{\detokenize{faq:frequently-asked-questions}}\label{\detokenize{faq::doc}}

\subsection{\sphinxstyleliteralintitle{\sphinxupquote{ImportError: /lib64/libstdc++.so.6}}}
\label{\detokenize{faq:importerror-lib64-libstdc-so-6}}\begin{itemize}
\item {} 
reason: the Gcc dynamic library version is too old.

\item {} 
how to solve::
\#edit bash
LD\_LIBRARY\_PATH=/home/feng/anaconda3/lib:\$LD\_LIBRARY\_PATH
export LD\_LIBRARY\_PATH

\end{itemize}


\subsection{\sphinxstyleliteralintitle{\sphinxupquote{ERROR: setuptools 1.0 or later is required by astropy\sphinxhyphen{}helpers}}}
\label{\detokenize{faq:error-setuptools-1-0-or-later-is-required-by-astropy-helpers}}\begin{itemize}
\item {} 
reason: when installing astroquery, setuptools version is low

\item {} 
how to solve:

\begin{sphinxVerbatim}[commandchars=\\\{\}]
\PYG{g+gp}{\PYGZgt{}\PYGZgt{}\PYGZgt{} }\PYG{k+kn}{import} \PYG{n+nn}{setuptools}
\PYG{g+gp}{\PYGZgt{}\PYGZgt{}\PYGZgt{} }\PYG{n}{setuptools}\PYG{o}{.}\PYG{n}{\PYGZus{}\PYGZus{}path\PYGZus{}\PYGZus{}}

\PYG{g+go}{see where setuptools is called, and then upgrade it: conda install setuptools/pip install setuptools \PYGZhy{}\PYGZhy{}upgrade/etc}
\end{sphinxVerbatim}

\end{itemize}


\subsection{Install multiprocessing}
\label{\detokenize{faq:install-multiprocessing}}\begin{itemize}
\item {} 
reason: for pip, default version of multiprocessing is 2.x, while we use python 3.

\item {} 
how to solve:

\begin{sphinxVerbatim}[commandchars=\\\{\}]
\PYG{g+gp}{\PYGZgt{}\PYGZgt{}\PYGZgt{} }\PYG{n}{pip3} \PYG{n}{install} \PYG{n}{multiprocessing}
\end{sphinxVerbatim}

\end{itemize}


\subsection{No plots showing in the Jupyter notebook}
\label{\detokenize{faq:no-plots-showing-in-the-jupyter-notebook}}\begin{itemize}
\item {} 
reason: \sphinxurl{https://stackoverflow.com/questions/43027980/purpose-of-matplotlib-inline}

\item {} 
how to solve:

\begin{sphinxVerbatim}[commandchars=\\\{\}]
\PYG{g+gp}{\PYGZgt{}\PYGZgt{}\PYGZgt{} }\PYG{o}{\PYGZpc{}}\PYG{n}{matplotlib} \PYG{n}{inline}

\PYG{g+go}{add this in your juperte notebook, when you import matplotlib}
\end{sphinxVerbatim}

\end{itemize}


\section{Licenses}
\label{\detokenize{license:licenses}}\label{\detokenize{license::doc}}

\subsection{HAFFET License}
\label{\detokenize{license:haffet-license}}
HAFFET is licensed under the GNU General Public License.
\begin{quote}
\begin{quote}
\begin{description}
\item[{GNU GENERAL PUBLIC LICENSE}] \leavevmode
Version 2, June 1991

\end{description}
\end{quote}
\begin{description}
\item[{Copyright (C) 1989, 1991 Free Software Foundation, Inc.}] \leavevmode
51 Franklin St, Fifth Floor, Boston, MA  02110\sphinxhyphen{}1301  USA

\end{description}

Everyone is permitted to copy and distribute verbatim copies
of this license document, but changing it is not allowed.
\begin{quote}
\begin{quote}

Preamble
\end{quote}

The licenses for most software are designed to take away your
\end{quote}
\end{quote}

freedom to share and change it.  By contrast, the GNU General Public
License is intended to guarantee your freedom to share and change free
software\textendash{}to make sure the software is free for all its users.  This
General Public License applies to most of the Free Software
Foundation’s software and to any other program whose authors commit to
using it.  (Some other Free Software Foundation software is covered by
the GNU Library General Public License instead.)  You can apply it to
your programs, too.
\begin{quote}

When we speak of free software, we are referring to freedom, not
\end{quote}

price.  Our General Public Licenses are designed to make sure that you
have the freedom to distribute copies of free software (and charge for
this service if you wish), that you receive source code or can get it
if you want it, that you can change the software or use pieces of it
in new free programs; and that you know you can do these things.
\begin{quote}

To protect your rights, we need to make restrictions that forbid
\end{quote}

anyone to deny you these rights or to ask you to surrender the rights.
These restrictions translate to certain responsibilities for you if you
distribute copies of the software, or if you modify it.
\begin{quote}

For example, if you distribute copies of such a program, whether
\end{quote}

gratis or for a fee, you must give the recipients all the rights that
you have.  You must make sure that they, too, receive or can get the
source code.  And you must show them these terms so they know their
rights.
\begin{quote}

We protect your rights with two steps: (1) copyright the software, and
\end{quote}

(2) offer you this license which gives you legal permission to copy,
distribute and/or modify the software.
\begin{quote}

Also, for each author’s protection and ours, we want to make certain
\end{quote}

that everyone understands that there is no warranty for this free
software.  If the software is modified by someone else and passed on, we
want its recipients to know that what they have is not the original, so
that any problems introduced by others will not reflect on the original
authors’ reputations.
\begin{quote}

Finally, any free program is threatened constantly by software
\end{quote}

patents.  We wish to avoid the danger that redistributors of a free
program will individually obtain patent licenses, in effect making the
program proprietary.  To prevent this, we have made it clear that any
patent must be licensed for everyone’s free use or not licensed at all.
\begin{quote}

The precise terms and conditions for copying, distribution and
\end{quote}

modification follow.
\begin{quote}
\begin{quote}
\begin{quote}

GNU GENERAL PUBLIC LICENSE
\end{quote}

TERMS AND CONDITIONS FOR COPYING, DISTRIBUTION AND MODIFICATION
\end{quote}
\begin{enumerate}
\sphinxsetlistlabels{\arabic}{enumi}{enumii}{}{.}%
\setcounter{enumi}{-1}
\item {} 
This License applies to any program or other work which contains

\end{enumerate}
\end{quote}

a notice placed by the copyright holder saying it may be distributed
under the terms of this General Public License.  The “Program”, below,
refers to any such program or work, and a “work based on the Program”
means either the Program or any derivative work under copyright law:
that is to say, a work containing the Program or a portion of it,
either verbatim or with modifications and/or translated into another
language.  (Hereinafter, translation is included without limitation in
the term “modification”.)  Each licensee is addressed as “you”.

Activities other than copying, distribution and modification are not
covered by this License; they are outside its scope.  The act of
running the Program is not restricted, and the output from the Program
is covered only if its contents constitute a work based on the
Program (independent of having been made by running the Program).
Whether that is true depends on what the Program does.
\begin{enumerate}
\sphinxsetlistlabels{\arabic}{enumi}{enumii}{}{.}%
\item {} 
You may copy and distribute verbatim copies of the Program’s

\end{enumerate}

source code as you receive it, in any medium, provided that you
conspicuously and appropriately publish on each copy an appropriate
copyright notice and disclaimer of warranty; keep intact all the
notices that refer to this License and to the absence of any warranty;
and give any other recipients of the Program a copy of this License
along with the Program.

You may charge a fee for the physical act of transferring a copy, and
you may at your option offer warranty protection in exchange for a fee.
\begin{enumerate}
\sphinxsetlistlabels{\arabic}{enumi}{enumii}{}{.}%
\setcounter{enumi}{1}
\item {} 
You may modify your copy or copies of the Program or any portion

\end{enumerate}

of it, thus forming a work based on the Program, and copy and
distribute such modifications or work under the terms of Section 1
above, provided that you also meet all of these conditions:
\begin{quote}

a) You must cause the modified files to carry prominent notices
stating that you changed the files and the date of any change.

b) You must cause any work that you distribute or publish, that in
whole or in part contains or is derived from the Program or any
part thereof, to be licensed as a whole at no charge to all third
parties under the terms of this License.

c) If the modified program normally reads commands interactively
when run, you must cause it, when started running for such
interactive use in the most ordinary way, to print or display an
announcement including an appropriate copyright notice and a
notice that there is no warranty (or else, saying that you provide
a warranty) and that users may redistribute the program under
these conditions, and telling the user how to view a copy of this
License.  (Exception: if the Program itself is interactive but
does not normally print such an announcement, your work based on
the Program is not required to print an announcement.)
\end{quote}

These requirements apply to the modified work as a whole.  If
identifiable sections of that work are not derived from the Program,
and can be reasonably considered independent and separate works in
themselves, then this License, and its terms, do not apply to those
sections when you distribute them as separate works.  But when you
distribute the same sections as part of a whole which is a work based
on the Program, the distribution of the whole must be on the terms of
this License, whose permissions for other licensees extend to the
entire whole, and thus to each and every part regardless of who wrote it.

Thus, it is not the intent of this section to claim rights or contest
your rights to work written entirely by you; rather, the intent is to
exercise the right to control the distribution of derivative or
collective works based on the Program.

In addition, mere aggregation of another work not based on the Program
with the Program (or with a work based on the Program) on a volume of
a storage or distribution medium does not bring the other work under
the scope of this License.
\begin{enumerate}
\sphinxsetlistlabels{\arabic}{enumi}{enumii}{}{.}%
\setcounter{enumi}{2}
\item {} 
You may copy and distribute the Program (or a work based on it,

\end{enumerate}

under Section 2) in object code or executable form under the terms of
Sections 1 and 2 above provided that you also do one of the following:
\begin{quote}

a) Accompany it with the complete corresponding machine\sphinxhyphen{}readable
source code, which must be distributed under the terms of Sections
1 and 2 above on a medium customarily used for software interchange; or,

b) Accompany it with a written offer, valid for at least three
years, to give any third party, for a charge no more than your
cost of physically performing source distribution, a complete
machine\sphinxhyphen{}readable copy of the corresponding source code, to be
distributed under the terms of Sections 1 and 2 above on a medium
customarily used for software interchange; or,

c) Accompany it with the information you received as to the offer
to distribute corresponding source code.  (This alternative is
allowed only for noncommercial distribution and only if you
received the program in object code or executable form with such
an offer, in accord with Subsection b above.)
\end{quote}

The source code for a work means the preferred form of the work for
making modifications to it.  For an executable work, complete source
code means all the source code for all modules it contains, plus any
associated interface definition files, plus the scripts used to
control compilation and installation of the executable.  However, as a
special exception, the source code distributed need not include
anything that is normally distributed (in either source or binary
form) with the major components (compiler, kernel, and so on) of the
operating system on which the executable runs, unless that component
itself accompanies the executable.

If distribution of executable or object code is made by offering
access to copy from a designated place, then offering equivalent
access to copy the source code from the same place counts as
distribution of the source code, even though third parties are not
compelled to copy the source along with the object code.
\begin{enumerate}
\sphinxsetlistlabels{\arabic}{enumi}{enumii}{}{.}%
\setcounter{enumi}{3}
\item {} 
You may not copy, modify, sublicense, or distribute the Program

\end{enumerate}

except as expressly provided under this License.  Any attempt
otherwise to copy, modify, sublicense or distribute the Program is
void, and will automatically terminate your rights under this License.
However, parties who have received copies, or rights, from you under
this License will not have their licenses terminated so long as such
parties remain in full compliance.
\begin{enumerate}
\sphinxsetlistlabels{\arabic}{enumi}{enumii}{}{.}%
\setcounter{enumi}{4}
\item {} 
You are not required to accept this License, since you have not

\end{enumerate}

signed it.  However, nothing else grants you permission to modify or
distribute the Program or its derivative works.  These actions are
prohibited by law if you do not accept this License.  Therefore, by
modifying or distributing the Program (or any work based on the
Program), you indicate your acceptance of this License to do so, and
all its terms and conditions for copying, distributing or modifying
the Program or works based on it.
\begin{enumerate}
\sphinxsetlistlabels{\arabic}{enumi}{enumii}{}{.}%
\setcounter{enumi}{5}
\item {} 
Each time you redistribute the Program (or any work based on the

\end{enumerate}

Program), the recipient automatically receives a license from the
original licensor to copy, distribute or modify the Program subject to
these terms and conditions.  You may not impose any further
restrictions on the recipients’ exercise of the rights granted herein.
You are not responsible for enforcing compliance by third parties to
this License.
\begin{enumerate}
\sphinxsetlistlabels{\arabic}{enumi}{enumii}{}{.}%
\setcounter{enumi}{6}
\item {} 
If, as a consequence of a court judgment or allegation of patent

\end{enumerate}

infringement or for any other reason (not limited to patent issues),
conditions are imposed on you (whether by court order, agreement or
otherwise) that contradict the conditions of this License, they do not
excuse you from the conditions of this License.  If you cannot
distribute so as to satisfy simultaneously your obligations under this
License and any other pertinent obligations, then as a consequence you
may not distribute the Program at all.  For example, if a patent
license would not permit royalty\sphinxhyphen{}free redistribution of the Program by
all those who receive copies directly or indirectly through you, then
the only way you could satisfy both it and this License would be to
refrain entirely from distribution of the Program.

If any portion of this section is held invalid or unenforceable under
any particular circumstance, the balance of the section is intended to
apply and the section as a whole is intended to apply in other
circumstances.

It is not the purpose of this section to induce you to infringe any
patents or other property right claims or to contest validity of any
such claims; this section has the sole purpose of protecting the
integrity of the free software distribution system, which is
implemented by public license practices.  Many people have made
generous contributions to the wide range of software distributed
through that system in reliance on consistent application of that
system; it is up to the author/donor to decide if he or she is willing
to distribute software through any other system and a licensee cannot
impose that choice.

This section is intended to make thoroughly clear what is believed to
be a consequence of the rest of this License.
\begin{enumerate}
\sphinxsetlistlabels{\arabic}{enumi}{enumii}{}{.}%
\setcounter{enumi}{7}
\item {} 
If the distribution and/or use of the Program is restricted in

\end{enumerate}

certain countries either by patents or by copyrighted interfaces, the
original copyright holder who places the Program under this License
may add an explicit geographical distribution limitation excluding
those countries, so that distribution is permitted only in or among
countries not thus excluded.  In such case, this License incorporates
the limitation as if written in the body of this License.
\begin{enumerate}
\sphinxsetlistlabels{\arabic}{enumi}{enumii}{}{.}%
\setcounter{enumi}{8}
\item {} 
The Free Software Foundation may publish revised and/or new versions

\end{enumerate}

of the General Public License from time to time.  Such new versions will
be similar in spirit to the present version, but may differ in detail to
address new problems or concerns.

Each version is given a distinguishing version number.  If the Program
specifies a version number of this License which applies to it and “any
later version”, you have the option of following the terms and conditions
either of that version or of any later version published by the Free
Software Foundation.  If the Program does not specify a version number of
this License, you may choose any version ever published by the Free Software
Foundation.
\begin{enumerate}
\sphinxsetlistlabels{\arabic}{enumi}{enumii}{}{.}%
\setcounter{enumi}{9}
\item {} 
If you wish to incorporate parts of the Program into other free

\end{enumerate}

programs whose distribution conditions are different, write to the author
to ask for permission.  For software which is copyrighted by the Free
Software Foundation, write to the Free Software Foundation; we sometimes
make exceptions for this.  Our decision will be guided by the two goals
of preserving the free status of all derivatives of our free software and
of promoting the sharing and reuse of software generally.
\begin{quote}
\begin{quote}

NO WARRANTY
\end{quote}
\begin{enumerate}
\sphinxsetlistlabels{\arabic}{enumi}{enumii}{}{.}%
\setcounter{enumi}{10}
\item {} 
BECAUSE THE PROGRAM IS LICENSED FREE OF CHARGE, THERE IS NO WARRANTY

\end{enumerate}
\end{quote}

FOR THE PROGRAM, TO THE EXTENT PERMITTED BY APPLICABLE LAW.  EXCEPT WHEN
OTHERWISE STATED IN WRITING THE COPYRIGHT HOLDERS AND/OR OTHER PARTIES
PROVIDE THE PROGRAM “AS IS” WITHOUT WARRANTY OF ANY KIND, EITHER EXPRESSED
OR IMPLIED, INCLUDING, BUT NOT LIMITED TO, THE IMPLIED WARRANTIES OF
MERCHANTABILITY AND FITNESS FOR A PARTICULAR PURPOSE.  THE ENTIRE RISK AS
TO THE QUALITY AND PERFORMANCE OF THE PROGRAM IS WITH YOU.  SHOULD THE
PROGRAM PROVE DEFECTIVE, YOU ASSUME THE COST OF ALL NECESSARY SERVICING,
REPAIR OR CORRECTION.
\begin{enumerate}
\sphinxsetlistlabels{\arabic}{enumi}{enumii}{}{.}%
\setcounter{enumi}{11}
\item {} 
IN NO EVENT UNLESS REQUIRED BY APPLICABLE LAW OR AGREED TO IN WRITING

\end{enumerate}

WILL ANY COPYRIGHT HOLDER, OR ANY OTHER PARTY WHO MAY MODIFY AND/OR
REDISTRIBUTE THE PROGRAM AS PERMITTED ABOVE, BE LIABLE TO YOU FOR DAMAGES,
INCLUDING ANY GENERAL, SPECIAL, INCIDENTAL OR CONSEQUENTIAL DAMAGES ARISING
OUT OF THE USE OR INABILITY TO USE THE PROGRAM (INCLUDING BUT NOT LIMITED
TO LOSS OF DATA OR DATA BEING RENDERED INACCURATE OR LOSSES SUSTAINED BY
YOU OR THIRD PARTIES OR A FAILURE OF THE PROGRAM TO OPERATE WITH ANY OTHER
PROGRAMS), EVEN IF SUCH HOLDER OR OTHER PARTY HAS BEEN ADVISED OF THE
POSSIBILITY OF SUCH DAMAGES.
\begin{quote}
\begin{quote}
\begin{quote}

END OF TERMS AND CONDITIONS
\end{quote}

How to Apply These Terms to Your New Programs
\end{quote}

If you develop a new program, and you want it to be of the greatest
\end{quote}

possible use to the public, the best way to achieve this is to make it
free software which everyone can redistribute and change under these terms.
\begin{quote}

To do so, attach the following notices to the program.  It is safest
\end{quote}

to attach them to the start of each source file to most effectively
convey the exclusion of warranty; and each file should have at least
the “copyright” line and a pointer to where the full notice is found.
\begin{quote}

\textless{}one line to give the program’s name and a brief idea of what it does.\textgreater{}
Copyright (C) \textless{}year\textgreater{}  \textless{}name of author\textgreater{}

This program is free software; you can redistribute it and/or modify
it under the terms of the GNU General Public License as published by
the Free Software Foundation; either version 2 of the License, or
(at your option) any later version.

This program is distributed in the hope that it will be useful,
but WITHOUT ANY WARRANTY; without even the implied warranty of
MERCHANTABILITY or FITNESS FOR A PARTICULAR PURPOSE.  See the
GNU General Public License for more details.

You should have received a copy of the GNU General Public License
along with this program; if not, write to the Free Software
Foundation, Inc., 51 Franklin St, Fifth Floor, Boston, MA  02110\sphinxhyphen{}1301  USA
\end{quote}

Also add information on how to contact you by electronic and paper mail.

If the program is interactive, make it output a short notice like this
when it starts in an interactive mode:
\begin{quote}

Gnomovision version 69, Copyright (C) year name of author
Gnomovision comes with ABSOLUTELY NO WARRANTY; for details type {\color{red}\bfseries{}\textasciigrave{}}show w’.
This is free software, and you are welcome to redistribute it
under certain conditions; type {\color{red}\bfseries{}\textasciigrave{}}show c’ for details.
\end{quote}

The hypothetical commands {\color{red}\bfseries{}\textasciigrave{}}show w’ and {\color{red}\bfseries{}\textasciigrave{}}show c’ should show the appropriate
parts of the General Public License.  Of course, the commands you use may
be called something other than {\color{red}\bfseries{}\textasciigrave{}}show w’ and {\color{red}\bfseries{}\textasciigrave{}}show c’; they could even be
mouse\sphinxhyphen{}clicks or menu items\textendash{}whatever suits your program.

You should also get your employer (if you work as a programmer) or your
school, if any, to sign a “copyright disclaimer” for the program, if
necessary.  Here is a sample; alter the names:
\begin{quote}

Yoyodyne, Inc., hereby disclaims all copyright interest in the program
{\color{red}\bfseries{}\textasciigrave{}}Gnomovision’ (which makes passes at compilers) written by James Hacker.

\textless{}signature of Ty Coon\textgreater{}, 1 April 1989
Ty Coon, President of Vice
\end{quote}

This General Public License does not permit incorporating your program into
proprietary programs.  If your program is a subroutine library, you may
consider it more useful to permit linking proprietary applications with the
library.  If this is what you want to do, use the GNU Library General
Public License instead of this License.


\chapter{API}
\label{\detokenize{index:api}}

\section{\sphinxstyleliteralintitle{\sphinxupquote{snelist}} \textendash{} class deal with a list of objects}
\label{\detokenize{snelist:snelist-class-deal-with-a-list-of-objects}}\label{\detokenize{snelist:snelist}}\label{\detokenize{snelist::doc}}

\begin{savenotes}\sphinxatlongtablestart\begin{longtable}[c]{\X{1}{2}\X{1}{2}}
\hline

\endfirsthead

\multicolumn{2}{c}%
{\makebox[0pt]{\sphinxtablecontinued{\tablename\ \thetable{} \textendash{} continued from previous page}}}\\
\hline

\endhead

\hline
\multicolumn{2}{r}{\makebox[0pt][r]{\sphinxtablecontinued{Continued on next page}}}\\
\endfoot

\endlastfoot

{\hyperref[\detokenize{generated/sdapy.snerun.snelist:sdapy.snerun.snelist}]{\sphinxcrossref{\sphinxcode{\sphinxupquote{snelist}}}}}({[}ax{]})
&
snelist: define a list of \sphinxstyleemphasis{snobject}, handle their data and fittings, aimed for a population study
\\
\hline
\end{longtable}\sphinxatlongtableend\end{savenotes}


\subsection{sdapy.snerun.snelist}
\label{\detokenize{generated/sdapy.snerun.snelist:sdapy-snerun-snelist}}\label{\detokenize{generated/sdapy.snerun.snelist::doc}}\index{snelist (class in sdapy.snerun)@\spxentry{snelist}\spxextra{class in sdapy.snerun}}

\begin{fulllineitems}
\phantomsection\label{\detokenize{generated/sdapy.snerun.snelist:sdapy.snerun.snelist}}\pysiglinewithargsret{\sphinxbfcode{\sphinxupquote{class }}\sphinxcode{\sphinxupquote{sdapy.snerun.}}\sphinxbfcode{\sphinxupquote{snelist}}}{\emph{ax=None}, \emph{**kwargs}}{}
snelist: define a list of \sphinxstyleemphasis{snobject}, handle their data and fittings, aimed for a population study


\sphinxstrong{See also:}

\begin{description}
\item[{{\hyperref[\detokenize{generated/sdapy.snerun.snobject:sdapy.snerun.snobject}]{\sphinxcrossref{\sphinxcode{\sphinxupquote{snobject}}}}}}] \leavevmode
\end{description}


\subsubsection*{Notes}

Take careful of meta table of \sphinxstyleemphasis{snelist}, especially their types.
\subsubsection*{Methods}


\begin{savenotes}\sphinxatlongtablestart\begin{longtable}[c]{\X{1}{2}\X{1}{2}}
\hline

\endfirsthead

\multicolumn{2}{c}%
{\makebox[0pt]{\sphinxtablecontinued{\tablename\ \thetable{} \textendash{} continued from previous page}}}\\
\hline

\endhead

\hline
\multicolumn{2}{r}{\makebox[0pt][r]{\sphinxtablecontinued{Continued on next page}}}\\
\endfoot

\endlastfoot

{\hyperref[\detokenize{generated/sdapy.snerun.snelist.add_hist:sdapy.snerun.snelist.add_hist}]{\sphinxcrossref{\sphinxcode{\sphinxupquote{add\_hist}}}}}(self, x, y{[}, syntax, nbinx, nbiny, …{]})
&
make histograms
\\
\hline
{\hyperref[\detokenize{generated/sdapy.snerun.snelist.add_subset:sdapy.snerun.snelist.add_subset}]{\sphinxcrossref{\sphinxcode{\sphinxupquote{add\_subset}}}}}(self{[}, syntax{]})
&
create a data subset from self.meta
\\
\hline
{\hyperref[\detokenize{generated/sdapy.snerun.snelist.format_par:sdapy.snerun.snelist.format_par}]{\sphinxcrossref{\sphinxcode{\sphinxupquote{format\_par}}}}}(v, vlow, vup{[}, digits{]})
&
make parameter into latex format
\\
\hline
{\hyperref[\detokenize{generated/sdapy.snerun.snelist.get_par:sdapy.snerun.snelist.get_par}]{\sphinxcrossref{\sphinxcode{\sphinxupquote{get\_par}}}}}(self, objid, parname{[}, filt1, …{]})
&
get parameter of one SN
\\
\hline
{\hyperref[\detokenize{generated/sdapy.snerun.snelist.init_hist_axes:sdapy.snerun.snelist.init_hist_axes}]{\sphinxcrossref{\sphinxcode{\sphinxupquote{init\_hist\_axes}}}}}(self{[}, pad, labelbottom, …{]})
&
create 2 subplots as histograms for the scatter plots
\\
\hline
{\hyperref[\detokenize{generated/sdapy.snerun.snelist.load_data:sdapy.snerun.snelist.load_data}]{\sphinxcrossref{\sphinxcode{\sphinxupquote{load\_data}}}}}(self, objid{[}, force, datafile{]})
&
for each object, load their \sphinxstyleemphasis{snobject} classes if they’re cached before.
\\
\hline
{\hyperref[\detokenize{generated/sdapy.snerun.snelist.parse_meta:sdapy.snerun.snelist.parse_meta}]{\sphinxcrossref{\sphinxcode{\sphinxupquote{parse\_meta}}}}}(self{[}, withnew, source, metafile{]})
&
Read a meta table from local
\\
\hline
{\hyperref[\detokenize{generated/sdapy.snerun.snelist.parse_meta_all:sdapy.snerun.snelist.parse_meta_all}]{\sphinxcrossref{\sphinxcode{\sphinxupquote{parse\_meta\_all}}}}}(self, kwargs, objid)
&
properly read a list of meta infomations from self.meta, i.e.
\\
\hline
{\hyperref[\detokenize{generated/sdapy.snerun.snelist.parse_meta_one:sdapy.snerun.snelist.parse_meta_one}]{\sphinxcrossref{\sphinxcode{\sphinxupquote{parse\_meta\_one}}}}}(self, idkey, objid, key)
&
Obtain value with object ID and a meta key
\\
\hline
{\hyperref[\detokenize{generated/sdapy.snerun.snelist.parse_params:sdapy.snerun.snelist.parse_params}]{\sphinxcrossref{\sphinxcode{\sphinxupquote{parse\_params}}}}}(self{[}, force, parfile{]})
&
Besides the general parameter settings, for SNe with peculiar properties, a specific parameter is sometimes needed, and \sphinxstyleemphasis{parse\_params} can read a text file (individual\_par.txt) that includes all special settings for particular SNe.
\\
\hline
{\hyperref[\detokenize{generated/sdapy.snerun.snelist.read_kwargs:sdapy.snerun.snelist.read_kwargs}]{\sphinxcrossref{\sphinxcode{\sphinxupquote{read\_kwargs}}}}}(self, \textbackslash{}*\textbackslash{}*kwargs)
&
Define a proper way to read and update optional parameters
\\
\hline
{\hyperref[\detokenize{generated/sdapy.snerun.snelist.run:sdapy.snerun.snelist.run}]{\sphinxcrossref{\sphinxcode{\sphinxupquote{run}}}}}(self{[}, ax, ax1, ax2, ax3, ax4, debug{]})
&
get a list of SNe, for each SN, define a dedicated \sphinxstyleemphasis{snobject}, and run snobject.run() for all of them.
\\
\hline
{\hyperref[\detokenize{generated/sdapy.snerun.snelist.save_data:sdapy.snerun.snelist.save_data}]{\sphinxcrossref{\sphinxcode{\sphinxupquote{save\_data}}}}}(self, objid{[}, force, datafile{]})
&
for each object, save their \sphinxstyleemphasis{snobject} classes to local cached files.
\\
\hline
{\hyperref[\detokenize{generated/sdapy.snerun.snelist.show1d:sdapy.snerun.snelist.show1d}]{\sphinxcrossref{\sphinxcode{\sphinxupquote{show1d}}}}}(self, k{[}, nbin, fontsize, labelpad{]})
&
1D histograms plot for one parameter
\\
\hline
{\hyperref[\detokenize{generated/sdapy.snerun.snelist.show2d:sdapy.snerun.snelist.show2d}]{\sphinxcrossref{\sphinxcode{\sphinxupquote{show2d}}}}}(self, k1, k2{[}, fontsize, labelpad{]})
&
2D scatter plot for two parameter
\\
\hline
\sphinxcode{\sphinxupquote{showax}}(self{[}, syntax, showfilt, …{]})
&
make flux plot for a large set of SNe
\\
\hline
\sphinxcode{\sphinxupquote{showax2}}(self{[}, syntax, showfilt{]})
&
make mag plot for a large set of SNe
\\
\hline
{\hyperref[\detokenize{generated/sdapy.snerun.snelist.table:sdapy.snerun.snelist.table}]{\sphinxcrossref{\sphinxcode{\sphinxupquote{table}}}}}(self{[}, syntax, keys, tablename{]})
&
create a latex table for a subset of SNe
\\
\hline
\end{longtable}\sphinxatlongtableend\end{savenotes}


\subsubsection{sdapy.snerun.snelist.add\_hist}
\label{\detokenize{generated/sdapy.snerun.snelist.add_hist:sdapy-snerun-snelist-add-hist}}\label{\detokenize{generated/sdapy.snerun.snelist.add_hist::doc}}\index{add\_hist() (sdapy.snerun.snelist method)@\spxentry{add\_hist()}\spxextra{sdapy.snerun.snelist method}}

\begin{fulllineitems}
\phantomsection\label{\detokenize{generated/sdapy.snerun.snelist.add_hist:sdapy.snerun.snelist.add_hist}}\pysiglinewithargsret{\sphinxcode{\sphinxupquote{snelist.}}\sphinxbfcode{\sphinxupquote{add\_hist}}}{\emph{self}, \emph{x}, \emph{y}, \emph{syntax=None}, \emph{nbinx=10}, \emph{nbiny=10}, \emph{xticks=None}, \emph{yticks=None}, \emph{**kwargs}}{}
make histograms
\begin{quote}\begin{description}
\item[{Parameters}] \leavevmode\begin{description}
\item[{\sphinxstylestrong{x}}] \leavevmode{[}\sphinxtitleref{float}{]}
table column name for x axis

\item[{\sphinxstylestrong{y}}] \leavevmode{[}\sphinxtitleref{str}{]}
table column name for y axis

\item[{\sphinxstylestrong{syntax}}] \leavevmode{[}\sphinxtitleref{str}{]}
syntax used to make subset of meta table, e.g.
type in {[}“SN Ib”, “SN Ic”{]}, which will only parse all SNe Ibcs

\item[{\sphinxstylestrong{nbinx}}] \leavevmode{[}\sphinxtitleref{int}, \sphinxtitleref{list}{]}
histogram bins for x axis

\item[{\sphinxstylestrong{nbiny}}] \leavevmode{[}\sphinxtitleref{int}, \sphinxtitleref{list}{]}
histogram bins for y axis

\item[{\sphinxstylestrong{xticks}}] \leavevmode{[}\sphinxtitleref{int}, \sphinxtitleref{list}{]}
ticks for x axis

\item[{\sphinxstylestrong{yticks}}] \leavevmode{[}\sphinxtitleref{int}, \sphinxtitleref{list}{]}
ticks for y axis

\item[{\sphinxstylestrong{kwargs}}] \leavevmode{[}\sphinxtitleref{Keyword Arguments}{]}
\sphinxstylestrong{matplotlib.ax.hist} kwargs

\end{description}

\end{description}\end{quote}

\end{fulllineitems}



\subsubsection{sdapy.snerun.snelist.add\_subset}
\label{\detokenize{generated/sdapy.snerun.snelist.add_subset:sdapy-snerun-snelist-add-subset}}\label{\detokenize{generated/sdapy.snerun.snelist.add_subset::doc}}\index{add\_subset() (sdapy.snerun.snelist method)@\spxentry{add\_subset()}\spxextra{sdapy.snerun.snelist method}}

\begin{fulllineitems}
\phantomsection\label{\detokenize{generated/sdapy.snerun.snelist.add_subset:sdapy.snerun.snelist.add_subset}}\pysiglinewithargsret{\sphinxcode{\sphinxupquote{snelist.}}\sphinxbfcode{\sphinxupquote{add\_subset}}}{\emph{self}, \emph{syntax=\textquotesingle{}all\textquotesingle{}}, \emph{**kwargs}}{}
create a data subset from self.meta
\begin{quote}\begin{description}
\item[{Parameters}] \leavevmode\begin{description}
\item[{\sphinxstylestrong{syntax}}] \leavevmode{[}\sphinxtitleref{str}{]}
syntax used to make subset of meta table, e.g.
type in {[}“SN Ib”, “SN Ic”{]}, which will only parse all SNe Ibcs

\item[{\sphinxstylestrong{kwargs}}] \leavevmode{[}\sphinxtitleref{Keyword Arguments}{]}
for subset plotter

\end{description}

\end{description}\end{quote}

\end{fulllineitems}



\subsubsection{sdapy.snerun.snelist.format\_par}
\label{\detokenize{generated/sdapy.snerun.snelist.format_par:sdapy-snerun-snelist-format-par}}\label{\detokenize{generated/sdapy.snerun.snelist.format_par::doc}}\index{format\_par() (sdapy.snerun.snelist static method)@\spxentry{format\_par()}\spxextra{sdapy.snerun.snelist static method}}

\begin{fulllineitems}
\phantomsection\label{\detokenize{generated/sdapy.snerun.snelist.format_par:sdapy.snerun.snelist.format_par}}\pysiglinewithargsret{\sphinxbfcode{\sphinxupquote{static }}\sphinxcode{\sphinxupquote{snelist.}}\sphinxbfcode{\sphinxupquote{format\_par}}}{\emph{v}, \emph{vlow}, \emph{vup}, \emph{digits=3}}{}
make parameter into latex format
\begin{quote}\begin{description}
\item[{Parameters}] \leavevmode\begin{description}
\item[{\sphinxstylestrong{v}}] \leavevmode{[}\sphinxtitleref{float}{]}
best fit value of one parameter

\item[{\sphinxstylestrong{vup}}] \leavevmode{[}\sphinxtitleref{float}{]}
upper limit of one parameter

\item[{\sphinxstylestrong{vlow}}] \leavevmode{[}\sphinxtitleref{float}{]}
lower limit of one parameter

\item[{\sphinxstylestrong{digits}}] \leavevmode{[}\sphinxtitleref{int}{]}
number digits

\end{description}

\end{description}\end{quote}

\end{fulllineitems}



\subsubsection{sdapy.snerun.snelist.get\_par}
\label{\detokenize{generated/sdapy.snerun.snelist.get_par:sdapy-snerun-snelist-get-par}}\label{\detokenize{generated/sdapy.snerun.snelist.get_par::doc}}\index{get\_par() (sdapy.snerun.snelist method)@\spxentry{get\_par()}\spxextra{sdapy.snerun.snelist method}}

\begin{fulllineitems}
\phantomsection\label{\detokenize{generated/sdapy.snerun.snelist.get_par:sdapy.snerun.snelist.get_par}}\pysiglinewithargsret{\sphinxcode{\sphinxupquote{snelist.}}\sphinxbfcode{\sphinxupquote{get\_par}}}{\emph{self, objid, parname, filt1=\textquotesingle{}g\textquotesingle{}, filt2=\textquotesingle{}r\textquotesingle{}, quant={[}0.05, 0.5, 0.95{]}, interpolation=None, corr\_mkw=False, corr\_host=False}}{}
get parameter of one SN
\begin{quote}\begin{description}
\item[{Parameters}] \leavevmode\begin{description}
\item[{\sphinxstylestrong{objid}}] \leavevmode{[}\sphinxtitleref{str}{]}
object ID string

\item[{\sphinxstylestrong{parname}}] \leavevmode{[}\sphinxtitleref{str}{]}
column name of self.meta

\end{description}

\end{description}\end{quote}

\end{fulllineitems}



\subsubsection{sdapy.snerun.snelist.init\_hist\_axes}
\label{\detokenize{generated/sdapy.snerun.snelist.init_hist_axes:sdapy-snerun-snelist-init-hist-axes}}\label{\detokenize{generated/sdapy.snerun.snelist.init_hist_axes::doc}}\index{init\_hist\_axes() (sdapy.snerun.snelist method)@\spxentry{init\_hist\_axes()}\spxextra{sdapy.snerun.snelist method}}

\begin{fulllineitems}
\phantomsection\label{\detokenize{generated/sdapy.snerun.snelist.init_hist_axes:sdapy.snerun.snelist.init_hist_axes}}\pysiglinewithargsret{\sphinxcode{\sphinxupquote{snelist.}}\sphinxbfcode{\sphinxupquote{init\_hist\_axes}}}{\emph{self}, \emph{pad=0.1}, \emph{labelbottom=False}, \emph{labelleft=False}}{}
create 2 subplots as histograms for the scatter plots
\begin{quote}\begin{description}
\item[{Parameters}] \leavevmode\begin{description}
\item[{\sphinxstylestrong{pad}}] \leavevmode{[}\sphinxtitleref{float}{]}
distance between scatter plot and the histograms

\item[{\sphinxstylestrong{labelbottom}}] \leavevmode{[}\sphinxtitleref{str}{]}
bottom histogram label name

\item[{\sphinxstylestrong{labelletf}}] \leavevmode{[}\sphinxtitleref{str}{]}
left histogram label name

\end{description}

\end{description}\end{quote}

\end{fulllineitems}



\subsubsection{sdapy.snerun.snelist.load\_data}
\label{\detokenize{generated/sdapy.snerun.snelist.load_data:sdapy-snerun-snelist-load-data}}\label{\detokenize{generated/sdapy.snerun.snelist.load_data::doc}}\index{load\_data() (sdapy.snerun.snelist method)@\spxentry{load\_data()}\spxextra{sdapy.snerun.snelist method}}

\begin{fulllineitems}
\phantomsection\label{\detokenize{generated/sdapy.snerun.snelist.load_data:sdapy.snerun.snelist.load_data}}\pysiglinewithargsret{\sphinxcode{\sphinxupquote{snelist.}}\sphinxbfcode{\sphinxupquote{load\_data}}}{\emph{self}, \emph{objid}, \emph{force=False}, \emph{datafile=\textquotesingle{}\%s\_data.clf\textquotesingle{}}, \emph{**kwargs}}{}
for each object, load their \sphinxstyleemphasis{snobject} classes if they’re cached before.
\begin{quote}\begin{description}
\item[{Parameters}] \leavevmode\begin{description}
\item[{\sphinxstylestrong{objid}}] \leavevmode{[}\sphinxtitleref{str}{]}
object ID string

\item[{\sphinxstylestrong{force}}] \leavevmode{[}\sphinxtitleref{bool}{]}
if meta already exists, reload or skip

\item[{\sphinxstylestrong{datafile}}] \leavevmode{[}\sphinxtitleref{str}{]}
cached file name

\item[{\sphinxstylestrong{verbose}}] \leavevmode{[}\sphinxtitleref{bool}{]}
show detailed running informations

\end{description}

\item[{Returns}] \leavevmode\begin{description}
\item[{\sphinxstylestrong{flag}}] \leavevmode{[}\sphinxtitleref{bool}{]}
if cachefile exists return True, otehriwse False

\end{description}

\end{description}\end{quote}


\sphinxstrong{See also:}

\begin{description}
\item[{{\hyperref[\detokenize{generated/sdapy.snerun.snelist.save_data:sdapy.snerun.snelist.save_data}]{\sphinxcrossref{\sphinxcode{\sphinxupquote{snelist.save\_data}}}}}}] \leavevmode
\end{description}



\end{fulllineitems}



\subsubsection{sdapy.snerun.snelist.parse\_meta}
\label{\detokenize{generated/sdapy.snerun.snelist.parse_meta:sdapy-snerun-snelist-parse-meta}}\label{\detokenize{generated/sdapy.snerun.snelist.parse_meta::doc}}\index{parse\_meta() (sdapy.snerun.snelist method)@\spxentry{parse\_meta()}\spxextra{sdapy.snerun.snelist method}}

\begin{fulllineitems}
\phantomsection\label{\detokenize{generated/sdapy.snerun.snelist.parse_meta:sdapy.snerun.snelist.parse_meta}}\pysiglinewithargsret{\sphinxcode{\sphinxupquote{snelist.}}\sphinxbfcode{\sphinxupquote{parse\_meta}}}{\emph{self}, \emph{withnew=\textquotesingle{}skip\textquotesingle{}}, \emph{source=\textquotesingle{}BTS\textquotesingle{}}, \emph{metafile=None}, \emph{**kwargs}}{}
Read a meta table from local
\begin{quote}\begin{description}
\item[{Parameters}] \leavevmode\begin{description}
\item[{\sphinxstylestrong{withnew}}] \leavevmode{[}\sphinxtitleref{bool}{]}\begin{description}
\item[{if meta already exists, and a new meta table comimg:}] \leavevmode
use {[}new{]} meta instead of the original one
or {[}skip{]} the new one 
or {[}merge{]} them together

\end{description}

\item[{\sphinxstylestrong{source}}] \leavevmode{[}\sphinxtitleref{str}{]}
which source for metadata: OAC or BTS

\item[{\sphinxstylestrong{metafile}}] \leavevmode{[}\sphinxtitleref{str}{]}
if you want to use your own meta, set source to None
and input a existing table metafile

\item[{\sphinxstylestrong{syntax}}] \leavevmode{[}\sphinxtitleref{str}{]}
syntax used to make subset of meta table, e.g.
type in {[}“SN Ib”, “SN Ic”{]}, which will only parse all SNe Ibcs

\item[{\sphinxstylestrong{verbose}}] \leavevmode{[}\sphinxtitleref{bool}{]}
show detailed running informations

\item[{\sphinxstylestrong{idkey}}] \leavevmode{[}\sphinxtitleref{str}{]}
object ID column name, e.g. ZTF name or IAU name

\item[{\sphinxstylestrong{sortkey}}] \leavevmode{[}\sphinxtitleref{str}{]}
the column used to sort self.meta table

\end{description}

\end{description}\end{quote}


\sphinxstrong{See also:}

\begin{description}
\item[{{\hyperref[\detokenize{generated/sdapy.snerun.snelist.parse_meta_one:sdapy.snerun.snelist.parse_meta_one}]{\sphinxcrossref{\sphinxcode{\sphinxupquote{snelist.parse\_meta\_one}}}}}, {\hyperref[\detokenize{generated/sdapy.snerun.snelist.parse_meta_all:sdapy.snerun.snelist.parse_meta_all}]{\sphinxcrossref{\sphinxcode{\sphinxupquote{snelist.parse\_meta\_all}}}}}}] \leavevmode
\end{description}



\end{fulllineitems}



\subsubsection{sdapy.snerun.snelist.parse\_meta\_all}
\label{\detokenize{generated/sdapy.snerun.snelist.parse_meta_all:sdapy-snerun-snelist-parse-meta-all}}\label{\detokenize{generated/sdapy.snerun.snelist.parse_meta_all::doc}}\index{parse\_meta\_all() (sdapy.snerun.snelist method)@\spxentry{parse\_meta\_all()}\spxextra{sdapy.snerun.snelist method}}

\begin{fulllineitems}
\phantomsection\label{\detokenize{generated/sdapy.snerun.snelist.parse_meta_all:sdapy.snerun.snelist.parse_meta_all}}\pysiglinewithargsret{\sphinxcode{\sphinxupquote{snelist.}}\sphinxbfcode{\sphinxupquote{parse\_meta\_all}}}{\emph{self}, \emph{kwargs}, \emph{objid}}{}
properly read a list of meta infomations from self.meta, i.e. coordinates self.ra, self.dec,
redshift self.z, distance self.dist, distance module self.dm, mkily way extinction self.mkwebv,
host galaxy extinction self.hostebv, type self.sntype and peak time self.jdpeak.
If ra dec missed, user should manully input lightcurve later instead of build\sphinxhyphen{}in sources.
If redshift missed, will make all analysis in obervational frame instead of rest frame.
If distance missed, will calculate it from redshift with a standard cosmology (astropy.cosmology.Planck13).
If milky way E(B\sphinxhyphen{}V) missed, will check if A\_V available, if not, make sure you had dustmaps.sfd installed,
and SFD dust map is downloaded propoerly with dustmaps, Otherwise will ignore milky way extinction.
If host galaxy E(B\sphinxhyphen{}V) missed, will check if A\_V available, otherwise temporarily assign 0 to host E(B\sphinxhyphen{}V),
which can be updated later from colour comparison or Na Id fittings.
Type is used by colour comparison (which template colour should be compared) and line measurements 
(which line should be fitted), if missed, will make trouble in these 2 parts.
jdpeak is can be decided by \sphinxstyleemphasis{snobject} in many ways, but a prior input is important to guess the JD range
to query photometry. jdpeak can be shifted by \sphinxstylestrong{jdpeak\_shift} 
(\sphinxurl{https://github.com/saberyoung/HAFFET/blob/master/sdapy/data/default\_par.txt})
\begin{quote}\begin{description}
\item[{Parameters}] \leavevmode\begin{description}
\item[{\sphinxstylestrong{kwargs}}] \leavevmode{[}\sphinxtitleref{Keyword Arguments}{]}\begin{description}
\item[{optional parameters, will use  \sphinxtitleref{idkey},  \sphinxtitleref{rakey},  \sphinxtitleref{deckey},  \sphinxtitleref{zkey}, }] \leavevmode
\sphinxtitleref{distkey},  \sphinxtitleref{dmkey},  \sphinxtitleref{mkwebvkey},  \sphinxtitleref{mkwavkey},  \sphinxtitleref{hostebvkey}, 
\sphinxtitleref{hostavkey},  \sphinxtitleref{typekey},  \sphinxtitleref{peaktkey},  \sphinxtitleref{idkey},  \sphinxtitleref{rv}, and  \sphinxtitleref{jdpeak\_shift}

\end{description}

\item[{\sphinxstylestrong{objid}}] \leavevmode{[}\sphinxtitleref{str}{]}
object ID string

\end{description}

\end{description}\end{quote}


\sphinxstrong{See also:}

\begin{description}
\item[{{\hyperref[\detokenize{generated/sdapy.snerun.snelist.parse_meta:sdapy.snerun.snelist.parse_meta}]{\sphinxcrossref{\sphinxcode{\sphinxupquote{snelist.parse\_meta}}}}}, {\hyperref[\detokenize{generated/sdapy.snerun.snelist.parse_meta_one:sdapy.snerun.snelist.parse_meta_one}]{\sphinxcrossref{\sphinxcode{\sphinxupquote{snelist.parse\_meta\_one}}}}}}] \leavevmode
\end{description}



\end{fulllineitems}



\subsubsection{sdapy.snerun.snelist.parse\_meta\_one}
\label{\detokenize{generated/sdapy.snerun.snelist.parse_meta_one:sdapy-snerun-snelist-parse-meta-one}}\label{\detokenize{generated/sdapy.snerun.snelist.parse_meta_one::doc}}\index{parse\_meta\_one() (sdapy.snerun.snelist method)@\spxentry{parse\_meta\_one()}\spxextra{sdapy.snerun.snelist method}}

\begin{fulllineitems}
\phantomsection\label{\detokenize{generated/sdapy.snerun.snelist.parse_meta_one:sdapy.snerun.snelist.parse_meta_one}}\pysiglinewithargsret{\sphinxcode{\sphinxupquote{snelist.}}\sphinxbfcode{\sphinxupquote{parse\_meta\_one}}}{\emph{self}, \emph{idkey}, \emph{objid}, \emph{key}}{}
Obtain value with object ID and a meta key
\begin{quote}\begin{description}
\item[{Parameters}] \leavevmode\begin{description}
\item[{\sphinxstylestrong{idkey}}] \leavevmode{[}\sphinxtitleref{str}{]}
object ID column name, e.g. ZTF name or IAU name

\item[{\sphinxstylestrong{objid}}] \leavevmode{[}\sphinxtitleref{str}{]}
object ID string

\item[{\sphinxstylestrong{key}}] \leavevmode{[}\sphinxtitleref{str}{]}
a column key of self.meta

\end{description}

\item[{Returns}] \leavevmode\begin{description}
\item[{\sphinxstylestrong{meta}}] \leavevmode{[}\sphinxtitleref{pandas.dataframe}{]}
\end{description}

\end{description}\end{quote}


\sphinxstrong{See also:}

\begin{description}
\item[{{\hyperref[\detokenize{generated/sdapy.snerun.snelist.parse_meta:sdapy.snerun.snelist.parse_meta}]{\sphinxcrossref{\sphinxcode{\sphinxupquote{snelist.parse\_meta}}}}}, {\hyperref[\detokenize{generated/sdapy.snerun.snelist.parse_meta_all:sdapy.snerun.snelist.parse_meta_all}]{\sphinxcrossref{\sphinxcode{\sphinxupquote{snelist.parse\_meta\_all}}}}}}] \leavevmode
\end{description}



\end{fulllineitems}



\subsubsection{sdapy.snerun.snelist.parse\_params}
\label{\detokenize{generated/sdapy.snerun.snelist.parse_params:sdapy-snerun-snelist-parse-params}}\label{\detokenize{generated/sdapy.snerun.snelist.parse_params::doc}}\index{parse\_params() (sdapy.snerun.snelist method)@\spxentry{parse\_params()}\spxextra{sdapy.snerun.snelist method}}

\begin{fulllineitems}
\phantomsection\label{\detokenize{generated/sdapy.snerun.snelist.parse_params:sdapy.snerun.snelist.parse_params}}\pysiglinewithargsret{\sphinxcode{\sphinxupquote{snelist.}}\sphinxbfcode{\sphinxupquote{parse\_params}}}{\emph{self}, \emph{force=False}, \emph{parfile=\textquotesingle{}individual\_par.txt\textquotesingle{}}}{}
Besides the general parameter settings, for SNe with peculiar properties, a specific parameter is
sometimes needed, and \sphinxstyleemphasis{parse\_params} can read a text file (individual\_par.txt) that includes all special
settings for particular SNe.
\begin{quote}\begin{description}
\item[{Parameters}] \leavevmode\begin{description}
\item[{\sphinxstylestrong{force}}] \leavevmode{[}\sphinxtitleref{bool}{]}
if meta already exists, reload or skip

\item[{\sphinxstylestrong{metafile}}] \leavevmode{[}\sphinxtitleref{str}{]}
meta filename, file folder is os.getenv(‘ZTFDATA’,”./Data/”)

\end{description}

\end{description}\end{quote}

\end{fulllineitems}



\subsubsection{sdapy.snerun.snelist.read\_kwargs}
\label{\detokenize{generated/sdapy.snerun.snelist.read_kwargs:sdapy-snerun-snelist-read-kwargs}}\label{\detokenize{generated/sdapy.snerun.snelist.read_kwargs::doc}}\index{read\_kwargs() (sdapy.snerun.snelist method)@\spxentry{read\_kwargs()}\spxextra{sdapy.snerun.snelist method}}

\begin{fulllineitems}
\phantomsection\label{\detokenize{generated/sdapy.snerun.snelist.read_kwargs:sdapy.snerun.snelist.read_kwargs}}\pysiglinewithargsret{\sphinxcode{\sphinxupquote{snelist.}}\sphinxbfcode{\sphinxupquote{read\_kwargs}}}{\emph{self}, \emph{**kwargs}}{}
Define a proper way to read and update optional parameters
\begin{quote}\begin{description}
\item[{Parameters}] \leavevmode\begin{description}
\item[{\sphinxstylestrong{kwargs}}] \leavevmode{[}\sphinxtitleref{Keyword Arguments}{]}
optional parameters

\end{description}

\item[{Returns}] \leavevmode\begin{description}
\item[{\sphinxstylestrong{kwargs}}] \leavevmode{[}\sphinxtitleref{Keyword Arguments}{]}
optional parameters

\end{description}

\end{description}\end{quote}


\sphinxstrong{See also:}

\begin{description}
\item[{{\hyperref[\detokenize{generated/sdapy.snerun.snelist.__init__:sdapy.snerun.snelist.__init__}]{\sphinxcrossref{\sphinxcode{\sphinxupquote{snelist.\_\_init\_\_}}}}}}] \leavevmode
\end{description}



\end{fulllineitems}



\subsubsection{sdapy.snerun.snelist.run}
\label{\detokenize{generated/sdapy.snerun.snelist.run:sdapy-snerun-snelist-run}}\label{\detokenize{generated/sdapy.snerun.snelist.run::doc}}\index{run() (sdapy.snerun.snelist method)@\spxentry{run()}\spxextra{sdapy.snerun.snelist method}}

\begin{fulllineitems}
\phantomsection\label{\detokenize{generated/sdapy.snerun.snelist.run:sdapy.snerun.snelist.run}}\pysiglinewithargsret{\sphinxcode{\sphinxupquote{snelist.}}\sphinxbfcode{\sphinxupquote{run}}}{\emph{self}, \emph{ax=None}, \emph{ax1=None}, \emph{ax2=None}, \emph{ax3=None}, \emph{ax4=None}, \emph{debug=False}, \emph{**kwargs}}{}
get a list of SNe, for each SN, define a dedicated \sphinxstyleemphasis{snobject}, and run snobject.run() for
all of them.
\begin{quote}\begin{description}
\item[{Parameters}] \leavevmode\begin{description}
\item[{\sphinxstylestrong{ax/ax1/ax2/ax3/ax4}}] \leavevmode{[}\sphinxtitleref{str}{]}
matplotlib.axes, used for histogram/scatter or other population plots

\item[{\sphinxstylestrong{verbose}}] \leavevmode{[}\sphinxtitleref{bool}{]}
show detailed running informations

\item[{\sphinxstylestrong{clobber}}] \leavevmode{[}\sphinxtitleref{bool}{]}
if cached file (from \sphinxstyleemphasis{save\_data}) exists, redo snobject.run() or just read the cach

\end{description}

\end{description}\end{quote}


\sphinxstrong{See also:}

\begin{description}
\item[{{\hyperref[\detokenize{generated/sdapy.snerun.snobject.run:sdapy.snerun.snobject.run}]{\sphinxcrossref{\sphinxcode{\sphinxupquote{snobject.run}}}}}}] \leavevmode
\end{description}



\end{fulllineitems}



\subsubsection{sdapy.snerun.snelist.save\_data}
\label{\detokenize{generated/sdapy.snerun.snelist.save_data:sdapy-snerun-snelist-save-data}}\label{\detokenize{generated/sdapy.snerun.snelist.save_data::doc}}\index{save\_data() (sdapy.snerun.snelist method)@\spxentry{save\_data()}\spxextra{sdapy.snerun.snelist method}}

\begin{fulllineitems}
\phantomsection\label{\detokenize{generated/sdapy.snerun.snelist.save_data:sdapy.snerun.snelist.save_data}}\pysiglinewithargsret{\sphinxcode{\sphinxupquote{snelist.}}\sphinxbfcode{\sphinxupquote{save\_data}}}{\emph{self}, \emph{objid}, \emph{force=False}, \emph{datafile=\textquotesingle{}\%s\_data.clf\textquotesingle{}}, \emph{**kwargs}}{}
for each object, save their \sphinxstyleemphasis{snobject} classes to local cached files.
\begin{quote}\begin{description}
\item[{Parameters}] \leavevmode\begin{description}
\item[{\sphinxstylestrong{objid}}] \leavevmode{[}\sphinxtitleref{str}{]}
object ID string

\item[{\sphinxstylestrong{force}}] \leavevmode{[}\sphinxtitleref{bool}{]}
if meta already exists, reload or skip

\item[{\sphinxstylestrong{datafile}}] \leavevmode{[}\sphinxtitleref{str}{]}
cached file name

\item[{\sphinxstylestrong{verbose}}] \leavevmode{[}\sphinxtitleref{bool}{]}
show detailed running informations

\end{description}

\item[{Returns}] \leavevmode\begin{description}
\item[{\sphinxstylestrong{flag}}] \leavevmode{[}\sphinxtitleref{bool}{]}
if saved cache return True, otehrwise False

\end{description}

\end{description}\end{quote}


\sphinxstrong{See also:}

\begin{description}
\item[{{\hyperref[\detokenize{generated/sdapy.snerun.snelist.load_data:sdapy.snerun.snelist.load_data}]{\sphinxcrossref{\sphinxcode{\sphinxupquote{snelist.load\_data}}}}}}] \leavevmode
\end{description}



\end{fulllineitems}



\subsubsection{sdapy.snerun.snelist.show1d}
\label{\detokenize{generated/sdapy.snerun.snelist.show1d:sdapy-snerun-snelist-show1d}}\label{\detokenize{generated/sdapy.snerun.snelist.show1d::doc}}\index{show1d() (sdapy.snerun.snelist method)@\spxentry{show1d()}\spxextra{sdapy.snerun.snelist method}}

\begin{fulllineitems}
\phantomsection\label{\detokenize{generated/sdapy.snerun.snelist.show1d:sdapy.snerun.snelist.show1d}}\pysiglinewithargsret{\sphinxcode{\sphinxupquote{snelist.}}\sphinxbfcode{\sphinxupquote{show1d}}}{\emph{self}, \emph{k}, \emph{nbin=10}, \emph{fontsize=12}, \emph{labelpad=12}, \emph{**kwargs}}{}
1D histograms plot for one parameter
\begin{quote}\begin{description}
\item[{Parameters}] \leavevmode\begin{description}
\item[{\sphinxstylestrong{k}}] \leavevmode{[}\sphinxtitleref{str}{]}
column name

\item[{\sphinxstylestrong{nbin}}] \leavevmode{[}\sphinxtitleref{int}, \sphinxtitleref{list}{]}
histogram bins

\item[{\sphinxstylestrong{fontsize}}] \leavevmode{[}\sphinxtitleref{str}{]}
figure label font size

\item[{\sphinxstylestrong{labelpad}}] \leavevmode{[}\sphinxtitleref{str}{]}
figure label pad

\item[{\sphinxstylestrong{kwargs}}] \leavevmode{[}\sphinxtitleref{Keyword Arguments}{]}
\sphinxstylestrong{matplotlib.ax.hist} kwargs

\end{description}

\end{description}\end{quote}

\end{fulllineitems}



\subsubsection{sdapy.snerun.snelist.show2d}
\label{\detokenize{generated/sdapy.snerun.snelist.show2d:sdapy-snerun-snelist-show2d}}\label{\detokenize{generated/sdapy.snerun.snelist.show2d::doc}}\index{show2d() (sdapy.snerun.snelist method)@\spxentry{show2d()}\spxextra{sdapy.snerun.snelist method}}

\begin{fulllineitems}
\phantomsection\label{\detokenize{generated/sdapy.snerun.snelist.show2d:sdapy.snerun.snelist.show2d}}\pysiglinewithargsret{\sphinxcode{\sphinxupquote{snelist.}}\sphinxbfcode{\sphinxupquote{show2d}}}{\emph{self}, \emph{k1}, \emph{k2}, \emph{fontsize=12}, \emph{labelpad=12}, \emph{**kwargs}}{}
2D scatter plot for two parameter
\begin{quote}\begin{description}
\item[{Parameters}] \leavevmode\begin{description}
\item[{\sphinxstylestrong{k1}}] \leavevmode{[}\sphinxtitleref{str}{]}
column name x axis

\item[{\sphinxstylestrong{k2}}] \leavevmode{[}\sphinxtitleref{str}{]}
column name y axis

\item[{\sphinxstylestrong{fontsize}}] \leavevmode{[}\sphinxtitleref{str}{]}
figure label font size

\item[{\sphinxstylestrong{labelpad}}] \leavevmode{[}\sphinxtitleref{str}{]}
figure label pad

\item[{\sphinxstylestrong{kwargs}}] \leavevmode{[}\sphinxtitleref{Keyword Arguments}{]}
\sphinxstylestrong{matplotlib.ax.plot} kwargs

\end{description}

\end{description}\end{quote}

\end{fulllineitems}



\subsubsection{sdapy.snerun.snelist.table}
\label{\detokenize{generated/sdapy.snerun.snelist.table:sdapy-snerun-snelist-table}}\label{\detokenize{generated/sdapy.snerun.snelist.table::doc}}\index{table() (sdapy.snerun.snelist method)@\spxentry{table()}\spxextra{sdapy.snerun.snelist method}}

\begin{fulllineitems}
\phantomsection\label{\detokenize{generated/sdapy.snerun.snelist.table:sdapy.snerun.snelist.table}}\pysiglinewithargsret{\sphinxcode{\sphinxupquote{snelist.}}\sphinxbfcode{\sphinxupquote{table}}}{\emph{self, syntax=\textquotesingle{}all\textquotesingle{}, keys={[}\textquotesingle{}ra\textquotesingle{}, \textquotesingle{}dec\textquotesingle{}, \textquotesingle{}dm\textquotesingle{}{]}, tablename=None}}{}
create a latex table for a subset of SNe
\begin{quote}\begin{description}
\item[{Parameters}] \leavevmode\begin{description}
\item[{\sphinxstylestrong{syntax}}] \leavevmode{[}\sphinxtitleref{str}{]}
syntax used to make subset of meta table, e.g.
type in {[}“SN Ib”, “SN Ic”{]}, which will only parse all SNe Ibcs

\item[{\sphinxstylestrong{keys}}] \leavevmode{[}\sphinxtitleref{list}{]}
a number of column names of self.meta table

\item[{\sphinxstylestrong{tablename}}] \leavevmode{[}\sphinxtitleref{str}{]}
output file name

\end{description}

\item[{Returns}] \leavevmode\begin{description}
\item[{\sphinxstylestrong{table}}] \leavevmode{[}\sphinxtitleref{str}{]}
if tablename is None, return a table, otherwise, store the table to tablename

\end{description}

\end{description}\end{quote}

\end{fulllineitems}



\begin{savenotes}\sphinxattablestart
\centering
\begin{tabulary}{\linewidth}[t]{|T|T|}
\hline

\sphinxstylestrong{showax3}
&\\
\hline
\sphinxstylestrong{showax4}
&\\
\hline
\sphinxstylestrong{showvelocity}
&\\
\hline
\end{tabulary}
\par
\sphinxattableend\end{savenotes}
\index{\_\_init\_\_() (sdapy.snerun.snelist method)@\spxentry{\_\_init\_\_()}\spxextra{sdapy.snerun.snelist method}}

\begin{fulllineitems}
\phantomsection\label{\detokenize{generated/sdapy.snerun.snelist:sdapy.snerun.snelist.__init__}}\pysiglinewithargsret{\sphinxbfcode{\sphinxupquote{\_\_init\_\_}}}{\emph{self}, \emph{ax=None}, \emph{**kwargs}}{}
initialize \sphinxstyleemphasis{snelist}
\begin{quote}\begin{description}
\item[{Parameters}] \leavevmode\begin{description}
\item[{\sphinxstylestrong{ax}}] \leavevmode{[}matplotlib.axes{]}
matplotlib.axes, used for histogram/scatter or other population plots

\item[{\sphinxstylestrong{kwargs}}] \leavevmode{[}\sphinxtitleref{Keyword Arguments}{]}
see \sphinxurl{https://github.com/saberyoung/HAFFET/blob/master/sdapy/data/default\_par.txt},
\sphinxstylestrong{snelist} part

\end{description}

\end{description}\end{quote}
\subsubsection*{Examples}

\begin{sphinxVerbatim}[commandchars=\\\{\}]
\PYG{g+gp}{\PYGZgt{}\PYGZgt{}\PYGZgt{} }\PYG{k+kn}{from} \PYG{n+nn}{sdapy} \PYG{k+kn}{import} \PYG{n}{snerun}
\PYG{g+gp}{\PYGZgt{}\PYGZgt{}\PYGZgt{} }\PYG{n}{a} \PYG{o}{=} \PYG{n}{snerun}\PYG{o}{.}\PYG{n}{snelist}\PYG{p}{(}\PYG{p}{)}
\PYG{g+gp}{\PYGZgt{}\PYGZgt{}\PYGZgt{} }\PYG{n}{a}
\PYG{g+go}{\PYGZlt{}sdapy.snerun.snelist object at 0x7fd6fb805f60\PYGZgt{}}
\end{sphinxVerbatim}

\end{fulllineitems}

\subsubsection*{Methods}


\begin{savenotes}\sphinxatlongtablestart\begin{longtable}[c]{\X{1}{2}\X{1}{2}}
\hline

\endfirsthead

\multicolumn{2}{c}%
{\makebox[0pt]{\sphinxtablecontinued{\tablename\ \thetable{} \textendash{} continued from previous page}}}\\
\hline

\endhead

\hline
\multicolumn{2}{r}{\makebox[0pt][r]{\sphinxtablecontinued{Continued on next page}}}\\
\endfoot

\endlastfoot

{\hyperref[\detokenize{generated/sdapy.snerun.snelist.__init__:sdapy.snerun.snelist.__init__}]{\sphinxcrossref{\sphinxcode{\sphinxupquote{\_\_init\_\_}}}}}(self{[}, ax{]})
&
initialize \sphinxstyleemphasis{snelist}
\\
\hline
{\hyperref[\detokenize{generated/sdapy.snerun.snelist.add_hist:sdapy.snerun.snelist.add_hist}]{\sphinxcrossref{\sphinxcode{\sphinxupquote{add\_hist}}}}}(self, x, y{[}, syntax, nbinx, nbiny, …{]})
&
make histograms
\\
\hline
{\hyperref[\detokenize{generated/sdapy.snerun.snelist.add_subset:sdapy.snerun.snelist.add_subset}]{\sphinxcrossref{\sphinxcode{\sphinxupquote{add\_subset}}}}}(self{[}, syntax{]})
&
create a data subset from self.meta
\\
\hline
{\hyperref[\detokenize{generated/sdapy.snerun.snelist.format_par:sdapy.snerun.snelist.format_par}]{\sphinxcrossref{\sphinxcode{\sphinxupquote{format\_par}}}}}(v, vlow, vup{[}, digits{]})
&
make parameter into latex format
\\
\hline
{\hyperref[\detokenize{generated/sdapy.snerun.snelist.get_par:sdapy.snerun.snelist.get_par}]{\sphinxcrossref{\sphinxcode{\sphinxupquote{get\_par}}}}}(self, objid, parname{[}, filt1, …{]})
&
get parameter of one SN
\\
\hline
{\hyperref[\detokenize{generated/sdapy.snerun.snelist.init_hist_axes:sdapy.snerun.snelist.init_hist_axes}]{\sphinxcrossref{\sphinxcode{\sphinxupquote{init\_hist\_axes}}}}}(self{[}, pad, labelbottom, …{]})
&
create 2 subplots as histograms for the scatter plots
\\
\hline
{\hyperref[\detokenize{generated/sdapy.snerun.snelist.load_data:sdapy.snerun.snelist.load_data}]{\sphinxcrossref{\sphinxcode{\sphinxupquote{load\_data}}}}}(self, objid{[}, force, datafile{]})
&
for each object, load their \sphinxstyleemphasis{snobject} classes if they’re cached before.
\\
\hline
{\hyperref[\detokenize{generated/sdapy.snerun.snelist.parse_meta:sdapy.snerun.snelist.parse_meta}]{\sphinxcrossref{\sphinxcode{\sphinxupquote{parse\_meta}}}}}(self{[}, withnew, source, metafile{]})
&
Read a meta table from local
\\
\hline
{\hyperref[\detokenize{generated/sdapy.snerun.snelist.parse_meta_all:sdapy.snerun.snelist.parse_meta_all}]{\sphinxcrossref{\sphinxcode{\sphinxupquote{parse\_meta\_all}}}}}(self, kwargs, objid)
&
properly read a list of meta infomations from self.meta, i.e.
\\
\hline
{\hyperref[\detokenize{generated/sdapy.snerun.snelist.parse_meta_one:sdapy.snerun.snelist.parse_meta_one}]{\sphinxcrossref{\sphinxcode{\sphinxupquote{parse\_meta\_one}}}}}(self, idkey, objid, key)
&
Obtain value with object ID and a meta key
\\
\hline
{\hyperref[\detokenize{generated/sdapy.snerun.snelist.parse_params:sdapy.snerun.snelist.parse_params}]{\sphinxcrossref{\sphinxcode{\sphinxupquote{parse\_params}}}}}(self{[}, force, parfile{]})
&
Besides the general parameter settings, for SNe with peculiar properties, a specific parameter is sometimes needed, and \sphinxstyleemphasis{parse\_params} can read a text file (individual\_par.txt) that includes all special settings for particular SNe.
\\
\hline
{\hyperref[\detokenize{generated/sdapy.snerun.snelist.read_kwargs:sdapy.snerun.snelist.read_kwargs}]{\sphinxcrossref{\sphinxcode{\sphinxupquote{read\_kwargs}}}}}(self, \textbackslash{}*\textbackslash{}*kwargs)
&
Define a proper way to read and update optional parameters
\\
\hline
{\hyperref[\detokenize{generated/sdapy.snerun.snelist.run:sdapy.snerun.snelist.run}]{\sphinxcrossref{\sphinxcode{\sphinxupquote{run}}}}}(self{[}, ax, ax1, ax2, ax3, ax4, debug{]})
&
get a list of SNe, for each SN, define a dedicated \sphinxstyleemphasis{snobject}, and run snobject.run() for all of them.
\\
\hline
{\hyperref[\detokenize{generated/sdapy.snerun.snelist.save_data:sdapy.snerun.snelist.save_data}]{\sphinxcrossref{\sphinxcode{\sphinxupquote{save\_data}}}}}(self, objid{[}, force, datafile{]})
&
for each object, save their \sphinxstyleemphasis{snobject} classes to local cached files.
\\
\hline
{\hyperref[\detokenize{generated/sdapy.snerun.snelist.show1d:sdapy.snerun.snelist.show1d}]{\sphinxcrossref{\sphinxcode{\sphinxupquote{show1d}}}}}(self, k{[}, nbin, fontsize, labelpad{]})
&
1D histograms plot for one parameter
\\
\hline
{\hyperref[\detokenize{generated/sdapy.snerun.snelist.show2d:sdapy.snerun.snelist.show2d}]{\sphinxcrossref{\sphinxcode{\sphinxupquote{show2d}}}}}(self, k1, k2{[}, fontsize, labelpad{]})
&
2D scatter plot for two parameter
\\
\hline
\sphinxcode{\sphinxupquote{showax}}(self{[}, syntax, showfilt, …{]})
&
make flux plot for a large set of SNe
\\
\hline
\sphinxcode{\sphinxupquote{showax2}}(self{[}, syntax, showfilt{]})
&
make mag plot for a large set of SNe
\\
\hline
\sphinxcode{\sphinxupquote{showax3}}(self{[}, syntax{]})
&

\\
\hline
\sphinxcode{\sphinxupquote{showax4}}(self{[}, syntax, show\_points, show\_fit{]})
&

\\
\hline
\sphinxcode{\sphinxupquote{showvelocity}}(self{[}, syntax, quant{]})
&

\\
\hline
{\hyperref[\detokenize{generated/sdapy.snerun.snelist.table:sdapy.snerun.snelist.table}]{\sphinxcrossref{\sphinxcode{\sphinxupquote{table}}}}}(self{[}, syntax, keys, tablename{]})
&
create a latex table for a subset of SNe
\\
\hline
\end{longtable}\sphinxatlongtableend\end{savenotes}
\subsubsection*{Attributes}


\begin{savenotes}\sphinxatlongtablestart\begin{longtable}[c]{\X{1}{2}\X{1}{2}}
\hline

\endfirsthead

\multicolumn{2}{c}%
{\makebox[0pt]{\sphinxtablecontinued{\tablename\ \thetable{} \textendash{} continued from previous page}}}\\
\hline

\endhead

\hline
\multicolumn{2}{r}{\makebox[0pt][r]{\sphinxtablecontinued{Continued on next page}}}\\
\endfoot

\endlastfoot

\sphinxcode{\sphinxupquote{version}}
&

\\
\hline
\end{longtable}\sphinxatlongtableend\end{savenotes}

\end{fulllineitems}


Below provide various of functions for \sphinxtitleref{snelist}:


\begin{savenotes}\sphinxatlongtablestart\begin{longtable}[c]{\X{1}{2}\X{1}{2}}
\hline

\endfirsthead

\multicolumn{2}{c}%
{\makebox[0pt]{\sphinxtablecontinued{\tablename\ \thetable{} \textendash{} continued from previous page}}}\\
\hline

\endhead

\hline
\multicolumn{2}{r}{\makebox[0pt][r]{\sphinxtablecontinued{Continued on next page}}}\\
\endfoot

\endlastfoot

{\hyperref[\detokenize{generated/sdapy.snerun.snelist.__init__:sdapy.snerun.snelist.__init__}]{\sphinxcrossref{\sphinxcode{\sphinxupquote{snelist.\_\_init\_\_}}}}}(self{[}, ax{]})
&
initialize \sphinxstyleemphasis{snelist}
\\
\hline
{\hyperref[\detokenize{generated/sdapy.snerun.snelist.read_kwargs:sdapy.snerun.snelist.read_kwargs}]{\sphinxcrossref{\sphinxcode{\sphinxupquote{snelist.read\_kwargs}}}}}(self, \textbackslash{}*\textbackslash{}*kwargs)
&
Define a proper way to read and update optional parameters
\\
\hline
{\hyperref[\detokenize{generated/sdapy.snerun.snelist.parse_meta:sdapy.snerun.snelist.parse_meta}]{\sphinxcrossref{\sphinxcode{\sphinxupquote{snelist.parse\_meta}}}}}(self{[}, withnew, source, …{]})
&
Read a meta table from local
\\
\hline
{\hyperref[\detokenize{generated/sdapy.snerun.snelist.parse_meta_one:sdapy.snerun.snelist.parse_meta_one}]{\sphinxcrossref{\sphinxcode{\sphinxupquote{snelist.parse\_meta\_one}}}}}(self, idkey, objid, key)
&
Obtain value with object ID and a meta key
\\
\hline
{\hyperref[\detokenize{generated/sdapy.snerun.snelist.parse_meta_all:sdapy.snerun.snelist.parse_meta_all}]{\sphinxcrossref{\sphinxcode{\sphinxupquote{snelist.parse\_meta\_all}}}}}(self, kwargs, objid)
&
properly read a list of meta infomations from self.meta, i.e.
\\
\hline
{\hyperref[\detokenize{generated/sdapy.snerun.snelist.parse_params:sdapy.snerun.snelist.parse_params}]{\sphinxcrossref{\sphinxcode{\sphinxupquote{snelist.parse\_params}}}}}(self{[}, force, parfile{]})
&
Besides the general parameter settings, for SNe with peculiar properties, a specific parameter is sometimes needed, and \sphinxstyleemphasis{parse\_params} can read a text file (individual\_par.txt) that includes all special settings for particular SNe.
\\
\hline
{\hyperref[\detokenize{generated/sdapy.snerun.snelist.load_data:sdapy.snerun.snelist.load_data}]{\sphinxcrossref{\sphinxcode{\sphinxupquote{snelist.load\_data}}}}}(self, objid{[}, force, datafile{]})
&
for each object, load their \sphinxstyleemphasis{snobject} classes if they’re cached before.
\\
\hline
{\hyperref[\detokenize{generated/sdapy.snerun.snelist.save_data:sdapy.snerun.snelist.save_data}]{\sphinxcrossref{\sphinxcode{\sphinxupquote{snelist.save\_data}}}}}(self, objid{[}, force, datafile{]})
&
for each object, save their \sphinxstyleemphasis{snobject} classes to local cached files.
\\
\hline
{\hyperref[\detokenize{generated/sdapy.snerun.snelist.run:sdapy.snerun.snelist.run}]{\sphinxcrossref{\sphinxcode{\sphinxupquote{snelist.run}}}}}(self{[}, ax, ax1, ax2, ax3, ax4, …{]})
&
get a list of SNe, for each SN, define a dedicated \sphinxstyleemphasis{snobject}, and run snobject.run() for all of them.
\\
\hline
{\hyperref[\detokenize{generated/sdapy.snerun.snelist.add_subset:sdapy.snerun.snelist.add_subset}]{\sphinxcrossref{\sphinxcode{\sphinxupquote{snelist.add\_subset}}}}}(self{[}, syntax{]})
&
create a data subset from self.meta
\\
\hline
{\hyperref[\detokenize{generated/sdapy.snerun.snelist.table:sdapy.snerun.snelist.table}]{\sphinxcrossref{\sphinxcode{\sphinxupquote{snelist.table}}}}}(self{[}, syntax, keys, tablename{]})
&
create a latex table for a subset of SNe
\\
\hline
{\hyperref[\detokenize{generated/sdapy.snerun.snelist.get_par:sdapy.snerun.snelist.get_par}]{\sphinxcrossref{\sphinxcode{\sphinxupquote{snelist.get\_par}}}}}(self, objid, parname{[}, …{]})
&
get parameter of one SN
\\
\hline
{\hyperref[\detokenize{generated/sdapy.snerun.snelist.format_par:sdapy.snerun.snelist.format_par}]{\sphinxcrossref{\sphinxcode{\sphinxupquote{snelist.format\_par}}}}}(v, vlow, vup{[}, digits{]})
&
make parameter into latex format
\\
\hline
{\hyperref[\detokenize{generated/sdapy.snerun.snelist.show1d:sdapy.snerun.snelist.show1d}]{\sphinxcrossref{\sphinxcode{\sphinxupquote{snelist.show1d}}}}}(self, k{[}, nbin, fontsize, …{]})
&
1D histograms plot for one parameter
\\
\hline
{\hyperref[\detokenize{generated/sdapy.snerun.snelist.show2d:sdapy.snerun.snelist.show2d}]{\sphinxcrossref{\sphinxcode{\sphinxupquote{snelist.show2d}}}}}(self, k1, k2{[}, fontsize, …{]})
&
2D scatter plot for two parameter
\\
\hline
{\hyperref[\detokenize{generated/sdapy.snerun.snelist.init_hist_axes:sdapy.snerun.snelist.init_hist_axes}]{\sphinxcrossref{\sphinxcode{\sphinxupquote{snelist.init\_hist\_axes}}}}}(self{[}, pad, …{]})
&
create 2 subplots as histograms for the scatter plots
\\
\hline
{\hyperref[\detokenize{generated/sdapy.snerun.snelist.add_hist:sdapy.snerun.snelist.add_hist}]{\sphinxcrossref{\sphinxcode{\sphinxupquote{snelist.add\_hist}}}}}(self, x, y{[}, syntax, …{]})
&
make histograms
\\
\hline
\end{longtable}\sphinxatlongtableend\end{savenotes}


\subsection{sdapy.snerun.snelist.\_\_init\_\_}
\label{\detokenize{generated/sdapy.snerun.snelist.__init__:sdapy-snerun-snelist-init}}\label{\detokenize{generated/sdapy.snerun.snelist.__init__::doc}}\index{\_\_init\_\_() (sdapy.snerun.snelist method)@\spxentry{\_\_init\_\_()}\spxextra{sdapy.snerun.snelist method}}

\begin{fulllineitems}
\phantomsection\label{\detokenize{generated/sdapy.snerun.snelist.__init__:sdapy.snerun.snelist.__init__}}\pysiglinewithargsret{\sphinxcode{\sphinxupquote{snelist.}}\sphinxbfcode{\sphinxupquote{\_\_init\_\_}}}{\emph{self}, \emph{ax=None}, \emph{**kwargs}}{}
initialize \sphinxstyleemphasis{snelist}
\begin{quote}\begin{description}
\item[{Parameters}] \leavevmode\begin{description}
\item[{\sphinxstylestrong{ax}}] \leavevmode{[}matplotlib.axes{]}
matplotlib.axes, used for histogram/scatter or other population plots

\item[{\sphinxstylestrong{kwargs}}] \leavevmode{[}\sphinxtitleref{Keyword Arguments}{]}
see \sphinxurl{https://github.com/saberyoung/HAFFET/blob/master/sdapy/data/default\_par.txt},
\sphinxstylestrong{snelist} part

\end{description}

\end{description}\end{quote}
\subsubsection*{Examples}

\begin{sphinxVerbatim}[commandchars=\\\{\}]
\PYG{g+gp}{\PYGZgt{}\PYGZgt{}\PYGZgt{} }\PYG{k+kn}{from} \PYG{n+nn}{sdapy} \PYG{k+kn}{import} \PYG{n}{snerun}
\PYG{g+gp}{\PYGZgt{}\PYGZgt{}\PYGZgt{} }\PYG{n}{a} \PYG{o}{=} \PYG{n}{snerun}\PYG{o}{.}\PYG{n}{snelist}\PYG{p}{(}\PYG{p}{)}
\PYG{g+gp}{\PYGZgt{}\PYGZgt{}\PYGZgt{} }\PYG{n}{a}
\PYG{g+go}{\PYGZlt{}sdapy.snerun.snelist object at 0x7fd6fb805f60\PYGZgt{}}
\end{sphinxVerbatim}

\end{fulllineitems}



\section{\sphinxstyleliteralintitle{\sphinxupquote{snobject}} \textendash{} class deal with one single object}
\label{\detokenize{snobject:snobject-class-deal-with-one-single-object}}\label{\detokenize{snobject:snobject}}\label{\detokenize{snobject::doc}}

\begin{savenotes}\sphinxatlongtablestart\begin{longtable}[c]{\X{1}{2}\X{1}{2}}
\hline

\endfirsthead

\multicolumn{2}{c}%
{\makebox[0pt]{\sphinxtablecontinued{\tablename\ \thetable{} \textendash{} continued from previous page}}}\\
\hline

\endhead

\hline
\multicolumn{2}{r}{\makebox[0pt][r]{\sphinxtablecontinued{Continued on next page}}}\\
\endfoot

\endlastfoot

{\hyperref[\detokenize{generated/sdapy.snerun.snobject:sdapy.snerun.snobject}]{\sphinxcrossref{\sphinxcode{\sphinxupquote{snobject}}}}}(objid{[}, aliasid, z, ra, dec, …{]})
&
snobject: define \sphinxstyleemphasis{snobject} for one SN, handle its data and fittings.
\\
\hline
\end{longtable}\sphinxatlongtableend\end{savenotes}


\subsection{sdapy.snerun.snobject}
\label{\detokenize{generated/sdapy.snerun.snobject:sdapy-snerun-snobject}}\label{\detokenize{generated/sdapy.snerun.snobject::doc}}\index{snobject (class in sdapy.snerun)@\spxentry{snobject}\spxextra{class in sdapy.snerun}}

\begin{fulllineitems}
\phantomsection\label{\detokenize{generated/sdapy.snerun.snobject:sdapy.snerun.snobject}}\pysiglinewithargsret{\sphinxbfcode{\sphinxupquote{class }}\sphinxcode{\sphinxupquote{sdapy.snerun.}}\sphinxbfcode{\sphinxupquote{snobject}}}{\emph{objid}, \emph{aliasid=None}, \emph{z=None}, \emph{ra=None}, \emph{dec=None}, \emph{mkwebv=None}, \emph{hostebv=None}, \emph{sntype=None}, \emph{dm=None}, \emph{jdpeak=None}, \emph{fig=None}, \emph{ax=None}, \emph{ax1=None}, \emph{ax2=None}, \emph{ax3=None}, \emph{ax4=None}, \emph{**kwargs}}{}
snobject: define \sphinxstyleemphasis{snobject} for one SN, handle its data and fittings.


\sphinxstrong{See also:}

\begin{description}
\item[{{\hyperref[\detokenize{generated/sdapy.snerun.snelist:sdapy.snerun.snelist}]{\sphinxcrossref{\sphinxcode{\sphinxupquote{snelist}}}}}}] \leavevmode
\end{description}


\subsubsection*{Methods}


\begin{savenotes}\sphinxatlongtablestart\begin{longtable}[c]{\X{1}{2}\X{1}{2}}
\hline

\endfirsthead

\multicolumn{2}{c}%
{\makebox[0pt]{\sphinxtablecontinued{\tablename\ \thetable{} \textendash{} continued from previous page}}}\\
\hline

\endhead

\hline
\multicolumn{2}{r}{\makebox[0pt][r]{\sphinxtablecontinued{Continued on next page}}}\\
\endfoot

\endlastfoot

{\hyperref[\detokenize{generated/sdapy.snerun.snobject.add_flux:sdapy.snerun.snobject.add_flux}]{\sphinxcrossref{\sphinxcode{\sphinxupquote{add\_flux}}}}}(self{[}, zp, source{]})
&
add \sphinxstylestrong{flux} and \sphinxstylestrong{eflux} column, based on \sphinxstylestrong{mag}, \sphinxstylestrong{emag}, and/or \sphinxstylestrong{limmag} column.
\\
\hline
{\hyperref[\detokenize{generated/sdapy.snerun.snobject.add_lc:sdapy.snerun.snobject.add_lc}]{\sphinxcrossref{\sphinxcode{\sphinxupquote{add\_lc}}}}}(self, df{[}, source{]})
&
add lightcurve data into self.lc
\\
\hline
{\hyperref[\detokenize{generated/sdapy.snerun.snobject.add_mag:sdapy.snerun.snobject.add_mag}]{\sphinxcrossref{\sphinxcode{\sphinxupquote{add\_mag}}}}}(self{[}, zp, source{]})
&
add \sphinxstylestrong{mag} and/or \sphinxstyleemphasis{limmag}/\sphinxstylestrong{emag} column, based on \sphinxstylestrong{flux}, \sphinxstylestrong{eflux} column.
\\
\hline
{\hyperref[\detokenize{generated/sdapy.snerun.snobject.bb_bol:sdapy.snerun.snobject.bb_bol}]{\sphinxcrossref{\sphinxcode{\sphinxupquote{bb\_bol}}}}}(self{[}, do\_Kcorr, ab2vega{]})
&
calculate bolometric LC with diluted blackbody
\\
\hline
\sphinxcode{\sphinxupquote{bb\_bol\_show}}(self, ax, phase{[}, filters, sep, …{]})
&
show constructed SED of specific epochs
\\
\hline
{\hyperref[\detokenize{generated/sdapy.snerun.snobject.bb_colors:sdapy.snerun.snobject.bb_colors}]{\sphinxcrossref{\sphinxcode{\sphinxupquote{bb\_colors}}}}}(self, \textbackslash{}*\textbackslash{}*kwargs)
&
match colours for multiple bands, for blackbody (BB) construction
\\
\hline
{\hyperref[\detokenize{generated/sdapy.snerun.snobject.bin_fp_atlas:sdapy.snerun.snobject.bin_fp_atlas}]{\sphinxcrossref{\sphinxcode{\sphinxupquote{bin\_fp\_atlas}}}}}(self{[}, binDays, resultsPath, …{]})
&
binning for ATLAS forced photometry with Dave’s code, \sphinxurl{https://gist.github.com/thespacedoctor/86777fa5a9567b7939e8d84fd8cf6a76}
\\
\hline
{\hyperref[\detokenize{generated/sdapy.snerun.snobject.calc_colors:sdapy.snerun.snobject.calc_colors}]{\sphinxcrossref{\sphinxcode{\sphinxupquote{calc\_colors}}}}}(self, \textbackslash{}*\textbackslash{}*kwargs)
&
calculate colors for two bands
\\
\hline
{\hyperref[\detokenize{generated/sdapy.snerun.snobject.calibrate_baseline:sdapy.snerun.snobject.calibrate_baseline}]{\sphinxcrossref{\sphinxcode{\sphinxupquote{calibrate\_baseline}}}}}(self{[}, ax, key, source, …{]})
&
Baseline calibration for ZTF forced photometry
\\
\hline
{\hyperref[\detokenize{generated/sdapy.snerun.snobject.clip_lc:sdapy.snerun.snobject.clip_lc}]{\sphinxcrossref{\sphinxcode{\sphinxupquote{clip\_lc}}}}}(self, \textbackslash{}*\textbackslash{}*kwargs)
&
Removes outlier data points using GP interpolation
\\
\hline
{\hyperref[\detokenize{generated/sdapy.snerun.snobject.combine_multi_obs:sdapy.snerun.snobject.combine_multi_obs}]{\sphinxcrossref{\sphinxcode{\sphinxupquote{combine\_multi\_obs}}}}}(self, \textbackslash{}*\textbackslash{}*kwargs)
&
(static) bin and combine observations from common epoch
\\
\hline
{\hyperref[\detokenize{generated/sdapy.snerun.snobject.config_ztfquery:sdapy.snerun.snobject.config_ztfquery}]{\sphinxcrossref{\sphinxcode{\sphinxupquote{config\_ztfquery}}}}}(self)
&
check and set ztfquery accounts
\\
\hline
{\hyperref[\detokenize{generated/sdapy.snerun.snobject.correct_baseline:sdapy.snerun.snobject.correct_baseline}]{\sphinxcrossref{\sphinxcode{\sphinxupquote{correct\_baseline}}}}}(self, baseline{[}, key, source{]})
&
Baseline correction for ZTF forced photometry
\\
\hline
{\hyperref[\detokenize{generated/sdapy.snerun.snobject.est_hostebv_with_c10:sdapy.snerun.snobject.est_hostebv_with_c10}]{\sphinxcrossref{\sphinxcode{\sphinxupquote{est\_hostebv\_with\_c10}}}}}(self{[}, interpolation{]})
&
estimate host ebv with color comparison approach
\\
\hline
{\hyperref[\detokenize{generated/sdapy.snerun.snobject.flux_to_mag:sdapy.snerun.snobject.flux_to_mag}]{\sphinxcrossref{\sphinxcode{\sphinxupquote{flux\_to\_mag}}}}}(flux{[}, dflux, sigma, units, zp, …{]})
&
Converts fluxes (erg/s/cm2/A) into AB magnitudes with flux/mag zerop
\\
\hline
{\hyperref[\detokenize{generated/sdapy.snerun.snobject.get_alert_ztf:sdapy.snerun.snobject.get_alert_ztf}]{\sphinxcrossref{\sphinxcode{\sphinxupquote{get\_alert\_ztf}}}}}(self{[}, source{]})
&
parse local ZTF alert photometic data to \sphinxstyleemphasis{self.lc}
\\
\hline
{\hyperref[\detokenize{generated/sdapy.snerun.snobject.get_external_phot:sdapy.snerun.snobject.get_external_phot}]{\sphinxcrossref{\sphinxcode{\sphinxupquote{get\_external\_phot}}}}}(self, filename, source, …)
&
parse user defined photometric data to \sphinxstyleemphasis{self.lc}
\\
\hline
{\hyperref[\detokenize{generated/sdapy.snerun.snobject.get_external_spectra:sdapy.snerun.snobject.get_external_spectra}]{\sphinxcrossref{\sphinxcode{\sphinxupquote{get\_external\_spectra}}}}}(self, filename, epoch)
&
parse \sphinxstylestrong{spectra} via a local file
\\
\hline
{\hyperref[\detokenize{generated/sdapy.snerun.snobject.get_fp_atlas:sdapy.snerun.snobject.get_fp_atlas}]{\sphinxcrossref{\sphinxcode{\sphinxupquote{get\_fp\_atlas}}}}}(self{[}, binDays, clobber{]})
&
parse local ATLAS forced (binned or not binned) photometric data to \sphinxstyleemphasis{self.lc}
\\
\hline
{\hyperref[\detokenize{generated/sdapy.snerun.snobject.get_fp_ztf:sdapy.snerun.snobject.get_fp_ztf}]{\sphinxcrossref{\sphinxcode{\sphinxupquote{get\_fp\_ztf}}}}}(self{[}, seeing\_cut{]})
&
parse local ZTF forced photometic data to \sphinxstyleemphasis{self.lc}
\\
\hline
{\hyperref[\detokenize{generated/sdapy.snerun.snobject.get_local_spectra:sdapy.snerun.snobject.get_local_spectra}]{\sphinxcrossref{\sphinxcode{\sphinxupquote{get\_local\_spectra}}}}}(self{[}, source{]})
&
get local ZTF fritz/marshal spectra         \sphinxstyleemphasis{Note}: this is only available for ZTF internal collaborators.
\\
\hline
{\hyperref[\detokenize{generated/sdapy.snerun.snobject.get_oac:sdapy.snerun.snobject.get_oac}]{\sphinxcrossref{\sphinxcode{\sphinxupquote{get\_oac}}}}}(self{[}, which{]})
&
parse local Open Astronomical Catalog data to \sphinxstyleemphasis{self.lc}
\\
\hline
{\hyperref[\detokenize{generated/sdapy.snerun.snobject.lyman_bol:sdapy.snerun.snobject.lyman_bol}]{\sphinxcrossref{\sphinxcode{\sphinxupquote{lyman\_bol}}}}}(self{[}, interpolation{]})
&
calculate bolometric LC from colours, with Lyman bolometric correction
\\
\hline
{\hyperref[\detokenize{generated/sdapy.snerun.snobject.mag_to_flux:sdapy.snerun.snobject.mag_to_flux}]{\sphinxcrossref{\sphinxcode{\sphinxupquote{mag\_to\_flux}}}}}(mag{[}, magerr, limmag, sigma, …{]})
&
with flux/mag zerop
\\
\hline
{\hyperref[\detokenize{generated/sdapy.snerun.snobject.match_colors:sdapy.snerun.snobject.match_colors}]{\sphinxcrossref{\sphinxcode{\sphinxupquote{match\_colors}}}}}(self, \textbackslash{}*\textbackslash{}*kwargs)
&
match colors
\\
\hline
{\hyperref[\detokenize{generated/sdapy.snerun.snobject.merge_df_cols:sdapy.snerun.snobject.merge_df_cols}]{\sphinxcrossref{\sphinxcode{\sphinxupquote{merge\_df\_cols}}}}}(\_snobject\_\_df)
&
will bug when combining columns from differents sources, i.e.
\\
\hline
{\hyperref[\detokenize{generated/sdapy.snerun.snobject.mjd_now:sdapy.snerun.snobject.mjd_now}]{\sphinxcrossref{\sphinxcode{\sphinxupquote{mjd\_now}}}}}(self{[}, jd{]})
&
get current juliand date via astropy.time
\\
\hline
{\hyperref[\detokenize{generated/sdapy.snerun.snobject.oac_phot_url:sdapy.snerun.snobject.oac_phot_url}]{\sphinxcrossref{\sphinxcode{\sphinxupquote{oac\_phot\_url}}}}}(self, url)
&
(static) make url to query OAC
\\
\hline
{\hyperref[\detokenize{generated/sdapy.snerun.snobject.parse_coo:sdapy.snerun.snobject.parse_coo}]{\sphinxcrossref{\sphinxcode{\sphinxupquote{parse\_coo}}}}}(self{[}, verbose, deg, hpx, nside{]})
&
handle SN coordinate from self.ra and self.dec
\\
\hline
{\hyperref[\detokenize{generated/sdapy.snerun.snobject.query_alert_ztf:sdapy.snerun.snobject.query_alert_ztf}]{\sphinxcrossref{\sphinxcode{\sphinxupquote{query\_alert\_ztf}}}}}(self{[}, source{]})
&
qeury ZTF laert photometry via ztfquery, see \sphinxurl{https://github.com/MickaelRigault/ztfquery} \sphinxstyleemphasis{Note}: this is only available for ZTF internal collaborators.
\\
\hline
{\hyperref[\detokenize{generated/sdapy.snerun.snobject.query_fp_atlas:sdapy.snerun.snobject.query_fp_atlas}]{\sphinxcrossref{\sphinxcode{\sphinxupquote{query\_fp\_atlas}}}}}(self, \textbackslash{}*\textbackslash{}*kwargs)
&
qeury ATLAS forced photometry,  see \sphinxurl{https://fallingstar-data.com/forcedphot/static/apiexample.py}
\\
\hline
{\hyperref[\detokenize{generated/sdapy.snerun.snobject.query_fp_ztf:sdapy.snerun.snobject.query_fp_ztf}]{\sphinxcrossref{\sphinxcode{\sphinxupquote{query\_fp\_ztf}}}}}(self{[}, get\_email{]})
&
qeury ZTF forced photometry,  see documentation: \sphinxurl{https://irsa.ipac.caltech.edu/data/ZTF/docs/forcedphot.pdf}.
\\
\hline
{\hyperref[\detokenize{generated/sdapy.snerun.snobject.query_oac:sdapy.snerun.snobject.query_oac}]{\sphinxcrossref{\sphinxcode{\sphinxupquote{query\_oac}}}}}(self{[}, db, which{]})
&
query Open Astronomical Catalog data,  via OACAPI, \sphinxurl{https://github.com/astrocatalogs/OACAPI}
\\
\hline
{\hyperref[\detokenize{generated/sdapy.snerun.snobject.query_spectra:sdapy.snerun.snobject.query_spectra}]{\sphinxcrossref{\sphinxcode{\sphinxupquote{query\_spectra}}}}}(self{[}, source{]})
&
qeury ZTF spectra via ztfquery, see \sphinxurl{https://github.com/MickaelRigault/ztfquery} \sphinxstyleemphasis{Note}: this is only available for ZTF internal collaborators.
\\
\hline
{\hyperref[\detokenize{generated/sdapy.snerun.snobject.rapid:sdapy.snerun.snobject.rapid}]{\sphinxcrossref{\sphinxcode{\sphinxupquote{rapid}}}}}(self{[}, ax{]})
&
run astrorapid codes (a Deep learning classifier to distiinguish LCs between different SNe type) on \sphinxstylestrong{self.lc}   \sphinxstyleemphasis{Note}: this is only available when astrorapid packagte is correcly installed
\\
\hline
{\hyperref[\detokenize{generated/sdapy.snerun.snobject.rapid_plot:sdapy.snerun.snobject.rapid_plot}]{\sphinxcrossref{\sphinxcode{\sphinxupquote{rapid\_plot}}}}}(self, ax, class\_names, class\_color)
&
(static) make astrorapid plot
\\
\hline
{\hyperref[\detokenize{generated/sdapy.snerun.snobject.read_kwargs:sdapy.snerun.snobject.read_kwargs}]{\sphinxcrossref{\sphinxcode{\sphinxupquote{read\_kwargs}}}}}(self, \textbackslash{}*\textbackslash{}*kwargs)
&
Define a proper way to read and update optional parameters
\\
\hline
{\hyperref[\detokenize{generated/sdapy.snerun.snobject.run:sdapy.snerun.snobject.run}]{\sphinxcrossref{\sphinxcode{\sphinxupquote{run}}}}}(self, \textbackslash{}*\textbackslash{}*kwargs)
&
actions for one SN.
\\
\hline
{\hyperref[\detokenize{generated/sdapy.snerun.snobject.run_fit:sdapy.snerun.snobject.run_fit}]{\sphinxcrossref{\sphinxcode{\sphinxupquote{run\_fit}}}}}(self, enginetype{[}, source{]})
&
run model fittings on varies of data from different engines
\\
\hline
{\hyperref[\detokenize{generated/sdapy.snerun.snobject.run_gp:sdapy.snerun.snobject.run_gp}]{\sphinxcrossref{\sphinxcode{\sphinxupquote{run\_gp}}}}}(self{[}, source{]})
&
run Gaussian Process interpolation via george package on \sphinxstylestrong{self.lc}
\\
\hline
{\hyperref[\detokenize{generated/sdapy.snerun.snobject.set_peak_bol_main:sdapy.snerun.snobject.set_peak_bol_main}]{\sphinxcrossref{\sphinxcode{\sphinxupquote{set\_peak\_bol\_main}}}}}(self{[}, model\_name, …{]})
&
set peak time and fluxes with bolometric LC fittings
\\
\hline
{\hyperref[\detokenize{generated/sdapy.snerun.snobject.set_peak_gp:sdapy.snerun.snobject.set_peak_gp}]{\sphinxcrossref{\sphinxcode{\sphinxupquote{set\_peak\_gp}}}}}(self, filt)
&
set peak time and fluxes with GP interpolation
\\
\hline
{\hyperref[\detokenize{generated/sdapy.snerun.snobject.set_peak_multiband_main:sdapy.snerun.snobject.set_peak_multiband_main}]{\sphinxcrossref{\sphinxcode{\sphinxupquote{set\_peak\_multiband\_main}}}}}(self, filt{[}, model\_name{]})
&
set peak time and fluxes with multiband LC fittings
\\
\hline
{\hyperref[\detokenize{generated/sdapy.snerun.snobject.set_texp_bol_main:sdapy.snerun.snobject.set_texp_bol_main}]{\sphinxcrossref{\sphinxcode{\sphinxupquote{set\_texp\_bol\_main}}}}}(self{[}, model\_name, …{]})
&
set the explosion epoch with bolometric LC fittings
\\
\hline
{\hyperref[\detokenize{generated/sdapy.snerun.snobject.set_texp_midway:sdapy.snerun.snobject.set_texp_midway}]{\sphinxcrossref{\sphinxcode{\sphinxupquote{set\_texp\_midway}}}}}(self)
&
set the explosion epoch with the first detection and the last non\sphinxhyphen{}detection
\\
\hline
{\hyperref[\detokenize{generated/sdapy.snerun.snobject.set_texp_pl:sdapy.snerun.snobject.set_texp_pl}]{\sphinxcrossref{\sphinxcode{\sphinxupquote{set\_texp\_pl}}}}}(self, filt{[}, model\_name{]})
&
set the explosion epoch with multiband LC fittings
\\
\hline
{\hyperref[\detokenize{generated/sdapy.snerun.snobject.summarize_plot:sdapy.snerun.snobject.summarize_plot}]{\sphinxcrossref{\sphinxcode{\sphinxupquote{summarize\_plot}}}}}(self{[}, ax{]})
&
show a summarize plot: flux/absolute/bolometric LCs, colour curve and the spectra
\\
\hline
{\hyperref[\detokenize{generated/sdapy.snerun.snobject.sym_mag:sdapy.snerun.snobject.sym_mag}]{\sphinxcrossref{\sphinxcode{\sphinxupquote{sym\_mag}}}}}(self, w, f, filt)
&
(static) synthetic magnitudes with filter transmission curves
\\
\hline
\end{longtable}\sphinxatlongtableend\end{savenotes}


\subsubsection{sdapy.snerun.snobject.add\_flux}
\label{\detokenize{generated/sdapy.snerun.snobject.add_flux:sdapy-snerun-snobject-add-flux}}\label{\detokenize{generated/sdapy.snerun.snobject.add_flux::doc}}\index{add\_flux() (sdapy.snerun.snobject method)@\spxentry{add\_flux()}\spxextra{sdapy.snerun.snobject method}}

\begin{fulllineitems}
\phantomsection\label{\detokenize{generated/sdapy.snerun.snobject.add_flux:sdapy.snerun.snobject.add_flux}}\pysiglinewithargsret{\sphinxcode{\sphinxupquote{snobject.}}\sphinxbfcode{\sphinxupquote{add\_flux}}}{\emph{self}, \emph{zp=23.9}, \emph{source=None}, \emph{**kwargs}}{}
add \sphinxstylestrong{flux} and \sphinxstylestrong{eflux} column, based on \sphinxstylestrong{mag}, \sphinxstylestrong{emag}, and/or \sphinxstylestrong{limmag} column.
\begin{quote}\begin{description}
\item[{Parameters}] \leavevmode\begin{description}
\item[{\sphinxstylestrong{df}}] \leavevmode{[}\sphinxtitleref{panda.dataframe}{]}
lightcurve data

\item[{\sphinxstylestrong{zp}}] \leavevmode{[}\sphinxtitleref{float}{]}
zeropoint to convert magnitude to flux.
zp = 23.9 for micro Jansky to AB mag
zp = 48.6 for ergs/s/cm2/Hz to AB mag
e.g. mab = \sphinxhyphen{}2.5 * log10(fv{[}Jy{]}/3631) = \sphinxhyphen{}2.5 * log10(fv{[}mJy{]}) + 2.5*log10(3631*1e6)

\item[{\sphinxstylestrong{snrt}}] \leavevmode{[}\sphinxtitleref{float}{]}
SNR threshold to distinguish detection/limit

\end{description}

\end{description}\end{quote}

\end{fulllineitems}



\subsubsection{sdapy.snerun.snobject.add\_lc}
\label{\detokenize{generated/sdapy.snerun.snobject.add_lc:sdapy-snerun-snobject-add-lc}}\label{\detokenize{generated/sdapy.snerun.snobject.add_lc::doc}}\index{add\_lc() (sdapy.snerun.snobject method)@\spxentry{add\_lc()}\spxextra{sdapy.snerun.snobject method}}

\begin{fulllineitems}
\phantomsection\label{\detokenize{generated/sdapy.snerun.snobject.add_lc:sdapy.snerun.snobject.add_lc}}\pysiglinewithargsret{\sphinxcode{\sphinxupquote{snobject.}}\sphinxbfcode{\sphinxupquote{add\_lc}}}{\emph{self}, \emph{df}, \emph{source=None}}{}
add lightcurve data into self.lc
\begin{quote}\begin{description}
\item[{Parameters}] \leavevmode\begin{description}
\item[{\sphinxstylestrong{df}}] \leavevmode{[}\sphinxtitleref{panda.dataframe}{]}
lightcurve data

\item[{\sphinxstylestrong{source}}] \leavevmode{[}\sphinxtitleref{str}{]}
source name, e.g. ztffp for ZTF forced photometry

\end{description}

\end{description}\end{quote}

\end{fulllineitems}



\subsubsection{sdapy.snerun.snobject.add\_mag}
\label{\detokenize{generated/sdapy.snerun.snobject.add_mag:sdapy-snerun-snobject-add-mag}}\label{\detokenize{generated/sdapy.snerun.snobject.add_mag::doc}}\index{add\_mag() (sdapy.snerun.snobject method)@\spxentry{add\_mag()}\spxextra{sdapy.snerun.snobject method}}

\begin{fulllineitems}
\phantomsection\label{\detokenize{generated/sdapy.snerun.snobject.add_mag:sdapy.snerun.snobject.add_mag}}\pysiglinewithargsret{\sphinxcode{\sphinxupquote{snobject.}}\sphinxbfcode{\sphinxupquote{add\_mag}}}{\emph{self}, \emph{zp=23.9}, \emph{source=None}, \emph{**kwargs}}{}
add \sphinxstylestrong{mag} and/or \sphinxstyleemphasis{limmag}/\sphinxstylestrong{emag} column, based on \sphinxstylestrong{flux}, \sphinxstylestrong{eflux} column.
\begin{quote}\begin{description}
\item[{Parameters}] \leavevmode\begin{description}
\item[{\sphinxstylestrong{df}}] \leavevmode{[}\sphinxtitleref{panda.dataframe}{]}
lightcurve data

\item[{\sphinxstylestrong{zp}}] \leavevmode{[}\sphinxtitleref{float}{]}
zeropoint to convert flux to magnitude.
zp = 23.9 for micro Jansky to AB mag
zp = 48.6 for ergs/s/cm2/Hz to AB mag
e.g. mab = \sphinxhyphen{}2.5 * log10(fv{[}Jy{]}/3631) = \sphinxhyphen{}2.5 * log10(fv{[}mJy{]}) + 2.5*log10(3631*1e6)

\item[{\sphinxstylestrong{snrt}}] \leavevmode{[}\sphinxtitleref{float}{]}
SNR threshold to distinguish detection/limit

\end{description}

\end{description}\end{quote}

\end{fulllineitems}



\subsubsection{sdapy.snerun.snobject.bb\_bol}
\label{\detokenize{generated/sdapy.snerun.snobject.bb_bol:sdapy-snerun-snobject-bb-bol}}\label{\detokenize{generated/sdapy.snerun.snobject.bb_bol::doc}}\index{bb\_bol() (sdapy.snerun.snobject method)@\spxentry{bb\_bol()}\spxextra{sdapy.snerun.snobject method}}

\begin{fulllineitems}
\phantomsection\label{\detokenize{generated/sdapy.snerun.snobject.bb_bol:sdapy.snerun.snobject.bb_bol}}\pysiglinewithargsret{\sphinxcode{\sphinxupquote{snobject.}}\sphinxbfcode{\sphinxupquote{bb\_bol}}}{\emph{self}, \emph{do\_Kcorr=True}, \emph{ab2vega=True}, \emph{**kwargs}}{}
calculate bolometric LC with diluted blackbody
\begin{quote}\begin{description}
\item[{Parameters}] \leavevmode\begin{description}
\item[{\sphinxstylestrong{bb\_bands}}] \leavevmode{[}\sphinxtitleref{list}{]}
multiple bands of the BB

\item[{\sphinxstylestrong{do\_Kcorr}}] \leavevmode{[}\sphinxtitleref{bol}{]}
if K corrected

\item[{\sphinxstylestrong{ab2vega}}] \leavevmode{[}\sphinxtitleref{bol}{]}
vega mag should be used for Swift data. If inputs are in AB mag, set this var to True

\item[{\sphinxstylestrong{maxfev}}] \leavevmode{[}\sphinxtitleref{int}{]}
for scipy. The maximum number of calls to the function.

\item[{\sphinxstylestrong{verbose}}] \leavevmode{[}\sphinxtitleref{bol}{]}
show process

\end{description}

\end{description}\end{quote}

\end{fulllineitems}



\subsubsection{sdapy.snerun.snobject.bb\_colors}
\label{\detokenize{generated/sdapy.snerun.snobject.bb_colors:sdapy-snerun-snobject-bb-colors}}\label{\detokenize{generated/sdapy.snerun.snobject.bb_colors::doc}}\index{bb\_colors() (sdapy.snerun.snobject method)@\spxentry{bb\_colors()}\spxextra{sdapy.snerun.snobject method}}

\begin{fulllineitems}
\phantomsection\label{\detokenize{generated/sdapy.snerun.snobject.bb_colors:sdapy.snerun.snobject.bb_colors}}\pysiglinewithargsret{\sphinxcode{\sphinxupquote{snobject.}}\sphinxbfcode{\sphinxupquote{bb\_colors}}}{\emph{self}, \emph{**kwargs}}{}
match colours for multiple bands, for blackbody (BB) construction
\begin{quote}\begin{description}
\item[{Parameters}] \leavevmode\begin{description}
\item[{\sphinxstylestrong{bb\_bands}}] \leavevmode{[}\sphinxtitleref{list}{]}
multiple bands of the BB

\item[{\sphinxstylestrong{tdbin}}] \leavevmode{[}\sphinxtitleref{float}{]}
threshold for binning

\item[{\sphinxstylestrong{bb\_interp}}] \leavevmode{[}\sphinxtitleref{str}{]}
estimate flux with data epoch less than than \sphinxstylestrong{tdbin}, or interpolation from GP/fits

\item[{\sphinxstylestrong{quantile}}] \leavevmode{[}\sphinxtitleref{list}{]}
use 50 percentile as mean, and 1 sigma (68\%\% \sphinxhyphen{}\textgreater{} 16\%\% \sphinxhyphen{} 84\%\%) as errors

\item[{\sphinxstylestrong{snrt}}] \leavevmode{[}\sphinxtitleref{float}{]}
SNR threshold to distinguish detection/limit

\end{description}

\end{description}\end{quote}


\sphinxstrong{See also:}

\begin{description}
\item[{{\hyperref[\detokenize{generated/sdapy.snerun.snobject.calc_colors:sdapy.snerun.snobject.calc_colors}]{\sphinxcrossref{\sphinxcode{\sphinxupquote{snobject.calc\_colors}}}}}}] \leavevmode
\end{description}


\subsubsection*{Notes}

haven’t included any ebv value

\end{fulllineitems}



\subsubsection{sdapy.snerun.snobject.bin\_fp\_atlas}
\label{\detokenize{generated/sdapy.snerun.snobject.bin_fp_atlas:sdapy-snerun-snobject-bin-fp-atlas}}\label{\detokenize{generated/sdapy.snerun.snobject.bin_fp_atlas::doc}}\index{bin\_fp\_atlas() (sdapy.snerun.snobject method)@\spxentry{bin\_fp\_atlas()}\spxextra{sdapy.snerun.snobject method}}

\begin{fulllineitems}
\phantomsection\label{\detokenize{generated/sdapy.snerun.snobject.bin_fp_atlas:sdapy.snerun.snobject.bin_fp_atlas}}\pysiglinewithargsret{\sphinxcode{\sphinxupquote{snobject.}}\sphinxbfcode{\sphinxupquote{bin\_fp\_atlas}}}{\emph{self}, \emph{binDays=3}, \emph{resultsPath=None}, \emph{outPath=None}, \emph{**kwargs}}{}
binning for ATLAS forced photometry with Dave’s code,
\sphinxurl{https://gist.github.com/thespacedoctor/86777fa5a9567b7939e8d84fd8cf6a76}
\begin{quote}\begin{description}
\item[{Parameters}] \leavevmode\begin{description}
\item[{\sphinxstylestrong{binDays}}] \leavevmode{[}\sphinxtitleref{int}{]}
days bin

\item[{\sphinxstylestrong{resultsPath}}] \leavevmode{[}\sphinxtitleref{str}{]}
path to store the binning plots

\item[{\sphinxstylestrong{outPath}}] \leavevmode{[}\sphinxtitleref{str}{]}
path to store the binned data

\item[{\sphinxstylestrong{verbose:   \textasciigrave{}bool\textasciigrave{}}}] \leavevmode
show detailed running informations

\end{description}

\end{description}\end{quote}

\end{fulllineitems}



\subsubsection{sdapy.snerun.snobject.calc\_colors}
\label{\detokenize{generated/sdapy.snerun.snobject.calc_colors:sdapy-snerun-snobject-calc-colors}}\label{\detokenize{generated/sdapy.snerun.snobject.calc_colors::doc}}\index{calc\_colors() (sdapy.snerun.snobject method)@\spxentry{calc\_colors()}\spxextra{sdapy.snerun.snobject method}}

\begin{fulllineitems}
\phantomsection\label{\detokenize{generated/sdapy.snerun.snobject.calc_colors:sdapy.snerun.snobject.calc_colors}}\pysiglinewithargsret{\sphinxcode{\sphinxupquote{snobject.}}\sphinxbfcode{\sphinxupquote{calc\_colors}}}{\emph{self}, \emph{**kwargs}}{}
calculate colors for two bands
\begin{quote}\begin{description}
\item[{Parameters}] \leavevmode\begin{description}
\item[{\sphinxstylestrong{color\_bands}}] \leavevmode{[}\sphinxtitleref{list}{]}
two bands of the colour

\item[{\sphinxstylestrong{tdbin}}] \leavevmode{[}\sphinxtitleref{float}{]}
threshold for binning

\item[{\sphinxstylestrong{color\_interp}}] \leavevmode{[}\sphinxtitleref{str}{]}
estimate flux with data epoch less than than \sphinxstylestrong{tdbin}, or interpolation from GP/fits

\item[{\sphinxstylestrong{quantile}}] \leavevmode{[}\sphinxtitleref{list}{]}
use 50 percentile as mean, and 1 sigma (68\%\% \sphinxhyphen{}\textgreater{} 16\%\% \sphinxhyphen{} 84\%\%) as errors

\item[{\sphinxstylestrong{snrt}}] \leavevmode{[}\sphinxtitleref{float}{]}
SNR threshold to distinguish detection/limit

\end{description}

\end{description}\end{quote}

\end{fulllineitems}



\subsubsection{sdapy.snerun.snobject.calibrate\_baseline}
\label{\detokenize{generated/sdapy.snerun.snobject.calibrate_baseline:sdapy-snerun-snobject-calibrate-baseline}}\label{\detokenize{generated/sdapy.snerun.snobject.calibrate_baseline::doc}}\index{calibrate\_baseline() (sdapy.snerun.snobject method)@\spxentry{calibrate\_baseline()}\spxextra{sdapy.snerun.snobject method}}

\begin{fulllineitems}
\phantomsection\label{\detokenize{generated/sdapy.snerun.snobject.calibrate_baseline:sdapy.snerun.snobject.calibrate_baseline}}\pysiglinewithargsret{\sphinxcode{\sphinxupquote{snobject.}}\sphinxbfcode{\sphinxupquote{calibrate\_baseline}}}{\emph{self}, \emph{ax=None}, \emph{key=\textquotesingle{}fcqfid\textquotesingle{}}, \emph{source=\textquotesingle{}ztffp\textquotesingle{}}, \emph{xmin=\sphinxhyphen{}100}, \emph{xmax=\sphinxhyphen{}20}, \emph{ax\_xlim=None}, \emph{ax\_ylim=None}}{}
Baseline calibration for ZTF forced photometry
\begin{quote}\begin{description}
\item[{Parameters}] \leavevmode\begin{description}
\item[{\sphinxstylestrong{ax}}] \leavevmode{[}\sphinxtitleref{matplotlib.axes}{]}
matplotlib subplot, is None will not show

\item[{\sphinxstylestrong{key}}] \leavevmode{[}\sphinxtitleref{str}{]}
which column to distinguish photometry between different fields/ccds

\item[{\sphinxstylestrong{source}}] \leavevmode{[}\sphinxtitleref{str}{]}
which source of lc to be corrected

\item[{\sphinxstylestrong{xmin}}] \leavevmode{[}\sphinxtitleref{float}{]}
left side of LC to be used to calibrate the baseline

\item[{\sphinxstylestrong{xmax}}] \leavevmode{[}\sphinxtitleref{float}{]}
right side of LC to be used to calibrate the baseline

\item[{\sphinxstylestrong{ax\_xlim}}] \leavevmode{[}\sphinxtitleref{range}{]}
x axis range for the plot

\item[{\sphinxstylestrong{ax\_ylim}}] \leavevmode{[}\sphinxtitleref{range}{]}
y axis range for the plot

\end{description}

\end{description}\end{quote}

\end{fulllineitems}



\subsubsection{sdapy.snerun.snobject.clip\_lc}
\label{\detokenize{generated/sdapy.snerun.snobject.clip_lc:sdapy-snerun-snobject-clip-lc}}\label{\detokenize{generated/sdapy.snerun.snobject.clip_lc::doc}}\index{clip\_lc() (sdapy.snerun.snobject method)@\spxentry{clip\_lc()}\spxextra{sdapy.snerun.snobject method}}

\begin{fulllineitems}
\phantomsection\label{\detokenize{generated/sdapy.snerun.snobject.clip_lc:sdapy.snerun.snobject.clip_lc}}\pysiglinewithargsret{\sphinxcode{\sphinxupquote{snobject.}}\sphinxbfcode{\sphinxupquote{clip\_lc}}}{\emph{self}, \emph{**kwargs}}{}
Removes outlier data points using GP interpolation
\begin{quote}\begin{description}
\item[{Parameters}] \leavevmode\begin{description}
\item[{\sphinxstylestrong{clipsigma}}] \leavevmode{[}\sphinxtitleref{float}{]}
sigma for LC clipping

\end{description}

\end{description}\end{quote}

\end{fulllineitems}



\subsubsection{sdapy.snerun.snobject.combine\_multi\_obs}
\label{\detokenize{generated/sdapy.snerun.snobject.combine_multi_obs:sdapy-snerun-snobject-combine-multi-obs}}\label{\detokenize{generated/sdapy.snerun.snobject.combine_multi_obs::doc}}\index{combine\_multi\_obs() (sdapy.snerun.snobject method)@\spxentry{combine\_multi\_obs()}\spxextra{sdapy.snerun.snobject method}}

\begin{fulllineitems}
\phantomsection\label{\detokenize{generated/sdapy.snerun.snobject.combine_multi_obs:sdapy.snerun.snobject.combine_multi_obs}}\pysiglinewithargsret{\sphinxcode{\sphinxupquote{snobject.}}\sphinxbfcode{\sphinxupquote{combine\_multi\_obs}}}{\emph{self}, \emph{**kwargs}}{}
(static) bin and combine observations from common epoch
\begin{quote}\begin{description}
\item[{Parameters}] \leavevmode\begin{description}
\item[{\sphinxstylestrong{tdbin}}] \leavevmode{[}\sphinxtitleref{float}{]}
threshold for binning

\end{description}

\end{description}\end{quote}


\sphinxstrong{See also:}

\begin{description}
\item[{{\hyperref[\detokenize{generated/sdapy.snerun.snobject.match_colors:sdapy.snerun.snobject.match_colors}]{\sphinxcrossref{\sphinxcode{\sphinxupquote{snobject.match\_colors}}}}}}] \leavevmode
\end{description}



\end{fulllineitems}



\subsubsection{sdapy.snerun.snobject.config\_ztfquery}
\label{\detokenize{generated/sdapy.snerun.snobject.config_ztfquery:sdapy-snerun-snobject-config-ztfquery}}\label{\detokenize{generated/sdapy.snerun.snobject.config_ztfquery::doc}}\index{config\_ztfquery() (sdapy.snerun.snobject method)@\spxentry{config\_ztfquery()}\spxextra{sdapy.snerun.snobject method}}

\begin{fulllineitems}
\phantomsection\label{\detokenize{generated/sdapy.snerun.snobject.config_ztfquery:sdapy.snerun.snobject.config_ztfquery}}\pysiglinewithargsret{\sphinxcode{\sphinxupquote{snobject.}}\sphinxbfcode{\sphinxupquote{config\_ztfquery}}}{\emph{self}}{}
check and set ztfquery accounts

\end{fulllineitems}



\subsubsection{sdapy.snerun.snobject.correct\_baseline}
\label{\detokenize{generated/sdapy.snerun.snobject.correct_baseline:sdapy-snerun-snobject-correct-baseline}}\label{\detokenize{generated/sdapy.snerun.snobject.correct_baseline::doc}}\index{correct\_baseline() (sdapy.snerun.snobject method)@\spxentry{correct\_baseline()}\spxextra{sdapy.snerun.snobject method}}

\begin{fulllineitems}
\phantomsection\label{\detokenize{generated/sdapy.snerun.snobject.correct_baseline:sdapy.snerun.snobject.correct_baseline}}\pysiglinewithargsret{\sphinxcode{\sphinxupquote{snobject.}}\sphinxbfcode{\sphinxupquote{correct\_baseline}}}{\emph{self}, \emph{baseline}, \emph{key=\textquotesingle{}fcqfid\textquotesingle{}}, \emph{source=\textquotesingle{}ztffp\textquotesingle{}}}{}
Baseline correction for ZTF forced photometry
\begin{quote}\begin{description}
\item[{Parameters}] \leavevmode\begin{description}
\item[{\sphinxstylestrong{baseline}}] \leavevmode{[}\sphinxtitleref{dictionary}{]}
data returned from \sphinxstylestrong{self.calibrate\_baseline}

\item[{\sphinxstylestrong{key}}] \leavevmode{[}\sphinxtitleref{str}{]}
which column to distinguish photometry between different fields/ccds

\item[{\sphinxstylestrong{source}}] \leavevmode{[}\sphinxtitleref{str}{]}
which source of lc to be corrected

\end{description}

\end{description}\end{quote}

\end{fulllineitems}



\subsubsection{sdapy.snerun.snobject.est\_hostebv\_with\_c10}
\label{\detokenize{generated/sdapy.snerun.snobject.est_hostebv_with_c10:sdapy-snerun-snobject-est-hostebv-with-c10}}\label{\detokenize{generated/sdapy.snerun.snobject.est_hostebv_with_c10::doc}}\index{est\_hostebv\_with\_c10() (sdapy.snerun.snobject method)@\spxentry{est\_hostebv\_with\_c10()}\spxextra{sdapy.snerun.snobject method}}

\begin{fulllineitems}
\phantomsection\label{\detokenize{generated/sdapy.snerun.snobject.est_hostebv_with_c10:sdapy.snerun.snobject.est_hostebv_with_c10}}\pysiglinewithargsret{\sphinxcode{\sphinxupquote{snobject.}}\sphinxbfcode{\sphinxupquote{est\_hostebv\_with\_c10}}}{\emph{self}, \emph{interpolation=\textquotesingle{}gp\textquotesingle{}}, \emph{**kwargs}}{}
estimate host ebv with color comparison approach
\begin{quote}\begin{description}
\item[{Parameters}] \leavevmode\begin{description}
\item[{\sphinxstylestrong{hostebv\_bands}}] \leavevmode{[}\sphinxtitleref{list}{]}
two bands of the colour

\item[{\sphinxstylestrong{interpolation}}] \leavevmode{[}\sphinxtitleref{str}{]}
which interpolation to be used to estimate fluxes at specific phase

\end{description}

\end{description}\end{quote}

\end{fulllineitems}



\subsubsection{sdapy.snerun.snobject.flux\_to\_mag}
\label{\detokenize{generated/sdapy.snerun.snobject.flux_to_mag:sdapy-snerun-snobject-flux-to-mag}}\label{\detokenize{generated/sdapy.snerun.snobject.flux_to_mag::doc}}\index{flux\_to\_mag() (sdapy.snerun.snobject static method)@\spxentry{flux\_to\_mag()}\spxextra{sdapy.snerun.snobject static method}}

\begin{fulllineitems}
\phantomsection\label{\detokenize{generated/sdapy.snerun.snobject.flux_to_mag:sdapy.snerun.snobject.flux_to_mag}}\pysiglinewithargsret{\sphinxbfcode{\sphinxupquote{static }}\sphinxcode{\sphinxupquote{snobject.}}\sphinxbfcode{\sphinxupquote{flux\_to\_mag}}}{\emph{flux}, \emph{dflux=None}, \emph{sigma=5.0}, \emph{units=\textquotesingle{}zp\textquotesingle{}}, \emph{zp=23.9}, \emph{wavelength=None}}{}
Converts fluxes (erg/s/cm2/A) into AB magnitudes with flux/mag zerop

\end{fulllineitems}



\subsubsection{sdapy.snerun.snobject.get\_alert\_ztf}
\label{\detokenize{generated/sdapy.snerun.snobject.get_alert_ztf:sdapy-snerun-snobject-get-alert-ztf}}\label{\detokenize{generated/sdapy.snerun.snobject.get_alert_ztf::doc}}\index{get\_alert\_ztf() (sdapy.snerun.snobject method)@\spxentry{get\_alert\_ztf()}\spxextra{sdapy.snerun.snobject method}}

\begin{fulllineitems}
\phantomsection\label{\detokenize{generated/sdapy.snerun.snobject.get_alert_ztf:sdapy.snerun.snobject.get_alert_ztf}}\pysiglinewithargsret{\sphinxcode{\sphinxupquote{snobject.}}\sphinxbfcode{\sphinxupquote{get\_alert\_ztf}}}{\emph{self}, \emph{source=\textquotesingle{}marshal\textquotesingle{}}, \emph{**kwargs}}{}
parse local ZTF alert photometic data to \sphinxstyleemphasis{self.lc}
\begin{quote}\begin{description}
\item[{Parameters}] \leavevmode\begin{description}
\item[{\sphinxstylestrong{source}}] \leavevmode{[}\sphinxtitleref{str}{]}
which one to parse: from Growth marshal, or fritz

\end{description}

\end{description}\end{quote}

\end{fulllineitems}



\subsubsection{sdapy.snerun.snobject.get\_external\_phot}
\label{\detokenize{generated/sdapy.snerun.snobject.get_external_phot:sdapy-snerun-snobject-get-external-phot}}\label{\detokenize{generated/sdapy.snerun.snobject.get_external_phot::doc}}\index{get\_external\_phot() (sdapy.snerun.snobject method)@\spxentry{get\_external\_phot()}\spxextra{sdapy.snerun.snobject method}}

\begin{fulllineitems}
\phantomsection\label{\detokenize{generated/sdapy.snerun.snobject.get_external_phot:sdapy.snerun.snobject.get_external_phot}}\pysiglinewithargsret{\sphinxcode{\sphinxupquote{snobject.}}\sphinxbfcode{\sphinxupquote{get\_external\_phot}}}{\emph{self}, \emph{filename}, \emph{source}, \emph{**kwargs}}{}
parse user defined photometric data to \sphinxstyleemphasis{self.lc}
\begin{quote}\begin{description}
\item[{Parameters}] \leavevmode\begin{description}
\item[{\sphinxstylestrong{filename}}] \leavevmode{[}\sphinxtitleref{str}{]}
path of the photometric file

\item[{\sphinxstylestrong{source}}] \leavevmode{[}\sphinxtitleref{float}{]}
light curve source ID

\end{description}

\end{description}\end{quote}

\end{fulllineitems}



\subsubsection{sdapy.snerun.snobject.get\_external\_spectra}
\label{\detokenize{generated/sdapy.snerun.snobject.get_external_spectra:sdapy-snerun-snobject-get-external-spectra}}\label{\detokenize{generated/sdapy.snerun.snobject.get_external_spectra::doc}}\index{get\_external\_spectra() (sdapy.snerun.snobject method)@\spxentry{get\_external\_spectra()}\spxextra{sdapy.snerun.snobject method}}

\begin{fulllineitems}
\phantomsection\label{\detokenize{generated/sdapy.snerun.snobject.get_external_spectra:sdapy.snerun.snobject.get_external_spectra}}\pysiglinewithargsret{\sphinxcode{\sphinxupquote{snobject.}}\sphinxbfcode{\sphinxupquote{get\_external\_spectra}}}{\emph{self}, \emph{filename}, \emph{epoch}, \emph{tel=\textquotesingle{}\textquotesingle{}}, \emph{**kwargs}}{}
parse \sphinxstylestrong{spectra} via a local file
\begin{quote}\begin{description}
\item[{Parameters}] \leavevmode\begin{description}
\item[{\sphinxstylestrong{filename}}] \leavevmode{[}\sphinxtitleref{str}{]}
filename.
all lines starting with \sphinxstylestrong{\#} will be skipped,
the first line should be: wave flux,
for the dataframe keywords, and in the following,
two columns are needed in each line, presetnting the wavelength and flu,
seperating with space.

\item[{\sphinxstylestrong{epoch}}] \leavevmode{[}\sphinxtitleref{str}{]}
astropy.time, that will be used to calculate the phase

\item[{\sphinxstylestrong{tel}}] \leavevmode{[}\sphinxtitleref{str}{]}
telescope/instrument

\end{description}

\end{description}\end{quote}

\end{fulllineitems}



\subsubsection{sdapy.snerun.snobject.get\_fp\_atlas}
\label{\detokenize{generated/sdapy.snerun.snobject.get_fp_atlas:sdapy-snerun-snobject-get-fp-atlas}}\label{\detokenize{generated/sdapy.snerun.snobject.get_fp_atlas::doc}}\index{get\_fp\_atlas() (sdapy.snerun.snobject method)@\spxentry{get\_fp\_atlas()}\spxextra{sdapy.snerun.snobject method}}

\begin{fulllineitems}
\phantomsection\label{\detokenize{generated/sdapy.snerun.snobject.get_fp_atlas:sdapy.snerun.snobject.get_fp_atlas}}\pysiglinewithargsret{\sphinxcode{\sphinxupquote{snobject.}}\sphinxbfcode{\sphinxupquote{get\_fp\_atlas}}}{\emph{self}, \emph{binDays=None}, \emph{clobber=False}, \emph{**kwargs}}{}
parse local ATLAS forced (binned or not binned) photometric data to \sphinxstyleemphasis{self.lc}
\begin{quote}\begin{description}
\item[{Parameters}] \leavevmode\begin{description}
\item[{\sphinxstylestrong{binDays}}] \leavevmode{[}\sphinxtitleref{int}{]}
days bin

\item[{\sphinxstylestrong{clobber}}] \leavevmode{[}\sphinxtitleref{bool}{]}
if cached file exists, redo or read the cache

\end{description}

\end{description}\end{quote}

\end{fulllineitems}



\subsubsection{sdapy.snerun.snobject.get\_fp\_ztf}
\label{\detokenize{generated/sdapy.snerun.snobject.get_fp_ztf:sdapy-snerun-snobject-get-fp-ztf}}\label{\detokenize{generated/sdapy.snerun.snobject.get_fp_ztf::doc}}\index{get\_fp\_ztf() (sdapy.snerun.snobject method)@\spxentry{get\_fp\_ztf()}\spxextra{sdapy.snerun.snobject method}}

\begin{fulllineitems}
\phantomsection\label{\detokenize{generated/sdapy.snerun.snobject.get_fp_ztf:sdapy.snerun.snobject.get_fp_ztf}}\pysiglinewithargsret{\sphinxcode{\sphinxupquote{snobject.}}\sphinxbfcode{\sphinxupquote{get\_fp\_ztf}}}{\emph{self}, \emph{seeing\_cut=7.0}, \emph{**kwargs}}{}
parse local ZTF forced photometic data to \sphinxstyleemphasis{self.lc}
\begin{quote}\begin{description}
\item[{Parameters}] \leavevmode\begin{description}
\item[{\sphinxstylestrong{seeing\_cut}}] \leavevmode{[}\sphinxtitleref{float}{]}
remove epochs with quite poor seeings

\item[{\sphinxstylestrong{snrt}}] \leavevmode{[}\sphinxtitleref{float}{]}
SNR threshold to distinguish detection/limit

\end{description}

\end{description}\end{quote}

\end{fulllineitems}



\subsubsection{sdapy.snerun.snobject.get\_local\_spectra}
\label{\detokenize{generated/sdapy.snerun.snobject.get_local_spectra:sdapy-snerun-snobject-get-local-spectra}}\label{\detokenize{generated/sdapy.snerun.snobject.get_local_spectra::doc}}\index{get\_local\_spectra() (sdapy.snerun.snobject method)@\spxentry{get\_local\_spectra()}\spxextra{sdapy.snerun.snobject method}}

\begin{fulllineitems}
\phantomsection\label{\detokenize{generated/sdapy.snerun.snobject.get_local_spectra:sdapy.snerun.snobject.get_local_spectra}}\pysiglinewithargsret{\sphinxcode{\sphinxupquote{snobject.}}\sphinxbfcode{\sphinxupquote{get\_local\_spectra}}}{\emph{self}, \emph{source=None}, \emph{**kwargs}}{}
get local ZTF fritz/marshal spectra        
\sphinxstyleemphasis{Note}: this is only available for ZTF internal collaborators.
\begin{quote}\begin{description}
\item[{Parameters}] \leavevmode\begin{description}
\item[{\sphinxstylestrong{source}}] \leavevmode{[}\sphinxtitleref{str}{]}
\sphinxstylestrong{fritz}, or \sphinxstylestrong{marshal}, or None for both

\item[{\sphinxstylestrong{verbose}}] \leavevmode{[}\sphinxtitleref{bool}{]}
show detailed running informations

\end{description}

\end{description}\end{quote}

\end{fulllineitems}



\subsubsection{sdapy.snerun.snobject.get\_oac}
\label{\detokenize{generated/sdapy.snerun.snobject.get_oac:sdapy-snerun-snobject-get-oac}}\label{\detokenize{generated/sdapy.snerun.snobject.get_oac::doc}}\index{get\_oac() (sdapy.snerun.snobject method)@\spxentry{get\_oac()}\spxextra{sdapy.snerun.snobject method}}

\begin{fulllineitems}
\phantomsection\label{\detokenize{generated/sdapy.snerun.snobject.get_oac:sdapy.snerun.snobject.get_oac}}\pysiglinewithargsret{\sphinxcode{\sphinxupquote{snobject.}}\sphinxbfcode{\sphinxupquote{get\_oac}}}{\emph{self}, \emph{which=\textquotesingle{}photometry\textquotesingle{}}, \emph{**kwargs}}{}
parse local Open Astronomical Catalog data to \sphinxstyleemphasis{self.lc}
\begin{quote}\begin{description}
\item[{Parameters}] \leavevmode\begin{description}
\item[{\sphinxstylestrong{which}}] \leavevmode{[}\sphinxtitleref{str}{]}
which to parse: photometry or spectra

\end{description}

\end{description}\end{quote}

\end{fulllineitems}



\subsubsection{sdapy.snerun.snobject.lyman\_bol}
\label{\detokenize{generated/sdapy.snerun.snobject.lyman_bol:sdapy-snerun-snobject-lyman-bol}}\label{\detokenize{generated/sdapy.snerun.snobject.lyman_bol::doc}}\index{lyman\_bol() (sdapy.snerun.snobject method)@\spxentry{lyman\_bol()}\spxextra{sdapy.snerun.snobject method}}

\begin{fulllineitems}
\phantomsection\label{\detokenize{generated/sdapy.snerun.snobject.lyman_bol:sdapy.snerun.snobject.lyman_bol}}\pysiglinewithargsret{\sphinxcode{\sphinxupquote{snobject.}}\sphinxbfcode{\sphinxupquote{lyman\_bol}}}{\emph{self, interpolation={[}\textquotesingle{}bin\textquotesingle{}, \textquotesingle{}fit\textquotesingle{}, \textquotesingle{}gp\textquotesingle{}{]}, **kwargs}}{}
calculate bolometric LC from colours, with Lyman bolometric correction
\begin{quote}\begin{description}
\item[{Parameters}] \leavevmode\begin{description}
\item[{\sphinxstylestrong{color\_bands}}] \leavevmode{[}\sphinxtitleref{list}{]}
two bands of the colour

\item[{\sphinxstylestrong{interpolation}}] \leavevmode{[}\sphinxtitleref{str}{]}
estimate flux with data epoch less than than \sphinxstylestrong{tdbin}, or interpolation from GP/fits

\item[{\sphinxstylestrong{verbose}}] \leavevmode{[}\sphinxtitleref{bool}{]}
show process

\end{description}

\end{description}\end{quote}


\sphinxstrong{See also:}

\begin{description}
\item[{{\hyperref[\detokenize{generated/sdapy.snerun.snobject.calc_colors:sdapy.snerun.snobject.calc_colors}]{\sphinxcrossref{\sphinxcode{\sphinxupquote{snobject.calc\_colors}}}}}}] \leavevmode
\end{description}



\end{fulllineitems}



\subsubsection{sdapy.snerun.snobject.mag\_to\_flux}
\label{\detokenize{generated/sdapy.snerun.snobject.mag_to_flux:sdapy-snerun-snobject-mag-to-flux}}\label{\detokenize{generated/sdapy.snerun.snobject.mag_to_flux::doc}}\index{mag\_to\_flux() (sdapy.snerun.snobject static method)@\spxentry{mag\_to\_flux()}\spxextra{sdapy.snerun.snobject static method}}

\begin{fulllineitems}
\phantomsection\label{\detokenize{generated/sdapy.snerun.snobject.mag_to_flux:sdapy.snerun.snobject.mag_to_flux}}\pysiglinewithargsret{\sphinxbfcode{\sphinxupquote{static }}\sphinxcode{\sphinxupquote{snobject.}}\sphinxbfcode{\sphinxupquote{mag\_to\_flux}}}{\emph{mag}, \emph{magerr=None}, \emph{limmag=None}, \emph{sigma=5.0}, \emph{units=\textquotesingle{}zp\textquotesingle{}}, \emph{zp=23.9}, \emph{wavelength=None}}{}
with flux/mag zerop

\end{fulllineitems}



\subsubsection{sdapy.snerun.snobject.match\_colors}
\label{\detokenize{generated/sdapy.snerun.snobject.match_colors:sdapy-snerun-snobject-match-colors}}\label{\detokenize{generated/sdapy.snerun.snobject.match_colors::doc}}\index{match\_colors() (sdapy.snerun.snobject method)@\spxentry{match\_colors()}\spxextra{sdapy.snerun.snobject method}}

\begin{fulllineitems}
\phantomsection\label{\detokenize{generated/sdapy.snerun.snobject.match_colors:sdapy.snerun.snobject.match_colors}}\pysiglinewithargsret{\sphinxcode{\sphinxupquote{snobject.}}\sphinxbfcode{\sphinxupquote{match\_colors}}}{\emph{self}, \emph{**kwargs}}{}
match colors
\begin{quote}\begin{description}
\item[{Parameters}] \leavevmode\begin{description}
\item[{\sphinxstylestrong{tdbin}}] \leavevmode{[}\sphinxtitleref{float}{]}
threshold for binning

\end{description}

\end{description}\end{quote}


\sphinxstrong{See also:}

\begin{description}
\item[{{\hyperref[\detokenize{generated/sdapy.snerun.snobject.combine_multi_obs:sdapy.snerun.snobject.combine_multi_obs}]{\sphinxcrossref{\sphinxcode{\sphinxupquote{snobject.combine\_multi\_obs}}}}}}] \leavevmode
\end{description}



\end{fulllineitems}



\subsubsection{sdapy.snerun.snobject.merge\_df\_cols}
\label{\detokenize{generated/sdapy.snerun.snobject.merge_df_cols:sdapy-snerun-snobject-merge-df-cols}}\label{\detokenize{generated/sdapy.snerun.snobject.merge_df_cols::doc}}\index{merge\_df\_cols() (sdapy.snerun.snobject static method)@\spxentry{merge\_df\_cols()}\spxextra{sdapy.snerun.snobject static method}}

\begin{fulllineitems}
\phantomsection\label{\detokenize{generated/sdapy.snerun.snobject.merge_df_cols:sdapy.snerun.snobject.merge_df_cols}}\pysiglinewithargsret{\sphinxbfcode{\sphinxupquote{static }}\sphinxcode{\sphinxupquote{snobject.}}\sphinxbfcode{\sphinxupquote{merge\_df\_cols}}}{\emph{\_snobject\_\_df}}{}
will bug when combining columns from differents sources, i.e. ZTF FP r and marshal r

\end{fulllineitems}



\subsubsection{sdapy.snerun.snobject.mjd\_now}
\label{\detokenize{generated/sdapy.snerun.snobject.mjd_now:sdapy-snerun-snobject-mjd-now}}\label{\detokenize{generated/sdapy.snerun.snobject.mjd_now::doc}}\index{mjd\_now() (sdapy.snerun.snobject method)@\spxentry{mjd\_now()}\spxextra{sdapy.snerun.snobject method}}

\begin{fulllineitems}
\phantomsection\label{\detokenize{generated/sdapy.snerun.snobject.mjd_now:sdapy.snerun.snobject.mjd_now}}\pysiglinewithargsret{\sphinxcode{\sphinxupquote{snobject.}}\sphinxbfcode{\sphinxupquote{mjd\_now}}}{\emph{self}, \emph{jd=False}}{}
get current juliand date via astropy.time
\begin{quote}\begin{description}
\item[{Parameters}] \leavevmode\begin{description}
\item[{\sphinxstylestrong{jd}}] \leavevmode{[}\sphinxtitleref{bool}{]}
get Julian date or midified Julian date

\end{description}

\end{description}\end{quote}

\end{fulllineitems}



\subsubsection{sdapy.snerun.snobject.oac\_phot\_url}
\label{\detokenize{generated/sdapy.snerun.snobject.oac_phot_url:sdapy-snerun-snobject-oac-phot-url}}\label{\detokenize{generated/sdapy.snerun.snobject.oac_phot_url::doc}}\index{oac\_phot\_url() (sdapy.snerun.snobject method)@\spxentry{oac\_phot\_url()}\spxextra{sdapy.snerun.snobject method}}

\begin{fulllineitems}
\phantomsection\label{\detokenize{generated/sdapy.snerun.snobject.oac_phot_url:sdapy.snerun.snobject.oac_phot_url}}\pysiglinewithargsret{\sphinxcode{\sphinxupquote{snobject.}}\sphinxbfcode{\sphinxupquote{oac\_phot\_url}}}{\emph{self}, \emph{url}}{}
(static) make url to query OAC
\begin{quote}\begin{description}
\item[{Parameters}] \leavevmode\begin{description}
\item[{\sphinxstylestrong{url}}] \leavevmode{[}\sphinxtitleref{str}{]}
basic url

\end{description}

\end{description}\end{quote}

\end{fulllineitems}



\subsubsection{sdapy.snerun.snobject.parse\_coo}
\label{\detokenize{generated/sdapy.snerun.snobject.parse_coo:sdapy-snerun-snobject-parse-coo}}\label{\detokenize{generated/sdapy.snerun.snobject.parse_coo::doc}}\index{parse\_coo() (sdapy.snerun.snobject method)@\spxentry{parse\_coo()}\spxextra{sdapy.snerun.snobject method}}

\begin{fulllineitems}
\phantomsection\label{\detokenize{generated/sdapy.snerun.snobject.parse_coo:sdapy.snerun.snobject.parse_coo}}\pysiglinewithargsret{\sphinxcode{\sphinxupquote{snobject.}}\sphinxbfcode{\sphinxupquote{parse\_coo}}}{\emph{self}, \emph{verbose=False}, \emph{deg=True}, \emph{hpx=False}, \emph{nside=128}}{}
handle SN coordinate from self.ra and self.dec
\begin{quote}\begin{description}
\item[{Parameters}] \leavevmode\begin{description}
\item[{\sphinxstylestrong{verbose}}] \leavevmode{[}\sphinxtitleref{bool}{]}
show detailed running informations

\item[{\sphinxstylestrong{def}}] \leavevmode{[}\sphinxtitleref{bool}{]}
return degrees, or hms, dms

\end{description}

\end{description}\end{quote}

\end{fulllineitems}



\subsubsection{sdapy.snerun.snobject.query\_alert\_ztf}
\label{\detokenize{generated/sdapy.snerun.snobject.query_alert_ztf:sdapy-snerun-snobject-query-alert-ztf}}\label{\detokenize{generated/sdapy.snerun.snobject.query_alert_ztf::doc}}\index{query\_alert\_ztf() (sdapy.snerun.snobject method)@\spxentry{query\_alert\_ztf()}\spxextra{sdapy.snerun.snobject method}}

\begin{fulllineitems}
\phantomsection\label{\detokenize{generated/sdapy.snerun.snobject.query_alert_ztf:sdapy.snerun.snobject.query_alert_ztf}}\pysiglinewithargsret{\sphinxcode{\sphinxupquote{snobject.}}\sphinxbfcode{\sphinxupquote{query\_alert\_ztf}}}{\emph{self}, \emph{source=None}, \emph{**kwargs}}{}
qeury ZTF laert photometry via ztfquery,
see \sphinxurl{https://github.com/MickaelRigault/ztfquery}
\sphinxstyleemphasis{Note}: this is only available for ZTF internal collaborators.
\begin{quote}\begin{description}
\item[{Parameters}] \leavevmode\begin{description}
\item[{\sphinxstylestrong{source}}] \leavevmode{[}\sphinxtitleref{str}{]}
\sphinxstylestrong{fritz}, or \sphinxstylestrong{marshal}, or None for both

\item[{\sphinxstylestrong{verbose}}] \leavevmode{[}\sphinxtitleref{bool}{]}
show detailed running informations

\end{description}

\end{description}\end{quote}

\end{fulllineitems}



\subsubsection{sdapy.snerun.snobject.query\_fp\_atlas}
\label{\detokenize{generated/sdapy.snerun.snobject.query_fp_atlas:sdapy-snerun-snobject-query-fp-atlas}}\label{\detokenize{generated/sdapy.snerun.snobject.query_fp_atlas::doc}}\index{query\_fp\_atlas() (sdapy.snerun.snobject method)@\spxentry{query\_fp\_atlas()}\spxextra{sdapy.snerun.snobject method}}

\begin{fulllineitems}
\phantomsection\label{\detokenize{generated/sdapy.snerun.snobject.query_fp_atlas:sdapy.snerun.snobject.query_fp_atlas}}\pysiglinewithargsret{\sphinxcode{\sphinxupquote{snobject.}}\sphinxbfcode{\sphinxupquote{query\_fp\_atlas}}}{\emph{self}, \emph{**kwargs}}{}
qeury ATLAS forced photometry, 
see \sphinxurl{https://fallingstar-data.com/forcedphot/static/apiexample.py}
\begin{quote}\begin{description}
\item[{Parameters}] \leavevmode\begin{description}
\item[{\sphinxstylestrong{verbose}}] \leavevmode{[}\sphinxtitleref{bool}{]}
show detailed running informations

\item[{\sphinxstylestrong{clobber}}] \leavevmode{[}\sphinxtitleref{bool}{]}
if ATLAS forced phot file exists, re\sphinxhyphen{}downloadit or not

\item[{\sphinxstylestrong{mjdstart}}] \leavevmode{[}\sphinxtitleref{float}{]}
start julian date to query

\item[{\sphinxstylestrong{dstart}}] \leavevmode{[}\sphinxtitleref{float}{]}
if \sphinxstylestrong{mjdstart} is None, how many days prior than \sphinxstylestrong{t0} to query

\item[{\sphinxstylestrong{mjdend}}] \leavevmode{[}\sphinxtitleref{float}{]}
end julian date to query

\item[{\sphinxstylestrong{dend}}] \leavevmode{[}\sphinxtitleref{float}{]}
if \sphinxstylestrong{mjdend} is None, how many days later than \sphinxstylestrong{t0} to query

\end{description}

\end{description}\end{quote}

\end{fulllineitems}



\subsubsection{sdapy.snerun.snobject.query\_fp\_ztf}
\label{\detokenize{generated/sdapy.snerun.snobject.query_fp_ztf:sdapy-snerun-snobject-query-fp-ztf}}\label{\detokenize{generated/sdapy.snerun.snobject.query_fp_ztf::doc}}\index{query\_fp\_ztf() (sdapy.snerun.snobject method)@\spxentry{query\_fp\_ztf()}\spxextra{sdapy.snerun.snobject method}}

\begin{fulllineitems}
\phantomsection\label{\detokenize{generated/sdapy.snerun.snobject.query_fp_ztf:sdapy.snerun.snobject.query_fp_ztf}}\pysiglinewithargsret{\sphinxcode{\sphinxupquote{snobject.}}\sphinxbfcode{\sphinxupquote{query\_fp\_ztf}}}{\emph{self}, \emph{get\_email=True}, \emph{**kwargs}}{}
qeury ZTF forced photometry, 
see documentation: \sphinxurl{https://irsa.ipac.caltech.edu/data/ZTF/docs/forcedphot.pdf}.
web query: \sphinxurl{https://ztfweb.ipac.caltech.edu/cgi-bin/requestForcedPhotometry.cgi}
\begin{quote}\begin{description}
\item[{Parameters}] \leavevmode\begin{description}
\item[{\sphinxstylestrong{get\_email}}] \leavevmode{[}\sphinxtitleref{bool}{]}
ZTF force phot query service is time consuming. 
One can set \sphinxstylestrong{get\_email} as True, to receive an email including data file
instead of query via API that would stuck the GUI for a while.

\item[{\sphinxstylestrong{verbose}}] \leavevmode{[}\sphinxtitleref{bool}{]}
show detailed running informations

\item[{\sphinxstylestrong{clobber}}] \leavevmode{[}\sphinxtitleref{bool}{]}
if ATLAS forced phot file exists, re\sphinxhyphen{}downloadit or not

\item[{\sphinxstylestrong{mjdstart}}] \leavevmode{[}\sphinxtitleref{float}{]}
start julian date to query

\item[{\sphinxstylestrong{dstart}}] \leavevmode{[}\sphinxtitleref{float}{]}
if \sphinxstylestrong{mjdstart} is None, how many days prior than \sphinxstylestrong{t0} to query

\item[{\sphinxstylestrong{mjdend}}] \leavevmode{[}\sphinxtitleref{float}{]}
end julian date to query

\item[{\sphinxstylestrong{dend}}] \leavevmode{[}\sphinxtitleref{float}{]}
if \sphinxstylestrong{mjdend} is None, how many days later than \sphinxstylestrong{t0} to query

\end{description}

\end{description}\end{quote}

\end{fulllineitems}



\subsubsection{sdapy.snerun.snobject.query\_oac}
\label{\detokenize{generated/sdapy.snerun.snobject.query_oac:sdapy-snerun-snobject-query-oac}}\label{\detokenize{generated/sdapy.snerun.snobject.query_oac::doc}}\index{query\_oac() (sdapy.snerun.snobject method)@\spxentry{query\_oac()}\spxextra{sdapy.snerun.snobject method}}

\begin{fulllineitems}
\phantomsection\label{\detokenize{generated/sdapy.snerun.snobject.query_oac:sdapy.snerun.snobject.query_oac}}\pysiglinewithargsret{\sphinxcode{\sphinxupquote{snobject.}}\sphinxbfcode{\sphinxupquote{query\_oac}}}{\emph{self}, \emph{db=\textquotesingle{}AL\textquotesingle{}}, \emph{which=\textquotesingle{}photometry\textquotesingle{}}, \emph{**kwargs}}{}
query Open Astronomical Catalog data, 
via OACAPI, \sphinxurl{https://github.com/astrocatalogs/OACAPI}
\begin{quote}\begin{description}
\item[{Parameters}] \leavevmode\begin{description}
\item[{\sphinxstylestrong{db}}] \leavevmode{[}\sphinxtitleref{str}{]}
which catalog to query: 
\sphinxstylestrong{AL}  : ‘\sphinxurl{https://api.astrocats.space/}’,
\sphinxstylestrong{SN}  : ‘\sphinxurl{https://api.sne.space/}’,
\sphinxstylestrong{TDE} : ‘\sphinxurl{https://api.tde.space/}’,
\sphinxstylestrong{KN}  : ‘\sphinxurl{https://api.kilonova.space/}’,
\sphinxstylestrong{FT}  : ‘\sphinxurl{https://api.faststars.space/}’

\item[{\sphinxstylestrong{which}}] \leavevmode{[}\sphinxtitleref{str}{]}
which to query: photometry or spectra

\item[{\sphinxstylestrong{clobber}}] \leavevmode{[}\sphinxtitleref{bool}{]}
if file exists, redo or read it

\item[{\sphinxstylestrong{verbose:   \textasciigrave{}bool\textasciigrave{}}}] \leavevmode
show detailed running informations

\end{description}

\end{description}\end{quote}

\end{fulllineitems}



\subsubsection{sdapy.snerun.snobject.query\_spectra}
\label{\detokenize{generated/sdapy.snerun.snobject.query_spectra:sdapy-snerun-snobject-query-spectra}}\label{\detokenize{generated/sdapy.snerun.snobject.query_spectra::doc}}\index{query\_spectra() (sdapy.snerun.snobject method)@\spxentry{query\_spectra()}\spxextra{sdapy.snerun.snobject method}}

\begin{fulllineitems}
\phantomsection\label{\detokenize{generated/sdapy.snerun.snobject.query_spectra:sdapy.snerun.snobject.query_spectra}}\pysiglinewithargsret{\sphinxcode{\sphinxupquote{snobject.}}\sphinxbfcode{\sphinxupquote{query\_spectra}}}{\emph{self}, \emph{source=None}, \emph{**kwargs}}{}
qeury ZTF spectra via ztfquery,
see \sphinxurl{https://github.com/MickaelRigault/ztfquery}
\sphinxstyleemphasis{Note}: this is only available for ZTF internal collaborators.
\begin{quote}\begin{description}
\item[{Parameters}] \leavevmode\begin{description}
\item[{\sphinxstylestrong{source}}] \leavevmode{[}\sphinxtitleref{str}{]}
\sphinxstylestrong{fritz}, or \sphinxstylestrong{marshal}, or None for both

\item[{\sphinxstylestrong{verbose}}] \leavevmode{[}\sphinxtitleref{bool}{]}
show detailed running informations

\end{description}

\end{description}\end{quote}

\end{fulllineitems}



\subsubsection{sdapy.snerun.snobject.rapid}
\label{\detokenize{generated/sdapy.snerun.snobject.rapid:sdapy-snerun-snobject-rapid}}\label{\detokenize{generated/sdapy.snerun.snobject.rapid::doc}}\index{rapid() (sdapy.snerun.snobject method)@\spxentry{rapid()}\spxextra{sdapy.snerun.snobject method}}

\begin{fulllineitems}
\phantomsection\label{\detokenize{generated/sdapy.snerun.snobject.rapid:sdapy.snerun.snobject.rapid}}\pysiglinewithargsret{\sphinxcode{\sphinxupquote{snobject.}}\sphinxbfcode{\sphinxupquote{rapid}}}{\emph{self}, \emph{ax=None}, \emph{**kwargs}}{}
run astrorapid codes (a Deep learning classifier to distiinguish LCs between different SNe type) on \sphinxstylestrong{self.lc}  
\sphinxstyleemphasis{Note}: this is only available when astrorapid packagte is correcly installed
\begin{quote}\begin{description}
\item[{Parameters}] \leavevmode\begin{description}
\item[{\sphinxstylestrong{ax}}] \leavevmode{[}\sphinxtitleref{matplotlib.axes}{]}
matplotlib subplot, is None will not show

\item[{\sphinxstylestrong{clobber}}] \leavevmode{[}\sphinxtitleref{bool}{]}
if astrorapid plot exists, re\sphinxhyphen{}do or load it

\item[{\sphinxstylestrong{verbose}}] \leavevmode{[}\sphinxtitleref{bool}{]}
show detailed running informations

\end{description}

\end{description}\end{quote}

\end{fulllineitems}



\subsubsection{sdapy.snerun.snobject.rapid\_plot}
\label{\detokenize{generated/sdapy.snerun.snobject.rapid_plot:sdapy-snerun-snobject-rapid-plot}}\label{\detokenize{generated/sdapy.snerun.snobject.rapid_plot::doc}}\index{rapid\_plot() (sdapy.snerun.snobject method)@\spxentry{rapid\_plot()}\spxextra{sdapy.snerun.snobject method}}

\begin{fulllineitems}
\phantomsection\label{\detokenize{generated/sdapy.snerun.snobject.rapid_plot:sdapy.snerun.snobject.rapid_plot}}\pysiglinewithargsret{\sphinxcode{\sphinxupquote{snobject.}}\sphinxbfcode{\sphinxupquote{rapid\_plot}}}{\emph{self}, \emph{ax}, \emph{class\_names}, \emph{class\_color}, \emph{figdir=\textquotesingle{}./\textquotesingle{}}}{}
(static) make astrorapid plot
\begin{quote}\begin{description}
\item[{Parameters}] \leavevmode\begin{description}
\item[{\sphinxstylestrong{ax}}] \leavevmode{[}\sphinxtitleref{matplotlib.axes}{]}
matplotlib subplot, is None will not show

\item[{\sphinxstylestrong{class\_names}}] \leavevmode{[}\sphinxtitleref{list}{]}
different transient classification type names

\item[{\sphinxstylestrong{class\_color}}] \leavevmode{[}\sphinxtitleref{matplotlib.axes}{]}
colors of them

\item[{\sphinxstylestrong{figdir}}] \leavevmode{[}\sphinxtitleref{matplotlib.axes}{]}
path to store the figure

\end{description}

\end{description}\end{quote}

\end{fulllineitems}



\subsubsection{sdapy.snerun.snobject.read\_kwargs}
\label{\detokenize{generated/sdapy.snerun.snobject.read_kwargs:sdapy-snerun-snobject-read-kwargs}}\label{\detokenize{generated/sdapy.snerun.snobject.read_kwargs::doc}}\index{read\_kwargs() (sdapy.snerun.snobject method)@\spxentry{read\_kwargs()}\spxextra{sdapy.snerun.snobject method}}

\begin{fulllineitems}
\phantomsection\label{\detokenize{generated/sdapy.snerun.snobject.read_kwargs:sdapy.snerun.snobject.read_kwargs}}\pysiglinewithargsret{\sphinxcode{\sphinxupquote{snobject.}}\sphinxbfcode{\sphinxupquote{read\_kwargs}}}{\emph{self}, \emph{**kwargs}}{}
Define a proper way to read and update optional parameters
\begin{quote}\begin{description}
\item[{Parameters}] \leavevmode\begin{description}
\item[{\sphinxstylestrong{kwargs}}] \leavevmode{[}\sphinxtitleref{Keyword Arguments}{]}
optional parameters

\end{description}

\end{description}\end{quote}


\sphinxstrong{See also:}

\begin{description}
\item[{{\hyperref[\detokenize{generated/sdapy.snerun.snobject.__init__:sdapy.snerun.snobject.__init__}]{\sphinxcrossref{\sphinxcode{\sphinxupquote{snobject.\_\_init\_\_}}}}}}] \leavevmode
\end{description}



\end{fulllineitems}



\subsubsection{sdapy.snerun.snobject.run}
\label{\detokenize{generated/sdapy.snerun.snobject.run:sdapy-snerun-snobject-run}}\label{\detokenize{generated/sdapy.snerun.snobject.run::doc}}\index{run() (sdapy.snerun.snobject method)@\spxentry{run()}\spxextra{sdapy.snerun.snobject method}}

\begin{fulllineitems}
\phantomsection\label{\detokenize{generated/sdapy.snerun.snobject.run:sdapy.snerun.snobject.run}}\pysiglinewithargsret{\sphinxcode{\sphinxupquote{snobject.}}\sphinxbfcode{\sphinxupquote{run}}}{\emph{self}, \emph{**kwargs}}{}
actions for one SN.
\begin{quote}\begin{description}
\item[{Parameters}] \leavevmode\begin{description}
\item[{\sphinxstylestrong{kwargs}}] \leavevmode{[}\sphinxtitleref{Keyword Arguments}{]}
optional parameters

\end{description}

\end{description}\end{quote}


\sphinxstrong{See also:}

\begin{description}
\item[{{\hyperref[\detokenize{generated/sdapy.snerun.snelist.run:sdapy.snerun.snelist.run}]{\sphinxcrossref{\sphinxcode{\sphinxupquote{snelist.run}}}}}}] \leavevmode
\end{description}



\end{fulllineitems}



\subsubsection{sdapy.snerun.snobject.run\_fit}
\label{\detokenize{generated/sdapy.snerun.snobject.run_fit:sdapy-snerun-snobject-run-fit}}\label{\detokenize{generated/sdapy.snerun.snobject.run_fit::doc}}\index{run\_fit() (sdapy.snerun.snobject method)@\spxentry{run\_fit()}\spxextra{sdapy.snerun.snobject method}}

\begin{fulllineitems}
\phantomsection\label{\detokenize{generated/sdapy.snerun.snobject.run_fit:sdapy.snerun.snobject.run_fit}}\pysiglinewithargsret{\sphinxcode{\sphinxupquote{snobject.}}\sphinxbfcode{\sphinxupquote{run\_fit}}}{\emph{self}, \emph{enginetype}, \emph{source=None}, \emph{**kwargs}}{}
run model fittings on varies of data from different engines
\begin{quote}\begin{description}
\item[{Parameters}] \leavevmode\begin{description}
\item[{\sphinxstylestrong{enginetype}}] \leavevmode{[}\sphinxtitleref{str}{]}
which engine to be used, e.g. multiband\_main, bol\_main, etc

\item[{\sphinxstylestrong{source}}] \leavevmode{[}\sphinxtitleref{str}{]}
optional, which source to be used

\item[{\sphinxstylestrong{fit\_methods}}] \leavevmode{[}\sphinxtitleref{str}{]}
which fitting model to be used, e.g. bazin, gauss, etc

\item[{\sphinxstylestrong{fit\_redo}}] \leavevmode{[}\sphinxtitleref{bool}{]}
if False by default, when cached file exists, read samples from cached sample file. for mcmc, if given nsteps is larger than that of cached sample, continue fits to nsteps, otherwise return samples. if True, when cached file exists, redo everything for new fitting samples

\item[{\sphinxstylestrong{**for engine x} :**}] \leavevmode
\item[{\sphinxstylestrong{x\_type}}] \leavevmode{[}\sphinxtitleref{str}{]}
what data to fit

\item[{\sphinxstylestrong{x\_xrange}}] \leavevmode{[}\sphinxtitleref{list}{]}
x range to fit the data

\item[{\sphinxstylestrong{x\_xrangep}}] \leavevmode{[}\sphinxtitleref{list}{]}
x range to predict the data

\item[{\sphinxstylestrong{x\_yrange}}] \leavevmode{[}\sphinxtitleref{list}{]}
y range to fit the data

\item[{\sphinxstylestrong{x\_bands}}] \leavevmode{[}\sphinxtitleref{list}{]}
data from which bands for the fit

\item[{\sphinxstylestrong{x\_routine: \textasciigrave{}str\textasciigrave{}}}] \leavevmode
Which technic to be used to realize optimization. Possible choices are: minimize, mcmc, leastsq

\item[{\sphinxstylestrong{**if routine is set to mcmc, then} :**}] \leavevmode
\item[{\sphinxstylestrong{ncores}}] \leavevmode{[}\sphinxtitleref{int}{]}
how many cores to run multi\sphinxhyphen{}processing

\item[{\sphinxstylestrong{nwalkers}}] \leavevmode{[}\sphinxtitleref{int}{]}
number of walkers

\item[{\sphinxstylestrong{nsteps}}] \leavevmode{[}\sphinxtitleref{int}{]}
number of MC steps

\item[{\sphinxstylestrong{nsteps\_burnin}}] \leavevmode{[}\sphinxtitleref{int}{]}
number of MC burn in steps

\item[{\sphinxstylestrong{thin\_by}}] \leavevmode{[}\sphinxtitleref{int}{]}
If you only want to store and yield every thin\_by samples in the chain, set thin\_by to an integer greater than 1. When this is set, iterations * thin\_by proposals will be made.

\item[{\sphinxstylestrong{emcee\_burnin}}] \leavevmode{[}\sphinxtitleref{bool}{]}
if use emcee to burnin chains

\item[{\sphinxstylestrong{use\_emcee\_backend}}] \leavevmode{[}\sphinxtitleref{bool}{]}
if use emcee backend

\item[{\sphinxstylestrong{quantile}}] \leavevmode{[}\sphinxtitleref{list}{]}
use 50 percentile as mean, and 1 sigma (68\%\% \sphinxhyphen{}\textgreater{} 16\%\% \sphinxhyphen{} 84\%\%) as errors

\item[{\sphinxstylestrong{**for minimize routine} :**}] \leavevmode
\item[{\sphinxstylestrong{maxfev}}] \leavevmode{[}\sphinxtitleref{int}{]}
for scipy. The maximum number of calls to the function.

\item[{\sphinxstylestrong{scipysamples}}] \leavevmode{[}\sphinxtitleref{int}{]}
generated sampling numbers for scipy approach

\item[{\sphinxstylestrong{quantile}}] \leavevmode{[}\sphinxtitleref{list}{]}
use 50 percentile as mean, and 1 sigma (68\%\% \sphinxhyphen{}\textgreater{} 16\%\% \sphinxhyphen{} 84\%\%) as errors

\item[{\sphinxstylestrong{**there’re more options for specline engine} :**}] \leavevmode
\item[{\sphinxstylestrong{spec\_snr}}] \leavevmode{[}\sphinxtitleref{int}{]}
if no error from spectra, add noise level of snr

\item[{\sphinxstylestrong{bin\_method}}] \leavevmode{[}\sphinxtitleref{str}{]}
method used to bin spectrum

\item[{\sphinxstylestrong{bin\_size}}] \leavevmode{[}\sphinxtitleref{int}{]}
size used to bin spectrum, in AA

\item[{\sphinxstylestrong{savgol\_order}}] \leavevmode{[}\sphinxtitleref{int}{]}
polynomial order for savgol filter

\item[{\sphinxstylestrong{continuum\_method}}] \leavevmode{[}\sphinxtitleref{str}{]}
The function type for continumm fitting, valid functions are “scalar”, “linear”, “quadratic”, “cubic”, “poly”, and “exponential”

\item[{\sphinxstylestrong{continuum\_degree}}] \leavevmode{[}\sphinxtitleref{int}{]}
degree of polynomial when method=”poly”, for continuum fitting

\item[{\sphinxstylestrong{pfactor}}] \leavevmode{[}\sphinxtitleref{int}{]}
threshold used for peak detection

\item[{\sphinxstylestrong{sn\_line}}] \leavevmode{[}\sphinxtitleref{str}{]}
which line to fit, e.g. ‘H\textasciitilde{}\$lpha\$’, ‘He\textasciitilde{}5876\$AA\$’, ‘O\textasciitilde{}7774\$AA\$’

\item[{\sphinxstylestrong{specfit\_phase}}] \leavevmode{[}\sphinxtitleref{list}{]}
phase range to decide which spectra should be fitted

\item[{\sphinxstylestrong{spec\_shift}}] \leavevmode{[}\sphinxtitleref{float}{]}
define shift as fnorm / \sphinxstyleemphasis{shift}

\item[{\sphinxstylestrong{v\_p}}] \leavevmode{[}\sphinxtitleref{float}{]}
guessed velocity (unit: 1e3 km/s)

\item[{\sphinxstylestrong{v\_bounds}}] \leavevmode{[}\sphinxtitleref{list}{]}
guessed velocity range (unit: 1e3 km/s)

\end{description}

\end{description}\end{quote}

\end{fulllineitems}



\subsubsection{sdapy.snerun.snobject.run\_gp}
\label{\detokenize{generated/sdapy.snerun.snobject.run_gp:sdapy-snerun-snobject-run-gp}}\label{\detokenize{generated/sdapy.snerun.snobject.run_gp::doc}}\index{run\_gp() (sdapy.snerun.snobject method)@\spxentry{run\_gp()}\spxextra{sdapy.snerun.snobject method}}

\begin{fulllineitems}
\phantomsection\label{\detokenize{generated/sdapy.snerun.snobject.run_gp:sdapy.snerun.snobject.run_gp}}\pysiglinewithargsret{\sphinxcode{\sphinxupquote{snobject.}}\sphinxbfcode{\sphinxupquote{run\_gp}}}{\emph{self}, \emph{source=None}, \emph{**kwargs}}{}
run Gaussian Process interpolation via george package on \sphinxstylestrong{self.lc}
\begin{quote}\begin{description}
\item[{Parameters}] \leavevmode\begin{description}
\item[{\sphinxstylestrong{source}}] \leavevmode{[}\sphinxtitleref{str}{]}
which source of \sphinxstylestrong{self.lc} to be used

\item[{\sphinxstylestrong{gp\_bands}}] \leavevmode{[}\sphinxtitleref{list}{]}
which bands of \sphinxstylestrong{self.lc} to be used

\item[{\sphinxstylestrong{gp\_xrange}}] \leavevmode{[}\sphinxtitleref{list}{]}
which bands of \sphinxstylestrong{self.lc} to be used

\item[{\sphinxstylestrong{gp\_xrangep}}] \leavevmode{[}\sphinxtitleref{list}{]}
which bands of \sphinxstylestrong{self.lc} to be used

\item[{\sphinxstylestrong{kernel}}] \leavevmode{[}\sphinxtitleref{list}{]}
which bands of \sphinxstylestrong{self.lc} to be used

\item[{\sphinxstylestrong{gp\_mean}}] \leavevmode{[}\sphinxtitleref{list}{]}
which bands of \sphinxstylestrong{self.lc} to be used

\item[{\sphinxstylestrong{fix\_scale}}] \leavevmode{[}\sphinxtitleref{list}{]}
which bands of \sphinxstylestrong{self.lc} to be used

\item[{\sphinxstylestrong{gp\_routine}}] \leavevmode{[}\sphinxtitleref{list}{]}
which bands of \sphinxstylestrong{self.lc} to be used

\item[{\sphinxstylestrong{nsteps}}] \leavevmode{[}\sphinxtitleref{list}{]}
which bands of \sphinxstylestrong{self.lc} to be used

\item[{\sphinxstylestrong{nsteps\_burnin}}] \leavevmode{[}\sphinxtitleref{list}{]}
which bands of \sphinxstylestrong{self.lc} to be used

\item[{\sphinxstylestrong{nwalkers}}] \leavevmode{[}\sphinxtitleref{list}{]}
which bands of \sphinxstylestrong{self.lc} to be used

\item[{\sphinxstylestrong{gp\_redo}}] \leavevmode{[}\sphinxtitleref{list}{]}
which bands of \sphinxstylestrong{self.lc} to be used

\item[{\sphinxstylestrong{verbose}}] \leavevmode{[}\sphinxtitleref{list}{]}
which bands of \sphinxstylestrong{self.lc} to be used

\item[{\sphinxstylestrong{set\_tpeak\_method}}] \leavevmode{[}\sphinxtitleref{list}{]}
which bands of \sphinxstylestrong{self.lc} to be used

\item[{\sphinxstylestrong{set\_tpeak\_filter}}] \leavevmode{[}\sphinxtitleref{list}{]}
which bands of \sphinxstylestrong{self.lc} to be used

\end{description}

\end{description}\end{quote}

\end{fulllineitems}



\subsubsection{sdapy.snerun.snobject.set\_peak\_bol\_main}
\label{\detokenize{generated/sdapy.snerun.snobject.set_peak_bol_main:sdapy-snerun-snobject-set-peak-bol-main}}\label{\detokenize{generated/sdapy.snerun.snobject.set_peak_bol_main::doc}}\index{set\_peak\_bol\_main() (sdapy.snerun.snobject method)@\spxentry{set\_peak\_bol\_main()}\spxextra{sdapy.snerun.snobject method}}

\begin{fulllineitems}
\phantomsection\label{\detokenize{generated/sdapy.snerun.snobject.set_peak_bol_main:sdapy.snerun.snobject.set_peak_bol_main}}\pysiglinewithargsret{\sphinxcode{\sphinxupquote{snobject.}}\sphinxbfcode{\sphinxupquote{set\_peak\_bol\_main}}}{\emph{self}, \emph{model\_name=None}, \emph{source\_name=None}}{}
set peak time and fluxes with bolometric LC fittings
\begin{quote}\begin{description}
\item[{Parameters}] \leavevmode\begin{description}
\item[{\sphinxstylestrong{model\_name}}] \leavevmode{[}\sphinxtitleref{str}{]}
optional, which bolometric model to be used

\item[{\sphinxstylestrong{source\_name}}] \leavevmode{[}\sphinxtitleref{str}{]}
optional, which source to be used, e.g. mbol or mbolbb

\end{description}

\end{description}\end{quote}

\end{fulllineitems}



\subsubsection{sdapy.snerun.snobject.set\_peak\_gp}
\label{\detokenize{generated/sdapy.snerun.snobject.set_peak_gp:sdapy-snerun-snobject-set-peak-gp}}\label{\detokenize{generated/sdapy.snerun.snobject.set_peak_gp::doc}}\index{set\_peak\_gp() (sdapy.snerun.snobject method)@\spxentry{set\_peak\_gp()}\spxextra{sdapy.snerun.snobject method}}

\begin{fulllineitems}
\phantomsection\label{\detokenize{generated/sdapy.snerun.snobject.set_peak_gp:sdapy.snerun.snobject.set_peak_gp}}\pysiglinewithargsret{\sphinxcode{\sphinxupquote{snobject.}}\sphinxbfcode{\sphinxupquote{set\_peak\_gp}}}{\emph{self}, \emph{filt}}{}
set peak time and fluxes with GP interpolation
\begin{quote}\begin{description}
\item[{Parameters}] \leavevmode\begin{description}
\item[{\sphinxstylestrong{filt}}] \leavevmode{[}\sphinxtitleref{str}{]}
t0 from which filter

\end{description}

\end{description}\end{quote}

\end{fulllineitems}



\subsubsection{sdapy.snerun.snobject.set\_peak\_multiband\_main}
\label{\detokenize{generated/sdapy.snerun.snobject.set_peak_multiband_main:sdapy-snerun-snobject-set-peak-multiband-main}}\label{\detokenize{generated/sdapy.snerun.snobject.set_peak_multiband_main::doc}}\index{set\_peak\_multiband\_main() (sdapy.snerun.snobject method)@\spxentry{set\_peak\_multiband\_main()}\spxextra{sdapy.snerun.snobject method}}

\begin{fulllineitems}
\phantomsection\label{\detokenize{generated/sdapy.snerun.snobject.set_peak_multiband_main:sdapy.snerun.snobject.set_peak_multiband_main}}\pysiglinewithargsret{\sphinxcode{\sphinxupquote{snobject.}}\sphinxbfcode{\sphinxupquote{set\_peak\_multiband\_main}}}{\emph{self}, \emph{filt}, \emph{model\_name=None}}{}
set peak time and fluxes with multiband LC fittings
\begin{quote}\begin{description}
\item[{Parameters}] \leavevmode\begin{description}
\item[{\sphinxstylestrong{filt}}] \leavevmode{[}\sphinxtitleref{str}{]}
t0 from which filter

\item[{\sphinxstylestrong{model\_name}}] \leavevmode{[}\sphinxtitleref{str}{]}
optional, which multiband model to be used

\end{description}

\end{description}\end{quote}

\end{fulllineitems}



\subsubsection{sdapy.snerun.snobject.set\_texp\_bol\_main}
\label{\detokenize{generated/sdapy.snerun.snobject.set_texp_bol_main:sdapy-snerun-snobject-set-texp-bol-main}}\label{\detokenize{generated/sdapy.snerun.snobject.set_texp_bol_main::doc}}\index{set\_texp\_bol\_main() (sdapy.snerun.snobject method)@\spxentry{set\_texp\_bol\_main()}\spxextra{sdapy.snerun.snobject method}}

\begin{fulllineitems}
\phantomsection\label{\detokenize{generated/sdapy.snerun.snobject.set_texp_bol_main:sdapy.snerun.snobject.set_texp_bol_main}}\pysiglinewithargsret{\sphinxcode{\sphinxupquote{snobject.}}\sphinxbfcode{\sphinxupquote{set\_texp\_bol\_main}}}{\emph{self}, \emph{model\_name=None}, \emph{source\_name=None}}{}
set the explosion epoch with bolometric LC fittings
\begin{quote}\begin{description}
\item[{Parameters}] \leavevmode\begin{description}
\item[{\sphinxstylestrong{model\_name}}] \leavevmode{[}\sphinxtitleref{str}{]}
optional, which bolometric model to be used

\item[{\sphinxstylestrong{source\_name}}] \leavevmode{[}\sphinxtitleref{str}{]}
optional, which source to be used, e.g. mbol or mbolbb

\end{description}

\end{description}\end{quote}

\end{fulllineitems}



\subsubsection{sdapy.snerun.snobject.set\_texp\_midway}
\label{\detokenize{generated/sdapy.snerun.snobject.set_texp_midway:sdapy-snerun-snobject-set-texp-midway}}\label{\detokenize{generated/sdapy.snerun.snobject.set_texp_midway::doc}}\index{set\_texp\_midway() (sdapy.snerun.snobject method)@\spxentry{set\_texp\_midway()}\spxextra{sdapy.snerun.snobject method}}

\begin{fulllineitems}
\phantomsection\label{\detokenize{generated/sdapy.snerun.snobject.set_texp_midway:sdapy.snerun.snobject.set_texp_midway}}\pysiglinewithargsret{\sphinxcode{\sphinxupquote{snobject.}}\sphinxbfcode{\sphinxupquote{set\_texp\_midway}}}{\emph{self}}{}
set the explosion epoch with the first detection and the last non\sphinxhyphen{}detection
\begin{quote}\begin{description}
\item[{Parameters}] \leavevmode\begin{description}
\item[{\sphinxstylestrong{N/A}}] \leavevmode
\end{description}

\end{description}\end{quote}

\end{fulllineitems}



\subsubsection{sdapy.snerun.snobject.set\_texp\_pl}
\label{\detokenize{generated/sdapy.snerun.snobject.set_texp_pl:sdapy-snerun-snobject-set-texp-pl}}\label{\detokenize{generated/sdapy.snerun.snobject.set_texp_pl::doc}}\index{set\_texp\_pl() (sdapy.snerun.snobject method)@\spxentry{set\_texp\_pl()}\spxextra{sdapy.snerun.snobject method}}

\begin{fulllineitems}
\phantomsection\label{\detokenize{generated/sdapy.snerun.snobject.set_texp_pl:sdapy.snerun.snobject.set_texp_pl}}\pysiglinewithargsret{\sphinxcode{\sphinxupquote{snobject.}}\sphinxbfcode{\sphinxupquote{set\_texp\_pl}}}{\emph{self}, \emph{filt}, \emph{model\_name=None}}{}
set the explosion epoch with multiband LC fittings
\begin{quote}\begin{description}
\item[{Parameters}] \leavevmode\begin{description}
\item[{\sphinxstylestrong{filt}}] \leavevmode{[}\sphinxtitleref{str}{]}
t0 from which filter

\item[{\sphinxstylestrong{model\_name}}] \leavevmode{[}\sphinxtitleref{str}{]}
optional, which multiband model to be used

\end{description}

\end{description}\end{quote}

\end{fulllineitems}



\subsubsection{sdapy.snerun.snobject.summarize\_plot}
\label{\detokenize{generated/sdapy.snerun.snobject.summarize_plot:sdapy-snerun-snobject-summarize-plot}}\label{\detokenize{generated/sdapy.snerun.snobject.summarize_plot::doc}}\index{summarize\_plot() (sdapy.snerun.snobject method)@\spxentry{summarize\_plot()}\spxextra{sdapy.snerun.snobject method}}

\begin{fulllineitems}
\phantomsection\label{\detokenize{generated/sdapy.snerun.snobject.summarize_plot:sdapy.snerun.snobject.summarize_plot}}\pysiglinewithargsret{\sphinxcode{\sphinxupquote{snobject.}}\sphinxbfcode{\sphinxupquote{summarize\_plot}}}{\emph{self}, \emph{ax=None}, \emph{**kwargs}}{}
show a summarize plot: flux/absolute/bolometric LCs, colour curve and the spectra
\begin{quote}\begin{description}
\item[{Parameters}] \leavevmode\begin{description}
\item[{\sphinxstylestrong{ax}}] \leavevmode{[}\sphinxtitleref{matplotlib.subplot}{]}
subplot to show

\end{description}

\end{description}\end{quote}

\end{fulllineitems}



\subsubsection{sdapy.snerun.snobject.sym\_mag}
\label{\detokenize{generated/sdapy.snerun.snobject.sym_mag:sdapy-snerun-snobject-sym-mag}}\label{\detokenize{generated/sdapy.snerun.snobject.sym_mag::doc}}\index{sym\_mag() (sdapy.snerun.snobject method)@\spxentry{sym\_mag()}\spxextra{sdapy.snerun.snobject method}}

\begin{fulllineitems}
\phantomsection\label{\detokenize{generated/sdapy.snerun.snobject.sym_mag:sdapy.snerun.snobject.sym_mag}}\pysiglinewithargsret{\sphinxcode{\sphinxupquote{snobject.}}\sphinxbfcode{\sphinxupquote{sym\_mag}}}{\emph{self}, \emph{w}, \emph{f}, \emph{filt}}{}
(static) synthetic magnitudes with filter transmission curves
\begin{quote}\begin{description}
\item[{Parameters}] \leavevmode\begin{description}
\item[{\sphinxstylestrong{w}}] \leavevmode{[}\sphinxtitleref{list}{]}
spectral wavelengths

\item[{\sphinxstylestrong{f}}] \leavevmode{[}\sphinxtitleref{list}{]}
spectral fluxes

\item[{\sphinxstylestrong{filt}}] \leavevmode{[}\sphinxtitleref{float}{]}
filter

\end{description}

\end{description}\end{quote}

\end{fulllineitems}



\begin{savenotes}\sphinxattablestart
\centering
\begin{tabulary}{\linewidth}[t]{|T|T|}
\hline

\sphinxstylestrong{bin\_df}
&\\
\hline
\sphinxstylestrong{dm\_error}
&\\
\hline
\sphinxstylestrong{get\_model}
&\\
\hline
\sphinxstylestrong{read\_c10}
&\\
\hline
\sphinxstylestrong{savefig}
&\\
\hline
\sphinxstylestrong{show\_corner}
&\\
\hline
\sphinxstylestrong{showfig}
&\\
\hline
\end{tabulary}
\par
\sphinxattableend\end{savenotes}
\index{\_\_init\_\_() (sdapy.snerun.snobject method)@\spxentry{\_\_init\_\_()}\spxextra{sdapy.snerun.snobject method}}

\begin{fulllineitems}
\phantomsection\label{\detokenize{generated/sdapy.snerun.snobject:sdapy.snerun.snobject.__init__}}\pysiglinewithargsret{\sphinxbfcode{\sphinxupquote{\_\_init\_\_}}}{\emph{self}, \emph{objid}, \emph{aliasid=None}, \emph{z=None}, \emph{ra=None}, \emph{dec=None}, \emph{mkwebv=None}, \emph{hostebv=None}, \emph{sntype=None}, \emph{dm=None}, \emph{jdpeak=None}, \emph{fig=None}, \emph{ax=None}, \emph{ax1=None}, \emph{ax2=None}, \emph{ax3=None}, \emph{ax4=None}, \emph{**kwargs}}{}
initialize \sphinxstyleemphasis{snelist}
\begin{quote}\begin{description}
\item[{Parameters}] \leavevmode\begin{description}
\item[{\sphinxstylestrong{objid}}] \leavevmode{[}\sphinxtitleref{str}{]}
object ID string

\item[{\sphinxstylestrong{aliasid}}] \leavevmode{[}\sphinxtitleref{str}{]}
other name of object

\item[{\sphinxstylestrong{z}}] \leavevmode{[}\sphinxtitleref{float}{]}
redshift

\item[{\sphinxstylestrong{ra}}] \leavevmode{[}\sphinxtitleref{str}{]}
R.A. in hh:mm:ss

\item[{\sphinxstylestrong{dec}}] \leavevmode{[}\sphinxtitleref{str}{]}
Dec. in dd:mm:dd

\item[{\sphinxstylestrong{mkwebv}}] \leavevmode{[}\sphinxtitleref{float}{]}
milky way E(B\sphinxhyphen{}V)

\item[{\sphinxstylestrong{hostebv}}] \leavevmode{[}\sphinxtitleref{float}{]}
host galaxy E(B\sphinxhyphen{}V)

\item[{\sphinxstylestrong{sntype}}] \leavevmode{[}\sphinxtitleref{str}{]}
supernova type

\item[{\sphinxstylestrong{dm}}] \leavevmode{[}\sphinxtitleref{float}{]}
distance module

\item[{\sphinxstylestrong{jdpeak}}] \leavevmode{[}\sphinxtitleref{float}{]}
julian date of peak epoch

\item[{\sphinxstylestrong{ax}}] \leavevmode{[}matplotlib.axes{]}
for flux lightcurve plot

\item[{\sphinxstylestrong{ax1}}] \leavevmode{[}matplotlib.axes{]}
for spectra plot

\item[{\sphinxstylestrong{ax2}}] \leavevmode{[}matplotlib.axes{]}
for magnitude lightcurve plot

\item[{\sphinxstylestrong{ax3}}] \leavevmode{[}matplotlib.axes{]}
for colour plot

\item[{\sphinxstylestrong{ax4}}] \leavevmode{[}matplotlib.axes{]}
for luminosoty plot

\item[{\sphinxstylestrong{kwargs}}] \leavevmode{[}\sphinxtitleref{Keyword Arguments}{]}
see \sphinxurl{https://github.com/saberyoung/HAFFET/blob/master/sdapy/data/default\_par.txt},
\sphinxstylestrong{snobject} part

\end{description}

\end{description}\end{quote}
\subsubsection*{Examples}

\begin{sphinxVerbatim}[commandchars=\\\{\}]
\PYG{g+gp}{\PYGZgt{}\PYGZgt{}\PYGZgt{} }\PYG{k+kn}{from} \PYG{n+nn}{sdapy} \PYG{k+kn}{import} \PYG{n}{snerun}
\PYG{g+gp}{\PYGZgt{}\PYGZgt{}\PYGZgt{} }\PYG{n}{a} \PYG{o}{=} \PYG{n}{snerun}\PYG{o}{.}\PYG{n}{snobject}\PYG{p}{(}\PYG{l+s+s1}{\PYGZsq{}}\PYG{l+s+s1}{sss}\PYG{l+s+s1}{\PYGZsq{}}\PYG{p}{)}
\PYG{g+gp}{\PYGZgt{}\PYGZgt{}\PYGZgt{} }\PYG{n}{a}
\PYG{g+go}{\PYGZlt{}sdapy.snerun.snobject object at 0x7fa4c8284780\PYGZgt{}}
\end{sphinxVerbatim}

\end{fulllineitems}

\subsubsection*{Methods}


\begin{savenotes}\sphinxatlongtablestart\begin{longtable}[c]{\X{1}{2}\X{1}{2}}
\hline

\endfirsthead

\multicolumn{2}{c}%
{\makebox[0pt]{\sphinxtablecontinued{\tablename\ \thetable{} \textendash{} continued from previous page}}}\\
\hline

\endhead

\hline
\multicolumn{2}{r}{\makebox[0pt][r]{\sphinxtablecontinued{Continued on next page}}}\\
\endfoot

\endlastfoot

{\hyperref[\detokenize{generated/sdapy.snerun.snobject.__init__:sdapy.snerun.snobject.__init__}]{\sphinxcrossref{\sphinxcode{\sphinxupquote{\_\_init\_\_}}}}}(self, objid{[}, aliasid, z, ra, dec, …{]})
&
initialize \sphinxstyleemphasis{snelist}
\\
\hline
{\hyperref[\detokenize{generated/sdapy.snerun.snobject.add_flux:sdapy.snerun.snobject.add_flux}]{\sphinxcrossref{\sphinxcode{\sphinxupquote{add\_flux}}}}}(self{[}, zp, source{]})
&
add \sphinxstylestrong{flux} and \sphinxstylestrong{eflux} column, based on \sphinxstylestrong{mag}, \sphinxstylestrong{emag}, and/or \sphinxstylestrong{limmag} column.
\\
\hline
{\hyperref[\detokenize{generated/sdapy.snerun.snobject.add_lc:sdapy.snerun.snobject.add_lc}]{\sphinxcrossref{\sphinxcode{\sphinxupquote{add\_lc}}}}}(self, df{[}, source{]})
&
add lightcurve data into self.lc
\\
\hline
{\hyperref[\detokenize{generated/sdapy.snerun.snobject.add_mag:sdapy.snerun.snobject.add_mag}]{\sphinxcrossref{\sphinxcode{\sphinxupquote{add\_mag}}}}}(self{[}, zp, source{]})
&
add \sphinxstylestrong{mag} and/or \sphinxstyleemphasis{limmag}/\sphinxstylestrong{emag} column, based on \sphinxstylestrong{flux}, \sphinxstylestrong{eflux} column.
\\
\hline
{\hyperref[\detokenize{generated/sdapy.snerun.snobject.bb_bol:sdapy.snerun.snobject.bb_bol}]{\sphinxcrossref{\sphinxcode{\sphinxupquote{bb\_bol}}}}}(self{[}, do\_Kcorr, ab2vega{]})
&
calculate bolometric LC with diluted blackbody
\\
\hline
\sphinxcode{\sphinxupquote{bb\_bol\_show}}(self, ax, phase{[}, filters, sep, …{]})
&
show constructed SED of specific epochs
\\
\hline
{\hyperref[\detokenize{generated/sdapy.snerun.snobject.bb_colors:sdapy.snerun.snobject.bb_colors}]{\sphinxcrossref{\sphinxcode{\sphinxupquote{bb\_colors}}}}}(self, \textbackslash{}*\textbackslash{}*kwargs)
&
match colours for multiple bands, for blackbody (BB) construction
\\
\hline
{\hyperref[\detokenize{generated/sdapy.snerun.snobject.bin_df:sdapy.snerun.snobject.bin_df}]{\sphinxcrossref{\sphinxcode{\sphinxupquote{bin\_df}}}}}(df{[}, deltah, xkey, fkey{]})
&

\\
\hline
{\hyperref[\detokenize{generated/sdapy.snerun.snobject.bin_fp_atlas:sdapy.snerun.snobject.bin_fp_atlas}]{\sphinxcrossref{\sphinxcode{\sphinxupquote{bin\_fp\_atlas}}}}}(self{[}, binDays, resultsPath, …{]})
&
binning for ATLAS forced photometry with Dave’s code, \sphinxurl{https://gist.github.com/thespacedoctor/86777fa5a9567b7939e8d84fd8cf6a76}
\\
\hline
{\hyperref[\detokenize{generated/sdapy.snerun.snobject.calc_colors:sdapy.snerun.snobject.calc_colors}]{\sphinxcrossref{\sphinxcode{\sphinxupquote{calc\_colors}}}}}(self, \textbackslash{}*\textbackslash{}*kwargs)
&
calculate colors for two bands
\\
\hline
{\hyperref[\detokenize{generated/sdapy.snerun.snobject.calibrate_baseline:sdapy.snerun.snobject.calibrate_baseline}]{\sphinxcrossref{\sphinxcode{\sphinxupquote{calibrate\_baseline}}}}}(self{[}, ax, key, source, …{]})
&
Baseline calibration for ZTF forced photometry
\\
\hline
{\hyperref[\detokenize{generated/sdapy.snerun.snobject.clip_lc:sdapy.snerun.snobject.clip_lc}]{\sphinxcrossref{\sphinxcode{\sphinxupquote{clip\_lc}}}}}(self, \textbackslash{}*\textbackslash{}*kwargs)
&
Removes outlier data points using GP interpolation
\\
\hline
{\hyperref[\detokenize{generated/sdapy.snerun.snobject.combine_multi_obs:sdapy.snerun.snobject.combine_multi_obs}]{\sphinxcrossref{\sphinxcode{\sphinxupquote{combine\_multi\_obs}}}}}(self, \textbackslash{}*\textbackslash{}*kwargs)
&
(static) bin and combine observations from common epoch
\\
\hline
{\hyperref[\detokenize{generated/sdapy.snerun.snobject.config_ztfquery:sdapy.snerun.snobject.config_ztfquery}]{\sphinxcrossref{\sphinxcode{\sphinxupquote{config\_ztfquery}}}}}(self)
&
check and set ztfquery accounts
\\
\hline
{\hyperref[\detokenize{generated/sdapy.snerun.snobject.correct_baseline:sdapy.snerun.snobject.correct_baseline}]{\sphinxcrossref{\sphinxcode{\sphinxupquote{correct\_baseline}}}}}(self, baseline{[}, key, source{]})
&
Baseline correction for ZTF forced photometry
\\
\hline
{\hyperref[\detokenize{generated/sdapy.snerun.snobject.dm_error:sdapy.snerun.snobject.dm_error}]{\sphinxcrossref{\sphinxcode{\sphinxupquote{dm\_error}}}}}(self, filt)
&

\\
\hline
{\hyperref[\detokenize{generated/sdapy.snerun.snobject.est_hostebv_with_c10:sdapy.snerun.snobject.est_hostebv_with_c10}]{\sphinxcrossref{\sphinxcode{\sphinxupquote{est\_hostebv\_with\_c10}}}}}(self{[}, interpolation{]})
&
estimate host ebv with color comparison approach
\\
\hline
\sphinxcode{\sphinxupquote{snobject.ext\_lc}}
&

\\
\hline
{\hyperref[\detokenize{generated/sdapy.snerun.snobject.flux_to_mag:sdapy.snerun.snobject.flux_to_mag}]{\sphinxcrossref{\sphinxcode{\sphinxupquote{flux\_to\_mag}}}}}(flux{[}, dflux, sigma, units, zp, …{]})
&
Converts fluxes (erg/s/cm2/A) into AB magnitudes with flux/mag zerop
\\
\hline
{\hyperref[\detokenize{generated/sdapy.snerun.snobject.get_alert_ztf:sdapy.snerun.snobject.get_alert_ztf}]{\sphinxcrossref{\sphinxcode{\sphinxupquote{get\_alert\_ztf}}}}}(self{[}, source{]})
&
parse local ZTF alert photometic data to \sphinxstyleemphasis{self.lc}
\\
\hline
{\hyperref[\detokenize{generated/sdapy.snerun.snobject.get_external_phot:sdapy.snerun.snobject.get_external_phot}]{\sphinxcrossref{\sphinxcode{\sphinxupquote{get\_external\_phot}}}}}(self, filename, source, …)
&
parse user defined photometric data to \sphinxstyleemphasis{self.lc}
\\
\hline
{\hyperref[\detokenize{generated/sdapy.snerun.snobject.get_external_spectra:sdapy.snerun.snobject.get_external_spectra}]{\sphinxcrossref{\sphinxcode{\sphinxupquote{get\_external\_spectra}}}}}(self, filename, epoch)
&
parse \sphinxstylestrong{spectra} via a local file
\\
\hline
{\hyperref[\detokenize{generated/sdapy.snerun.snobject.get_fp_atlas:sdapy.snerun.snobject.get_fp_atlas}]{\sphinxcrossref{\sphinxcode{\sphinxupquote{get\_fp\_atlas}}}}}(self{[}, binDays, clobber{]})
&
parse local ATLAS forced (binned or not binned) photometric data to \sphinxstyleemphasis{self.lc}
\\
\hline
{\hyperref[\detokenize{generated/sdapy.snerun.snobject.get_fp_ztf:sdapy.snerun.snobject.get_fp_ztf}]{\sphinxcrossref{\sphinxcode{\sphinxupquote{get\_fp\_ztf}}}}}(self{[}, seeing\_cut{]})
&
parse local ZTF forced photometic data to \sphinxstyleemphasis{self.lc}
\\
\hline
{\hyperref[\detokenize{generated/sdapy.snerun.snobject.get_local_spectra:sdapy.snerun.snobject.get_local_spectra}]{\sphinxcrossref{\sphinxcode{\sphinxupquote{get\_local\_spectra}}}}}(self{[}, source{]})
&
get local ZTF fritz/marshal spectra         \sphinxstyleemphasis{Note}: this is only available for ZTF internal collaborators.
\\
\hline
{\hyperref[\detokenize{generated/sdapy.snerun.snobject.get_model:sdapy.snerun.snobject.get_model}]{\sphinxcrossref{\sphinxcode{\sphinxupquote{get\_model}}}}}(self{[}, engine, model, source{]})
&

\\
\hline
{\hyperref[\detokenize{generated/sdapy.snerun.snobject.get_oac:sdapy.snerun.snobject.get_oac}]{\sphinxcrossref{\sphinxcode{\sphinxupquote{get\_oac}}}}}(self{[}, which{]})
&
parse local Open Astronomical Catalog data to \sphinxstyleemphasis{self.lc}
\\
\hline
{\hyperref[\detokenize{generated/sdapy.snerun.snobject.lyman_bol:sdapy.snerun.snobject.lyman_bol}]{\sphinxcrossref{\sphinxcode{\sphinxupquote{lyman\_bol}}}}}(self{[}, interpolation{]})
&
calculate bolometric LC from colours, with Lyman bolometric correction
\\
\hline
{\hyperref[\detokenize{generated/sdapy.snerun.snobject.mag_to_flux:sdapy.snerun.snobject.mag_to_flux}]{\sphinxcrossref{\sphinxcode{\sphinxupquote{mag\_to\_flux}}}}}(mag{[}, magerr, limmag, sigma, …{]})
&
with flux/mag zerop
\\
\hline
{\hyperref[\detokenize{generated/sdapy.snerun.snobject.match_colors:sdapy.snerun.snobject.match_colors}]{\sphinxcrossref{\sphinxcode{\sphinxupquote{match\_colors}}}}}(self, \textbackslash{}*\textbackslash{}*kwargs)
&
match colors
\\
\hline
{\hyperref[\detokenize{generated/sdapy.snerun.snobject.merge_df_cols:sdapy.snerun.snobject.merge_df_cols}]{\sphinxcrossref{\sphinxcode{\sphinxupquote{merge\_df\_cols}}}}}(\_snobject\_\_df)
&
will bug when combining columns from differents sources, i.e.
\\
\hline
{\hyperref[\detokenize{generated/sdapy.snerun.snobject.mjd_now:sdapy.snerun.snobject.mjd_now}]{\sphinxcrossref{\sphinxcode{\sphinxupquote{mjd\_now}}}}}(self{[}, jd{]})
&
get current juliand date via astropy.time
\\
\hline
{\hyperref[\detokenize{generated/sdapy.snerun.snobject.oac_phot_url:sdapy.snerun.snobject.oac_phot_url}]{\sphinxcrossref{\sphinxcode{\sphinxupquote{oac\_phot\_url}}}}}(self, url)
&
(static) make url to query OAC
\\
\hline
{\hyperref[\detokenize{generated/sdapy.snerun.snobject.parse_coo:sdapy.snerun.snobject.parse_coo}]{\sphinxcrossref{\sphinxcode{\sphinxupquote{parse\_coo}}}}}(self{[}, verbose, deg, hpx, nside{]})
&
handle SN coordinate from self.ra and self.dec
\\
\hline
{\hyperref[\detokenize{generated/sdapy.snerun.snobject.query_alert_ztf:sdapy.snerun.snobject.query_alert_ztf}]{\sphinxcrossref{\sphinxcode{\sphinxupquote{query\_alert\_ztf}}}}}(self{[}, source{]})
&
qeury ZTF laert photometry via ztfquery, see \sphinxurl{https://github.com/MickaelRigault/ztfquery} \sphinxstyleemphasis{Note}: this is only available for ZTF internal collaborators.
\\
\hline
{\hyperref[\detokenize{generated/sdapy.snerun.snobject.query_fp_atlas:sdapy.snerun.snobject.query_fp_atlas}]{\sphinxcrossref{\sphinxcode{\sphinxupquote{query\_fp\_atlas}}}}}(self, \textbackslash{}*\textbackslash{}*kwargs)
&
qeury ATLAS forced photometry,  see \sphinxurl{https://fallingstar-data.com/forcedphot/static/apiexample.py}
\\
\hline
{\hyperref[\detokenize{generated/sdapy.snerun.snobject.query_fp_ztf:sdapy.snerun.snobject.query_fp_ztf}]{\sphinxcrossref{\sphinxcode{\sphinxupquote{query\_fp\_ztf}}}}}(self{[}, get\_email{]})
&
qeury ZTF forced photometry,  see documentation: \sphinxurl{https://irsa.ipac.caltech.edu/data/ZTF/docs/forcedphot.pdf}.
\\
\hline
{\hyperref[\detokenize{generated/sdapy.snerun.snobject.query_oac:sdapy.snerun.snobject.query_oac}]{\sphinxcrossref{\sphinxcode{\sphinxupquote{query\_oac}}}}}(self{[}, db, which{]})
&
query Open Astronomical Catalog data,  via OACAPI, \sphinxurl{https://github.com/astrocatalogs/OACAPI}
\\
\hline
{\hyperref[\detokenize{generated/sdapy.snerun.snobject.query_spectra:sdapy.snerun.snobject.query_spectra}]{\sphinxcrossref{\sphinxcode{\sphinxupquote{query\_spectra}}}}}(self{[}, source{]})
&
qeury ZTF spectra via ztfquery, see \sphinxurl{https://github.com/MickaelRigault/ztfquery} \sphinxstyleemphasis{Note}: this is only available for ZTF internal collaborators.
\\
\hline
{\hyperref[\detokenize{generated/sdapy.snerun.snobject.rapid:sdapy.snerun.snobject.rapid}]{\sphinxcrossref{\sphinxcode{\sphinxupquote{rapid}}}}}(self{[}, ax{]})
&
run astrorapid codes (a Deep learning classifier to distiinguish LCs between different SNe type) on \sphinxstylestrong{self.lc}   \sphinxstyleemphasis{Note}: this is only available when astrorapid packagte is correcly installed
\\
\hline
{\hyperref[\detokenize{generated/sdapy.snerun.snobject.rapid_plot:sdapy.snerun.snobject.rapid_plot}]{\sphinxcrossref{\sphinxcode{\sphinxupquote{rapid\_plot}}}}}(self, ax, class\_names, class\_color)
&
(static) make astrorapid plot
\\
\hline
{\hyperref[\detokenize{generated/sdapy.snerun.snobject.read_c10:sdapy.snerun.snobject.read_c10}]{\sphinxcrossref{\sphinxcode{\sphinxupquote{read\_c10}}}}}({[}filename{]})
&

\\
\hline
{\hyperref[\detokenize{generated/sdapy.snerun.snobject.read_kwargs:sdapy.snerun.snobject.read_kwargs}]{\sphinxcrossref{\sphinxcode{\sphinxupquote{read\_kwargs}}}}}(self, \textbackslash{}*\textbackslash{}*kwargs)
&
Define a proper way to read and update optional parameters
\\
\hline
{\hyperref[\detokenize{generated/sdapy.snerun.snobject.run:sdapy.snerun.snobject.run}]{\sphinxcrossref{\sphinxcode{\sphinxupquote{run}}}}}(self, \textbackslash{}*\textbackslash{}*kwargs)
&
actions for one SN.
\\
\hline
{\hyperref[\detokenize{generated/sdapy.snerun.snobject.run_fit:sdapy.snerun.snobject.run_fit}]{\sphinxcrossref{\sphinxcode{\sphinxupquote{run\_fit}}}}}(self, enginetype{[}, source{]})
&
run model fittings on varies of data from different engines
\\
\hline
{\hyperref[\detokenize{generated/sdapy.snerun.snobject.run_gp:sdapy.snerun.snobject.run_gp}]{\sphinxcrossref{\sphinxcode{\sphinxupquote{run\_gp}}}}}(self{[}, source{]})
&
run Gaussian Process interpolation via george package on \sphinxstylestrong{self.lc}
\\
\hline
{\hyperref[\detokenize{generated/sdapy.snerun.snobject.savefig:sdapy.snerun.snobject.savefig}]{\sphinxcrossref{\sphinxcode{\sphinxupquote{savefig}}}}}(self, \textbackslash{}*\textbackslash{}*kwargs)
&

\\
\hline
{\hyperref[\detokenize{generated/sdapy.snerun.snobject.set_peak_bol_main:sdapy.snerun.snobject.set_peak_bol_main}]{\sphinxcrossref{\sphinxcode{\sphinxupquote{set\_peak\_bol\_main}}}}}(self{[}, model\_name, …{]})
&
set peak time and fluxes with bolometric LC fittings
\\
\hline
{\hyperref[\detokenize{generated/sdapy.snerun.snobject.set_peak_gp:sdapy.snerun.snobject.set_peak_gp}]{\sphinxcrossref{\sphinxcode{\sphinxupquote{set\_peak\_gp}}}}}(self, filt)
&
set peak time and fluxes with GP interpolation
\\
\hline
{\hyperref[\detokenize{generated/sdapy.snerun.snobject.set_peak_multiband_main:sdapy.snerun.snobject.set_peak_multiband_main}]{\sphinxcrossref{\sphinxcode{\sphinxupquote{set\_peak\_multiband\_main}}}}}(self, filt{[}, model\_name{]})
&
set peak time and fluxes with multiband LC fittings
\\
\hline
{\hyperref[\detokenize{generated/sdapy.snerun.snobject.set_texp_bol_main:sdapy.snerun.snobject.set_texp_bol_main}]{\sphinxcrossref{\sphinxcode{\sphinxupquote{set\_texp\_bol\_main}}}}}(self{[}, model\_name, …{]})
&
set the explosion epoch with bolometric LC fittings
\\
\hline
{\hyperref[\detokenize{generated/sdapy.snerun.snobject.set_texp_midway:sdapy.snerun.snobject.set_texp_midway}]{\sphinxcrossref{\sphinxcode{\sphinxupquote{set\_texp\_midway}}}}}(self)
&
set the explosion epoch with the first detection and the last non\sphinxhyphen{}detection
\\
\hline
{\hyperref[\detokenize{generated/sdapy.snerun.snobject.set_texp_pl:sdapy.snerun.snobject.set_texp_pl}]{\sphinxcrossref{\sphinxcode{\sphinxupquote{set\_texp\_pl}}}}}(self, filt{[}, model\_name{]})
&
set the explosion epoch with multiband LC fittings
\\
\hline
{\hyperref[\detokenize{generated/sdapy.snerun.snobject.show_corner:sdapy.snerun.snobject.show_corner}]{\sphinxcrossref{\sphinxcode{\sphinxupquote{show\_corner}}}}}(self, ax{[}, index, gp, engine, …{]})
&

\\
\hline
{\hyperref[\detokenize{generated/sdapy.snerun.snobject.showfig:sdapy.snerun.snobject.showfig}]{\sphinxcrossref{\sphinxcode{\sphinxupquote{showfig}}}}}(self{[}, ax{]})
&

\\
\hline
{\hyperref[\detokenize{generated/sdapy.snerun.snobject.summarize_plot:sdapy.snerun.snobject.summarize_plot}]{\sphinxcrossref{\sphinxcode{\sphinxupquote{summarize\_plot}}}}}(self{[}, ax{]})
&
show a summarize plot: flux/absolute/bolometric LCs, colour curve and the spectra
\\
\hline
{\hyperref[\detokenize{generated/sdapy.snerun.snobject.sym_mag:sdapy.snerun.snobject.sym_mag}]{\sphinxcrossref{\sphinxcode{\sphinxupquote{sym\_mag}}}}}(self, w, f, filt)
&
(static) synthetic magnitudes with filter transmission curves
\\
\hline
\end{longtable}\sphinxatlongtableend\end{savenotes}
\subsubsection*{Attributes}


\begin{savenotes}\sphinxatlongtablestart\begin{longtable}[c]{\X{1}{2}\X{1}{2}}
\hline

\endfirsthead

\multicolumn{2}{c}%
{\makebox[0pt]{\sphinxtablecontinued{\tablename\ \thetable{} \textendash{} continued from previous page}}}\\
\hline

\endhead

\hline
\multicolumn{2}{r}{\makebox[0pt][r]{\sphinxtablecontinued{Continued on next page}}}\\
\endfoot

\endlastfoot

\sphinxcode{\sphinxupquote{keys2query\_lc}}
&

\\
\hline
\sphinxcode{\sphinxupquote{keys2query\_spec}}
&

\\
\hline
\sphinxcode{\sphinxupquote{urllist}}
&

\\
\hline
\sphinxcode{\sphinxupquote{version}}
&

\\
\hline
\end{longtable}\sphinxatlongtableend\end{savenotes}

\end{fulllineitems}


Below provide various of functions for \sphinxtitleref{snobject}:


\begin{savenotes}\sphinxatlongtablestart\begin{longtable}[c]{\X{1}{2}\X{1}{2}}
\hline

\endfirsthead

\multicolumn{2}{c}%
{\makebox[0pt]{\sphinxtablecontinued{\tablename\ \thetable{} \textendash{} continued from previous page}}}\\
\hline

\endhead

\hline
\multicolumn{2}{r}{\makebox[0pt][r]{\sphinxtablecontinued{Continued on next page}}}\\
\endfoot

\endlastfoot

{\hyperref[\detokenize{generated/sdapy.snerun.snobject.__init__:sdapy.snerun.snobject.__init__}]{\sphinxcrossref{\sphinxcode{\sphinxupquote{snobject.\_\_init\_\_}}}}}(self, objid{[}, aliasid, z, …{]})
&
initialize \sphinxstyleemphasis{snelist}
\\
\hline
{\hyperref[\detokenize{generated/sdapy.snerun.snobject.read_kwargs:sdapy.snerun.snobject.read_kwargs}]{\sphinxcrossref{\sphinxcode{\sphinxupquote{snobject.read\_kwargs}}}}}(self, \textbackslash{}*\textbackslash{}*kwargs)
&
Define a proper way to read and update optional parameters
\\
\hline
{\hyperref[\detokenize{generated/sdapy.snerun.snobject.run:sdapy.snerun.snobject.run}]{\sphinxcrossref{\sphinxcode{\sphinxupquote{snobject.run}}}}}(self, \textbackslash{}*\textbackslash{}*kwargs)
&
actions for one SN.
\\
\hline
{\hyperref[\detokenize{generated/sdapy.snerun.snobject.config_ztfquery:sdapy.snerun.snobject.config_ztfquery}]{\sphinxcrossref{\sphinxcode{\sphinxupquote{snobject.config\_ztfquery}}}}}(self)
&
check and set ztfquery accounts
\\
\hline
{\hyperref[\detokenize{generated/sdapy.snerun.snobject.parse_coo:sdapy.snerun.snobject.parse_coo}]{\sphinxcrossref{\sphinxcode{\sphinxupquote{snobject.parse\_coo}}}}}(self{[}, verbose, deg, …{]})
&
handle SN coordinate from self.ra and self.dec
\\
\hline
{\hyperref[\detokenize{generated/sdapy.snerun.snobject.mjd_now:sdapy.snerun.snobject.mjd_now}]{\sphinxcrossref{\sphinxcode{\sphinxupquote{snobject.mjd\_now}}}}}(self{[}, jd{]})
&
get current juliand date via astropy.time
\\
\hline
{\hyperref[\detokenize{generated/sdapy.snerun.snobject.add_lc:sdapy.snerun.snobject.add_lc}]{\sphinxcrossref{\sphinxcode{\sphinxupquote{snobject.add\_lc}}}}}(self, df{[}, source{]})
&
add lightcurve data into self.lc
\\
\hline
{\hyperref[\detokenize{generated/sdapy.snerun.snobject.add_flux:sdapy.snerun.snobject.add_flux}]{\sphinxcrossref{\sphinxcode{\sphinxupquote{snobject.add\_flux}}}}}(self{[}, zp, source{]})
&
add \sphinxstylestrong{flux} and \sphinxstylestrong{eflux} column, based on \sphinxstylestrong{mag}, \sphinxstylestrong{emag}, and/or \sphinxstylestrong{limmag} column.
\\
\hline
{\hyperref[\detokenize{generated/sdapy.snerun.snobject.add_mag:sdapy.snerun.snobject.add_mag}]{\sphinxcrossref{\sphinxcode{\sphinxupquote{snobject.add\_mag}}}}}(self{[}, zp, source{]})
&
add \sphinxstylestrong{mag} and/or \sphinxstyleemphasis{limmag}/\sphinxstylestrong{emag} column, based on \sphinxstylestrong{flux}, \sphinxstylestrong{eflux} column.
\\
\hline
{\hyperref[\detokenize{generated/sdapy.snerun.snobject.get_external_phot:sdapy.snerun.snobject.get_external_phot}]{\sphinxcrossref{\sphinxcode{\sphinxupquote{snobject.get\_external\_phot}}}}}(self, filename, …)
&
parse user defined photometric data to \sphinxstyleemphasis{self.lc}
\\
\hline
{\hyperref[\detokenize{generated/sdapy.snerun.snobject.bin_fp_atlas:sdapy.snerun.snobject.bin_fp_atlas}]{\sphinxcrossref{\sphinxcode{\sphinxupquote{snobject.bin\_fp\_atlas}}}}}(self{[}, binDays, …{]})
&
binning for ATLAS forced photometry with Dave’s code, \sphinxurl{https://gist.github.com/thespacedoctor/86777fa5a9567b7939e8d84fd8cf6a76}
\\
\hline
{\hyperref[\detokenize{generated/sdapy.snerun.snobject.get_fp_atlas:sdapy.snerun.snobject.get_fp_atlas}]{\sphinxcrossref{\sphinxcode{\sphinxupquote{snobject.get\_fp\_atlas}}}}}(self{[}, binDays, clobber{]})
&
parse local ATLAS forced (binned or not binned) photometric data to \sphinxstyleemphasis{self.lc}
\\
\hline
{\hyperref[\detokenize{generated/sdapy.snerun.snobject.get_alert_ztf:sdapy.snerun.snobject.get_alert_ztf}]{\sphinxcrossref{\sphinxcode{\sphinxupquote{snobject.get\_alert\_ztf}}}}}(self{[}, source{]})
&
parse local ZTF alert photometic data to \sphinxstyleemphasis{self.lc}
\\
\hline
{\hyperref[\detokenize{generated/sdapy.snerun.snobject.get_fp_ztf:sdapy.snerun.snobject.get_fp_ztf}]{\sphinxcrossref{\sphinxcode{\sphinxupquote{snobject.get\_fp\_ztf}}}}}(self{[}, seeing\_cut{]})
&
parse local ZTF forced photometic data to \sphinxstyleemphasis{self.lc}
\\
\hline
{\hyperref[\detokenize{generated/sdapy.snerun.snobject.query_fp_atlas:sdapy.snerun.snobject.query_fp_atlas}]{\sphinxcrossref{\sphinxcode{\sphinxupquote{snobject.query\_fp\_atlas}}}}}(self, \textbackslash{}*\textbackslash{}*kwargs)
&
qeury ATLAS forced photometry,  see \sphinxurl{https://fallingstar-data.com/forcedphot/static/apiexample.py}
\\
\hline
{\hyperref[\detokenize{generated/sdapy.snerun.snobject.query_fp_ztf:sdapy.snerun.snobject.query_fp_ztf}]{\sphinxcrossref{\sphinxcode{\sphinxupquote{snobject.query\_fp\_ztf}}}}}(self{[}, get\_email{]})
&
qeury ZTF forced photometry,  see documentation: \sphinxurl{https://irsa.ipac.caltech.edu/data/ZTF/docs/forcedphot.pdf}.
\\
\hline
{\hyperref[\detokenize{generated/sdapy.snerun.snobject.query_alert_ztf:sdapy.snerun.snobject.query_alert_ztf}]{\sphinxcrossref{\sphinxcode{\sphinxupquote{snobject.query\_alert\_ztf}}}}}(self{[}, source{]})
&
qeury ZTF laert photometry via ztfquery, see \sphinxurl{https://github.com/MickaelRigault/ztfquery} \sphinxstyleemphasis{Note}: this is only available for ZTF internal collaborators.
\\
\hline
{\hyperref[\detokenize{generated/sdapy.snerun.snobject.query_spectra:sdapy.snerun.snobject.query_spectra}]{\sphinxcrossref{\sphinxcode{\sphinxupquote{snobject.query\_spectra}}}}}(self{[}, source{]})
&
qeury ZTF spectra via ztfquery, see \sphinxurl{https://github.com/MickaelRigault/ztfquery} \sphinxstyleemphasis{Note}: this is only available for ZTF internal collaborators.
\\
\hline
{\hyperref[\detokenize{generated/sdapy.snerun.snobject.get_local_spectra:sdapy.snerun.snobject.get_local_spectra}]{\sphinxcrossref{\sphinxcode{\sphinxupquote{snobject.get\_local\_spectra}}}}}(self{[}, source{]})
&
get local ZTF fritz/marshal spectra         \sphinxstyleemphasis{Note}: this is only available for ZTF internal collaborators.
\\
\hline
{\hyperref[\detokenize{generated/sdapy.snerun.snobject.correct_baseline:sdapy.snerun.snobject.correct_baseline}]{\sphinxcrossref{\sphinxcode{\sphinxupquote{snobject.correct\_baseline}}}}}(self, baseline{[}, …{]})
&
Baseline correction for ZTF forced photometry
\\
\hline
{\hyperref[\detokenize{generated/sdapy.snerun.snobject.calibrate_baseline:sdapy.snerun.snobject.calibrate_baseline}]{\sphinxcrossref{\sphinxcode{\sphinxupquote{snobject.calibrate\_baseline}}}}}(self{[}, ax, key, …{]})
&
Baseline calibration for ZTF forced photometry
\\
\hline
{\hyperref[\detokenize{generated/sdapy.snerun.snobject.get_external_spectra:sdapy.snerun.snobject.get_external_spectra}]{\sphinxcrossref{\sphinxcode{\sphinxupquote{snobject.get\_external\_spectra}}}}}(self, …{[}, tel{]})
&
parse \sphinxstylestrong{spectra} via a local file
\\
\hline
{\hyperref[\detokenize{generated/sdapy.snerun.snobject.rapid:sdapy.snerun.snobject.rapid}]{\sphinxcrossref{\sphinxcode{\sphinxupquote{snobject.rapid}}}}}(self{[}, ax{]})
&
run astrorapid codes (a Deep learning classifier to distiinguish LCs between different SNe type) on \sphinxstylestrong{self.lc}   \sphinxstyleemphasis{Note}: this is only available when astrorapid packagte is correcly installed
\\
\hline
{\hyperref[\detokenize{generated/sdapy.snerun.snobject.rapid_plot:sdapy.snerun.snobject.rapid_plot}]{\sphinxcrossref{\sphinxcode{\sphinxupquote{snobject.rapid\_plot}}}}}(self, ax, class\_names, …)
&
(static) make astrorapid plot
\\
\hline
{\hyperref[\detokenize{generated/sdapy.snerun.snobject.run_gp:sdapy.snerun.snobject.run_gp}]{\sphinxcrossref{\sphinxcode{\sphinxupquote{snobject.run\_gp}}}}}(self{[}, source{]})
&
run Gaussian Process interpolation via george package on \sphinxstylestrong{self.lc}
\\
\hline
{\hyperref[\detokenize{generated/sdapy.snerun.snobject.run_fit:sdapy.snerun.snobject.run_fit}]{\sphinxcrossref{\sphinxcode{\sphinxupquote{snobject.run\_fit}}}}}(self, enginetype{[}, source{]})
&
run model fittings on varies of data from different engines
\\
\hline
{\hyperref[\detokenize{generated/sdapy.snerun.snobject.set_peak_gp:sdapy.snerun.snobject.set_peak_gp}]{\sphinxcrossref{\sphinxcode{\sphinxupquote{snobject.set\_peak\_gp}}}}}(self, filt)
&
set peak time and fluxes with GP interpolation
\\
\hline
{\hyperref[\detokenize{generated/sdapy.snerun.snobject.set_peak_bol_main:sdapy.snerun.snobject.set_peak_bol_main}]{\sphinxcrossref{\sphinxcode{\sphinxupquote{snobject.set\_peak\_bol\_main}}}}}(self{[}, …{]})
&
set peak time and fluxes with bolometric LC fittings
\\
\hline
{\hyperref[\detokenize{generated/sdapy.snerun.snobject.set_peak_multiband_main:sdapy.snerun.snobject.set_peak_multiband_main}]{\sphinxcrossref{\sphinxcode{\sphinxupquote{snobject.set\_peak\_multiband\_main}}}}}(self, filt)
&
set peak time and fluxes with multiband LC fittings
\\
\hline
{\hyperref[\detokenize{generated/sdapy.snerun.snobject.set_texp_pl:sdapy.snerun.snobject.set_texp_pl}]{\sphinxcrossref{\sphinxcode{\sphinxupquote{snobject.set\_texp\_pl}}}}}(self, filt{[}, model\_name{]})
&
set the explosion epoch with multiband LC fittings
\\
\hline
{\hyperref[\detokenize{generated/sdapy.snerun.snobject.set_texp_bol_main:sdapy.snerun.snobject.set_texp_bol_main}]{\sphinxcrossref{\sphinxcode{\sphinxupquote{snobject.set\_texp\_bol\_main}}}}}(self{[}, …{]})
&
set the explosion epoch with bolometric LC fittings
\\
\hline
{\hyperref[\detokenize{generated/sdapy.snerun.snobject.set_texp_midway:sdapy.snerun.snobject.set_texp_midway}]{\sphinxcrossref{\sphinxcode{\sphinxupquote{snobject.set\_texp\_midway}}}}}(self)
&
set the explosion epoch with the first detection and the last non\sphinxhyphen{}detection
\\
\hline
{\hyperref[\detokenize{generated/sdapy.snerun.snobject._flux_at:sdapy.snerun.snobject._flux_at}]{\sphinxcrossref{\sphinxcode{\sphinxupquote{snobject.\_flux\_at}}}}}(self, filt, phase{[}, …{]})
&
estimate flux (unit in uJy)
\\
\hline
{\hyperref[\detokenize{generated/sdapy.snerun.snobject._flux_at_list:sdapy.snerun.snobject._flux_at_list}]{\sphinxcrossref{\sphinxcode{\sphinxupquote{snobject.\_flux\_at\_list}}}}}(self, filt, phaselist)
&
estimate fluxes (unit in uJy) for a list of phases
\\
\hline
{\hyperref[\detokenize{generated/sdapy.snerun.snobject._mag_at:sdapy.snerun.snobject._mag_at}]{\sphinxcrossref{\sphinxcode{\sphinxupquote{snobject.\_mag\_at}}}}}(self, filt, phase{[}, …{]})
&
estimate apparent magnitude
\\
\hline
{\hyperref[\detokenize{generated/sdapy.snerun.snobject._mag_at_list:sdapy.snerun.snobject._mag_at_list}]{\sphinxcrossref{\sphinxcode{\sphinxupquote{snobject.\_mag\_at\_list}}}}}(self, filt, phaselist)
&
estimate apparent magnitudes from a list of phases
\\
\hline
{\hyperref[\detokenize{generated/sdapy.snerun.snobject._absmag_at:sdapy.snerun.snobject._absmag_at}]{\sphinxcrossref{\sphinxcode{\sphinxupquote{snobject.\_absmag\_at}}}}}(self, filt, phase{[}, …{]})
&
estimate absolute magnitude
\\
\hline
{\hyperref[\detokenize{generated/sdapy.snerun.snobject._absmag_at_list:sdapy.snerun.snobject._absmag_at_list}]{\sphinxcrossref{\sphinxcode{\sphinxupquote{snobject.\_absmag\_at\_list}}}}}(self, filt, phaselist)
&
estimate absolute magnitudes from a list of phases
\\
\hline
{\hyperref[\detokenize{generated/sdapy.snerun.snobject._color_at:sdapy.snerun.snobject._color_at}]{\sphinxcrossref{\sphinxcode{\sphinxupquote{snobject.\_color\_at}}}}}(self, filt1, filt2, phase)
&
estimate colour
\\
\hline
{\hyperref[\detokenize{generated/sdapy.snerun.snobject._rate_at:sdapy.snerun.snobject._rate_at}]{\sphinxcrossref{\sphinxcode{\sphinxupquote{snobject.\_rate\_at}}}}}(self, filt, phase1, phase2)
&
estimate magnitude rate
\\
\hline
{\hyperref[\detokenize{generated/sdapy.snerun.snobject.est_hostebv_with_c10:sdapy.snerun.snobject.est_hostebv_with_c10}]{\sphinxcrossref{\sphinxcode{\sphinxupquote{snobject.est\_hostebv\_with\_c10}}}}}(self{[}, …{]})
&
estimate host ebv with color comparison approach
\\
\hline
{\hyperref[\detokenize{generated/sdapy.snerun.snobject.calc_colors:sdapy.snerun.snobject.calc_colors}]{\sphinxcrossref{\sphinxcode{\sphinxupquote{snobject.calc\_colors}}}}}(self, \textbackslash{}*\textbackslash{}*kwargs)
&
calculate colors for two bands
\\
\hline
{\hyperref[\detokenize{generated/sdapy.snerun.snobject.lyman_bol:sdapy.snerun.snobject.lyman_bol}]{\sphinxcrossref{\sphinxcode{\sphinxupquote{snobject.lyman\_bol}}}}}(self{[}, interpolation{]})
&
calculate bolometric LC from colours, with Lyman bolometric correction
\\
\hline
{\hyperref[\detokenize{generated/sdapy.snerun.snobject.bb_colors:sdapy.snerun.snobject.bb_colors}]{\sphinxcrossref{\sphinxcode{\sphinxupquote{snobject.bb\_colors}}}}}(self, \textbackslash{}*\textbackslash{}*kwargs)
&
match colours for multiple bands, for blackbody (BB) construction
\\
\hline
{\hyperref[\detokenize{generated/sdapy.snerun.snobject.bb_bol:sdapy.snerun.snobject.bb_bol}]{\sphinxcrossref{\sphinxcode{\sphinxupquote{snobject.bb\_bol}}}}}(self{[}, do\_Kcorr, ab2vega{]})
&
calculate bolometric LC with diluted blackbody
\\
\hline
{\hyperref[\detokenize{generated/sdapy.snerun.snobject.match_colors:sdapy.snerun.snobject.match_colors}]{\sphinxcrossref{\sphinxcode{\sphinxupquote{snobject.match\_colors}}}}}(self, \textbackslash{}*\textbackslash{}*kwargs)
&
match colors
\\
\hline
{\hyperref[\detokenize{generated/sdapy.snerun.snobject.combine_multi_obs:sdapy.snerun.snobject.combine_multi_obs}]{\sphinxcrossref{\sphinxcode{\sphinxupquote{snobject.combine\_multi\_obs}}}}}(self, \textbackslash{}*\textbackslash{}*kwargs)
&
(static) bin and combine observations from common epoch
\\
\hline
{\hyperref[\detokenize{generated/sdapy.snerun.snobject.clip_lc:sdapy.snerun.snobject.clip_lc}]{\sphinxcrossref{\sphinxcode{\sphinxupquote{snobject.clip\_lc}}}}}(self, \textbackslash{}*\textbackslash{}*kwargs)
&
Removes outlier data points using GP interpolation
\\
\hline
{\hyperref[\detokenize{generated/sdapy.snerun.snobject._nepochs:sdapy.snerun.snobject._nepochs}]{\sphinxcrossref{\sphinxcode{\sphinxupquote{snobject.\_nepochs}}}}}(self, \textbackslash{}*\textbackslash{}*kwargs)
&
how many epochs of photometry in either band.
\\
\hline
{\hyperref[\detokenize{generated/sdapy.snerun.snobject._ncolors:sdapy.snerun.snobject._ncolors}]{\sphinxcrossref{\sphinxcode{\sphinxupquote{snobject.\_ncolors}}}}}(self, \textbackslash{}*\textbackslash{}*kwargs)
&
how many color epochs
\\
\hline
{\hyperref[\detokenize{generated/sdapy.snerun.snobject._peak_accuracy:sdapy.snerun.snobject._peak_accuracy}]{\sphinxcrossref{\sphinxcode{\sphinxupquote{snobject.\_peak\_accuracy}}}}}(self{[}, within{]})
&
how accurate a peak can be determined: how many photometric points available within a range to the peak
\\
\hline
{\hyperref[\detokenize{generated/sdapy.snerun.snobject.summarize_plot:sdapy.snerun.snobject.summarize_plot}]{\sphinxcrossref{\sphinxcode{\sphinxupquote{snobject.summarize\_plot}}}}}(self{[}, ax{]})
&
show a summarize plot: flux/absolute/bolometric LCs, colour curve and the spectra
\\
\hline
{\hyperref[\detokenize{generated/sdapy.snerun.snobject._ax:sdapy.snerun.snobject._ax}]{\sphinxcrossref{\sphinxcode{\sphinxupquote{snobject.\_ax}}}}}(self{[}, to\_t0, show\_title, …{]})
&

\\
\hline
{\hyperref[\detokenize{generated/sdapy.snerun.snobject._ax1:sdapy.snerun.snobject._ax1}]{\sphinxcrossref{\sphinxcode{\sphinxupquote{snobject.\_ax1}}}}}(self{[}, stype, show\_title, …{]})
&
Spectra plot
\\
\hline
{\hyperref[\detokenize{generated/sdapy.snerun.snobject._ax2:sdapy.snerun.snobject._ax2}]{\sphinxcrossref{\sphinxcode{\sphinxupquote{snobject.\_ax2}}}}}(self{[}, show\_title, …{]})
&

\\
\hline
{\hyperref[\detokenize{generated/sdapy.snerun.snobject._ax3:sdapy.snerun.snobject._ax3}]{\sphinxcrossref{\sphinxcode{\sphinxupquote{snobject.\_ax3}}}}}(self{[}, show\_title, …{]})
&

\\
\hline
{\hyperref[\detokenize{generated/sdapy.snerun.snobject._ax4:sdapy.snerun.snobject._ax4}]{\sphinxcrossref{\sphinxcode{\sphinxupquote{snobject.\_ax4}}}}}(self{[}, show\_title, …{]})
&

\\
\hline
{\hyperref[\detokenize{generated/sdapy.snerun.snobject.show_corner:sdapy.snerun.snobject.show_corner}]{\sphinxcrossref{\sphinxcode{\sphinxupquote{snobject.show\_corner}}}}}(self, ax{[}, index, gp, …{]})
&

\\
\hline
{\hyperref[\detokenize{generated/sdapy.snerun.snobject.get_model:sdapy.snerun.snobject.get_model}]{\sphinxcrossref{\sphinxcode{\sphinxupquote{snobject.get\_model}}}}}(self{[}, engine, model, source{]})
&

\\
\hline
{\hyperref[\detokenize{generated/sdapy.snerun.snobject.savefig:sdapy.snerun.snobject.savefig}]{\sphinxcrossref{\sphinxcode{\sphinxupquote{snobject.savefig}}}}}(self, \textbackslash{}*\textbackslash{}*kwargs)
&

\\
\hline
{\hyperref[\detokenize{generated/sdapy.snerun.snobject.showfig:sdapy.snerun.snobject.showfig}]{\sphinxcrossref{\sphinxcode{\sphinxupquote{snobject.showfig}}}}}(self{[}, ax{]})
&

\\
\hline
{\hyperref[\detokenize{generated/sdapy.snerun.snobject.dm_error:sdapy.snerun.snobject.dm_error}]{\sphinxcrossref{\sphinxcode{\sphinxupquote{snobject.dm\_error}}}}}(self, filt)
&

\\
\hline
\sphinxcode{\sphinxupquote{snobject.clip\_df}}
&

\\
\hline
{\hyperref[\detokenize{generated/sdapy.snerun.snobject.read_c10:sdapy.snerun.snobject.read_c10}]{\sphinxcrossref{\sphinxcode{\sphinxupquote{snobject.read\_c10}}}}}({[}filename{]})
&

\\
\hline
{\hyperref[\detokenize{generated/sdapy.snerun.snobject.mag_to_flux:sdapy.snerun.snobject.mag_to_flux}]{\sphinxcrossref{\sphinxcode{\sphinxupquote{snobject.mag\_to\_flux}}}}}(mag{[}, magerr, limmag, …{]})
&
with flux/mag zerop
\\
\hline
{\hyperref[\detokenize{generated/sdapy.snerun.snobject.flux_to_mag:sdapy.snerun.snobject.flux_to_mag}]{\sphinxcrossref{\sphinxcode{\sphinxupquote{snobject.flux\_to\_mag}}}}}(flux{[}, dflux, sigma, …{]})
&
Converts fluxes (erg/s/cm2/A) into AB magnitudes with flux/mag zerop
\\
\hline
{\hyperref[\detokenize{generated/sdapy.snerun.snobject.bin_df:sdapy.snerun.snobject.bin_df}]{\sphinxcrossref{\sphinxcode{\sphinxupquote{snobject.bin\_df}}}}}(df{[}, deltah, xkey, fkey{]})
&

\\
\hline
{\hyperref[\detokenize{generated/sdapy.snerun.snobject.merge_df_cols:sdapy.snerun.snobject.merge_df_cols}]{\sphinxcrossref{\sphinxcode{\sphinxupquote{snobject.merge\_df\_cols}}}}}(\_snobject\_\_df)
&
will bug when combining columns from differents sources, i.e.
\\
\hline
{\hyperref[\detokenize{generated/sdapy.snerun.snobject.oac_phot_url:sdapy.snerun.snobject.oac_phot_url}]{\sphinxcrossref{\sphinxcode{\sphinxupquote{snobject.oac\_phot\_url}}}}}(self, url)
&
(static) make url to query OAC
\\
\hline
{\hyperref[\detokenize{generated/sdapy.snerun.snobject.get_oac:sdapy.snerun.snobject.get_oac}]{\sphinxcrossref{\sphinxcode{\sphinxupquote{snobject.get\_oac}}}}}(self{[}, which{]})
&
parse local Open Astronomical Catalog data to \sphinxstyleemphasis{self.lc}
\\
\hline
{\hyperref[\detokenize{generated/sdapy.snerun.snobject.query_oac:sdapy.snerun.snobject.query_oac}]{\sphinxcrossref{\sphinxcode{\sphinxupquote{snobject.query\_oac}}}}}(self{[}, db, which{]})
&
query Open Astronomical Catalog data,  via OACAPI, \sphinxurl{https://github.com/astrocatalogs/OACAPI}
\\
\hline
{\hyperref[\detokenize{generated/sdapy.snerun.snobject.sym_mag:sdapy.snerun.snobject.sym_mag}]{\sphinxcrossref{\sphinxcode{\sphinxupquote{snobject.sym\_mag}}}}}(self, w, f, filt)
&
(static) synthetic magnitudes with filter transmission curves
\\
\hline
\end{longtable}\sphinxatlongtableend\end{savenotes}


\subsection{sdapy.snerun.snobject.\_\_init\_\_}
\label{\detokenize{generated/sdapy.snerun.snobject.__init__:sdapy-snerun-snobject-init}}\label{\detokenize{generated/sdapy.snerun.snobject.__init__::doc}}\index{\_\_init\_\_() (sdapy.snerun.snobject method)@\spxentry{\_\_init\_\_()}\spxextra{sdapy.snerun.snobject method}}

\begin{fulllineitems}
\phantomsection\label{\detokenize{generated/sdapy.snerun.snobject.__init__:sdapy.snerun.snobject.__init__}}\pysiglinewithargsret{\sphinxcode{\sphinxupquote{snobject.}}\sphinxbfcode{\sphinxupquote{\_\_init\_\_}}}{\emph{self}, \emph{objid}, \emph{aliasid=None}, \emph{z=None}, \emph{ra=None}, \emph{dec=None}, \emph{mkwebv=None}, \emph{hostebv=None}, \emph{sntype=None}, \emph{dm=None}, \emph{jdpeak=None}, \emph{fig=None}, \emph{ax=None}, \emph{ax1=None}, \emph{ax2=None}, \emph{ax3=None}, \emph{ax4=None}, \emph{**kwargs}}{}
initialize \sphinxstyleemphasis{snelist}
\begin{quote}\begin{description}
\item[{Parameters}] \leavevmode\begin{description}
\item[{\sphinxstylestrong{objid}}] \leavevmode{[}\sphinxtitleref{str}{]}
object ID string

\item[{\sphinxstylestrong{aliasid}}] \leavevmode{[}\sphinxtitleref{str}{]}
other name of object

\item[{\sphinxstylestrong{z}}] \leavevmode{[}\sphinxtitleref{float}{]}
redshift

\item[{\sphinxstylestrong{ra}}] \leavevmode{[}\sphinxtitleref{str}{]}
R.A. in hh:mm:ss

\item[{\sphinxstylestrong{dec}}] \leavevmode{[}\sphinxtitleref{str}{]}
Dec. in dd:mm:dd

\item[{\sphinxstylestrong{mkwebv}}] \leavevmode{[}\sphinxtitleref{float}{]}
milky way E(B\sphinxhyphen{}V)

\item[{\sphinxstylestrong{hostebv}}] \leavevmode{[}\sphinxtitleref{float}{]}
host galaxy E(B\sphinxhyphen{}V)

\item[{\sphinxstylestrong{sntype}}] \leavevmode{[}\sphinxtitleref{str}{]}
supernova type

\item[{\sphinxstylestrong{dm}}] \leavevmode{[}\sphinxtitleref{float}{]}
distance module

\item[{\sphinxstylestrong{jdpeak}}] \leavevmode{[}\sphinxtitleref{float}{]}
julian date of peak epoch

\item[{\sphinxstylestrong{ax}}] \leavevmode{[}matplotlib.axes{]}
for flux lightcurve plot

\item[{\sphinxstylestrong{ax1}}] \leavevmode{[}matplotlib.axes{]}
for spectra plot

\item[{\sphinxstylestrong{ax2}}] \leavevmode{[}matplotlib.axes{]}
for magnitude lightcurve plot

\item[{\sphinxstylestrong{ax3}}] \leavevmode{[}matplotlib.axes{]}
for colour plot

\item[{\sphinxstylestrong{ax4}}] \leavevmode{[}matplotlib.axes{]}
for luminosoty plot

\item[{\sphinxstylestrong{kwargs}}] \leavevmode{[}\sphinxtitleref{Keyword Arguments}{]}
see \sphinxurl{https://github.com/saberyoung/HAFFET/blob/master/sdapy/data/default\_par.txt},
\sphinxstylestrong{snobject} part

\end{description}

\end{description}\end{quote}
\subsubsection*{Examples}

\begin{sphinxVerbatim}[commandchars=\\\{\}]
\PYG{g+gp}{\PYGZgt{}\PYGZgt{}\PYGZgt{} }\PYG{k+kn}{from} \PYG{n+nn}{sdapy} \PYG{k+kn}{import} \PYG{n}{snerun}
\PYG{g+gp}{\PYGZgt{}\PYGZgt{}\PYGZgt{} }\PYG{n}{a} \PYG{o}{=} \PYG{n}{snerun}\PYG{o}{.}\PYG{n}{snobject}\PYG{p}{(}\PYG{l+s+s1}{\PYGZsq{}}\PYG{l+s+s1}{sss}\PYG{l+s+s1}{\PYGZsq{}}\PYG{p}{)}
\PYG{g+gp}{\PYGZgt{}\PYGZgt{}\PYGZgt{} }\PYG{n}{a}
\PYG{g+go}{\PYGZlt{}sdapy.snerun.snobject object at 0x7fa4c8284780\PYGZgt{}}
\end{sphinxVerbatim}

\end{fulllineitems}



\subsection{sdapy.snerun.snobject.\_flux\_at}
\label{\detokenize{generated/sdapy.snerun.snobject._flux_at:sdapy-snerun-snobject-flux-at}}\label{\detokenize{generated/sdapy.snerun.snobject._flux_at::doc}}\index{\_flux\_at() (sdapy.snerun.snobject method)@\spxentry{\_flux\_at()}\spxextra{sdapy.snerun.snobject method}}

\begin{fulllineitems}
\phantomsection\label{\detokenize{generated/sdapy.snerun.snobject._flux_at:sdapy.snerun.snobject._flux_at}}\pysiglinewithargsret{\sphinxcode{\sphinxupquote{snobject.}}\sphinxbfcode{\sphinxupquote{\_flux\_at}}}{\emph{self}, \emph{filt}, \emph{phase}, \emph{interpolation=None}, \emph{fitmodel=0}, \emph{**kwargs}}{}
estimate flux (unit in uJy)
\begin{quote}\begin{description}
\item[{Parameters}] \leavevmode\begin{description}
\item[{\sphinxstylestrong{filt}}] \leavevmode{[}\sphinxtitleref{str}{]}
filter

\item[{\sphinxstylestrong{phase}}] \leavevmode{[}\sphinxtitleref{float}{]}
rest frame phase (days) relative to t0

\item[{\sphinxstylestrong{tdbin}}] \leavevmode{[}\sphinxtitleref{float}{]}
threshold for binning

\item[{\sphinxstylestrong{interpolation}}] \leavevmode{[}\sphinxtitleref{str}{]}
estimate flux with data epoch less than than \sphinxstylestrong{tdbin}, or interpolation from GP/fits

\item[{\sphinxstylestrong{fitmodel}}] \leavevmode{[}\sphinxtitleref{int}{]}
if multiple models available, which of them to be used

\end{description}

\item[{Returns}] \leavevmode\begin{description}
\item[{\sphinxstylestrong{flux}}] \leavevmode{[}\sphinxtitleref{float}{]}
\item[{\sphinxstylestrong{flux error}}] \leavevmode{[}\sphinxtitleref{float}{]}
\end{description}

\end{description}\end{quote}


\sphinxstrong{See also:}

\begin{description}
\item[{\sphinxcode{\sphinxupquote{sbobject.\_flux\_at\_list}}}] \leavevmode
\end{description}



\end{fulllineitems}



\subsection{sdapy.snerun.snobject.\_flux\_at\_list}
\label{\detokenize{generated/sdapy.snerun.snobject._flux_at_list:sdapy-snerun-snobject-flux-at-list}}\label{\detokenize{generated/sdapy.snerun.snobject._flux_at_list::doc}}\index{\_flux\_at\_list() (sdapy.snerun.snobject method)@\spxentry{\_flux\_at\_list()}\spxextra{sdapy.snerun.snobject method}}

\begin{fulllineitems}
\phantomsection\label{\detokenize{generated/sdapy.snerun.snobject._flux_at_list:sdapy.snerun.snobject._flux_at_list}}\pysiglinewithargsret{\sphinxcode{\sphinxupquote{snobject.}}\sphinxbfcode{\sphinxupquote{\_flux\_at\_list}}}{\emph{self}, \emph{filt}, \emph{phaselist}, \emph{interpolation=None}, \emph{**kwargs}}{}
estimate fluxes (unit in uJy) for a list of phases
\begin{quote}\begin{description}
\item[{Parameters}] \leavevmode\begin{description}
\item[{\sphinxstylestrong{filt}}] \leavevmode{[}\sphinxtitleref{str}{]}
filter

\item[{\sphinxstylestrong{phaselist}}] \leavevmode{[}\sphinxtitleref{list}{]}
rest frame phases (days) relative to t0

\item[{\sphinxstylestrong{tdbin}}] \leavevmode{[}\sphinxtitleref{float}{]}
threshold for binning

\item[{\sphinxstylestrong{interpolation}}] \leavevmode{[}\sphinxtitleref{str}{]}
estimate flux with data epoch less than than \sphinxstylestrong{tdbin}, or interpolation from GP/fits

\item[{\sphinxstylestrong{fitmodel}}] \leavevmode{[}\sphinxtitleref{int}{]}
if multiple models available, which of them to be used

\end{description}

\item[{Returns}] \leavevmode\begin{description}
\item[{\sphinxstylestrong{flux list}}] \leavevmode{[}\sphinxtitleref{list}{]}
\item[{\sphinxstylestrong{flux error list}}] \leavevmode{[}\sphinxtitleref{list}{]}
\end{description}

\end{description}\end{quote}


\sphinxstrong{See also:}

\begin{description}
\item[{\sphinxcode{\sphinxupquote{sbobject.\_flux\_at}}}] \leavevmode
\end{description}



\end{fulllineitems}



\subsection{sdapy.snerun.snobject.\_mag\_at}
\label{\detokenize{generated/sdapy.snerun.snobject._mag_at:sdapy-snerun-snobject-mag-at}}\label{\detokenize{generated/sdapy.snerun.snobject._mag_at::doc}}\index{\_mag\_at() (sdapy.snerun.snobject method)@\spxentry{\_mag\_at()}\spxextra{sdapy.snerun.snobject method}}

\begin{fulllineitems}
\phantomsection\label{\detokenize{generated/sdapy.snerun.snobject._mag_at:sdapy.snerun.snobject._mag_at}}\pysiglinewithargsret{\sphinxcode{\sphinxupquote{snobject.}}\sphinxbfcode{\sphinxupquote{\_mag\_at}}}{\emph{self}, \emph{filt}, \emph{phase}, \emph{interpolation=None}, \emph{corr\_mkw=False}, \emph{corr\_host=False}, \emph{**kwargs}}{}
estimate apparent magnitude
\begin{quote}\begin{description}
\item[{Parameters}] \leavevmode\begin{description}
\item[{\sphinxstylestrong{filt}}] \leavevmode{[}\sphinxtitleref{str}{]}
filter

\item[{\sphinxstylestrong{phase}}] \leavevmode{[}\sphinxtitleref{float}{]}
rest frame phase (days) relative to t0

\item[{\sphinxstylestrong{tdbin}}] \leavevmode{[}\sphinxtitleref{float}{]}
threshold for binning

\item[{\sphinxstylestrong{interpolation}}] \leavevmode{[}\sphinxtitleref{str}{]}
estimate flux with data epoch less than than \sphinxstylestrong{tdbin}, or interpolation from GP/fits

\item[{\sphinxstylestrong{corr\_mkw}}] \leavevmode{[}\sphinxtitleref{bool}{]}
if correct milky way extinction

\item[{\sphinxstylestrong{corr\_host}}] \leavevmode{[}\sphinxtitleref{bool}{]}
if host galaxy extinction

\item[{\sphinxstylestrong{fitmodel}}] \leavevmode{[}\sphinxtitleref{int}{]}
if multiple models available, which of them to be used

\end{description}

\item[{Returns}] \leavevmode\begin{description}
\item[{\sphinxstylestrong{mag}}] \leavevmode{[}\sphinxtitleref{float}{]}
\item[{\sphinxstylestrong{mag error}}] \leavevmode{[}\sphinxtitleref{float}{]}
\end{description}

\end{description}\end{quote}


\sphinxstrong{See also:}

\begin{description}
\item[{\sphinxcode{\sphinxupquote{sbobject.\_flux\_at}}, \sphinxcode{\sphinxupquote{sbobject.\_mag\_at\_list}}}] \leavevmode
\end{description}



\end{fulllineitems}



\subsection{sdapy.snerun.snobject.\_mag\_at\_list}
\label{\detokenize{generated/sdapy.snerun.snobject._mag_at_list:sdapy-snerun-snobject-mag-at-list}}\label{\detokenize{generated/sdapy.snerun.snobject._mag_at_list::doc}}\index{\_mag\_at\_list() (sdapy.snerun.snobject method)@\spxentry{\_mag\_at\_list()}\spxextra{sdapy.snerun.snobject method}}

\begin{fulllineitems}
\phantomsection\label{\detokenize{generated/sdapy.snerun.snobject._mag_at_list:sdapy.snerun.snobject._mag_at_list}}\pysiglinewithargsret{\sphinxcode{\sphinxupquote{snobject.}}\sphinxbfcode{\sphinxupquote{\_mag\_at\_list}}}{\emph{self}, \emph{filt}, \emph{phaselist}, \emph{interpolation=None}, \emph{corr\_mkw=False}, \emph{corr\_host=False}, \emph{**kwargs}}{}
estimate apparent magnitudes from a list of phases
\begin{quote}\begin{description}
\item[{Parameters}] \leavevmode\begin{description}
\item[{\sphinxstylestrong{filt}}] \leavevmode{[}\sphinxtitleref{str}{]}
filter

\item[{\sphinxstylestrong{phaselist}}] \leavevmode{[}\sphinxtitleref{list}{]}
rest frame phases (days) relative to t0

\item[{\sphinxstylestrong{tdbin}}] \leavevmode{[}\sphinxtitleref{float}{]}
threshold for binning

\item[{\sphinxstylestrong{interpolation}}] \leavevmode{[}\sphinxtitleref{str}{]}
estimate flux with data epoch less than than \sphinxstylestrong{tdbin}, or interpolation from GP/fits

\item[{\sphinxstylestrong{corr\_mkw}}] \leavevmode{[}\sphinxtitleref{bool}{]}
if correct milky way extinction

\item[{\sphinxstylestrong{corr\_host}}] \leavevmode{[}\sphinxtitleref{bool}{]}
if host galaxy extinction

\item[{\sphinxstylestrong{fitmodel}}] \leavevmode{[}\sphinxtitleref{int}{]}
if multiple models available, which of them to be used

\end{description}

\item[{Returns}] \leavevmode\begin{description}
\item[{\sphinxstylestrong{mag list}}] \leavevmode{[}\sphinxtitleref{list}{]}
\item[{\sphinxstylestrong{mag error list}}] \leavevmode{[}\sphinxtitleref{list}{]}
\end{description}

\end{description}\end{quote}


\sphinxstrong{See also:}

\begin{description}
\item[{\sphinxcode{\sphinxupquote{sbobject.\_flux\_at\_list}}, \sphinxcode{\sphinxupquote{sbobject.\_mag\_at}}}] \leavevmode
\end{description}



\end{fulllineitems}



\subsection{sdapy.snerun.snobject.\_absmag\_at}
\label{\detokenize{generated/sdapy.snerun.snobject._absmag_at:sdapy-snerun-snobject-absmag-at}}\label{\detokenize{generated/sdapy.snerun.snobject._absmag_at::doc}}\index{\_absmag\_at() (sdapy.snerun.snobject method)@\spxentry{\_absmag\_at()}\spxextra{sdapy.snerun.snobject method}}

\begin{fulllineitems}
\phantomsection\label{\detokenize{generated/sdapy.snerun.snobject._absmag_at:sdapy.snerun.snobject._absmag_at}}\pysiglinewithargsret{\sphinxcode{\sphinxupquote{snobject.}}\sphinxbfcode{\sphinxupquote{\_absmag\_at}}}{\emph{self}, \emph{filt}, \emph{phase}, \emph{interpolation=None}, \emph{corr\_mkw=False}, \emph{corr\_host=False}, \emph{**kwargs}}{}
estimate absolute magnitude
\begin{quote}\begin{description}
\item[{Parameters}] \leavevmode\begin{description}
\item[{\sphinxstylestrong{filt}}] \leavevmode{[}\sphinxtitleref{str}{]}
filter

\item[{\sphinxstylestrong{phase}}] \leavevmode{[}\sphinxtitleref{float}{]}
rest frame phase (days) relative to t0

\item[{\sphinxstylestrong{tdbin}}] \leavevmode{[}\sphinxtitleref{float}{]}
threshold for binning

\item[{\sphinxstylestrong{interpolation}}] \leavevmode{[}\sphinxtitleref{str}{]}
estimate flux with data epoch less than than \sphinxstylestrong{tdbin}, or interpolation from GP/fits

\item[{\sphinxstylestrong{corr\_mkw}}] \leavevmode{[}\sphinxtitleref{bool}{]}
if correct milky way extinction

\item[{\sphinxstylestrong{corr\_host}}] \leavevmode{[}\sphinxtitleref{bool}{]}
if host galaxy extinction

\item[{\sphinxstylestrong{fitmodel}}] \leavevmode{[}\sphinxtitleref{int}{]}
if multiple models available, which of them to be used

\end{description}

\item[{Returns}] \leavevmode\begin{description}
\item[{\sphinxstylestrong{mag}}] \leavevmode{[}\sphinxtitleref{float}{]}
\item[{\sphinxstylestrong{mag error}}] \leavevmode{[}\sphinxtitleref{float}{]}
\end{description}

\end{description}\end{quote}


\sphinxstrong{See also:}

\begin{description}
\item[{\sphinxcode{\sphinxupquote{sbobject.\_flux\_at}}, \sphinxcode{\sphinxupquote{sbobject.\_mag\_at}}}] \leavevmode
\end{description}



\end{fulllineitems}



\subsection{sdapy.snerun.snobject.\_absmag\_at\_list}
\label{\detokenize{generated/sdapy.snerun.snobject._absmag_at_list:sdapy-snerun-snobject-absmag-at-list}}\label{\detokenize{generated/sdapy.snerun.snobject._absmag_at_list::doc}}\index{\_absmag\_at\_list() (sdapy.snerun.snobject method)@\spxentry{\_absmag\_at\_list()}\spxextra{sdapy.snerun.snobject method}}

\begin{fulllineitems}
\phantomsection\label{\detokenize{generated/sdapy.snerun.snobject._absmag_at_list:sdapy.snerun.snobject._absmag_at_list}}\pysiglinewithargsret{\sphinxcode{\sphinxupquote{snobject.}}\sphinxbfcode{\sphinxupquote{\_absmag\_at\_list}}}{\emph{self}, \emph{filt}, \emph{phaselist}, \emph{interpolation=None}, \emph{corr\_mkw=False}, \emph{corr\_host=False}, \emph{**kwargs}}{}
estimate absolute magnitudes from a list of phases
\begin{quote}\begin{description}
\item[{Parameters}] \leavevmode\begin{description}
\item[{\sphinxstylestrong{filt}}] \leavevmode{[}\sphinxtitleref{str}{]}
filter

\item[{\sphinxstylestrong{phaselist}}] \leavevmode{[}\sphinxtitleref{list}{]}
rest frame phases (days) relative to t0

\item[{\sphinxstylestrong{tdbin}}] \leavevmode{[}\sphinxtitleref{float}{]}
threshold for binning

\item[{\sphinxstylestrong{interpolation}}] \leavevmode{[}\sphinxtitleref{str}{]}
estimate flux with data epoch less than than \sphinxstylestrong{tdbin}, or interpolation from GP/fits

\item[{\sphinxstylestrong{corr\_mkw}}] \leavevmode{[}\sphinxtitleref{bool}{]}
if correct milky way extinction

\item[{\sphinxstylestrong{corr\_host}}] \leavevmode{[}\sphinxtitleref{bool}{]}
if host galaxy extinction

\item[{\sphinxstylestrong{fitmodel}}] \leavevmode{[}\sphinxtitleref{int}{]}
if multiple models available, which of them to be used

\end{description}

\item[{Returns}] \leavevmode\begin{description}
\item[{\sphinxstylestrong{mag list}}] \leavevmode{[}\sphinxtitleref{list}{]}
\item[{\sphinxstylestrong{mag error list}}] \leavevmode{[}\sphinxtitleref{list}{]}
\end{description}

\end{description}\end{quote}


\sphinxstrong{See also:}

\begin{description}
\item[{\sphinxcode{\sphinxupquote{sbobject.\_flux\_at}}, \sphinxcode{\sphinxupquote{sbobject.\_absmag\_at}}, \sphinxcode{\sphinxupquote{sbobject.\_mag\_at}}}] \leavevmode
\end{description}



\end{fulllineitems}



\subsection{sdapy.snerun.snobject.\_color\_at}
\label{\detokenize{generated/sdapy.snerun.snobject._color_at:sdapy-snerun-snobject-color-at}}\label{\detokenize{generated/sdapy.snerun.snobject._color_at::doc}}\index{\_color\_at() (sdapy.snerun.snobject method)@\spxentry{\_color\_at()}\spxextra{sdapy.snerun.snobject method}}

\begin{fulllineitems}
\phantomsection\label{\detokenize{generated/sdapy.snerun.snobject._color_at:sdapy.snerun.snobject._color_at}}\pysiglinewithargsret{\sphinxcode{\sphinxupquote{snobject.}}\sphinxbfcode{\sphinxupquote{\_color\_at}}}{\emph{self}, \emph{filt1}, \emph{filt2}, \emph{phase}, \emph{interpolation=None}, \emph{corr\_mkw=False}, \emph{corr\_host=False}, \emph{**kwargs}}{}
estimate colour
\begin{quote}\begin{description}
\item[{Parameters}] \leavevmode\begin{description}
\item[{\sphinxstylestrong{filt1}}] \leavevmode{[}\sphinxtitleref{str}{]}
filter1

\item[{\sphinxstylestrong{filt2}}] \leavevmode{[}\sphinxtitleref{str}{]}
filter2

\item[{\sphinxstylestrong{phase}}] \leavevmode{[}\sphinxtitleref{float}{]}
rest frame phase (days) relative to t0

\item[{\sphinxstylestrong{tdbin}}] \leavevmode{[}\sphinxtitleref{float}{]}
threshold for binning

\item[{\sphinxstylestrong{interpolation}}] \leavevmode{[}\sphinxtitleref{str}{]}
estimate flux with data epoch less than than \sphinxstylestrong{tdbin}, or interpolation from GP/fits

\item[{\sphinxstylestrong{corr\_mkw}}] \leavevmode{[}\sphinxtitleref{bool}{]}
if correct milky way extinction

\item[{\sphinxstylestrong{corr\_host}}] \leavevmode{[}\sphinxtitleref{bool}{]}
if host galaxy extinction

\item[{\sphinxstylestrong{fitmodel}}] \leavevmode{[}\sphinxtitleref{int}{]}
if multiple models available, which of them to be used

\end{description}

\item[{Returns}] \leavevmode\begin{description}
\item[{\sphinxstylestrong{color}}] \leavevmode{[}\sphinxtitleref{float}{]}
\item[{\sphinxstylestrong{color error}}] \leavevmode{[}\sphinxtitleref{float}{]}
\end{description}

\end{description}\end{quote}


\sphinxstrong{See also:}

\begin{description}
\item[{\sphinxcode{\sphinxupquote{sbobject.\_flux\_at}}, \sphinxcode{\sphinxupquote{sbobject.\_mag\_at}}}] \leavevmode
\end{description}



\end{fulllineitems}



\subsection{sdapy.snerun.snobject.\_rate\_at}
\label{\detokenize{generated/sdapy.snerun.snobject._rate_at:sdapy-snerun-snobject-rate-at}}\label{\detokenize{generated/sdapy.snerun.snobject._rate_at::doc}}\index{\_rate\_at() (sdapy.snerun.snobject method)@\spxentry{\_rate\_at()}\spxextra{sdapy.snerun.snobject method}}

\begin{fulllineitems}
\phantomsection\label{\detokenize{generated/sdapy.snerun.snobject._rate_at:sdapy.snerun.snobject._rate_at}}\pysiglinewithargsret{\sphinxcode{\sphinxupquote{snobject.}}\sphinxbfcode{\sphinxupquote{\_rate\_at}}}{\emph{self}, \emph{filt}, \emph{phase1}, \emph{phase2}, \emph{interpolation=None}, \emph{**kwargs}}{}
estimate magnitude rate
\begin{quote}\begin{description}
\item[{Parameters}] \leavevmode\begin{description}
\item[{\sphinxstylestrong{filt}}] \leavevmode{[}\sphinxtitleref{str}{]}
filter

\item[{\sphinxstylestrong{phase1}}] \leavevmode{[}\sphinxtitleref{float}{]}
rest frame phase 1 (days) relative to t0

\item[{\sphinxstylestrong{phase2}}] \leavevmode{[}\sphinxtitleref{float}{]}
rest frame phase 2 (days) relative to t0

\item[{\sphinxstylestrong{tdbin}}] \leavevmode{[}\sphinxtitleref{float}{]}
threshold for binning

\item[{\sphinxstylestrong{interpolation}}] \leavevmode{[}\sphinxtitleref{str}{]}
estimate flux with data epoch less than than \sphinxstylestrong{tdbin}, or interpolation from GP/fits

\item[{\sphinxstylestrong{fitmodel}}] \leavevmode{[}\sphinxtitleref{int}{]}
if multiple models available, which of them to be used

\end{description}

\item[{Returns}] \leavevmode\begin{description}
\item[{\sphinxstylestrong{rate}}] \leavevmode{[}\sphinxtitleref{float}{]}
\item[{\sphinxstylestrong{rate error}}] \leavevmode{[}\sphinxtitleref{float}{]}
\end{description}

\end{description}\end{quote}


\sphinxstrong{See also:}

\begin{description}
\item[{\sphinxcode{\sphinxupquote{sbobject.\_flux\_at}}, \sphinxcode{\sphinxupquote{sbobject.\_mag\_at}}}] \leavevmode
\end{description}



\end{fulllineitems}



\subsection{sdapy.snerun.snobject.\_nepochs}
\label{\detokenize{generated/sdapy.snerun.snobject._nepochs:sdapy-snerun-snobject-nepochs}}\label{\detokenize{generated/sdapy.snerun.snobject._nepochs::doc}}\index{\_nepochs() (sdapy.snerun.snobject method)@\spxentry{\_nepochs()}\spxextra{sdapy.snerun.snobject method}}

\begin{fulllineitems}
\phantomsection\label{\detokenize{generated/sdapy.snerun.snobject._nepochs:sdapy.snerun.snobject._nepochs}}\pysiglinewithargsret{\sphinxcode{\sphinxupquote{snobject.}}\sphinxbfcode{\sphinxupquote{\_nepochs}}}{\emph{self}, \emph{**kwargs}}{}
how many epochs of photometry in either band.
\begin{quote}\begin{description}
\item[{Parameters}] \leavevmode\begin{description}
\item[{\sphinxstylestrong{plot\_bands}}] \leavevmode{[}\sphinxtitleref{list}{]}
photometric filters

\end{description}

\end{description}\end{quote}

\end{fulllineitems}



\subsection{sdapy.snerun.snobject.\_ncolors}
\label{\detokenize{generated/sdapy.snerun.snobject._ncolors:sdapy-snerun-snobject-ncolors}}\label{\detokenize{generated/sdapy.snerun.snobject._ncolors::doc}}\index{\_ncolors() (sdapy.snerun.snobject method)@\spxentry{\_ncolors()}\spxextra{sdapy.snerun.snobject method}}

\begin{fulllineitems}
\phantomsection\label{\detokenize{generated/sdapy.snerun.snobject._ncolors:sdapy.snerun.snobject._ncolors}}\pysiglinewithargsret{\sphinxcode{\sphinxupquote{snobject.}}\sphinxbfcode{\sphinxupquote{\_ncolors}}}{\emph{self}, \emph{**kwargs}}{}
how many color epochs
\begin{quote}\begin{description}
\item[{Parameters}] \leavevmode\begin{description}
\item[{\sphinxstylestrong{plot\_bands}}] \leavevmode{[}\sphinxtitleref{list}{]}
photometric filters

\item[{\sphinxstylestrong{tdbin}}] \leavevmode{[}\sphinxtitleref{float}{]}
threshold for binning

\end{description}

\end{description}\end{quote}

\end{fulllineitems}



\subsection{sdapy.snerun.snobject.\_peak\_accuracy}
\label{\detokenize{generated/sdapy.snerun.snobject._peak_accuracy:sdapy-snerun-snobject-peak-accuracy}}\label{\detokenize{generated/sdapy.snerun.snobject._peak_accuracy::doc}}\index{\_peak\_accuracy() (sdapy.snerun.snobject method)@\spxentry{\_peak\_accuracy()}\spxextra{sdapy.snerun.snobject method}}

\begin{fulllineitems}
\phantomsection\label{\detokenize{generated/sdapy.snerun.snobject._peak_accuracy:sdapy.snerun.snobject._peak_accuracy}}\pysiglinewithargsret{\sphinxcode{\sphinxupquote{snobject.}}\sphinxbfcode{\sphinxupquote{\_peak\_accuracy}}}{\emph{self}, \emph{within=3}, \emph{**kwargs}}{}
how accurate a peak can be determined: how many photometric points available within a range to the peak
\begin{quote}\begin{description}
\item[{Parameters}] \leavevmode\begin{description}
\item[{\sphinxstylestrong{within}}] \leavevmode{[}\sphinxtitleref{int}{]}
the distance in days

\item[{\sphinxstylestrong{plot\_bands}}] \leavevmode{[}\sphinxtitleref{list}{]}
photometric filters

\end{description}

\end{description}\end{quote}

\end{fulllineitems}



\subsection{sdapy.snerun.snobject.\_ax}
\label{\detokenize{generated/sdapy.snerun.snobject._ax:sdapy-snerun-snobject-ax}}\label{\detokenize{generated/sdapy.snerun.snobject._ax::doc}}\index{\_ax() (sdapy.snerun.snobject method)@\spxentry{\_ax()}\spxextra{sdapy.snerun.snobject method}}

\begin{fulllineitems}
\phantomsection\label{\detokenize{generated/sdapy.snerun.snobject._ax:sdapy.snerun.snobject._ax}}\pysiglinewithargsret{\sphinxcode{\sphinxupquote{snobject.}}\sphinxbfcode{\sphinxupquote{\_ax}}}{\emph{self}, \emph{to\_t0=False}, \emph{show\_title=True}, \emph{show\_legend=True}, \emph{ylabel\_2right=False}, \emph{x0=2458000}, \emph{source=None}, \emph{showfilt=None}, \emph{show\_points=True}, \emph{show\_fit=True}, \emph{show\_gp=True}, \emph{show\_texp=True}, \emph{show\_fit\_error=True}, \emph{**kwargs}}{}~
\end{fulllineitems}



\subsection{sdapy.snerun.snobject.\_ax1}
\label{\detokenize{generated/sdapy.snerun.snobject._ax1:sdapy-snerun-snobject-ax1}}\label{\detokenize{generated/sdapy.snerun.snobject._ax1::doc}}\index{\_ax1() (sdapy.snerun.snobject method)@\spxentry{\_ax1()}\spxextra{sdapy.snerun.snobject method}}

\begin{fulllineitems}
\phantomsection\label{\detokenize{generated/sdapy.snerun.snobject._ax1:sdapy.snerun.snobject._ax1}}\pysiglinewithargsret{\sphinxcode{\sphinxupquote{snobject.}}\sphinxbfcode{\sphinxupquote{\_ax1}}}{\emph{self}, \emph{stype=\textquotesingle{}flat\textquotesingle{}}, \emph{show\_title=False}, \emph{show\_legend=False}, \emph{element=None}, \emph{source=None}, \emph{show\_text=True}, \emph{**kwargs}}{}
Spectra plot
\begin{quote}\begin{description}
\item[{Parameters}] \leavevmode\begin{description}
\item[{\sphinxstylestrong{stype}}] \leavevmode{[}\sphinxtitleref{str}{]}
spectral data type, options: ‘original’, ‘rest’, ‘bin’, ‘continuum’, ‘flat’

\item[{\sphinxstylestrong{show\_title}}] \leavevmode{[}\sphinxtitleref{bool}{]}
show subplot title or not

\item[{\sphinxstylestrong{show\_legend}}] \leavevmode{[}\sphinxtitleref{bool}{]}
show subplot legend or not

\end{description}

\end{description}\end{quote}


\sphinxstrong{See also:}

\begin{description}
\item[{{\hyperref[\detokenize{generated/sdapy.snerun.snelist.__init__:sdapy.snerun.snelist.__init__}]{\sphinxcrossref{\sphinxcode{\sphinxupquote{snelist.\_\_init\_\_}}}}}}] \leavevmode
\end{description}



\end{fulllineitems}



\subsection{sdapy.snerun.snobject.\_ax2}
\label{\detokenize{generated/sdapy.snerun.snobject._ax2:sdapy-snerun-snobject-ax2}}\label{\detokenize{generated/sdapy.snerun.snobject._ax2::doc}}\index{\_ax2() (sdapy.snerun.snobject method)@\spxentry{\_ax2()}\spxextra{sdapy.snerun.snobject method}}

\begin{fulllineitems}
\phantomsection\label{\detokenize{generated/sdapy.snerun.snobject._ax2:sdapy.snerun.snobject._ax2}}\pysiglinewithargsret{\sphinxcode{\sphinxupquote{snobject.}}\sphinxbfcode{\sphinxupquote{\_ax2}}}{\emph{self}, \emph{show\_title=False}, \emph{show\_legend=False}, \emph{ylabel\_2right=False}, \emph{corr\_mkw=False}, \emph{corr\_host=False}, \emph{source=None}, \emph{showfilt=None}, \emph{show\_texp=True}, \emph{show\_fit=True}, \emph{show\_gp=True}, \emph{show\_fit\_error=False}, \emph{**kwargs}}{}~
\end{fulllineitems}



\subsection{sdapy.snerun.snobject.\_ax3}
\label{\detokenize{generated/sdapy.snerun.snobject._ax3:sdapy-snerun-snobject-ax3}}\label{\detokenize{generated/sdapy.snerun.snobject._ax3::doc}}\index{\_ax3() (sdapy.snerun.snobject method)@\spxentry{\_ax3()}\spxextra{sdapy.snerun.snobject method}}

\begin{fulllineitems}
\phantomsection\label{\detokenize{generated/sdapy.snerun.snobject._ax3:sdapy.snerun.snobject._ax3}}\pysiglinewithargsret{\sphinxcode{\sphinxupquote{snobject.}}\sphinxbfcode{\sphinxupquote{\_ax3}}}{\emph{self, show\_title=False, show\_legend=False, ylabel\_2right=True, corr\_mkw=False, corr\_host=False, source=None, show\_texp=True, show\_interp={[}\textquotesingle{}bin\textquotesingle{}, \textquotesingle{}gp\textquotesingle{}, \textquotesingle{}fit\textquotesingle{}{]}, **kwargs}}{}~
\end{fulllineitems}



\subsection{sdapy.snerun.snobject.\_ax4}
\label{\detokenize{generated/sdapy.snerun.snobject._ax4:sdapy-snerun-snobject-ax4}}\label{\detokenize{generated/sdapy.snerun.snobject._ax4::doc}}\index{\_ax4() (sdapy.snerun.snobject method)@\spxentry{\_ax4()}\spxextra{sdapy.snerun.snobject method}}

\begin{fulllineitems}
\phantomsection\label{\detokenize{generated/sdapy.snerun.snobject._ax4:sdapy.snerun.snobject._ax4}}\pysiglinewithargsret{\sphinxcode{\sphinxupquote{snobject.}}\sphinxbfcode{\sphinxupquote{\_ax4}}}{\emph{self, show\_title=False, show\_legend=False, ylabel\_2right=False, logscale=True, source=None, show\_texp=True, show\_points=True, show\_fit=True, show\_interp={[}\textquotesingle{}bin\textquotesingle{}, \textquotesingle{}fit\textquotesingle{}, \textquotesingle{}gp\textquotesingle{}{]}, **kwargs}}{}~
\end{fulllineitems}



\subsection{sdapy.snerun.snobject.show\_corner}
\label{\detokenize{generated/sdapy.snerun.snobject.show_corner:sdapy-snerun-snobject-show-corner}}\label{\detokenize{generated/sdapy.snerun.snobject.show_corner::doc}}\index{show\_corner() (sdapy.snerun.snobject method)@\spxentry{show\_corner()}\spxextra{sdapy.snerun.snobject method}}

\begin{fulllineitems}
\phantomsection\label{\detokenize{generated/sdapy.snerun.snobject.show_corner:sdapy.snerun.snobject.show_corner}}\pysiglinewithargsret{\sphinxcode{\sphinxupquote{snobject.}}\sphinxbfcode{\sphinxupquote{show\_corner}}}{\emph{self}, \emph{ax}, \emph{index=0}, \emph{gp=False}, \emph{engine=None}, \emph{model=None}, \emph{source=None}, \emph{filts=None}, \emph{**kwargs}}{}~
\end{fulllineitems}



\subsection{sdapy.snerun.snobject.get\_model}
\label{\detokenize{generated/sdapy.snerun.snobject.get_model:sdapy-snerun-snobject-get-model}}\label{\detokenize{generated/sdapy.snerun.snobject.get_model::doc}}\index{get\_model() (sdapy.snerun.snobject method)@\spxentry{get\_model()}\spxextra{sdapy.snerun.snobject method}}

\begin{fulllineitems}
\phantomsection\label{\detokenize{generated/sdapy.snerun.snobject.get_model:sdapy.snerun.snobject.get_model}}\pysiglinewithargsret{\sphinxcode{\sphinxupquote{snobject.}}\sphinxbfcode{\sphinxupquote{get\_model}}}{\emph{self}, \emph{engine=None}, \emph{model=None}, \emph{source=None}}{}~
\end{fulllineitems}



\subsection{sdapy.snerun.snobject.savefig}
\label{\detokenize{generated/sdapy.snerun.snobject.savefig:sdapy-snerun-snobject-savefig}}\label{\detokenize{generated/sdapy.snerun.snobject.savefig::doc}}\index{savefig() (sdapy.snerun.snobject method)@\spxentry{savefig()}\spxextra{sdapy.snerun.snobject method}}

\begin{fulllineitems}
\phantomsection\label{\detokenize{generated/sdapy.snerun.snobject.savefig:sdapy.snerun.snobject.savefig}}\pysiglinewithargsret{\sphinxcode{\sphinxupquote{snobject.}}\sphinxbfcode{\sphinxupquote{savefig}}}{\emph{self}, \emph{**kwargs}}{}~
\end{fulllineitems}



\subsection{sdapy.snerun.snobject.showfig}
\label{\detokenize{generated/sdapy.snerun.snobject.showfig:sdapy-snerun-snobject-showfig}}\label{\detokenize{generated/sdapy.snerun.snobject.showfig::doc}}\index{showfig() (sdapy.snerun.snobject method)@\spxentry{showfig()}\spxextra{sdapy.snerun.snobject method}}

\begin{fulllineitems}
\phantomsection\label{\detokenize{generated/sdapy.snerun.snobject.showfig:sdapy.snerun.snobject.showfig}}\pysiglinewithargsret{\sphinxcode{\sphinxupquote{snobject.}}\sphinxbfcode{\sphinxupquote{showfig}}}{\emph{self}, \emph{ax=None}, \emph{**kwargs}}{}~
\end{fulllineitems}



\subsection{sdapy.snerun.snobject.dm\_error}
\label{\detokenize{generated/sdapy.snerun.snobject.dm_error:sdapy-snerun-snobject-dm-error}}\label{\detokenize{generated/sdapy.snerun.snobject.dm_error::doc}}\index{dm\_error() (sdapy.snerun.snobject method)@\spxentry{dm\_error()}\spxextra{sdapy.snerun.snobject method}}

\begin{fulllineitems}
\phantomsection\label{\detokenize{generated/sdapy.snerun.snobject.dm_error:sdapy.snerun.snobject.dm_error}}\pysiglinewithargsret{\sphinxcode{\sphinxupquote{snobject.}}\sphinxbfcode{\sphinxupquote{dm\_error}}}{\emph{self}, \emph{filt}}{}~
\end{fulllineitems}



\subsection{sdapy.snerun.snobject.read\_c10}
\label{\detokenize{generated/sdapy.snerun.snobject.read_c10:sdapy-snerun-snobject-read-c10}}\label{\detokenize{generated/sdapy.snerun.snobject.read_c10::doc}}\index{read\_c10() (sdapy.snerun.snobject static method)@\spxentry{read\_c10()}\spxextra{sdapy.snerun.snobject static method}}

\begin{fulllineitems}
\phantomsection\label{\detokenize{generated/sdapy.snerun.snobject.read_c10:sdapy.snerun.snobject.read_c10}}\pysiglinewithargsret{\sphinxbfcode{\sphinxupquote{static }}\sphinxcode{\sphinxupquote{snobject.}}\sphinxbfcode{\sphinxupquote{read\_c10}}}{\emph{filename=\textquotesingle{}c10\_template.txt\textquotesingle{}}}{}~
\end{fulllineitems}



\subsection{sdapy.snerun.snobject.bin\_df}
\label{\detokenize{generated/sdapy.snerun.snobject.bin_df:sdapy-snerun-snobject-bin-df}}\label{\detokenize{generated/sdapy.snerun.snobject.bin_df::doc}}\index{bin\_df() (sdapy.snerun.snobject static method)@\spxentry{bin\_df()}\spxextra{sdapy.snerun.snobject static method}}

\begin{fulllineitems}
\phantomsection\label{\detokenize{generated/sdapy.snerun.snobject.bin_df:sdapy.snerun.snobject.bin_df}}\pysiglinewithargsret{\sphinxbfcode{\sphinxupquote{static }}\sphinxcode{\sphinxupquote{snobject.}}\sphinxbfcode{\sphinxupquote{bin\_df}}}{\emph{df}, \emph{deltah=1.0}, \emph{xkey=\textquotesingle{}jdobs\textquotesingle{}}, \emph{fkey=\textquotesingle{}filter\textquotesingle{}}}{}~
\end{fulllineitems}



\section{\sphinxstyleliteralintitle{\sphinxupquote{Fitting and Gaussian Process}} \textendash{} Fitting packages}
\label{\detokenize{fitters:fitting-and-gaussian-process-fitting-packages}}\label{\detokenize{fitters:fitters}}\label{\detokenize{fitters::doc}}

\subsection{Gaussian Process}
\label{\detokenize{fitters:gaussian-process}}

\begin{savenotes}\sphinxatlongtablestart\begin{longtable}[c]{\X{1}{2}\X{1}{2}}
\hline

\endfirsthead

\multicolumn{2}{c}%
{\makebox[0pt]{\sphinxtablecontinued{\tablename\ \thetable{} \textendash{} continued from previous page}}}\\
\hline

\endhead

\hline
\multicolumn{2}{r}{\makebox[0pt][r]{\sphinxtablecontinued{Continued on next page}}}\\
\endfoot

\endlastfoot

{\hyperref[\detokenize{generated/sdapy.gaussian_process.fit_gp:sdapy.gaussian_process.fit_gp}]{\sphinxcrossref{\sphinxcode{\sphinxupquote{fit\_gp}}}}}(x\_data, y\_data{[}, yerr\_data, filters{]})
&
Fits data with gaussian process.
\\
\hline
\end{longtable}\sphinxatlongtableend\end{savenotes}


\subsubsection{sdapy.gaussian\_process.fit\_gp}
\label{\detokenize{generated/sdapy.gaussian_process.fit_gp:sdapy-gaussian-process-fit-gp}}\label{\detokenize{generated/sdapy.gaussian_process.fit_gp::doc}}\index{fit\_gp (class in sdapy.gaussian\_process)@\spxentry{fit\_gp}\spxextra{class in sdapy.gaussian\_process}}

\begin{fulllineitems}
\phantomsection\label{\detokenize{generated/sdapy.gaussian_process.fit_gp:sdapy.gaussian_process.fit_gp}}\pysiglinewithargsret{\sphinxbfcode{\sphinxupquote{class }}\sphinxcode{\sphinxupquote{sdapy.gaussian\_process.}}\sphinxbfcode{\sphinxupquote{fit\_gp}}}{\emph{x\_data}, \emph{y\_data}, \emph{yerr\_data=1e\sphinxhyphen{}08}, \emph{filters=None}}{}
Fits data with gaussian process.

The package ‘george’ is used for the gaussian process fit.
\begin{quote}\begin{description}
\item[{Parameters}] \leavevmode\begin{description}
\item[{\sphinxstylestrong{x\_data}}] \leavevmode{[}array{]}
Independent values.

\item[{\sphinxstylestrong{y\_data}}] \leavevmode{[}array{]}
Dependent values.

\item[{\sphinxstylestrong{yerr\_data}}] \leavevmode{[}array, int{]}
Dependent value errors.

\item[{\sphinxstylestrong{filters}}] \leavevmode{[}array{]}
If filters available, will convolve wavelengths to x\_data to train gaussian process.

\item[{\sphinxstylestrong{kernel}}] \leavevmode{[}str, default ‘matern52’{]}
Kernel to be used with the gaussian process. 
Possible choices are: ‘matern52’, ‘matern32’, ‘squaredexp’.

\item[{\sphinxstylestrong{fix\_scale}}] \leavevmode{[}bool{]}
If fix default gaussian process param

\item[{\sphinxstylestrong{gp\_mean: str, default ‘mean’}}] \leavevmode
Mean y\_data function.
Possible choices are: ‘mean’, ‘gaussian’, ‘bazin’, ‘villar’.

\item[{\sphinxstylestrong{opt\_routine}}] \leavevmode{[}str,{]}
Which technic to be used to realize optimization.
Possible choices are: ‘minimize’, ‘mcmc’, ‘leastsq’.

\item[{\sphinxstylestrong{nwalkers}}] \leavevmode{[}int{]}
if mcmc adopted, set walker number

\item[{\sphinxstylestrong{nsteps:  int}}] \leavevmode
if mcmc adopted, set step

\item[{\sphinxstylestrong{nsteps\_burnin: int}}] \leavevmode
if mcmc adopted, set burnin step

\item[{\sphinxstylestrong{clobber: bool}}] \leavevmode
if gp already done, if redo it or not

\end{description}

\item[{Returns}] \leavevmode\begin{description}
\item[{Returns the interpolated independent and dependent values with the 1\sphinxhyphen{}sigma standard deviation.}] \leavevmode
\item[{Examples}] \leavevmode
\end{description}
\begin{quote}
\end{quote}
\begin{description}
\item[{gp = fit\_gp(jd, flux, fluxerr, central\_wavelength)}] \leavevmode
\item[{gp.train(gp\_mean=’bazin’, opt\_routine = ‘mcmc’)}] \leavevmode
\item[{gp\_jd, gp\_flux, gp\_flux\_errors, gp\_ws = gp.predict()}] \leavevmode
\item[{gp.save\_corner(saveplotas=’tmp.png’)}] \leavevmode
\end{description}

\end{description}\end{quote}
\subsubsection*{Methods}


\begin{savenotes}\sphinxatlongtablestart\begin{longtable}[c]{\X{1}{2}\X{1}{2}}
\hline

\endfirsthead

\multicolumn{2}{c}%
{\makebox[0pt]{\sphinxtablecontinued{\tablename\ \thetable{} \textendash{} continued from previous page}}}\\
\hline

\endhead

\hline
\multicolumn{2}{r}{\makebox[0pt][r]{\sphinxtablecontinued{Continued on next page}}}\\
\endfoot

\endlastfoot

\sphinxcode{\sphinxupquote{predict}}(self{[}, x\_pred, step, clobber, returnv{]})
&
output GP products
\\
\hline
\sphinxcode{\sphinxupquote{save\_corner}}(self, figpath{[}, datadir, …{]})
&
generate corner plots
\\
\hline
\end{longtable}\sphinxatlongtableend\end{savenotes}


\begin{savenotes}\sphinxattablestart
\centering
\begin{tabulary}{\linewidth}[t]{|T|T|}
\hline

\sphinxstylestrong{parse\_filters}
&\\
\hline
\sphinxstylestrong{parse\_wavelength}
&\\
\hline
\sphinxstylestrong{set\_peak}
&\\
\hline
\sphinxstylestrong{train}
&\\
\hline
\end{tabulary}
\par
\sphinxattableend\end{savenotes}
\index{\_\_init\_\_() (sdapy.gaussian\_process.fit\_gp method)@\spxentry{\_\_init\_\_()}\spxextra{sdapy.gaussian\_process.fit\_gp method}}

\begin{fulllineitems}
\phantomsection\label{\detokenize{generated/sdapy.gaussian_process.fit_gp:sdapy.gaussian_process.fit_gp.__init__}}\pysiglinewithargsret{\sphinxbfcode{\sphinxupquote{\_\_init\_\_}}}{\emph{self}, \emph{x\_data}, \emph{y\_data}, \emph{yerr\_data=1e\sphinxhyphen{}08}, \emph{filters=None}}{}
Initialize self.  See help(type(self)) for accurate signature.

\end{fulllineitems}

\subsubsection*{Methods}


\begin{savenotes}\sphinxatlongtablestart\begin{longtable}[c]{\X{1}{2}\X{1}{2}}
\hline

\endfirsthead

\multicolumn{2}{c}%
{\makebox[0pt]{\sphinxtablecontinued{\tablename\ \thetable{} \textendash{} continued from previous page}}}\\
\hline

\endhead

\hline
\multicolumn{2}{r}{\makebox[0pt][r]{\sphinxtablecontinued{Continued on next page}}}\\
\endfoot

\endlastfoot

{\hyperref[\detokenize{generated/sdapy.gaussian_process.fit_gp:sdapy.gaussian_process.fit_gp.__init__}]{\sphinxcrossref{\sphinxcode{\sphinxupquote{\_\_init\_\_}}}}}(self, x\_data, y\_data{[}, yerr\_data, …{]})
&
Initialize self.
\\
\hline
\sphinxcode{\sphinxupquote{parse\_filters}}(self)
&

\\
\hline
\sphinxcode{\sphinxupquote{parse\_wavelength}}(w)
&

\\
\hline
\sphinxcode{\sphinxupquote{predict}}(self{[}, x\_pred, step, clobber, returnv{]})
&
output GP products
\\
\hline
\sphinxcode{\sphinxupquote{save\_corner}}(self, figpath{[}, datadir, …{]})
&
generate corner plots
\\
\hline
\sphinxcode{\sphinxupquote{set\_peak}}(self{[}, clobber{]})
&

\\
\hline
\sphinxcode{\sphinxupquote{train}}(self{[}, kernel, fix\_scale, gp\_mean, …{]})
&

\\
\hline
\end{longtable}\sphinxatlongtableend\end{savenotes}

\end{fulllineitems}



\subsection{Model Fitters}
\label{\detokenize{fitters:model-fitters}}

\begin{savenotes}\sphinxatlongtablestart\begin{longtable}[c]{\X{1}{2}\X{1}{2}}
\hline

\endfirsthead

\multicolumn{2}{c}%
{\makebox[0pt]{\sphinxtablecontinued{\tablename\ \thetable{} \textendash{} continued from previous page}}}\\
\hline

\endhead

\hline
\multicolumn{2}{r}{\makebox[0pt][r]{\sphinxtablecontinued{Continued on next page}}}\\
\endfoot

\endlastfoot

{\hyperref[\detokenize{generated/sdapy.model_fitters.fit_model:sdapy.model_fitters.fit_model}]{\sphinxcrossref{\sphinxcode{\sphinxupquote{fit\_model}}}}}(x\_data, y\_data, yerr\_data{[}, filters{]})
&
Fits data with power law.
\\
\hline
{\hyperref[\detokenize{generated/sdapy.model_fitters.get_engine:sdapy.model_fitters.get_engine}]{\sphinxcrossref{\sphinxcode{\sphinxupquote{get\_engine}}}}}()
&

\\
\hline
{\hyperref[\detokenize{generated/sdapy.model_fitters.get_model:sdapy.model_fitters.get_model}]{\sphinxcrossref{\sphinxcode{\sphinxupquote{get\_model}}}}}({[}which\_engine, with\_alias{]})
&

\\
\hline
{\hyperref[\detokenize{generated/sdapy.model_fitters.get_pars:sdapy.model_fitters.get_pars}]{\sphinxcrossref{\sphinxcode{\sphinxupquote{get\_pars}}}}}(which{[}, with\_alias{]})
&

\\
\hline
\end{longtable}\sphinxatlongtableend\end{savenotes}


\subsubsection{sdapy.model\_fitters.fit\_model}
\label{\detokenize{generated/sdapy.model_fitters.fit_model:sdapy-model-fitters-fit-model}}\label{\detokenize{generated/sdapy.model_fitters.fit_model::doc}}\index{fit\_model (class in sdapy.model\_fitters)@\spxentry{fit\_model}\spxextra{class in sdapy.model\_fitters}}

\begin{fulllineitems}
\phantomsection\label{\detokenize{generated/sdapy.model_fitters.fit_model:sdapy.model_fitters.fit_model}}\pysiglinewithargsret{\sphinxbfcode{\sphinxupquote{class }}\sphinxcode{\sphinxupquote{sdapy.model\_fitters.}}\sphinxbfcode{\sphinxupquote{fit\_model}}}{\emph{x\_data}, \emph{y\_data}, \emph{yerr\_data}, \emph{filters=None}}{}
Fits data with power law.
power law parts were from:
\begin{quote}

Miller et al, \sphinxhref{https://ui.adsabs.harvard.edu/abs/2020ApJ...902...47M/abstract}{https://ui.adsabs.harvard.edu/abs/2020ApJ…902…47M/abstract}
\end{quote}
\begin{quote}\begin{description}
\item[{Parameters}] \leavevmode\begin{description}
\item[{\sphinxstylestrong{x\_data}}] \leavevmode{[}array{]}
Independent values, e.g. (rest frame) phase relative to peak

\item[{\sphinxstylestrong{y\_data}}] \leavevmode{[}array{]}
Dependent values, e.g. fluxes

\item[{\sphinxstylestrong{yerr\_data}}] \leavevmode{[}array, int{]}
Dependent value errors, e.g. flux errors

\item[{\sphinxstylestrong{filters}}] \leavevmode{[}array{]}
If filters available, will fit for each band simultaneously.

\item[{\sphinxstylestrong{opt\_routine}}] \leavevmode{[}str,{]}
Which technic to be used to realize optimization.
Possible choices are: ‘mcmc’, ‘minimize’, ‘leastsq’.

\item[{\sphinxstylestrong{nwalkers}}] \leavevmode{[}int{]}
if mcmc adopted, set walker number

\item[{\sphinxstylestrong{ncores}}] \leavevmode{[}int{]}
core numbers to be used for multi processing

\item[{\sphinxstylestrong{nsteps:  int}}] \leavevmode
if mcmc adopted, set step

\item[{\sphinxstylestrong{clobber: bool}}] \leavevmode
if power law already done, if redo it or not

\item[{\sphinxstylestrong{verbose: bool}}] \leavevmode
show progress or not

\end{description}

\item[{Returns}] \leavevmode\begin{description}
\item[{Returns the interpolated independent and dependent values with the 1\sphinxhyphen{}sigma standard deviation.}] \leavevmode
\end{description}

\end{description}\end{quote}
\subsubsection*{Methods}


\begin{savenotes}\sphinxatlongtablestart\begin{longtable}[c]{\X{1}{2}\X{1}{2}}
\hline

\endfirsthead

\multicolumn{2}{c}%
{\makebox[0pt]{\sphinxtablecontinued{\tablename\ \thetable{} \textendash{} continued from previous page}}}\\
\hline

\endhead

\hline
\multicolumn{2}{r}{\makebox[0pt][r]{\sphinxtablecontinued{Continued on next page}}}\\
\endfoot

\endlastfoot

\sphinxcode{\sphinxupquote{continue\_chains}}(self, t\_data, f\_data, …{[}, …{]})
&
Run MCMC for longer than initial fit
\\
\hline
\sphinxcode{\sphinxupquote{predict}}(self{[}, x\_pred, step, returnv, quant{]})
&
output fitting products
\\
\hline
\sphinxcode{\sphinxupquote{predict\_random}}(self{[}, limit, plotnsamples, …{]})
&
output fitting products
\\
\hline
\sphinxcode{\sphinxupquote{run\_mcmc}}(self, t\_data, f\_data, f\_unc\_data, …)
&
initial fit
\\
\hline
\sphinxcode{\sphinxupquote{save\_corner}}(self, figpath{[}, datadir, filts, …{]})
&
generate corner plots
\\
\hline
\end{longtable}\sphinxatlongtableend\end{savenotes}


\begin{savenotes}\sphinxattablestart
\centering
\begin{tabulary}{\linewidth}[t]{|T|T|}
\hline

\sphinxstylestrong{filter\_samples}
&\\
\hline
\sphinxstylestrong{get\_par}
&\\
\hline
\sphinxstylestrong{get\_random\_samples}
&\\
\hline
\sphinxstylestrong{lnlikelihood1}
&\\
\hline
\sphinxstylestrong{lnlikelihood2}
&\\
\hline
\sphinxstylestrong{lnposterior1}
&\\
\hline
\sphinxstylestrong{lnposterior2}
&\\
\hline
\sphinxstylestrong{lnprior1}
&\\
\hline
\sphinxstylestrong{lnprior2}
&\\
\hline
\sphinxstylestrong{run}
&\\
\hline
\sphinxstylestrong{run\_scipy}
&\\
\hline
\sphinxstylestrong{set\_peak}
&\\
\hline
\sphinxstylestrong{train}
&\\
\hline
\end{tabulary}
\par
\sphinxattableend\end{savenotes}
\index{\_\_init\_\_() (sdapy.model\_fitters.fit\_model method)@\spxentry{\_\_init\_\_()}\spxextra{sdapy.model\_fitters.fit\_model method}}

\begin{fulllineitems}
\phantomsection\label{\detokenize{generated/sdapy.model_fitters.fit_model:sdapy.model_fitters.fit_model.__init__}}\pysiglinewithargsret{\sphinxbfcode{\sphinxupquote{\_\_init\_\_}}}{\emph{self}, \emph{x\_data}, \emph{y\_data}, \emph{yerr\_data}, \emph{filters=None}}{}
Initialize self.  See help(type(self)) for accurate signature.

\end{fulllineitems}

\subsubsection*{Methods}


\begin{savenotes}\sphinxatlongtablestart\begin{longtable}[c]{\X{1}{2}\X{1}{2}}
\hline

\endfirsthead

\multicolumn{2}{c}%
{\makebox[0pt]{\sphinxtablecontinued{\tablename\ \thetable{} \textendash{} continued from previous page}}}\\
\hline

\endhead

\hline
\multicolumn{2}{r}{\makebox[0pt][r]{\sphinxtablecontinued{Continued on next page}}}\\
\endfoot

\endlastfoot

{\hyperref[\detokenize{generated/sdapy.model_fitters.fit_model:sdapy.model_fitters.fit_model.__init__}]{\sphinxcrossref{\sphinxcode{\sphinxupquote{\_\_init\_\_}}}}}(self, x\_data, y\_data, yerr\_data{[}, …{]})
&
Initialize self.
\\
\hline
\sphinxcode{\sphinxupquote{continue\_chains}}(self, t\_data, f\_data, …{[}, …{]})
&
Run MCMC for longer than initial fit
\\
\hline
\sphinxcode{\sphinxupquote{filter\_samples}}(samples, lnprob{[}, limit{]})
&

\\
\hline
\sphinxcode{\sphinxupquote{get\_par}}(self{[}, filt, quant, parname{]})
&

\\
\hline
\sphinxcode{\sphinxupquote{get\_random\_samples}}(self{[}, limit, plotnsamples{]})
&

\\
\hline
\sphinxcode{\sphinxupquote{lnlikelihood1}}(theta, f, t, f\_err, func, …)
&

\\
\hline
\sphinxcode{\sphinxupquote{lnlikelihood2}}(theta, f, t, f\_err, filters, …)
&

\\
\hline
\sphinxcode{\sphinxupquote{lnposterior1}}(theta, f, t, f\_err, func, …)
&

\\
\hline
\sphinxcode{\sphinxupquote{lnposterior2}}(theta, f, t, f\_err, filters, …)
&

\\
\hline
\sphinxcode{\sphinxupquote{lnprior1}}(theta, bounds)
&

\\
\hline
\sphinxcode{\sphinxupquote{lnprior2}}(theta, filters, cl, bounds)
&

\\
\hline
\sphinxcode{\sphinxupquote{predict}}(self{[}, x\_pred, step, returnv, quant{]})
&
output fitting products
\\
\hline
\sphinxcode{\sphinxupquote{predict\_random}}(self{[}, limit, plotnsamples, …{]})
&
output fitting products
\\
\hline
\sphinxcode{\sphinxupquote{run}}(self, bestv, bounds{[}, filt{]})
&

\\
\hline
\sphinxcode{\sphinxupquote{run\_mcmc}}(self, t\_data, f\_data, f\_unc\_data, …)
&
initial fit
\\
\hline
\sphinxcode{\sphinxupquote{run\_scipy}}(func, p0, bounds, x, y, yerr{[}, …{]})
&

\\
\hline
\sphinxcode{\sphinxupquote{save\_corner}}(self, figpath{[}, datadir, filts, …{]})
&
generate corner plots
\\
\hline
\sphinxcode{\sphinxupquote{set\_peak}}(self)
&

\\
\hline
\sphinxcode{\sphinxupquote{train}}(self{[}, opt\_routine, fit\_mean, …{]})
&

\\
\hline
\end{longtable}\sphinxatlongtableend\end{savenotes}

\end{fulllineitems}



\subsubsection{sdapy.model\_fitters.get\_engine}
\label{\detokenize{generated/sdapy.model_fitters.get_engine:sdapy-model-fitters-get-engine}}\label{\detokenize{generated/sdapy.model_fitters.get_engine::doc}}\index{get\_engine() (in module sdapy.model\_fitters)@\spxentry{get\_engine()}\spxextra{in module sdapy.model\_fitters}}

\begin{fulllineitems}
\phantomsection\label{\detokenize{generated/sdapy.model_fitters.get_engine:sdapy.model_fitters.get_engine}}\pysiglinewithargsret{\sphinxcode{\sphinxupquote{sdapy.model\_fitters.}}\sphinxbfcode{\sphinxupquote{get\_engine}}}{}{}~
\end{fulllineitems}



\subsubsection{sdapy.model\_fitters.get\_model}
\label{\detokenize{generated/sdapy.model_fitters.get_model:sdapy-model-fitters-get-model}}\label{\detokenize{generated/sdapy.model_fitters.get_model::doc}}\index{get\_model() (in module sdapy.model\_fitters)@\spxentry{get\_model()}\spxextra{in module sdapy.model\_fitters}}

\begin{fulllineitems}
\phantomsection\label{\detokenize{generated/sdapy.model_fitters.get_model:sdapy.model_fitters.get_model}}\pysiglinewithargsret{\sphinxcode{\sphinxupquote{sdapy.model\_fitters.}}\sphinxbfcode{\sphinxupquote{get\_model}}}{\emph{which\_engine=None}, \emph{with\_alias=False}}{}~
\end{fulllineitems}



\subsubsection{sdapy.model\_fitters.get\_pars}
\label{\detokenize{generated/sdapy.model_fitters.get_pars:sdapy-model-fitters-get-pars}}\label{\detokenize{generated/sdapy.model_fitters.get_pars::doc}}\index{get\_pars() (in module sdapy.model\_fitters)@\spxentry{get\_pars()}\spxextra{in module sdapy.model\_fitters}}

\begin{fulllineitems}
\phantomsection\label{\detokenize{generated/sdapy.model_fitters.get_pars:sdapy.model_fitters.get_pars}}\pysiglinewithargsret{\sphinxcode{\sphinxupquote{sdapy.model\_fitters.}}\sphinxbfcode{\sphinxupquote{get\_pars}}}{\emph{which}, \emph{with\_alias=True}}{}~
\end{fulllineitems}


More details about the fitting engines and models can be found at {\hyperref[\detokenize{models:models}]{\sphinxcrossref{\DUrole{std,std-ref}{Models}}}}


\section{\sphinxstyleliteralintitle{\sphinxupquote{Engines and Models}} \textendash{} Models}
\label{\detokenize{models:engines-and-models-models}}\label{\detokenize{models:models}}\label{\detokenize{models::doc}}
Below provide various of defined engines:


\subsection{Engines}
\label{\detokenize{models:engines}}

\begin{savenotes}\sphinxatlongtablestart\begin{longtable}[c]{\X{1}{2}\X{1}{2}}
\hline

\endfirsthead

\multicolumn{2}{c}%
{\makebox[0pt]{\sphinxtablecontinued{\tablename\ \thetable{} \textendash{} continued from previous page}}}\\
\hline

\endhead

\hline
\multicolumn{2}{r}{\makebox[0pt][r]{\sphinxtablecontinued{Continued on next page}}}\\
\endfoot

\endlastfoot

{\hyperref[\detokenize{generated/sdapy.engines.multiband_early.engine:sdapy.engines.multiband_early.engine}]{\sphinxcrossref{\sphinxcode{\sphinxupquote{multiband\_early.engine}}}}}(self, model\_name, …)
&
engine used to fit multiband lcs independently before peak
\\
\hline
{\hyperref[\detokenize{generated/sdapy.engines.multiband_main.engine:sdapy.engines.multiband_main.engine}]{\sphinxcrossref{\sphinxcode{\sphinxupquote{multiband\_main.engine}}}}}(self, model\_name, …)
&
engine used to fit multiband lcs independently around the main peak
\\
\hline
{\hyperref[\detokenize{generated/sdapy.engines.bol_early.engine:sdapy.engines.bol_early.engine}]{\sphinxcrossref{\sphinxcode{\sphinxupquote{bol\_early.engine}}}}}(self, model\_name, engine\_name)
&
engine used to fit bolometric lc around the main peak
\\
\hline
{\hyperref[\detokenize{generated/sdapy.engines.bol_main.engine:sdapy.engines.bol_main.engine}]{\sphinxcrossref{\sphinxcode{\sphinxupquote{bol\_main.engine}}}}}(self, model\_name, engine\_name)
&
engine used to fit bolometric lc around the main peak
\\
\hline
{\hyperref[\detokenize{generated/sdapy.engines.bol_tail.engine:sdapy.engines.bol_tail.engine}]{\sphinxcrossref{\sphinxcode{\sphinxupquote{bol\_tail.engine}}}}}(self, model\_name, engine\_name)
&
engine used to fit bolometric lc at the tail
\\
\hline
{\hyperref[\detokenize{generated/sdapy.engines.bol_full.engine:sdapy.engines.bol_full.engine}]{\sphinxcrossref{\sphinxcode{\sphinxupquote{bol\_full.engine}}}}}(self, model\_name, engine\_name)
&
engine used to fit bolometric lc around the main peak
\\
\hline
{\hyperref[\detokenize{generated/sdapy.engines.specline.engine:sdapy.engines.specline.engine}]{\sphinxcrossref{\sphinxcode{\sphinxupquote{specline.engine}}}}}(self, model\_name, engine\_name)
&
engine used to fit spectral lines
\\
\hline
{\hyperref[\detokenize{generated/sdapy.engines.specv_evolution.engine:sdapy.engines.specv_evolution.engine}]{\sphinxcrossref{\sphinxcode{\sphinxupquote{specv\_evolution.engine}}}}}(self, model\_name, …)
&
engine used to fit colours
\\
\hline
\end{longtable}\sphinxatlongtableend\end{savenotes}


\subsubsection{sdapy.engines.multiband\_early.engine}
\label{\detokenize{generated/sdapy.engines.multiband_early.engine:sdapy-engines-multiband-early-engine}}\label{\detokenize{generated/sdapy.engines.multiband_early.engine::doc}}\index{engine() (in module sdapy.engines.multiband\_early)@\spxentry{engine()}\spxextra{in module sdapy.engines.multiband\_early}}

\begin{fulllineitems}
\phantomsection\label{\detokenize{generated/sdapy.engines.multiband_early.engine:sdapy.engines.multiband_early.engine}}\pysiglinewithargsret{\sphinxcode{\sphinxupquote{sdapy.engines.multiband\_early.}}\sphinxbfcode{\sphinxupquote{engine}}}{\emph{self}, \emph{model\_name}, \emph{engine\_name}, \emph{sourcename=None}, \emph{**kwargs}}{}
engine used to fit multiband lcs independently before peak

\end{fulllineitems}



\subsubsection{sdapy.engines.multiband\_main.engine}
\label{\detokenize{generated/sdapy.engines.multiband_main.engine:sdapy-engines-multiband-main-engine}}\label{\detokenize{generated/sdapy.engines.multiband_main.engine::doc}}\index{engine() (in module sdapy.engines.multiband\_main)@\spxentry{engine()}\spxextra{in module sdapy.engines.multiband\_main}}

\begin{fulllineitems}
\phantomsection\label{\detokenize{generated/sdapy.engines.multiband_main.engine:sdapy.engines.multiband_main.engine}}\pysiglinewithargsret{\sphinxcode{\sphinxupquote{sdapy.engines.multiband\_main.}}\sphinxbfcode{\sphinxupquote{engine}}}{\emph{self}, \emph{model\_name}, \emph{engine\_name}, \emph{sourcename=None}, \emph{**kwargs}}{}
engine used to fit multiband lcs independently around the main peak

\end{fulllineitems}



\subsubsection{sdapy.engines.bol\_early.engine}
\label{\detokenize{generated/sdapy.engines.bol_early.engine:sdapy-engines-bol-early-engine}}\label{\detokenize{generated/sdapy.engines.bol_early.engine::doc}}\index{engine() (in module sdapy.engines.bol\_early)@\spxentry{engine()}\spxextra{in module sdapy.engines.bol\_early}}

\begin{fulllineitems}
\phantomsection\label{\detokenize{generated/sdapy.engines.bol_early.engine:sdapy.engines.bol_early.engine}}\pysiglinewithargsret{\sphinxcode{\sphinxupquote{sdapy.engines.bol\_early.}}\sphinxbfcode{\sphinxupquote{engine}}}{\emph{self}, \emph{model\_name}, \emph{engine\_name}, \emph{sourcename=None}, \emph{**kwargs}}{}
engine used to fit bolometric lc around the main peak

\end{fulllineitems}



\subsubsection{sdapy.engines.bol\_main.engine}
\label{\detokenize{generated/sdapy.engines.bol_main.engine:sdapy-engines-bol-main-engine}}\label{\detokenize{generated/sdapy.engines.bol_main.engine::doc}}\index{engine() (in module sdapy.engines.bol\_main)@\spxentry{engine()}\spxextra{in module sdapy.engines.bol\_main}}

\begin{fulllineitems}
\phantomsection\label{\detokenize{generated/sdapy.engines.bol_main.engine:sdapy.engines.bol_main.engine}}\pysiglinewithargsret{\sphinxcode{\sphinxupquote{sdapy.engines.bol\_main.}}\sphinxbfcode{\sphinxupquote{engine}}}{\emph{self}, \emph{model\_name}, \emph{engine\_name}, \emph{sourcename=None}, \emph{**kwargs}}{}
engine used to fit bolometric lc around the main peak

\end{fulllineitems}



\subsubsection{sdapy.engines.bol\_tail.engine}
\label{\detokenize{generated/sdapy.engines.bol_tail.engine:sdapy-engines-bol-tail-engine}}\label{\detokenize{generated/sdapy.engines.bol_tail.engine::doc}}\index{engine() (in module sdapy.engines.bol\_tail)@\spxentry{engine()}\spxextra{in module sdapy.engines.bol\_tail}}

\begin{fulllineitems}
\phantomsection\label{\detokenize{generated/sdapy.engines.bol_tail.engine:sdapy.engines.bol_tail.engine}}\pysiglinewithargsret{\sphinxcode{\sphinxupquote{sdapy.engines.bol\_tail.}}\sphinxbfcode{\sphinxupquote{engine}}}{\emph{self}, \emph{model\_name}, \emph{engine\_name}, \emph{sourcename=None}, \emph{**kwargs}}{}
engine used to fit bolometric lc at the tail

\end{fulllineitems}



\subsubsection{sdapy.engines.bol\_full.engine}
\label{\detokenize{generated/sdapy.engines.bol_full.engine:sdapy-engines-bol-full-engine}}\label{\detokenize{generated/sdapy.engines.bol_full.engine::doc}}\index{engine() (in module sdapy.engines.bol\_full)@\spxentry{engine()}\spxextra{in module sdapy.engines.bol\_full}}

\begin{fulllineitems}
\phantomsection\label{\detokenize{generated/sdapy.engines.bol_full.engine:sdapy.engines.bol_full.engine}}\pysiglinewithargsret{\sphinxcode{\sphinxupquote{sdapy.engines.bol\_full.}}\sphinxbfcode{\sphinxupquote{engine}}}{\emph{self}, \emph{model\_name}, \emph{engine\_name}, \emph{sourcename=None}, \emph{**kwargs}}{}
engine used to fit bolometric lc around the main peak

\end{fulllineitems}



\subsubsection{sdapy.engines.specline.engine}
\label{\detokenize{generated/sdapy.engines.specline.engine:sdapy-engines-specline-engine}}\label{\detokenize{generated/sdapy.engines.specline.engine::doc}}\index{engine() (in module sdapy.engines.specline)@\spxentry{engine()}\spxextra{in module sdapy.engines.specline}}

\begin{fulllineitems}
\phantomsection\label{\detokenize{generated/sdapy.engines.specline.engine:sdapy.engines.specline.engine}}\pysiglinewithargsret{\sphinxcode{\sphinxupquote{sdapy.engines.specline.}}\sphinxbfcode{\sphinxupquote{engine}}}{\emph{self}, \emph{model\_name}, \emph{engine\_name}, \emph{sourcename=None}, \emph{**kwargs}}{}
engine used to fit spectral lines

\end{fulllineitems}



\subsubsection{sdapy.engines.specv\_evolution.engine}
\label{\detokenize{generated/sdapy.engines.specv_evolution.engine:sdapy-engines-specv-evolution-engine}}\label{\detokenize{generated/sdapy.engines.specv_evolution.engine::doc}}\index{engine() (in module sdapy.engines.specv\_evolution)@\spxentry{engine()}\spxextra{in module sdapy.engines.specv\_evolution}}

\begin{fulllineitems}
\phantomsection\label{\detokenize{generated/sdapy.engines.specv_evolution.engine:sdapy.engines.specv_evolution.engine}}\pysiglinewithargsret{\sphinxcode{\sphinxupquote{sdapy.engines.specv\_evolution.}}\sphinxbfcode{\sphinxupquote{engine}}}{\emph{self, model\_name, engine\_name, sourcename=None, modelname=None, quant={[}0.05, 0.5, 0.95{]}, **kwargs}}{}
engine used to fit colours

\end{fulllineitems}


Below provide various of models working with different engines:


\subsection{Models}
\label{\detokenize{models:id1}}
\sphinxstylestrong{multiband\_early \sphinxhyphen{}\textgreater{} Power law Models}


\begin{savenotes}\sphinxatlongtablestart\begin{longtable}[c]{\X{1}{2}\X{1}{2}}
\hline

\endfirsthead

\multicolumn{2}{c}%
{\makebox[0pt]{\sphinxtablecontinued{\tablename\ \thetable{} \textendash{} continued from previous page}}}\\
\hline

\endhead

\hline
\multicolumn{2}{r}{\makebox[0pt][r]{\sphinxtablecontinued{Continued on next page}}}\\
\endfoot

\endlastfoot

{\hyperref[\detokenize{generated/sdapy.models.risepl.powerlaw_post_baseline:sdapy.models.risepl.powerlaw_post_baseline}]{\sphinxcrossref{\sphinxcode{\sphinxupquote{powerlaw\_post\_baseline}}}}}(times{[}, t\_0, …{]})
&

\\
\hline
{\hyperref[\detokenize{generated/sdapy.models.risepl.powerlaw_full:sdapy.models.risepl.powerlaw_full}]{\sphinxcrossref{\sphinxcode{\sphinxupquote{powerlaw\_full}}}}}(times{[}, t\_0, amplitude, …{]})
&

\\
\hline
\end{longtable}\sphinxatlongtableend\end{savenotes}


\subsubsection{sdapy.models.risepl.powerlaw\_post\_baseline}
\label{\detokenize{generated/sdapy.models.risepl.powerlaw_post_baseline:sdapy-models-risepl-powerlaw-post-baseline}}\label{\detokenize{generated/sdapy.models.risepl.powerlaw_post_baseline::doc}}\index{powerlaw\_post\_baseline() (in module sdapy.models.risepl)@\spxentry{powerlaw\_post\_baseline()}\spxextra{in module sdapy.models.risepl}}

\begin{fulllineitems}
\phantomsection\label{\detokenize{generated/sdapy.models.risepl.powerlaw_post_baseline:sdapy.models.risepl.powerlaw_post_baseline}}\pysiglinewithargsret{\sphinxcode{\sphinxupquote{sdapy.models.risepl.}}\sphinxbfcode{\sphinxupquote{powerlaw\_post\_baseline}}}{\emph{times}, \emph{t\_0=0}, \emph{amplitude=25}, \emph{alpha\_r=2}}{}~
\end{fulllineitems}



\subsubsection{sdapy.models.risepl.powerlaw\_full}
\label{\detokenize{generated/sdapy.models.risepl.powerlaw_full:sdapy-models-risepl-powerlaw-full}}\label{\detokenize{generated/sdapy.models.risepl.powerlaw_full::doc}}\index{powerlaw\_full() (in module sdapy.models.risepl)@\spxentry{powerlaw\_full()}\spxextra{in module sdapy.models.risepl}}

\begin{fulllineitems}
\phantomsection\label{\detokenize{generated/sdapy.models.risepl.powerlaw_full:sdapy.models.risepl.powerlaw_full}}\pysiglinewithargsret{\sphinxcode{\sphinxupquote{sdapy.models.risepl.}}\sphinxbfcode{\sphinxupquote{powerlaw\_full}}}{\emph{times}, \emph{t\_0=0}, \emph{amplitude=25}, \emph{alpha\_r=2}, \emph{c=0}}{}~
\end{fulllineitems}


\sphinxstylestrong{multiband\_main \sphinxhyphen{}\textgreater{} Bazin Models}


\begin{savenotes}\sphinxatlongtablestart\begin{longtable}[c]{\X{1}{2}\X{1}{2}}
\hline

\endfirsthead

\multicolumn{2}{c}%
{\makebox[0pt]{\sphinxtablecontinued{\tablename\ \thetable{} \textendash{} continued from previous page}}}\\
\hline

\endhead

\hline
\multicolumn{2}{r}{\makebox[0pt][r]{\sphinxtablecontinued{Continued on next page}}}\\
\endfoot

\endlastfoot

{\hyperref[\detokenize{generated/sdapy.models.bazin.bazin:sdapy.models.bazin.bazin}]{\sphinxcrossref{\sphinxcode{\sphinxupquote{bazin}}}}}(time, A, t0, tfall, trise, C)
&
bazin 2009 et al model
\\
\hline
{\hyperref[\detokenize{generated/sdapy.models.bazin.bazin1:sdapy.models.bazin.bazin1}]{\sphinxcrossref{\sphinxcode{\sphinxupquote{bazin1}}}}}(time, A, t0, tfall)
&
bazin model part I
\\
\hline
{\hyperref[\detokenize{generated/sdapy.models.bazin.bazin2:sdapy.models.bazin.bazin2}]{\sphinxcrossref{\sphinxcode{\sphinxupquote{bazin2}}}}}(time, A, t0, trise)
&
bazin model part II
\\
\hline
\end{longtable}\sphinxatlongtableend\end{savenotes}


\subsubsection{sdapy.models.bazin.bazin}
\label{\detokenize{generated/sdapy.models.bazin.bazin:sdapy-models-bazin-bazin}}\label{\detokenize{generated/sdapy.models.bazin.bazin::doc}}\index{bazin() (in module sdapy.models.bazin)@\spxentry{bazin()}\spxextra{in module sdapy.models.bazin}}

\begin{fulllineitems}
\phantomsection\label{\detokenize{generated/sdapy.models.bazin.bazin:sdapy.models.bazin.bazin}}\pysiglinewithargsret{\sphinxcode{\sphinxupquote{sdapy.models.bazin.}}\sphinxbfcode{\sphinxupquote{bazin}}}{\emph{time}, \emph{A}, \emph{t0}, \emph{tfall}, \emph{trise}, \emph{C}}{}
bazin 2009 et al model

\end{fulllineitems}



\subsubsection{sdapy.models.bazin.bazin1}
\label{\detokenize{generated/sdapy.models.bazin.bazin1:sdapy-models-bazin-bazin1}}\label{\detokenize{generated/sdapy.models.bazin.bazin1::doc}}\index{bazin1() (in module sdapy.models.bazin)@\spxentry{bazin1()}\spxextra{in module sdapy.models.bazin}}

\begin{fulllineitems}
\phantomsection\label{\detokenize{generated/sdapy.models.bazin.bazin1:sdapy.models.bazin.bazin1}}\pysiglinewithargsret{\sphinxcode{\sphinxupquote{sdapy.models.bazin.}}\sphinxbfcode{\sphinxupquote{bazin1}}}{\emph{time}, \emph{A}, \emph{t0}, \emph{tfall}}{}
bazin model part I

\end{fulllineitems}



\subsubsection{sdapy.models.bazin.bazin2}
\label{\detokenize{generated/sdapy.models.bazin.bazin2:sdapy-models-bazin-bazin2}}\label{\detokenize{generated/sdapy.models.bazin.bazin2::doc}}\index{bazin2() (in module sdapy.models.bazin)@\spxentry{bazin2()}\spxextra{in module sdapy.models.bazin}}

\begin{fulllineitems}
\phantomsection\label{\detokenize{generated/sdapy.models.bazin.bazin2:sdapy.models.bazin.bazin2}}\pysiglinewithargsret{\sphinxcode{\sphinxupquote{sdapy.models.bazin.}}\sphinxbfcode{\sphinxupquote{bazin2}}}{\emph{time}, \emph{A}, \emph{t0}, \emph{trise}}{}
bazin model part II

\end{fulllineitems}


\sphinxstylestrong{multiband\_main \sphinxhyphen{}\textgreater{} Villar Models}


\begin{savenotes}\sphinxatlongtablestart\begin{longtable}[c]{\X{1}{2}\X{1}{2}}
\hline

\endfirsthead

\multicolumn{2}{c}%
{\makebox[0pt]{\sphinxtablecontinued{\tablename\ \thetable{} \textendash{} continued from previous page}}}\\
\hline

\endhead

\hline
\multicolumn{2}{r}{\makebox[0pt][r]{\sphinxtablecontinued{Continued on next page}}}\\
\endfoot

\endlastfoot

{\hyperref[\detokenize{generated/sdapy.models.villar.villar:sdapy.models.villar.villar}]{\sphinxcrossref{\sphinxcode{\sphinxupquote{villar}}}}}(time, a, b, t0, t1, tfall, trise, c)
&
Villar et al 2019 (\sphinxurl{https://iopscience.iop.org/article/10.3847/1538-4357/ab418c/pdf}), function 1
\\
\hline
\end{longtable}\sphinxatlongtableend\end{savenotes}


\subsubsection{sdapy.models.villar.villar}
\label{\detokenize{generated/sdapy.models.villar.villar:sdapy-models-villar-villar}}\label{\detokenize{generated/sdapy.models.villar.villar::doc}}\index{villar() (in module sdapy.models.villar)@\spxentry{villar()}\spxextra{in module sdapy.models.villar}}

\begin{fulllineitems}
\phantomsection\label{\detokenize{generated/sdapy.models.villar.villar:sdapy.models.villar.villar}}\pysiglinewithargsret{\sphinxcode{\sphinxupquote{sdapy.models.villar.}}\sphinxbfcode{\sphinxupquote{villar}}}{\emph{time}, \emph{a}, \emph{b}, \emph{t0}, \emph{t1}, \emph{tfall}, \emph{trise}, \emph{c}}{}
Villar et al 2019 (\sphinxurl{https://iopscience.iop.org/article/10.3847/1538-4357/ab418c/pdf}), function 1
\begin{quote}\begin{description}
\item[{Parameters}] \leavevmode\begin{description}
\item[{\sphinxstylestrong{time}}] \leavevmode{[}array{]}
Independent values.

\item[{\sphinxstylestrong{a}}] \leavevmode{[}float{]}
A, Amplitude

\item[{\sphinxstylestrong{b}}] \leavevmode{[}float{]}
beta (flux/day), Plateau slope

\item[{\sphinxstylestrong{t0}}] \leavevmode{[}float{]}
“Start” time

\item[{\sphinxstylestrong{t1}}] \leavevmode{[}float{]}
plateau onset of LC

\item[{\sphinxstylestrong{tfall: float}}] \leavevmode
rise time of LC

\item[{\sphinxstylestrong{trise: float}}] \leavevmode
fall time of LC

\item[{\sphinxstylestrong{c}}] \leavevmode{[}float{]}
baseline flux

\end{description}

\end{description}\end{quote}

\end{fulllineitems}


\sphinxstylestrong{bol\_early \sphinxhyphen{}\textgreater{} Piro Models}


\begin{savenotes}\sphinxatlongtablestart\begin{longtable}[c]{\X{1}{2}\X{1}{2}}
\hline

\endfirsthead

\multicolumn{2}{c}%
{\makebox[0pt]{\sphinxtablecontinued{\tablename\ \thetable{} \textendash{} continued from previous page}}}\\
\hline

\endhead

\hline
\multicolumn{2}{r}{\makebox[0pt][r]{\sphinxtablecontinued{Continued on next page}}}\\
\endfoot

\endlastfoot

{\hyperref[\detokenize{generated/sdapy.models.sbo.shock_fit:sdapy.models.sbo.shock_fit}]{\sphinxcrossref{\sphinxcode{\sphinxupquote{shock\_fit}}}}}(time, Me, Re, Ee{[}, texp{]})
&
shock cooling fit with Piro et al 2020 model (\sphinxurl{https://arxiv.org/pdf/2007.08543.pdf})
\\
\hline
\sphinxcode{\sphinxupquote{shock\_fit\_texp}}
&

\\
\hline
{\hyperref[\detokenize{generated/sdapy.models.sbo.shock_arnett_fit:sdapy.models.sbo.shock_arnett_fit}]{\sphinxcrossref{\sphinxcode{\sphinxupquote{shock\_arnett\_fit}}}}}(time, Me, Re, Ee, mni, taum)
&
shock cooling + Arnett (mni, taum) fit
\\
\hline
\sphinxcode{\sphinxupquote{shock\_arnett\_fit\_texp}}
&

\\
\hline
{\hyperref[\detokenize{generated/sdapy.models.sbo.shock_arnett_mejek_fit:sdapy.models.sbo.shock_arnett_mejek_fit}]{\sphinxcrossref{\sphinxcode{\sphinxupquote{shock\_arnett\_mejek\_fit}}}}}(time, Me, Re, Ee, …)
&
shock cooling + Arnett (mni, mej, ek) fit
\\
\hline
\sphinxcode{\sphinxupquote{shock\_arnett\_fit\_mejek\_texp}}
&

\\
\hline
\end{longtable}\sphinxatlongtableend\end{savenotes}


\subsubsection{sdapy.models.sbo.shock\_fit}
\label{\detokenize{generated/sdapy.models.sbo.shock_fit:sdapy-models-sbo-shock-fit}}\label{\detokenize{generated/sdapy.models.sbo.shock_fit::doc}}\index{shock\_fit() (in module sdapy.models.sbo)@\spxentry{shock\_fit()}\spxextra{in module sdapy.models.sbo}}

\begin{fulllineitems}
\phantomsection\label{\detokenize{generated/sdapy.models.sbo.shock_fit:sdapy.models.sbo.shock_fit}}\pysiglinewithargsret{\sphinxcode{\sphinxupquote{sdapy.models.sbo.}}\sphinxbfcode{\sphinxupquote{shock\_fit}}}{\emph{time}, \emph{Me}, \emph{Re}, \emph{Ee}, \emph{texp=None}}{}
shock cooling fit with Piro et al 2020 model (\sphinxurl{https://arxiv.org/pdf/2007.08543.pdf})
\begin{quote}\begin{description}
\item[{Parameters}] \leavevmode\begin{description}
\item[{\sphinxstylestrong{time}}] \leavevmode{[}\sphinxtitleref{array}{]}
Independent values.

\item[{\sphinxstylestrong{Me}}] \leavevmode{[}\sphinxtitleref{float}{]}
envolop mass (unit: solar mass)

\item[{\sphinxstylestrong{Re}}] \leavevmode{[}\sphinxtitleref{float}{]}
envolop radius (unit: solar radius)

\item[{\sphinxstylestrong{Ee}}] \leavevmode{[}\sphinxtitleref{float}{]}
envolop energy (unit: foe)

\item[{\sphinxstylestrong{texp}}] \leavevmode{[}\sphinxtitleref{float}{]}
explosion time, time between first light to the peak epoch.

\end{description}

\end{description}\end{quote}

\end{fulllineitems}



\subsubsection{sdapy.models.sbo.shock\_arnett\_fit}
\label{\detokenize{generated/sdapy.models.sbo.shock_arnett_fit:sdapy-models-sbo-shock-arnett-fit}}\label{\detokenize{generated/sdapy.models.sbo.shock_arnett_fit::doc}}\index{shock\_arnett\_fit() (in module sdapy.models.sbo)@\spxentry{shock\_arnett\_fit()}\spxextra{in module sdapy.models.sbo}}

\begin{fulllineitems}
\phantomsection\label{\detokenize{generated/sdapy.models.sbo.shock_arnett_fit:sdapy.models.sbo.shock_arnett_fit}}\pysiglinewithargsret{\sphinxcode{\sphinxupquote{sdapy.models.sbo.}}\sphinxbfcode{\sphinxupquote{shock\_arnett\_fit}}}{\emph{time}, \emph{Me}, \emph{Re}, \emph{Ee}, \emph{mni}, \emph{taum}, \emph{texp=None}}{}
shock cooling + Arnett (mni, taum) fit
\begin{quote}\begin{description}
\item[{Parameters}] \leavevmode\begin{description}
\item[{\sphinxstylestrong{time}}] \leavevmode{[}\sphinxtitleref{array}{]}
Independent values.

\item[{\sphinxstylestrong{Me}}] \leavevmode{[}\sphinxtitleref{float}{]}
envolop mass (unit: solar mass)

\item[{\sphinxstylestrong{Re}}] \leavevmode{[}\sphinxtitleref{float}{]}
envolop radius (unit: solar radius)

\item[{\sphinxstylestrong{Ee}}] \leavevmode{[}\sphinxtitleref{float}{]}
envolop energy (unit: foe)

\item[{\sphinxstylestrong{mni}}] \leavevmode{[}\sphinxtitleref{float}{]}
Arnett model parameter, Nickel mass, unit in solar mass.

\item[{\sphinxstylestrong{taum}}] \leavevmode{[}\sphinxtitleref{float}{]}
Arnett model parameter, characteristic time, unit in days, decided by ejecta mass and kinetic enegies.

\item[{\sphinxstylestrong{texp}}] \leavevmode{[}\sphinxtitleref{float}{]}
explosion time, time between first light to the peak epoch.

\end{description}

\end{description}\end{quote}

\end{fulllineitems}



\subsubsection{sdapy.models.sbo.shock\_arnett\_mejek\_fit}
\label{\detokenize{generated/sdapy.models.sbo.shock_arnett_mejek_fit:sdapy-models-sbo-shock-arnett-mejek-fit}}\label{\detokenize{generated/sdapy.models.sbo.shock_arnett_mejek_fit::doc}}\index{shock\_arnett\_mejek\_fit() (in module sdapy.models.sbo)@\spxentry{shock\_arnett\_mejek\_fit()}\spxextra{in module sdapy.models.sbo}}

\begin{fulllineitems}
\phantomsection\label{\detokenize{generated/sdapy.models.sbo.shock_arnett_mejek_fit:sdapy.models.sbo.shock_arnett_mejek_fit}}\pysiglinewithargsret{\sphinxcode{\sphinxupquote{sdapy.models.sbo.}}\sphinxbfcode{\sphinxupquote{shock\_arnett\_mejek\_fit}}}{\emph{time}, \emph{Me}, \emph{Re}, \emph{Ee}, \emph{f\_ni}, \emph{mej}, \emph{ek}, \emph{texp=None}, \emph{k\_opt=None}}{}
shock cooling + Arnett (mni, mej, ek) fit
\begin{quote}\begin{description}
\item[{Parameters}] \leavevmode\begin{description}
\item[{\sphinxstylestrong{time}}] \leavevmode{[}\sphinxtitleref{array}{]}
Independent values.

\item[{\sphinxstylestrong{Me}}] \leavevmode{[}\sphinxtitleref{float}{]}
envolop mass (unit: solar mass)

\item[{\sphinxstylestrong{Re}}] \leavevmode{[}\sphinxtitleref{float}{]}
envolop radius (unit: solar radius)

\item[{\sphinxstylestrong{Ee}}] \leavevmode{[}\sphinxtitleref{float}{]}
envolop energy (unit: foe)

\item[{\sphinxstylestrong{f\_ni}}] \leavevmode{[}\sphinxtitleref{float}{]}
fraction of nickel mass

\item[{\sphinxstylestrong{Mej}}] \leavevmode{[}\sphinxtitleref{float}{]}
ejecta mass (unit: solar mass)

\item[{\sphinxstylestrong{Ek}}] \leavevmode{[}\sphinxtitleref{float}{]}
kinetic energy (unit: foe)

\item[{\sphinxstylestrong{texp}}] \leavevmode{[}\sphinxtitleref{float}{]}
explosion time, time between first light to the peak epoch.

\item[{\sphinxstylestrong{k\_opt}}] \leavevmode{[}\sphinxtitleref{float}{]}
diffusion opacity

\end{description}

\end{description}\end{quote}

\end{fulllineitems}


\sphinxstylestrong{bol\_main \sphinxhyphen{}\textgreater{} Arnett Models}


\begin{savenotes}\sphinxatlongtablestart\begin{longtable}[c]{\X{1}{2}\X{1}{2}}
\hline

\endfirsthead

\multicolumn{2}{c}%
{\makebox[0pt]{\sphinxtablecontinued{\tablename\ \thetable{} \textendash{} continued from previous page}}}\\
\hline

\endhead

\hline
\multicolumn{2}{r}{\makebox[0pt][r]{\sphinxtablecontinued{Continued on next page}}}\\
\endfoot

\endlastfoot

{\hyperref[\detokenize{generated/sdapy.models.arnett_tail.Arnett_fit_taum:sdapy.models.arnett_tail.Arnett_fit_taum}]{\sphinxcrossref{\sphinxcode{\sphinxupquote{Arnett\_fit\_taum}}}}}(times, m\_ni, taum{[}, texp{]})
&
output Arnett bolometric luminosities
\\
\hline
{\hyperref[\detokenize{generated/sdapy.models.arnett_tail.taum_to_MejEk:sdapy.models.arnett_tail.taum_to_MejEk}]{\sphinxcrossref{\sphinxcode{\sphinxupquote{taum\_to\_MejEk}}}}}(tau\_m{[}, taum\_err, k\_opt{]})
&
from characteristic time, taum (unit: day) to the product of kinetic energy (unit: foe) and ejecta mass (unit: solar mass), i.e.
\\
\hline
{\hyperref[\detokenize{generated/sdapy.models.arnett_tail.taum_to_Mej_Ek:sdapy.models.arnett_tail.taum_to_Mej_Ek}]{\sphinxcrossref{\sphinxcode{\sphinxupquote{taum\_to\_Mej\_Ek}}}}}(tau\_m, v\_ej{[}, taum\_err, …{]})
&
break the degenracy of kinetic energy and ejecta mass, with the help of velocity.
\\
\hline
{\hyperref[\detokenize{generated/sdapy.models.arnett_tail.taum_to_Mej_Ek_1:sdapy.models.arnett_tail.taum_to_Mej_Ek_1}]{\sphinxcrossref{\sphinxcode{\sphinxupquote{taum\_to\_Mej\_Ek\_1}}}}}(tau\_m, v\_ej{[}, taum\_err, …{]})
&
break the degenracy of kinetic energy and ejecta mass, with the help of velocity.
\\
\hline
{\hyperref[\detokenize{generated/sdapy.models.arnett_tail.Mej_Ek_to_taum:sdapy.models.arnett_tail.Mej_Ek_to_taum}]{\sphinxcrossref{\sphinxcode{\sphinxupquote{Mej\_Ek\_to\_taum}}}}}(Mej, Ek{[}, k\_opt{]})
&
from kinetic energy (unit: foe) and ejecta mass (unit: solar mass) to taum (unit: day)
\\
\hline
{\hyperref[\detokenize{generated/sdapy.models.arnett_tail.Mej_Ek_to_vej:sdapy.models.arnett_tail.Mej_Ek_to_vej}]{\sphinxcrossref{\sphinxcode{\sphinxupquote{Mej\_Ek\_to\_vej}}}}}(mej, ek)
&
from kinetic energy (unit: foe) and ejecta mass (unit: solar mass) to photospheric velocity (unit: 10**3 km/s)
\\
\hline
{\hyperref[\detokenize{generated/sdapy.models.arnett_tail.Arnett_fit_Mej_Ek:sdapy.models.arnett_tail.Arnett_fit_Mej_Ek}]{\sphinxcrossref{\sphinxcode{\sphinxupquote{Arnett\_fit\_Mej\_Ek}}}}}(times, f\_ni, Ek, Mej{[}, …{]})
&
fit Arnett model with Ek (unit: foe), Mej (unit: solar mass),
\\
\hline
\sphinxcode{\sphinxupquote{Arnett\_fit\_taum\_texp}}
&

\\
\hline
\sphinxcode{\sphinxupquote{Arnett\_fit\_Mej\_Ek\_texp}}
&

\\
\hline
\end{longtable}\sphinxatlongtableend\end{savenotes}


\subsubsection{sdapy.models.arnett\_tail.Arnett\_fit\_taum}
\label{\detokenize{generated/sdapy.models.arnett_tail.Arnett_fit_taum:sdapy-models-arnett-tail-arnett-fit-taum}}\label{\detokenize{generated/sdapy.models.arnett_tail.Arnett_fit_taum::doc}}\index{Arnett\_fit\_taum() (in module sdapy.models.arnett\_tail)@\spxentry{Arnett\_fit\_taum()}\spxextra{in module sdapy.models.arnett\_tail}}

\begin{fulllineitems}
\phantomsection\label{\detokenize{generated/sdapy.models.arnett_tail.Arnett_fit_taum:sdapy.models.arnett_tail.Arnett_fit_taum}}\pysiglinewithargsret{\sphinxcode{\sphinxupquote{sdapy.models.arnett\_tail.}}\sphinxbfcode{\sphinxupquote{Arnett\_fit\_taum}}}{\emph{times}, \emph{m\_ni}, \emph{taum}, \emph{texp=None}}{}
output Arnett bolometric luminosities
\begin{quote}\begin{description}
\item[{Parameters}] \leavevmode\begin{description}
\item[{\sphinxstylestrong{time}}] \leavevmode{[}\sphinxtitleref{array}{]}
Independent values.

\item[{\sphinxstylestrong{m\_ni}}] \leavevmode{[}\sphinxtitleref{float}{]}
Arnett model parameter, Nickel mass, unit in solar mass.

\item[{\sphinxstylestrong{taum}}] \leavevmode{[}\sphinxtitleref{float}{]}
Arnett model parameter, characteristic time, unit in days, decided by ejecta mass and kinetic enegies.

\item[{\sphinxstylestrong{texp}}] \leavevmode{[}\sphinxtitleref{float}{]}
explosion time, time between first light to the peak epoch.

\end{description}

\end{description}\end{quote}

\end{fulllineitems}



\subsubsection{sdapy.models.arnett\_tail.taum\_to\_MejEk}
\label{\detokenize{generated/sdapy.models.arnett_tail.taum_to_MejEk:sdapy-models-arnett-tail-taum-to-mejek}}\label{\detokenize{generated/sdapy.models.arnett_tail.taum_to_MejEk::doc}}\index{taum\_to\_MejEk() (in module sdapy.models.arnett\_tail)@\spxentry{taum\_to\_MejEk()}\spxextra{in module sdapy.models.arnett\_tail}}

\begin{fulllineitems}
\phantomsection\label{\detokenize{generated/sdapy.models.arnett_tail.taum_to_MejEk:sdapy.models.arnett_tail.taum_to_MejEk}}\pysiglinewithargsret{\sphinxcode{\sphinxupquote{sdapy.models.arnett\_tail.}}\sphinxbfcode{\sphinxupquote{taum\_to\_MejEk}}}{\emph{tau\_m}, \emph{taum\_err=None}, \emph{k\_opt=None}}{}
from characteristic time, taum (unit: day) to the product of kinetic energy (unit: foe) and ejecta mass (unit: solar mass),
i.e. M\$\_\{ej\}\textasciicircum{}\{3/4\}\$ E\$\_\{kin\}\textasciicircum{}\{\sphinxhyphen{}1/4\}\$
\begin{quote}\begin{description}
\item[{Parameters}] \leavevmode\begin{description}
\item[{\sphinxstylestrong{tau\_m}}] \leavevmode{[}\sphinxtitleref{float}{]}
Arnett model parameter, characteristic time, unit in days, decided by ejecta mass and kinetic enegies.

\item[{\sphinxstylestrong{taum\_err}}] \leavevmode{[}\sphinxtitleref{float}{]}
Error of taum, if not setted, will return only value, otherwise will return value as well as error.

\item[{\sphinxstylestrong{k\_opt}}] \leavevmode{[}\sphinxtitleref{float}{]}
diffusion opacity

\end{description}

\end{description}\end{quote}

\end{fulllineitems}



\subsubsection{sdapy.models.arnett\_tail.taum\_to\_Mej\_Ek}
\label{\detokenize{generated/sdapy.models.arnett_tail.taum_to_Mej_Ek:sdapy-models-arnett-tail-taum-to-mej-ek}}\label{\detokenize{generated/sdapy.models.arnett_tail.taum_to_Mej_Ek::doc}}\index{taum\_to\_Mej\_Ek() (in module sdapy.models.arnett\_tail)@\spxentry{taum\_to\_Mej\_Ek()}\spxextra{in module sdapy.models.arnett\_tail}}

\begin{fulllineitems}
\phantomsection\label{\detokenize{generated/sdapy.models.arnett_tail.taum_to_Mej_Ek:sdapy.models.arnett_tail.taum_to_Mej_Ek}}\pysiglinewithargsret{\sphinxcode{\sphinxupquote{sdapy.models.arnett\_tail.}}\sphinxbfcode{\sphinxupquote{taum\_to\_Mej\_Ek}}}{\emph{tau\_m}, \emph{v\_ej}, \emph{taum\_err=None}, \emph{vej\_err=None}, \emph{k\_opt=None}}{}~\begin{description}
\item[{break the degenracy of kinetic energy and ejecta mass, with the help of velocity.}] \leavevmode
for Thin Shell:  vej = sqrt(2*Ekin/Mej)

\end{description}
\begin{quote}\begin{description}
\item[{Parameters}] \leavevmode\begin{description}
\item[{\sphinxstylestrong{tau\_m}}] \leavevmode{[}\sphinxtitleref{float}{]}
Arnett model parameter, characteristic time, unit in days, decided by ejecta mass and kinetic enegies.

\item[{\sphinxstylestrong{v\_ej}}] \leavevmode{[}\sphinxtitleref{float}{]}
here vej is the photospheric velocity (unit: 10**3 km/s), which can be related to
line velocities (He 5876 and O 7772 line velocity at peak epoch for SN Ib and Ic correspondingly)
via Dessart et al 2014.

\item[{\sphinxstylestrong{taum\_err}}] \leavevmode{[}\sphinxtitleref{float}{]}
Error of taum.

\item[{\sphinxstylestrong{vej\_err}}] \leavevmode{[}\sphinxtitleref{float}{]}
Error of vej.

\item[{\sphinxstylestrong{k\_opt}}] \leavevmode{[}\sphinxtitleref{float}{]}
diffusion opacity

\end{description}

\end{description}\end{quote}

\end{fulllineitems}



\subsubsection{sdapy.models.arnett\_tail.taum\_to\_Mej\_Ek\_1}
\label{\detokenize{generated/sdapy.models.arnett_tail.taum_to_Mej_Ek_1:sdapy-models-arnett-tail-taum-to-mej-ek-1}}\label{\detokenize{generated/sdapy.models.arnett_tail.taum_to_Mej_Ek_1::doc}}\index{taum\_to\_Mej\_Ek\_1() (in module sdapy.models.arnett\_tail)@\spxentry{taum\_to\_Mej\_Ek\_1()}\spxextra{in module sdapy.models.arnett\_tail}}

\begin{fulllineitems}
\phantomsection\label{\detokenize{generated/sdapy.models.arnett_tail.taum_to_Mej_Ek_1:sdapy.models.arnett_tail.taum_to_Mej_Ek_1}}\pysiglinewithargsret{\sphinxcode{\sphinxupquote{sdapy.models.arnett\_tail.}}\sphinxbfcode{\sphinxupquote{taum\_to\_Mej\_Ek\_1}}}{\emph{tau\_m}, \emph{v\_ej}, \emph{taum\_err=None}, \emph{vej\_err=None}, \emph{k\_opt=None}}{}~\begin{description}
\item[{break the degenracy of kinetic energy and ejecta mass, with the help of velocity.}] \leavevmode
for Homologous Expansion: vm = vej = sqrt(Ekin/Mej*10./3.)

\end{description}
\begin{quote}\begin{description}
\item[{Parameters}] \leavevmode\begin{description}
\item[{\sphinxstylestrong{tau\_m}}] \leavevmode{[}\sphinxtitleref{float}{]}
Arnett model parameter, characteristic time, unit in days, decided by ejecta mass and kinetic enegies.

\item[{\sphinxstylestrong{v\_ej}}] \leavevmode{[}\sphinxtitleref{float}{]}
here vej is the photospheric velocity (unit: 10**3 km/s), which can be assumed as line velocities.

\item[{\sphinxstylestrong{taum\_err}}] \leavevmode{[}\sphinxtitleref{float}{]}
Error of taum.

\item[{\sphinxstylestrong{vej\_err}}] \leavevmode{[}\sphinxtitleref{float}{]}
Error of vej.

\item[{\sphinxstylestrong{k\_opt}}] \leavevmode{[}\sphinxtitleref{float}{]}
diffusion opacity

\end{description}

\end{description}\end{quote}

\end{fulllineitems}



\subsubsection{sdapy.models.arnett\_tail.Mej\_Ek\_to\_taum}
\label{\detokenize{generated/sdapy.models.arnett_tail.Mej_Ek_to_taum:sdapy-models-arnett-tail-mej-ek-to-taum}}\label{\detokenize{generated/sdapy.models.arnett_tail.Mej_Ek_to_taum::doc}}\index{Mej\_Ek\_to\_taum() (in module sdapy.models.arnett\_tail)@\spxentry{Mej\_Ek\_to\_taum()}\spxextra{in module sdapy.models.arnett\_tail}}

\begin{fulllineitems}
\phantomsection\label{\detokenize{generated/sdapy.models.arnett_tail.Mej_Ek_to_taum:sdapy.models.arnett_tail.Mej_Ek_to_taum}}\pysiglinewithargsret{\sphinxcode{\sphinxupquote{sdapy.models.arnett\_tail.}}\sphinxbfcode{\sphinxupquote{Mej\_Ek\_to\_taum}}}{\emph{Mej}, \emph{Ek}, \emph{k\_opt=None}}{}
from kinetic energy (unit: foe) and ejecta mass (unit: solar mass) to taum (unit: day)
\begin{quote}\begin{description}
\item[{Parameters}] \leavevmode\begin{description}
\item[{\sphinxstylestrong{Mej}}] \leavevmode{[}\sphinxtitleref{float}{]}
ejecta mass (unit: solar mass)

\item[{\sphinxstylestrong{Ek}}] \leavevmode{[}\sphinxtitleref{float}{]}
kinetic energy (unit: foe)

\item[{\sphinxstylestrong{k\_opt}}] \leavevmode{[}\sphinxtitleref{float}{]}
diffusion opacity

\end{description}

\end{description}\end{quote}

\end{fulllineitems}



\subsubsection{sdapy.models.arnett\_tail.Mej\_Ek\_to\_vej}
\label{\detokenize{generated/sdapy.models.arnett_tail.Mej_Ek_to_vej:sdapy-models-arnett-tail-mej-ek-to-vej}}\label{\detokenize{generated/sdapy.models.arnett_tail.Mej_Ek_to_vej::doc}}\index{Mej\_Ek\_to\_vej() (in module sdapy.models.arnett\_tail)@\spxentry{Mej\_Ek\_to\_vej()}\spxextra{in module sdapy.models.arnett\_tail}}

\begin{fulllineitems}
\phantomsection\label{\detokenize{generated/sdapy.models.arnett_tail.Mej_Ek_to_vej:sdapy.models.arnett_tail.Mej_Ek_to_vej}}\pysiglinewithargsret{\sphinxcode{\sphinxupquote{sdapy.models.arnett\_tail.}}\sphinxbfcode{\sphinxupquote{Mej\_Ek\_to\_vej}}}{\emph{mej}, \emph{ek}}{}
from kinetic energy (unit: foe) and ejecta mass (unit: solar mass) to photospheric velocity (unit: 10**3 km/s)
\begin{quote}\begin{description}
\item[{Parameters}] \leavevmode\begin{description}
\item[{\sphinxstylestrong{mej}}] \leavevmode{[}\sphinxtitleref{float}{]}
ejecta mass (unit: solar mass)

\item[{\sphinxstylestrong{ek}}] \leavevmode{[}\sphinxtitleref{float}{]}
kinetic energy (unit: foe)

\end{description}

\end{description}\end{quote}

\end{fulllineitems}



\subsubsection{sdapy.models.arnett\_tail.Arnett\_fit\_Mej\_Ek}
\label{\detokenize{generated/sdapy.models.arnett_tail.Arnett_fit_Mej_Ek:sdapy-models-arnett-tail-arnett-fit-mej-ek}}\label{\detokenize{generated/sdapy.models.arnett_tail.Arnett_fit_Mej_Ek::doc}}\index{Arnett\_fit\_Mej\_Ek() (in module sdapy.models.arnett\_tail)@\spxentry{Arnett\_fit\_Mej\_Ek()}\spxextra{in module sdapy.models.arnett\_tail}}

\begin{fulllineitems}
\phantomsection\label{\detokenize{generated/sdapy.models.arnett_tail.Arnett_fit_Mej_Ek:sdapy.models.arnett_tail.Arnett_fit_Mej_Ek}}\pysiglinewithargsret{\sphinxcode{\sphinxupquote{sdapy.models.arnett\_tail.}}\sphinxbfcode{\sphinxupquote{Arnett\_fit\_Mej\_Ek}}}{\emph{times}, \emph{f\_ni}, \emph{Ek}, \emph{Mej}, \emph{texp=None}, \emph{k\_opt=None}}{}~\begin{description}
\item[{fit Arnett model with Ek (unit: foe), Mej (unit: solar mass),}] \leavevmode
as free parameter

\end{description}
\begin{quote}\begin{description}
\item[{Parameters}] \leavevmode\begin{description}
\item[{\sphinxstylestrong{time}}] \leavevmode{[}\sphinxtitleref{array}{]}
Independent values.

\item[{\sphinxstylestrong{f\_ni}}] \leavevmode{[}\sphinxtitleref{float}{]}
fraction of nickel mass

\item[{\sphinxstylestrong{Mej}}] \leavevmode{[}\sphinxtitleref{float}{]}
ejecta mass (unit: solar mass)

\item[{\sphinxstylestrong{Ek}}] \leavevmode{[}\sphinxtitleref{float}{]}
kinetic energy (unit: foe)

\item[{\sphinxstylestrong{k\_opt}}] \leavevmode{[}\sphinxtitleref{float}{]}
diffusion opacity

\item[{\sphinxstylestrong{texp}}] \leavevmode{[}\sphinxtitleref{float}{]}
explosion time, time between first light to the peak epoch.

\end{description}

\end{description}\end{quote}

\end{fulllineitems}


\sphinxstylestrong{bol\_main \sphinxhyphen{}\textgreater{} Magnetar Models}


\begin{savenotes}\sphinxatlongtablestart\begin{longtable}[c]{\X{1}{2}\X{1}{2}}
\hline

\endfirsthead

\multicolumn{2}{c}%
{\makebox[0pt]{\sphinxtablecontinued{\tablename\ \thetable{} \textendash{} continued from previous page}}}\\
\hline

\endhead

\hline
\multicolumn{2}{r}{\makebox[0pt][r]{\sphinxtablecontinued{Continued on next page}}}\\
\endfoot

\endlastfoot

{\hyperref[\detokenize{generated/sdapy.models.magnetar.basic_magnetar:sdapy.models.magnetar.basic_magnetar}]{\sphinxcrossref{\sphinxcode{\sphinxupquote{basic\_magnetar}}}}}(time, p0, bp, mass\_ns, theta\_pb)
&
\sphinxhref{https://ui.adsabs.harvard.edu/abs/2006ApJ...648L..51S/abstract}{https://ui.adsabs.harvard.edu/abs/2006ApJ…648L..51S/abstract}
\\
\hline
{\hyperref[\detokenize{generated/sdapy.models.magnetar.magnetar_only:sdapy.models.magnetar.magnetar_only}]{\sphinxcrossref{\sphinxcode{\sphinxupquote{magnetar\_only}}}}}(time, l0, tau, nn)
&
\sphinxhref{https://ui.adsabs.harvard.edu/abs/2017ApJ...843L...1L/abstract}{https://ui.adsabs.harvard.edu/abs/2017ApJ…843L…1L/abstract}
\\
\hline
{\hyperref[\detokenize{generated/sdapy.models.magnetar.basic_magnetar_powered_bolometric:sdapy.models.magnetar.basic_magnetar_powered_bolometric}]{\sphinxcrossref{\sphinxcode{\sphinxupquote{basic\_magnetar\_powered\_bolometric}}}}}(times, p0, …)
&
\begin{quote}\begin{description}
\item[{param times}] \leavevmode
time in days in source frame

\end{description}\end{quote}

\\
\hline
{\hyperref[\detokenize{generated/sdapy.models.magnetar.general_magnetar_slsn_bolometric:sdapy.models.magnetar.general_magnetar_slsn_bolometric}]{\sphinxcrossref{\sphinxcode{\sphinxupquote{general\_magnetar\_slsn\_bolometric}}}}}(time, l0, …)
&
\begin{quote}\begin{description}
\item[{param time}] \leavevmode
time in days in source frame

\end{description}\end{quote}

\\
\hline
\end{longtable}\sphinxatlongtableend\end{savenotes}


\subsubsection{sdapy.models.magnetar.basic\_magnetar}
\label{\detokenize{generated/sdapy.models.magnetar.basic_magnetar:sdapy-models-magnetar-basic-magnetar}}\label{\detokenize{generated/sdapy.models.magnetar.basic_magnetar::doc}}\index{basic\_magnetar() (in module sdapy.models.magnetar)@\spxentry{basic\_magnetar()}\spxextra{in module sdapy.models.magnetar}}

\begin{fulllineitems}
\phantomsection\label{\detokenize{generated/sdapy.models.magnetar.basic_magnetar:sdapy.models.magnetar.basic_magnetar}}\pysiglinewithargsret{\sphinxcode{\sphinxupquote{sdapy.models.magnetar.}}\sphinxbfcode{\sphinxupquote{basic\_magnetar}}}{\emph{time}, \emph{p0}, \emph{bp}, \emph{mass\_ns}, \emph{theta\_pb}}{}
\sphinxhref{https://ui.adsabs.harvard.edu/abs/2006ApJ...648L..51S/abstract}{https://ui.adsabs.harvard.edu/abs/2006ApJ…648L..51S/abstract}
\begin{quote}\begin{description}
\item[{Parameters}] \leavevmode\begin{itemize}
\item {} 
\sphinxstyleliteralstrong{\sphinxupquote{time}} \textendash{} time in seconds in source frame

\item {} 
\sphinxstyleliteralstrong{\sphinxupquote{p0}} \textendash{} initial spin period in seconds

\item {} 
\sphinxstyleliteralstrong{\sphinxupquote{bp}} \textendash{} polar magnetic field strength in Gauss

\item {} 
\sphinxstyleliteralstrong{\sphinxupquote{mass\_ns}} \textendash{} mass of neutron star in solar masses

\item {} 
\sphinxstyleliteralstrong{\sphinxupquote{theta\_pb}} \textendash{} angle between spin and magnetic field axes

\end{itemize}

\item[{Returns}] \leavevmode
luminosity

\end{description}\end{quote}

\end{fulllineitems}



\subsubsection{sdapy.models.magnetar.magnetar\_only}
\label{\detokenize{generated/sdapy.models.magnetar.magnetar_only:sdapy-models-magnetar-magnetar-only}}\label{\detokenize{generated/sdapy.models.magnetar.magnetar_only::doc}}\index{magnetar\_only() (in module sdapy.models.magnetar)@\spxentry{magnetar\_only()}\spxextra{in module sdapy.models.magnetar}}

\begin{fulllineitems}
\phantomsection\label{\detokenize{generated/sdapy.models.magnetar.magnetar_only:sdapy.models.magnetar.magnetar_only}}\pysiglinewithargsret{\sphinxcode{\sphinxupquote{sdapy.models.magnetar.}}\sphinxbfcode{\sphinxupquote{magnetar\_only}}}{\emph{time}, \emph{l0}, \emph{tau}, \emph{nn}}{}
\sphinxhref{https://ui.adsabs.harvard.edu/abs/2017ApJ...843L...1L/abstract}{https://ui.adsabs.harvard.edu/abs/2017ApJ…843L…1L/abstract}
\begin{quote}\begin{description}
\item[{Parameters}] \leavevmode\begin{itemize}
\item {} 
\sphinxstyleliteralstrong{\sphinxupquote{time}} \textendash{} time in seconds

\item {} 
\sphinxstyleliteralstrong{\sphinxupquote{l0}} \textendash{} initial luminosity parameter

\item {} 
\sphinxstyleliteralstrong{\sphinxupquote{tau}} \textendash{} spin\sphinxhyphen{}down damping timescale

\item {} 
\sphinxstyleliteralstrong{\sphinxupquote{nn}} \textendash{} braking index

\end{itemize}

\item[{Returns}] \leavevmode
luminosity or flux (depending on scaling of l0) as a function of time.

\end{description}\end{quote}

\end{fulllineitems}



\subsubsection{sdapy.models.magnetar.basic\_magnetar\_powered\_bolometric}
\label{\detokenize{generated/sdapy.models.magnetar.basic_magnetar_powered_bolometric:sdapy-models-magnetar-basic-magnetar-powered-bolometric}}\label{\detokenize{generated/sdapy.models.magnetar.basic_magnetar_powered_bolometric::doc}}\index{basic\_magnetar\_powered\_bolometric() (in module sdapy.models.magnetar)@\spxentry{basic\_magnetar\_powered\_bolometric()}\spxextra{in module sdapy.models.magnetar}}

\begin{fulllineitems}
\phantomsection\label{\detokenize{generated/sdapy.models.magnetar.basic_magnetar_powered_bolometric:sdapy.models.magnetar.basic_magnetar_powered_bolometric}}\pysiglinewithargsret{\sphinxcode{\sphinxupquote{sdapy.models.magnetar.}}\sphinxbfcode{\sphinxupquote{basic\_magnetar\_powered\_bolometric}}}{\emph{times}, \emph{p0}, \emph{bp}, \emph{mass\_ns}, \emph{theta\_pb}, \emph{mej}, \emph{ek}, \emph{texp=None}, \emph{k\_opt=None}, \emph{k\_gamma=None}}{}~\begin{quote}\begin{description}
\item[{Parameters}] \leavevmode\begin{itemize}
\item {} 
\sphinxstyleliteralstrong{\sphinxupquote{times}} \textendash{} time in days in source frame

\item {} 
\sphinxstyleliteralstrong{\sphinxupquote{p0}} \textendash{} initial spin period

\item {} 
\sphinxstyleliteralstrong{\sphinxupquote{bp}} \textendash{} polar magnetic field strength in Gauss

\item {} 
\sphinxstyleliteralstrong{\sphinxupquote{mass\_ns}} \textendash{} mass of neutron star in solar masses

\item {} 
\sphinxstyleliteralstrong{\sphinxupquote{theta\_pb}} \textendash{} angle between spin and magnetic field axes

\item {} 
\sphinxstyleliteralstrong{\sphinxupquote{mej}} \textendash{} total ejecta mass in solar masses

\item {} 
\sphinxstyleliteralstrong{\sphinxupquote{ek}} \textendash{} total kinetic energy in foe

\end{itemize}

\item[{Texp}] \leavevmode
explosion epoch, unit in days

\item[{K\_opt}] \leavevmode
optical opacity

\item[{K\_gamma}] \leavevmode
gamma ray opacity

\item[{Returns}] \leavevmode
bolometric\_luminosity

\end{description}\end{quote}

\end{fulllineitems}



\subsubsection{sdapy.models.magnetar.general\_magnetar\_slsn\_bolometric}
\label{\detokenize{generated/sdapy.models.magnetar.general_magnetar_slsn_bolometric:sdapy-models-magnetar-general-magnetar-slsn-bolometric}}\label{\detokenize{generated/sdapy.models.magnetar.general_magnetar_slsn_bolometric::doc}}\index{general\_magnetar\_slsn\_bolometric() (in module sdapy.models.magnetar)@\spxentry{general\_magnetar\_slsn\_bolometric()}\spxextra{in module sdapy.models.magnetar}}

\begin{fulllineitems}
\phantomsection\label{\detokenize{generated/sdapy.models.magnetar.general_magnetar_slsn_bolometric:sdapy.models.magnetar.general_magnetar_slsn_bolometric}}\pysiglinewithargsret{\sphinxcode{\sphinxupquote{sdapy.models.magnetar.}}\sphinxbfcode{\sphinxupquote{general\_magnetar\_slsn\_bolometric}}}{\emph{time}, \emph{l0}, \emph{tsd}, \emph{nn}, \emph{mej}, \emph{ek}, \emph{k\_opt=None}, \emph{k\_gamma=None}}{}~\begin{quote}\begin{description}
\item[{Parameters}] \leavevmode\begin{itemize}
\item {} 
\sphinxstyleliteralstrong{\sphinxupquote{time}} \textendash{} time in days in source frame

\item {} 
\sphinxstyleliteralstrong{\sphinxupquote{l0}} \textendash{} magnetar energy normalisation in ergs

\item {} 
\sphinxstyleliteralstrong{\sphinxupquote{tsd}} \textendash{} magnetar spin down damping timescale in source frame days

\item {} 
\sphinxstyleliteralstrong{\sphinxupquote{nn}} \textendash{} braking index

\item {} 
\sphinxstyleliteralstrong{\sphinxupquote{mej}} \textendash{} total ejecta mass in solar masses

\item {} 
\sphinxstyleliteralstrong{\sphinxupquote{ek}} \textendash{} total kinetic energy in foe

\end{itemize}

\item[{K\_opt}] \leavevmode
optical opacity

\item[{K\_gamma}] \leavevmode
gamma ray opacity

\item[{Returns}] \leavevmode
bolometric\_luminosity

\end{description}\end{quote}

\end{fulllineitems}


\sphinxstylestrong{bol\_main \sphinxhyphen{}\textgreater{} CSM Models}


\begin{savenotes}\sphinxatlongtablestart\begin{longtable}[c]{\X{1}{2}\X{1}{2}}
\hline

\endfirsthead

\multicolumn{2}{c}%
{\makebox[0pt]{\sphinxtablecontinued{\tablename\ \thetable{} \textendash{} continued from previous page}}}\\
\hline

\endhead

\hline
\multicolumn{2}{r}{\makebox[0pt][r]{\sphinxtablecontinued{Continued on next page}}}\\
\endfoot

\endlastfoot

{\hyperref[\detokenize{generated/sdapy.models.csm.get_csm_properties:sdapy.models.csm.get_csm_properties}]{\sphinxcrossref{\sphinxcode{\sphinxupquote{get\_csm\_properties}}}}}(nn, eta)
&

\\
\hline
\sphinxcode{\sphinxupquote{\_csm\_engine}}
&

\\
\hline
{\hyperref[\detokenize{generated/sdapy.models.csm.csm_interaction_bolometric:sdapy.models.csm.csm_interaction_bolometric}]{\sphinxcrossref{\sphinxcode{\sphinxupquote{csm\_interaction\_bolometric}}}}}(times, mej, …)
&
\begin{quote}\begin{description}
\item[{param times}] \leavevmode
time in days in source frame

\end{description}\end{quote}

\\
\hline
\end{longtable}\sphinxatlongtableend\end{savenotes}


\subsubsection{sdapy.models.csm.get\_csm\_properties}
\label{\detokenize{generated/sdapy.models.csm.get_csm_properties:sdapy-models-csm-get-csm-properties}}\label{\detokenize{generated/sdapy.models.csm.get_csm_properties::doc}}\index{get\_csm\_properties() (in module sdapy.models.csm)@\spxentry{get\_csm\_properties()}\spxextra{in module sdapy.models.csm}}

\begin{fulllineitems}
\phantomsection\label{\detokenize{generated/sdapy.models.csm.get_csm_properties:sdapy.models.csm.get_csm_properties}}\pysiglinewithargsret{\sphinxcode{\sphinxupquote{sdapy.models.csm.}}\sphinxbfcode{\sphinxupquote{get\_csm\_properties}}}{\emph{nn}, \emph{eta}}{}~
\end{fulllineitems}



\subsubsection{sdapy.models.csm.csm\_interaction\_bolometric}
\label{\detokenize{generated/sdapy.models.csm.csm_interaction_bolometric:sdapy-models-csm-csm-interaction-bolometric}}\label{\detokenize{generated/sdapy.models.csm.csm_interaction_bolometric::doc}}\index{csm\_interaction\_bolometric() (in module sdapy.models.csm)@\spxentry{csm\_interaction\_bolometric()}\spxextra{in module sdapy.models.csm}}

\begin{fulllineitems}
\phantomsection\label{\detokenize{generated/sdapy.models.csm.csm_interaction_bolometric:sdapy.models.csm.csm_interaction_bolometric}}\pysiglinewithargsret{\sphinxcode{\sphinxupquote{sdapy.models.csm.}}\sphinxbfcode{\sphinxupquote{csm\_interaction\_bolometric}}}{\emph{times}, \emph{mej}, \emph{csm\_mass}, \emph{vej}, \emph{eta}, \emph{rho}, \emph{kappa}, \emph{r0}, \emph{texp=None}}{}~\begin{quote}\begin{description}
\item[{Parameters}] \leavevmode\begin{itemize}
\item {} 
\sphinxstyleliteralstrong{\sphinxupquote{times}} \textendash{} time in days in source frame

\item {} 
\sphinxstyleliteralstrong{\sphinxupquote{mej}} \textendash{} ejecta mass in solar masses

\item {} 
\sphinxstyleliteralstrong{\sphinxupquote{csm\_mass}} \textendash{} csm mass in solar masses

\item {} 
\sphinxstyleliteralstrong{\sphinxupquote{vej}} \textendash{} ejecta velocity in 1000 km/s

\item {} 
\sphinxstyleliteralstrong{\sphinxupquote{eta}} \textendash{} csm density profile exponent

\item {} 
\sphinxstyleliteralstrong{\sphinxupquote{rho}} \textendash{} csm density profile amplitude

\item {} 
\sphinxstyleliteralstrong{\sphinxupquote{kappa}} \textendash{} opacity

\item {} 
\sphinxstyleliteralstrong{\sphinxupquote{r0}} \textendash{} radius of csm shell in AU

\item {} 
\sphinxstyleliteralstrong{\sphinxupquote{texp}} \textendash{} time in days, time between first light to the peak epoch

\end{itemize}

\item[{Returns}] \leavevmode
bolometric\_luminosity

\end{description}\end{quote}

\end{fulllineitems}


\sphinxstylestrong{bol\_tail \sphinxhyphen{}\textgreater{} Tail Models}


\begin{savenotes}\sphinxatlongtablestart\begin{longtable}[c]{\X{1}{2}\X{1}{2}}
\hline

\endfirsthead

\multicolumn{2}{c}%
{\makebox[0pt]{\sphinxtablecontinued{\tablename\ \thetable{} \textendash{} continued from previous page}}}\\
\hline

\endhead

\hline
\multicolumn{2}{r}{\makebox[0pt][r]{\sphinxtablecontinued{Continued on next page}}}\\
\endfoot

\endlastfoot

{\hyperref[\detokenize{generated/sdapy.models.arnett_tail.tail_fit_t0:sdapy.models.arnett_tail.tail_fit_t0}]{\sphinxcrossref{\sphinxcode{\sphinxupquote{tail\_fit\_t0}}}}}(t, mni, t0)
&
fit radioactive tail with Wygoda 2019 Eq 10, 11 and 12 (\sphinxurl{https://arxiv.org/pdf/1711.00969.pdf})
\\
\hline
{\hyperref[\detokenize{generated/sdapy.models.arnett_tail.t0_to_Mej_Ek:sdapy.models.arnett_tail.t0_to_Mej_Ek}]{\sphinxcrossref{\sphinxcode{\sphinxupquote{t0\_to\_Mej\_Ek}}}}}(…)
&
from tail characteristic time, t0 to kinetic energy and ejecta mass seperately t0 = sqrt(c*kgamma*Mej**2/Ekin) vej = sqrt(2*Ekin/Mej)
\\
\hline
{\hyperref[\detokenize{generated/sdapy.models.arnett_tail.Mej_Ek_to_t0:sdapy.models.arnett_tail.Mej_Ek_to_t0}]{\sphinxcrossref{\sphinxcode{\sphinxupquote{Mej\_Ek\_to\_t0}}}}}(Mej, Ek{[}, k\_gamma{]})
&
from kinetic energy and ejecta mass to t0
\\
\hline
{\hyperref[\detokenize{generated/sdapy.models.arnett_tail.tail_fit_Mej_Ek:sdapy.models.arnett_tail.tail_fit_Mej_Ek}]{\sphinxcrossref{\sphinxcode{\sphinxupquote{tail\_fit\_Mej\_Ek}}}}}(times, f\_ni, Ek, Mej{[}, k\_gamma{]})
&
fit tail model on Ek, Mej, with photospheric velocity, vm as free parameter as well
\\
\hline
\sphinxcode{\sphinxupquote{tail\_fit\_t0\_ts}}
&

\\
\hline
\sphinxcode{\sphinxupquote{tail\_fit\_Mej\_Ek\_ts}}
&

\\
\hline
\end{longtable}\sphinxatlongtableend\end{savenotes}


\subsubsection{sdapy.models.arnett\_tail.tail\_fit\_t0}
\label{\detokenize{generated/sdapy.models.arnett_tail.tail_fit_t0:sdapy-models-arnett-tail-tail-fit-t0}}\label{\detokenize{generated/sdapy.models.arnett_tail.tail_fit_t0::doc}}\index{tail\_fit\_t0() (in module sdapy.models.arnett\_tail)@\spxentry{tail\_fit\_t0()}\spxextra{in module sdapy.models.arnett\_tail}}

\begin{fulllineitems}
\phantomsection\label{\detokenize{generated/sdapy.models.arnett_tail.tail_fit_t0:sdapy.models.arnett_tail.tail_fit_t0}}\pysiglinewithargsret{\sphinxcode{\sphinxupquote{sdapy.models.arnett\_tail.}}\sphinxbfcode{\sphinxupquote{tail\_fit\_t0}}}{\emph{t}, \emph{mni}, \emph{t0}}{}
fit radioactive tail with Wygoda 2019 Eq 10, 11 and 12
(\sphinxurl{https://arxiv.org/pdf/1711.00969.pdf})
\begin{quote}\begin{description}
\item[{Parameters}] \leavevmode\begin{description}
\item[{\sphinxstylestrong{t}}] \leavevmode{[}\sphinxtitleref{array}{]}
Independent values.

\item[{\sphinxstylestrong{m\_ni}}] \leavevmode{[}\sphinxtitleref{float}{]}
Tail model parameter, Nickel mass, unit in solar mass.

\item[{\sphinxstylestrong{t0}}] \leavevmode{[}\sphinxtitleref{float}{]}
Tail model parameter, characteristic time, unit in days, decided by ejecta mass and kinetic enegies.

\end{description}

\end{description}\end{quote}

\end{fulllineitems}



\subsubsection{sdapy.models.arnett\_tail.t0\_to\_Mej\_Ek}
\label{\detokenize{generated/sdapy.models.arnett_tail.t0_to_Mej_Ek:sdapy-models-arnett-tail-t0-to-mej-ek}}\label{\detokenize{generated/sdapy.models.arnett_tail.t0_to_Mej_Ek::doc}}\index{t0\_to\_Mej\_Ek() (in module sdapy.models.arnett\_tail)@\spxentry{t0\_to\_Mej\_Ek()}\spxextra{in module sdapy.models.arnett\_tail}}

\begin{fulllineitems}
\phantomsection\label{\detokenize{generated/sdapy.models.arnett_tail.t0_to_Mej_Ek:sdapy.models.arnett_tail.t0_to_Mej_Ek}}\pysiglinewithargsret{\sphinxcode{\sphinxupquote{sdapy.models.arnett\_tail.}}\sphinxbfcode{\sphinxupquote{t0\_to\_Mej\_Ek}}}{\emph{c*kgamma*Mej**2/Ekin) vej = sqrt(2*Ekin/Mej}}{}
from tail characteristic time, t0 to kinetic energy and ejecta mass seperately
t0 = sqrt(c*kgamma*Mej**2/Ekin)
vej = sqrt(2*Ekin/Mej)
\begin{quote}\begin{description}
\item[{Parameters}] \leavevmode\begin{description}
\item[{\sphinxstylestrong{t\_0}}] \leavevmode{[}\sphinxtitleref{float}{]}
Tail model parameter, characteristic time, unit in days, decided by ejecta mass and kinetic enegies.

\item[{\sphinxstylestrong{v\_ej}}] \leavevmode{[}\sphinxtitleref{float}{]}
here vej is the photospheric velocity (unit: 10**3 km/s), which can be assumed as line velocities.

\item[{\sphinxstylestrong{t0\_err}}] \leavevmode{[}\sphinxtitleref{float}{]}
Error of t0.

\item[{\sphinxstylestrong{vej\_err}}] \leavevmode{[}\sphinxtitleref{float}{]}
Error of vej.

\item[{\sphinxstylestrong{k\_gamma}}] \leavevmode{[}\sphinxtitleref{float}{]}
gamma ray opacity

\end{description}

\end{description}\end{quote}

\end{fulllineitems}



\subsubsection{sdapy.models.arnett\_tail.Mej\_Ek\_to\_t0}
\label{\detokenize{generated/sdapy.models.arnett_tail.Mej_Ek_to_t0:sdapy-models-arnett-tail-mej-ek-to-t0}}\label{\detokenize{generated/sdapy.models.arnett_tail.Mej_Ek_to_t0::doc}}\index{Mej\_Ek\_to\_t0() (in module sdapy.models.arnett\_tail)@\spxentry{Mej\_Ek\_to\_t0()}\spxextra{in module sdapy.models.arnett\_tail}}

\begin{fulllineitems}
\phantomsection\label{\detokenize{generated/sdapy.models.arnett_tail.Mej_Ek_to_t0:sdapy.models.arnett_tail.Mej_Ek_to_t0}}\pysiglinewithargsret{\sphinxcode{\sphinxupquote{sdapy.models.arnett\_tail.}}\sphinxbfcode{\sphinxupquote{Mej\_Ek\_to\_t0}}}{\emph{Mej}, \emph{Ek}, \emph{k\_gamma=None}}{}
from kinetic energy and ejecta mass to t0
\begin{quote}\begin{description}
\item[{Parameters}] \leavevmode\begin{description}
\item[{\sphinxstylestrong{Mej}}] \leavevmode{[}\sphinxtitleref{float}{]}
ejecta mass (unit: solar mass)

\item[{\sphinxstylestrong{Ek}}] \leavevmode{[}\sphinxtitleref{float}{]}
kinetic energy (unit: foe)

\item[{\sphinxstylestrong{k\_gamma}}] \leavevmode{[}\sphinxtitleref{float}{]}
gamma ray opacity

\end{description}

\end{description}\end{quote}

\end{fulllineitems}



\subsubsection{sdapy.models.arnett\_tail.tail\_fit\_Mej\_Ek}
\label{\detokenize{generated/sdapy.models.arnett_tail.tail_fit_Mej_Ek:sdapy-models-arnett-tail-tail-fit-mej-ek}}\label{\detokenize{generated/sdapy.models.arnett_tail.tail_fit_Mej_Ek::doc}}\index{tail\_fit\_Mej\_Ek() (in module sdapy.models.arnett\_tail)@\spxentry{tail\_fit\_Mej\_Ek()}\spxextra{in module sdapy.models.arnett\_tail}}

\begin{fulllineitems}
\phantomsection\label{\detokenize{generated/sdapy.models.arnett_tail.tail_fit_Mej_Ek:sdapy.models.arnett_tail.tail_fit_Mej_Ek}}\pysiglinewithargsret{\sphinxcode{\sphinxupquote{sdapy.models.arnett\_tail.}}\sphinxbfcode{\sphinxupquote{tail\_fit\_Mej\_Ek}}}{\emph{times}, \emph{f\_ni}, \emph{Ek}, \emph{Mej}, \emph{k\_gamma=None}}{}
fit tail model on Ek, Mej, with photospheric velocity, vm as free parameter as well
\begin{quote}\begin{description}
\item[{Parameters}] \leavevmode\begin{description}
\item[{\sphinxstylestrong{time}}] \leavevmode{[}\sphinxtitleref{array}{]}
Independent values.

\item[{\sphinxstylestrong{f\_ni}}] \leavevmode{[}\sphinxtitleref{float}{]}
fraction of nickel mass

\item[{\sphinxstylestrong{Mej}}] \leavevmode{[}\sphinxtitleref{float}{]}
ejecta mass (unit: solar mass)

\item[{\sphinxstylestrong{Ek}}] \leavevmode{[}\sphinxtitleref{float}{]}
kinetic energy (unit: foe)

\item[{\sphinxstylestrong{k\_gamma}}] \leavevmode{[}\sphinxtitleref{float}{]}
gamma ray opacity

\end{description}

\end{description}\end{quote}

\end{fulllineitems}


\sphinxstylestrong{bol\_full \sphinxhyphen{}\textgreater{} Arnett+Tail Models}


\begin{savenotes}\sphinxatlongtablestart\begin{longtable}[c]{\X{1}{2}\X{1}{2}}
\hline

\endfirsthead

\multicolumn{2}{c}%
{\makebox[0pt]{\sphinxtablecontinued{\tablename\ \thetable{} \textendash{} continued from previous page}}}\\
\hline

\endhead

\hline
\multicolumn{2}{r}{\makebox[0pt][r]{\sphinxtablecontinued{Continued on next page}}}\\
\endfoot

\endlastfoot

{\hyperref[\detokenize{generated/sdapy.models.arnett_tail.joint_fit_taum_t0:sdapy.models.arnett_tail.joint_fit_taum_t0}]{\sphinxcrossref{\sphinxcode{\sphinxupquote{joint\_fit\_taum\_t0}}}}}(times, mni, taum, t0, ts)
&
for the full range, \sphinxstyleemphasis{Arnett\_fit\_taum} fit + \sphinxstyleemphasis{tail\_fit\_t0} fit, with ts as free parameter
\\
\hline
\sphinxcode{\sphinxupquote{joint\_fit\_taum\_t0\_texp}}
&

\\
\hline
{\hyperref[\detokenize{generated/sdapy.models.arnett_tail.joint_fit_Mej_Ek:sdapy.models.arnett_tail.joint_fit_Mej_Ek}]{\sphinxcrossref{\sphinxcode{\sphinxupquote{joint\_fit\_Mej\_Ek}}}}}(times, mni, Mej, Ek, ts{[}, …{]})
&
for the full range, \sphinxstyleemphasis{Arnett\_fit\_Mej\_Ek} fit + \sphinxstyleemphasis{tail\_fit\_Mej\_Ek} fit, with ts as free parameter
\\
\hline
\sphinxcode{\sphinxupquote{joint\_fit\_Mej\_Ek\_texp}}
&

\\
\hline
\end{longtable}\sphinxatlongtableend\end{savenotes}


\subsubsection{sdapy.models.arnett\_tail.joint\_fit\_taum\_t0}
\label{\detokenize{generated/sdapy.models.arnett_tail.joint_fit_taum_t0:sdapy-models-arnett-tail-joint-fit-taum-t0}}\label{\detokenize{generated/sdapy.models.arnett_tail.joint_fit_taum_t0::doc}}\index{joint\_fit\_taum\_t0() (in module sdapy.models.arnett\_tail)@\spxentry{joint\_fit\_taum\_t0()}\spxextra{in module sdapy.models.arnett\_tail}}

\begin{fulllineitems}
\phantomsection\label{\detokenize{generated/sdapy.models.arnett_tail.joint_fit_taum_t0:sdapy.models.arnett_tail.joint_fit_taum_t0}}\pysiglinewithargsret{\sphinxcode{\sphinxupquote{sdapy.models.arnett\_tail.}}\sphinxbfcode{\sphinxupquote{joint\_fit\_taum\_t0}}}{\emph{times}, \emph{mni}, \emph{taum}, \emph{t0}, \emph{ts}, \emph{texp=None}}{}
for the full range, \sphinxstyleemphasis{Arnett\_fit\_taum} fit + \sphinxstyleemphasis{tail\_fit\_t0} fit, with ts as free parameter
\begin{quote}\begin{description}
\item[{Parameters}] \leavevmode\begin{description}
\item[{\sphinxstylestrong{time}}] \leavevmode{[}\sphinxtitleref{array}{]}
Independent values.

\item[{\sphinxstylestrong{m\_ni}}] \leavevmode{[}\sphinxtitleref{float}{]}
Arnett model parameter, Nickel mass, unit in solar mass.

\item[{\sphinxstylestrong{taum}}] \leavevmode{[}\sphinxtitleref{float}{]}
Arnett model parameter, characteristic time, unit in days, decided by ejecta mass and kinetic enegies.

\item[{\sphinxstylestrong{t0}}] \leavevmode{[}\sphinxtitleref{float}{]}
Tail model parameter, characteristic time, unit in days, decided by ejecta mass and kinetic enegies.

\item[{\sphinxstylestrong{ts}}] \leavevmode{[}\sphinxtitleref{float}{]}
boundary between Arnett model and tail model

\item[{\sphinxstylestrong{texp}}] \leavevmode{[}\sphinxtitleref{float}{]}
explosion time, time between first light to the peak epoch.

\end{description}

\end{description}\end{quote}

\end{fulllineitems}



\subsubsection{sdapy.models.arnett\_tail.joint\_fit\_Mej\_Ek}
\label{\detokenize{generated/sdapy.models.arnett_tail.joint_fit_Mej_Ek:sdapy-models-arnett-tail-joint-fit-mej-ek}}\label{\detokenize{generated/sdapy.models.arnett_tail.joint_fit_Mej_Ek::doc}}\index{joint\_fit\_Mej\_Ek() (in module sdapy.models.arnett\_tail)@\spxentry{joint\_fit\_Mej\_Ek()}\spxextra{in module sdapy.models.arnett\_tail}}

\begin{fulllineitems}
\phantomsection\label{\detokenize{generated/sdapy.models.arnett_tail.joint_fit_Mej_Ek:sdapy.models.arnett_tail.joint_fit_Mej_Ek}}\pysiglinewithargsret{\sphinxcode{\sphinxupquote{sdapy.models.arnett\_tail.}}\sphinxbfcode{\sphinxupquote{joint\_fit\_Mej\_Ek}}}{\emph{times}, \emph{mni}, \emph{Mej}, \emph{Ek}, \emph{ts}, \emph{texp=None}, \emph{k\_opt=None}}{}
for the full range, \sphinxstyleemphasis{Arnett\_fit\_Mej\_Ek} fit + \sphinxstyleemphasis{tail\_fit\_Mej\_Ek} fit, with ts as free parameter
\begin{quote}\begin{description}
\item[{Parameters}] \leavevmode\begin{description}
\item[{\sphinxstylestrong{time}}] \leavevmode{[}\sphinxtitleref{array}{]}
Independent values.

\item[{\sphinxstylestrong{m\_ni}}] \leavevmode{[}\sphinxtitleref{float}{]}
Arnett model parameter, Nickel mass, unit in solar mass.

\item[{\sphinxstylestrong{Mej}}] \leavevmode{[}\sphinxtitleref{float}{]}
ejecta mass (unit: solar mass)

\item[{\sphinxstylestrong{Ek}}] \leavevmode{[}\sphinxtitleref{float}{]}
kinetic energy (unit: foe)

\item[{\sphinxstylestrong{ts}}] \leavevmode{[}\sphinxtitleref{float}{]}
boundary between Arnett model and tail model

\item[{\sphinxstylestrong{texp}}] \leavevmode{[}\sphinxtitleref{float}{]}
explosion time, time between first light to the peak epoch.

\end{description}

\end{description}\end{quote}

\end{fulllineitems}


\sphinxstylestrong{bol engines \sphinxhyphen{}\textgreater{} Diffusion models}


\begin{savenotes}\sphinxatlongtablestart\begin{longtable}[c]{\X{1}{2}\X{1}{2}}
\hline

\endfirsthead

\multicolumn{2}{c}%
{\makebox[0pt]{\sphinxtablecontinued{\tablename\ \thetable{} \textendash{} continued from previous page}}}\\
\hline

\endhead

\hline
\multicolumn{2}{r}{\makebox[0pt][r]{\sphinxtablecontinued{Continued on next page}}}\\
\endfoot

\endlastfoot

{\hyperref[\detokenize{generated/sdapy.interaction_processes.Diffusion:sdapy.interaction_processes.Diffusion}]{\sphinxcrossref{\sphinxcode{\sphinxupquote{Diffusion}}}}}(time, luminosity, kappa, …)
&
\subsubsection*{Methods}

\\
\hline
{\hyperref[\detokenize{generated/sdapy.interaction_processes.AsphericalDiffusion:sdapy.interaction_processes.AsphericalDiffusion}]{\sphinxcrossref{\sphinxcode{\sphinxupquote{AsphericalDiffusion}}}}}(time, luminosity, kappa, …)
&
\subsubsection*{Methods}

\\
\hline
{\hyperref[\detokenize{generated/sdapy.interaction_processes.CSMDiffusion:sdapy.interaction_processes.CSMDiffusion}]{\sphinxcrossref{\sphinxcode{\sphinxupquote{CSMDiffusion}}}}}(time, luminosity, kappa, …)
&
\subsubsection*{Methods}

\\
\hline
{\hyperref[\detokenize{generated/sdapy.interaction_processes.Viscous:sdapy.interaction_processes.Viscous}]{\sphinxcrossref{\sphinxcode{\sphinxupquote{Viscous}}}}}(time, luminosity, t\_viscous)
&
\subsubsection*{Methods}

\\
\hline
\end{longtable}\sphinxatlongtableend\end{savenotes}


\subsubsection{sdapy.interaction\_processes.Diffusion}
\label{\detokenize{generated/sdapy.interaction_processes.Diffusion:sdapy-interaction-processes-diffusion}}\label{\detokenize{generated/sdapy.interaction_processes.Diffusion::doc}}\index{Diffusion (class in sdapy.interaction\_processes)@\spxentry{Diffusion}\spxextra{class in sdapy.interaction\_processes}}

\begin{fulllineitems}
\phantomsection\label{\detokenize{generated/sdapy.interaction_processes.Diffusion:sdapy.interaction_processes.Diffusion}}\pysiglinewithargsret{\sphinxbfcode{\sphinxupquote{class }}\sphinxcode{\sphinxupquote{sdapy.interaction\_processes.}}\sphinxbfcode{\sphinxupquote{Diffusion}}}{\emph{time}, \emph{luminosity}, \emph{kappa}, \emph{kappa\_gamma}, \emph{mej}, \emph{vej}}{}~\subsubsection*{Methods}


\begin{savenotes}\sphinxattablestart
\centering
\begin{tabulary}{\linewidth}[t]{|T|T|}
\hline

\sphinxstylestrong{convert\_input\_luminosity}
&\\
\hline
\end{tabulary}
\par
\sphinxattableend\end{savenotes}
\index{\_\_init\_\_() (sdapy.interaction\_processes.Diffusion method)@\spxentry{\_\_init\_\_()}\spxextra{sdapy.interaction\_processes.Diffusion method}}

\begin{fulllineitems}
\phantomsection\label{\detokenize{generated/sdapy.interaction_processes.Diffusion:sdapy.interaction_processes.Diffusion.__init__}}\pysiglinewithargsret{\sphinxbfcode{\sphinxupquote{\_\_init\_\_}}}{\emph{self}, \emph{time}, \emph{luminosity}, \emph{kappa}, \emph{kappa\_gamma}, \emph{mej}, \emph{vej}}{}~\begin{quote}\begin{description}
\item[{Parameters}] \leavevmode\begin{itemize}
\item {} 
\sphinxstyleliteralstrong{\sphinxupquote{time}} \textendash{} source frame time in days

\item {} 
\sphinxstyleliteralstrong{\sphinxupquote{luminosity}} \textendash{} luminosity

\item {} 
\sphinxstyleliteralstrong{\sphinxupquote{kappa}} \textendash{} opacity

\item {} 
\sphinxstyleliteralstrong{\sphinxupquote{kappa\_gamma}} \textendash{} gamma\sphinxhyphen{}ray opacity

\item {} 
\sphinxstyleliteralstrong{\sphinxupquote{mej}} \textendash{} ejecta mass, unit in solar mass

\item {} 
\sphinxstyleliteralstrong{\sphinxupquote{vej}} \textendash{} ejecta velocity, unit in km/s

\end{itemize}

\end{description}\end{quote}

Adds new attributes for tau\_diffusion and new luminosity accounting for the interaction process

\end{fulllineitems}

\subsubsection*{Methods}


\begin{savenotes}\sphinxatlongtablestart\begin{longtable}[c]{\X{1}{2}\X{1}{2}}
\hline

\endfirsthead

\multicolumn{2}{c}%
{\makebox[0pt]{\sphinxtablecontinued{\tablename\ \thetable{} \textendash{} continued from previous page}}}\\
\hline

\endhead

\hline
\multicolumn{2}{r}{\makebox[0pt][r]{\sphinxtablecontinued{Continued on next page}}}\\
\endfoot

\endlastfoot

{\hyperref[\detokenize{generated/sdapy.interaction_processes.Diffusion:sdapy.interaction_processes.Diffusion.__init__}]{\sphinxcrossref{\sphinxcode{\sphinxupquote{\_\_init\_\_}}}}}(self, time, luminosity, kappa, …)
&
\begin{quote}\begin{description}
\item[{param time}] \leavevmode
source frame time in days

\end{description}\end{quote}

\\
\hline
\sphinxcode{\sphinxupquote{convert\_input\_luminosity}}(self)
&

\\
\hline
\end{longtable}\sphinxatlongtableend\end{savenotes}

\end{fulllineitems}



\subsubsection{sdapy.interaction\_processes.AsphericalDiffusion}
\label{\detokenize{generated/sdapy.interaction_processes.AsphericalDiffusion:sdapy-interaction-processes-asphericaldiffusion}}\label{\detokenize{generated/sdapy.interaction_processes.AsphericalDiffusion::doc}}\index{AsphericalDiffusion (class in sdapy.interaction\_processes)@\spxentry{AsphericalDiffusion}\spxextra{class in sdapy.interaction\_processes}}

\begin{fulllineitems}
\phantomsection\label{\detokenize{generated/sdapy.interaction_processes.AsphericalDiffusion:sdapy.interaction_processes.AsphericalDiffusion}}\pysiglinewithargsret{\sphinxbfcode{\sphinxupquote{class }}\sphinxcode{\sphinxupquote{sdapy.interaction\_processes.}}\sphinxbfcode{\sphinxupquote{AsphericalDiffusion}}}{\emph{time}, \emph{luminosity}, \emph{kappa}, \emph{kappa\_gamma}, \emph{mej}, \emph{vej}, \emph{area\_projection}, \emph{area\_reference}}{}~\subsubsection*{Methods}


\begin{savenotes}\sphinxattablestart
\centering
\begin{tabulary}{\linewidth}[t]{|T|T|}
\hline

\sphinxstylestrong{convert\_input\_luminosity}
&\\
\hline
\end{tabulary}
\par
\sphinxattableend\end{savenotes}
\index{\_\_init\_\_() (sdapy.interaction\_processes.AsphericalDiffusion method)@\spxentry{\_\_init\_\_()}\spxextra{sdapy.interaction\_processes.AsphericalDiffusion method}}

\begin{fulllineitems}
\phantomsection\label{\detokenize{generated/sdapy.interaction_processes.AsphericalDiffusion:sdapy.interaction_processes.AsphericalDiffusion.__init__}}\pysiglinewithargsret{\sphinxbfcode{\sphinxupquote{\_\_init\_\_}}}{\emph{self}, \emph{time}, \emph{luminosity}, \emph{kappa}, \emph{kappa\_gamma}, \emph{mej}, \emph{vej}, \emph{area\_projection}, \emph{area\_reference}}{}~\begin{quote}\begin{description}
\item[{Parameters}] \leavevmode\begin{itemize}
\item {} 
\sphinxstyleliteralstrong{\sphinxupquote{time}} \textendash{} source frame time in days

\item {} 
\sphinxstyleliteralstrong{\sphinxupquote{luminosity}} \textendash{} luminosity

\item {} 
\sphinxstyleliteralstrong{\sphinxupquote{kappa}} \textendash{} opacity

\item {} 
\sphinxstyleliteralstrong{\sphinxupquote{kappa\_gamma}} \textendash{} gamma\sphinxhyphen{}ray opacity

\item {} 
\sphinxstyleliteralstrong{\sphinxupquote{mej}} \textendash{} ejecta mass

\item {} 
\sphinxstyleliteralstrong{\sphinxupquote{vej}} \textendash{} ejecta velocity

\item {} 
\sphinxstyleliteralstrong{\sphinxupquote{area\_projection}} \textendash{} projected area of cocoon/polar ejecta

\item {} 
\sphinxstyleliteralstrong{\sphinxupquote{area\_reference}} \textendash{} remaining reference area i.e., the equitorial ejecta

\end{itemize}

\end{description}\end{quote}

Adds new attributes for tau\_diffusion and new luminosity accounting for the interaction process

\end{fulllineitems}

\subsubsection*{Methods}


\begin{savenotes}\sphinxatlongtablestart\begin{longtable}[c]{\X{1}{2}\X{1}{2}}
\hline

\endfirsthead

\multicolumn{2}{c}%
{\makebox[0pt]{\sphinxtablecontinued{\tablename\ \thetable{} \textendash{} continued from previous page}}}\\
\hline

\endhead

\hline
\multicolumn{2}{r}{\makebox[0pt][r]{\sphinxtablecontinued{Continued on next page}}}\\
\endfoot

\endlastfoot

{\hyperref[\detokenize{generated/sdapy.interaction_processes.AsphericalDiffusion:sdapy.interaction_processes.AsphericalDiffusion.__init__}]{\sphinxcrossref{\sphinxcode{\sphinxupquote{\_\_init\_\_}}}}}(self, time, luminosity, kappa, …)
&
\begin{quote}\begin{description}
\item[{param time}] \leavevmode
source frame time in days

\end{description}\end{quote}

\\
\hline
\sphinxcode{\sphinxupquote{convert\_input\_luminosity}}(self)
&

\\
\hline
\end{longtable}\sphinxatlongtableend\end{savenotes}

\end{fulllineitems}



\subsubsection{sdapy.interaction\_processes.CSMDiffusion}
\label{\detokenize{generated/sdapy.interaction_processes.CSMDiffusion:sdapy-interaction-processes-csmdiffusion}}\label{\detokenize{generated/sdapy.interaction_processes.CSMDiffusion::doc}}\index{CSMDiffusion (class in sdapy.interaction\_processes)@\spxentry{CSMDiffusion}\spxextra{class in sdapy.interaction\_processes}}

\begin{fulllineitems}
\phantomsection\label{\detokenize{generated/sdapy.interaction_processes.CSMDiffusion:sdapy.interaction_processes.CSMDiffusion}}\pysiglinewithargsret{\sphinxbfcode{\sphinxupquote{class }}\sphinxcode{\sphinxupquote{sdapy.interaction\_processes.}}\sphinxbfcode{\sphinxupquote{CSMDiffusion}}}{\emph{time}, \emph{luminosity}, \emph{kappa}, \emph{r\_photosphere}, \emph{mass\_csm\_threshold}, \emph{csm\_mass}}{}~\subsubsection*{Methods}


\begin{savenotes}\sphinxattablestart
\centering
\begin{tabulary}{\linewidth}[t]{|T|T|}
\hline

\sphinxstylestrong{convert\_input\_luminosity}
&\\
\hline
\end{tabulary}
\par
\sphinxattableend\end{savenotes}
\index{\_\_init\_\_() (sdapy.interaction\_processes.CSMDiffusion method)@\spxentry{\_\_init\_\_()}\spxextra{sdapy.interaction\_processes.CSMDiffusion method}}

\begin{fulllineitems}
\phantomsection\label{\detokenize{generated/sdapy.interaction_processes.CSMDiffusion:sdapy.interaction_processes.CSMDiffusion.__init__}}\pysiglinewithargsret{\sphinxbfcode{\sphinxupquote{\_\_init\_\_}}}{\emph{self}, \emph{time}, \emph{luminosity}, \emph{kappa}, \emph{r\_photosphere}, \emph{mass\_csm\_threshold}, \emph{csm\_mass}}{}~\begin{quote}\begin{description}
\item[{Parameters}] \leavevmode\begin{itemize}
\item {} 
\sphinxstyleliteralstrong{\sphinxupquote{time}} \textendash{} source frame time in days

\item {} 
\sphinxstyleliteralstrong{\sphinxupquote{luminosity}} \textendash{} luminosity

\item {} 
\sphinxstyleliteralstrong{\sphinxupquote{kappa}} \textendash{} opacity

\item {} 
\sphinxstyleliteralstrong{\sphinxupquote{csm\_mass}} \textendash{} csm mass in solar masses

\item {} 
\sphinxstyleliteralstrong{\sphinxupquote{mej}} \textendash{} ejecta mass in solar masses

\item {} 
\sphinxstyleliteralstrong{\sphinxupquote{r0}} \textendash{} radius of csm shell in AU

\item {} 
\sphinxstyleliteralstrong{\sphinxupquote{eta}} \textendash{} csm density profile exponent

\item {} 
\sphinxstyleliteralstrong{\sphinxupquote{rho}} \textendash{} csm density profile amplitude

\end{itemize}

\end{description}\end{quote}

Adds new attribute for luminosity accounting for the interaction process

\end{fulllineitems}

\subsubsection*{Methods}


\begin{savenotes}\sphinxatlongtablestart\begin{longtable}[c]{\X{1}{2}\X{1}{2}}
\hline

\endfirsthead

\multicolumn{2}{c}%
{\makebox[0pt]{\sphinxtablecontinued{\tablename\ \thetable{} \textendash{} continued from previous page}}}\\
\hline

\endhead

\hline
\multicolumn{2}{r}{\makebox[0pt][r]{\sphinxtablecontinued{Continued on next page}}}\\
\endfoot

\endlastfoot

{\hyperref[\detokenize{generated/sdapy.interaction_processes.CSMDiffusion:sdapy.interaction_processes.CSMDiffusion.__init__}]{\sphinxcrossref{\sphinxcode{\sphinxupquote{\_\_init\_\_}}}}}(self, time, luminosity, kappa, …)
&
\begin{quote}\begin{description}
\item[{param time}] \leavevmode
source frame time in days

\end{description}\end{quote}

\\
\hline
\sphinxcode{\sphinxupquote{convert\_input\_luminosity}}(self)
&

\\
\hline
\end{longtable}\sphinxatlongtableend\end{savenotes}

\end{fulllineitems}



\subsubsection{sdapy.interaction\_processes.Viscous}
\label{\detokenize{generated/sdapy.interaction_processes.Viscous:sdapy-interaction-processes-viscous}}\label{\detokenize{generated/sdapy.interaction_processes.Viscous::doc}}\index{Viscous (class in sdapy.interaction\_processes)@\spxentry{Viscous}\spxextra{class in sdapy.interaction\_processes}}

\begin{fulllineitems}
\phantomsection\label{\detokenize{generated/sdapy.interaction_processes.Viscous:sdapy.interaction_processes.Viscous}}\pysiglinewithargsret{\sphinxbfcode{\sphinxupquote{class }}\sphinxcode{\sphinxupquote{sdapy.interaction\_processes.}}\sphinxbfcode{\sphinxupquote{Viscous}}}{\emph{time}, \emph{luminosity}, \emph{t\_viscous}}{}~\subsubsection*{Methods}


\begin{savenotes}\sphinxattablestart
\centering
\begin{tabulary}{\linewidth}[t]{|T|T|}
\hline

\sphinxstylestrong{convert\_input\_luminosity}
&\\
\hline
\end{tabulary}
\par
\sphinxattableend\end{savenotes}
\index{\_\_init\_\_() (sdapy.interaction\_processes.Viscous method)@\spxentry{\_\_init\_\_()}\spxextra{sdapy.interaction\_processes.Viscous method}}

\begin{fulllineitems}
\phantomsection\label{\detokenize{generated/sdapy.interaction_processes.Viscous:sdapy.interaction_processes.Viscous.__init__}}\pysiglinewithargsret{\sphinxbfcode{\sphinxupquote{\_\_init\_\_}}}{\emph{self}, \emph{time}, \emph{luminosity}, \emph{t\_viscous}}{}~\begin{quote}\begin{description}
\item[{Parameters}] \leavevmode\begin{itemize}
\item {} 
\sphinxstyleliteralstrong{\sphinxupquote{time}} \textendash{} source frame time in days

\item {} 
\sphinxstyleliteralstrong{\sphinxupquote{luminosity}} \textendash{} luminosity

\item {} 
\sphinxstyleliteralstrong{\sphinxupquote{t\_viscous}} \textendash{} viscous timescale

\end{itemize}

\end{description}\end{quote}

Adds new attribute for luminosity accounting for the interaction process

\end{fulllineitems}

\subsubsection*{Methods}


\begin{savenotes}\sphinxatlongtablestart\begin{longtable}[c]{\X{1}{2}\X{1}{2}}
\hline

\endfirsthead

\multicolumn{2}{c}%
{\makebox[0pt]{\sphinxtablecontinued{\tablename\ \thetable{} \textendash{} continued from previous page}}}\\
\hline

\endhead

\hline
\multicolumn{2}{r}{\makebox[0pt][r]{\sphinxtablecontinued{Continued on next page}}}\\
\endfoot

\endlastfoot

{\hyperref[\detokenize{generated/sdapy.interaction_processes.Viscous:sdapy.interaction_processes.Viscous.__init__}]{\sphinxcrossref{\sphinxcode{\sphinxupquote{\_\_init\_\_}}}}}(self, time, luminosity, t\_viscous)
&
\begin{quote}\begin{description}
\item[{param time}] \leavevmode
source frame time in days

\end{description}\end{quote}

\\
\hline
\sphinxcode{\sphinxupquote{convert\_input\_luminosity}}(self)
&

\\
\hline
\end{longtable}\sphinxatlongtableend\end{savenotes}

\end{fulllineitems}


\sphinxstylestrong{specline \sphinxhyphen{}\textgreater{} Gaussian Models}


\begin{savenotes}\sphinxatlongtablestart\begin{longtable}[c]{\X{1}{2}\X{1}{2}}
\hline

\endfirsthead

\multicolumn{2}{c}%
{\makebox[0pt]{\sphinxtablecontinued{\tablename\ \thetable{} \textendash{} continued from previous page}}}\\
\hline

\endhead

\hline
\multicolumn{2}{r}{\makebox[0pt][r]{\sphinxtablecontinued{Continued on next page}}}\\
\endfoot

\endlastfoot

{\hyperref[\detokenize{generated/sdapy.models.gauss.gauss:sdapy.models.gauss.gauss}]{\sphinxcrossref{\sphinxcode{\sphinxupquote{gauss}}}}}(x, H, A, x0, sigma)
&

\\
\hline
{\hyperref[\detokenize{generated/sdapy.models.gauss.double_gauss:sdapy.models.gauss.double_gauss}]{\sphinxcrossref{\sphinxcode{\sphinxupquote{double\_gauss}}}}}(x, H, A1, x01, sigma1, A2, x02, …)
&

\\
\hline
\end{longtable}\sphinxatlongtableend\end{savenotes}


\subsubsection{sdapy.models.gauss.gauss}
\label{\detokenize{generated/sdapy.models.gauss.gauss:sdapy-models-gauss-gauss}}\label{\detokenize{generated/sdapy.models.gauss.gauss::doc}}\index{gauss() (in module sdapy.models.gauss)@\spxentry{gauss()}\spxextra{in module sdapy.models.gauss}}

\begin{fulllineitems}
\phantomsection\label{\detokenize{generated/sdapy.models.gauss.gauss:sdapy.models.gauss.gauss}}\pysiglinewithargsret{\sphinxcode{\sphinxupquote{sdapy.models.gauss.}}\sphinxbfcode{\sphinxupquote{gauss}}}{\emph{x}, \emph{H}, \emph{A}, \emph{x0}, \emph{sigma}}{}~
\end{fulllineitems}



\subsubsection{sdapy.models.gauss.double\_gauss}
\label{\detokenize{generated/sdapy.models.gauss.double_gauss:sdapy-models-gauss-double-gauss}}\label{\detokenize{generated/sdapy.models.gauss.double_gauss::doc}}\index{double\_gauss() (in module sdapy.models.gauss)@\spxentry{double\_gauss()}\spxextra{in module sdapy.models.gauss}}

\begin{fulllineitems}
\phantomsection\label{\detokenize{generated/sdapy.models.gauss.double_gauss:sdapy.models.gauss.double_gauss}}\pysiglinewithargsret{\sphinxcode{\sphinxupquote{sdapy.models.gauss.}}\sphinxbfcode{\sphinxupquote{double\_gauss}}}{\emph{x}, \emph{H}, \emph{A1}, \emph{x01}, \emph{sigma1}, \emph{A2}, \emph{x02}, \emph{sigma2}}{}~
\end{fulllineitems}


\sphinxstylestrong{specline \sphinxhyphen{}\textgreater{} Voigt Models}


\begin{savenotes}\sphinxatlongtablestart\begin{longtable}[c]{\X{1}{2}\X{1}{2}}
\hline

\endfirsthead

\multicolumn{2}{c}%
{\makebox[0pt]{\sphinxtablecontinued{\tablename\ \thetable{} \textendash{} continued from previous page}}}\\
\hline

\endhead

\hline
\multicolumn{2}{r}{\makebox[0pt][r]{\sphinxtablecontinued{Continued on next page}}}\\
\endfoot

\endlastfoot

{\hyperref[\detokenize{generated/sdapy.models.voigt.voigt:sdapy.models.voigt.voigt}]{\sphinxcrossref{\sphinxcode{\sphinxupquote{voigt}}}}}(x, H, A, x0, sigma, gamma)
&

\\
\hline
\end{longtable}\sphinxatlongtableend\end{savenotes}


\subsubsection{sdapy.models.voigt.voigt}
\label{\detokenize{generated/sdapy.models.voigt.voigt:sdapy-models-voigt-voigt}}\label{\detokenize{generated/sdapy.models.voigt.voigt::doc}}\index{voigt() (in module sdapy.models.voigt)@\spxentry{voigt()}\spxextra{in module sdapy.models.voigt}}

\begin{fulllineitems}
\phantomsection\label{\detokenize{generated/sdapy.models.voigt.voigt:sdapy.models.voigt.voigt}}\pysiglinewithargsret{\sphinxcode{\sphinxupquote{sdapy.models.voigt.}}\sphinxbfcode{\sphinxupquote{voigt}}}{\emph{x}, \emph{H}, \emph{A}, \emph{x0}, \emph{sigma}, \emph{gamma}}{}~
\end{fulllineitems}


\sphinxstylestrong{specv\_evolution \sphinxhyphen{}\textgreater{} Exponential Models}


\begin{savenotes}\sphinxatlongtablestart\begin{longtable}[c]{\X{1}{2}\X{1}{2}}
\hline

\endfirsthead

\multicolumn{2}{c}%
{\makebox[0pt]{\sphinxtablecontinued{\tablename\ \thetable{} \textendash{} continued from previous page}}}\\
\hline

\endhead

\hline
\multicolumn{2}{r}{\makebox[0pt][r]{\sphinxtablecontinued{Continued on next page}}}\\
\endfoot

\endlastfoot

{\hyperref[\detokenize{generated/sdapy.models.exponential.exp:sdapy.models.exponential.exp}]{\sphinxcrossref{\sphinxcode{\sphinxupquote{exp}}}}}(t, a, t0, b, c)
&

\\
\hline
\end{longtable}\sphinxatlongtableend\end{savenotes}


\subsubsection{sdapy.models.exponential.exp}
\label{\detokenize{generated/sdapy.models.exponential.exp:sdapy-models-exponential-exp}}\label{\detokenize{generated/sdapy.models.exponential.exp::doc}}\index{exp() (in module sdapy.models.exponential)@\spxentry{exp()}\spxextra{in module sdapy.models.exponential}}

\begin{fulllineitems}
\phantomsection\label{\detokenize{generated/sdapy.models.exponential.exp:sdapy.models.exponential.exp}}\pysiglinewithargsret{\sphinxcode{\sphinxupquote{sdapy.models.exponential.}}\sphinxbfcode{\sphinxupquote{exp}}}{\emph{t}, \emph{a}, \emph{t0}, \emph{b}, \emph{c}}{}~
\end{fulllineitems}


\sphinxstylestrong{specv\_evolution \sphinxhyphen{}\textgreater{} Polynomial Models}


\begin{savenotes}\sphinxatlongtablestart\begin{longtable}[c]{\X{1}{2}\X{1}{2}}
\hline

\endfirsthead

\multicolumn{2}{c}%
{\makebox[0pt]{\sphinxtablecontinued{\tablename\ \thetable{} \textendash{} continued from previous page}}}\\
\hline

\endhead

\hline
\multicolumn{2}{r}{\makebox[0pt][r]{\sphinxtablecontinued{Continued on next page}}}\\
\endfoot

\endlastfoot

{\hyperref[\detokenize{generated/sdapy.models.polynomial.linear:sdapy.models.polynomial.linear}]{\sphinxcrossref{\sphinxcode{\sphinxupquote{linear}}}}}(x, a, b)
&

\\
\hline
{\hyperref[\detokenize{generated/sdapy.models.polynomial.poly2:sdapy.models.polynomial.poly2}]{\sphinxcrossref{\sphinxcode{\sphinxupquote{poly2}}}}}(x, a, b, c)
&

\\
\hline
{\hyperref[\detokenize{generated/sdapy.models.polynomial.poly3:sdapy.models.polynomial.poly3}]{\sphinxcrossref{\sphinxcode{\sphinxupquote{poly3}}}}}(x, a, b, c, d)
&

\\
\hline
{\hyperref[\detokenize{generated/sdapy.models.polynomial.poly4:sdapy.models.polynomial.poly4}]{\sphinxcrossref{\sphinxcode{\sphinxupquote{poly4}}}}}(x, a, b, c, d, e)
&

\\
\hline
{\hyperref[\detokenize{generated/sdapy.models.polynomial.poly5:sdapy.models.polynomial.poly5}]{\sphinxcrossref{\sphinxcode{\sphinxupquote{poly5}}}}}(x, a, b, c, d, e, f)
&

\\
\hline
{\hyperref[\detokenize{generated/sdapy.models.polynomial.poly6:sdapy.models.polynomial.poly6}]{\sphinxcrossref{\sphinxcode{\sphinxupquote{poly6}}}}}(x, a, b, c, d, e, f, g)
&

\\
\hline
\end{longtable}\sphinxatlongtableend\end{savenotes}


\subsubsection{sdapy.models.polynomial.linear}
\label{\detokenize{generated/sdapy.models.polynomial.linear:sdapy-models-polynomial-linear}}\label{\detokenize{generated/sdapy.models.polynomial.linear::doc}}\index{linear() (in module sdapy.models.polynomial)@\spxentry{linear()}\spxextra{in module sdapy.models.polynomial}}

\begin{fulllineitems}
\phantomsection\label{\detokenize{generated/sdapy.models.polynomial.linear:sdapy.models.polynomial.linear}}\pysiglinewithargsret{\sphinxcode{\sphinxupquote{sdapy.models.polynomial.}}\sphinxbfcode{\sphinxupquote{linear}}}{\emph{x}, \emph{a}, \emph{b}}{}~
\end{fulllineitems}



\subsubsection{sdapy.models.polynomial.poly2}
\label{\detokenize{generated/sdapy.models.polynomial.poly2:sdapy-models-polynomial-poly2}}\label{\detokenize{generated/sdapy.models.polynomial.poly2::doc}}\index{poly2() (in module sdapy.models.polynomial)@\spxentry{poly2()}\spxextra{in module sdapy.models.polynomial}}

\begin{fulllineitems}
\phantomsection\label{\detokenize{generated/sdapy.models.polynomial.poly2:sdapy.models.polynomial.poly2}}\pysiglinewithargsret{\sphinxcode{\sphinxupquote{sdapy.models.polynomial.}}\sphinxbfcode{\sphinxupquote{poly2}}}{\emph{x}, \emph{a}, \emph{b}, \emph{c}}{}~
\end{fulllineitems}



\subsubsection{sdapy.models.polynomial.poly3}
\label{\detokenize{generated/sdapy.models.polynomial.poly3:sdapy-models-polynomial-poly3}}\label{\detokenize{generated/sdapy.models.polynomial.poly3::doc}}\index{poly3() (in module sdapy.models.polynomial)@\spxentry{poly3()}\spxextra{in module sdapy.models.polynomial}}

\begin{fulllineitems}
\phantomsection\label{\detokenize{generated/sdapy.models.polynomial.poly3:sdapy.models.polynomial.poly3}}\pysiglinewithargsret{\sphinxcode{\sphinxupquote{sdapy.models.polynomial.}}\sphinxbfcode{\sphinxupquote{poly3}}}{\emph{x}, \emph{a}, \emph{b}, \emph{c}, \emph{d}}{}~
\end{fulllineitems}



\subsubsection{sdapy.models.polynomial.poly4}
\label{\detokenize{generated/sdapy.models.polynomial.poly4:sdapy-models-polynomial-poly4}}\label{\detokenize{generated/sdapy.models.polynomial.poly4::doc}}\index{poly4() (in module sdapy.models.polynomial)@\spxentry{poly4()}\spxextra{in module sdapy.models.polynomial}}

\begin{fulllineitems}
\phantomsection\label{\detokenize{generated/sdapy.models.polynomial.poly4:sdapy.models.polynomial.poly4}}\pysiglinewithargsret{\sphinxcode{\sphinxupquote{sdapy.models.polynomial.}}\sphinxbfcode{\sphinxupquote{poly4}}}{\emph{x}, \emph{a}, \emph{b}, \emph{c}, \emph{d}, \emph{e}}{}~
\end{fulllineitems}



\subsubsection{sdapy.models.polynomial.poly5}
\label{\detokenize{generated/sdapy.models.polynomial.poly5:sdapy-models-polynomial-poly5}}\label{\detokenize{generated/sdapy.models.polynomial.poly5::doc}}\index{poly5() (in module sdapy.models.polynomial)@\spxentry{poly5()}\spxextra{in module sdapy.models.polynomial}}

\begin{fulllineitems}
\phantomsection\label{\detokenize{generated/sdapy.models.polynomial.poly5:sdapy.models.polynomial.poly5}}\pysiglinewithargsret{\sphinxcode{\sphinxupquote{sdapy.models.polynomial.}}\sphinxbfcode{\sphinxupquote{poly5}}}{\emph{x}, \emph{a}, \emph{b}, \emph{c}, \emph{d}, \emph{e}, \emph{f}}{}~
\end{fulllineitems}



\subsubsection{sdapy.models.polynomial.poly6}
\label{\detokenize{generated/sdapy.models.polynomial.poly6:sdapy-models-polynomial-poly6}}\label{\detokenize{generated/sdapy.models.polynomial.poly6::doc}}\index{poly6() (in module sdapy.models.polynomial)@\spxentry{poly6()}\spxextra{in module sdapy.models.polynomial}}

\begin{fulllineitems}
\phantomsection\label{\detokenize{generated/sdapy.models.polynomial.poly6:sdapy.models.polynomial.poly6}}\pysiglinewithargsret{\sphinxcode{\sphinxupquote{sdapy.models.polynomial.}}\sphinxbfcode{\sphinxupquote{poly6}}}{\emph{x}, \emph{a}, \emph{b}, \emph{c}, \emph{d}, \emph{e}, \emph{f}, \emph{g}}{}~
\end{fulllineitems}



\chapter{Indices and tables}
\label{\detokenize{index:indices-and-tables}}\begin{itemize}
\item {} 
\DUrole{xref,std,std-ref}{genindex}

\item {} 
\DUrole{xref,std,std-ref}{modindex}

\item {} 
\DUrole{xref,std,std-ref}{search}

\end{itemize}


\chapter{Links}
\label{\detokenize{index:links}}\begin{itemize}
\item {} 
\sphinxhref{https://github.com/saberyoung/sn\_data\_analysis}{Source code}

\item {} 
\sphinxhref{https://haffet.readthedocs.io/}{Docs}

\item {} 
\sphinxhref{https://github.com/saberyoung/sn\_data\_analysis/issues}{Issues}

\end{itemize}


\chapter{Author}
\label{\detokenize{index:author}}
\sphinxhref{http://www.sngyang.com}{Sheng Yang}: \sphinxhref{mailto:saberyoung@gmail.com}{saberyoung@gmail.com}



\renewcommand{\indexname}{Index}
\printindex
\end{document}